\documentclass[dvipdfmx,a4paper,11pt]{article}
\usepackage[utf8]{inputenc}
%\usepackage[dvipdfmx]{hyperref} %リンクを有効にする
\usepackage{url} %同上
\usepackage{amsmath,amssymb} %もちろん
\usepackage{amsfonts,amsthm,mathtools} %もちろん
\usepackage{braket,physics} %あると便利なやつ
\usepackage{bm} %ラプラシアンで使った
\usepackage[top=30truemm,bottom=30truemm,left=25truemm,right=25truemm]{geometry} %余白設定
\usepackage{latexsym} %ごくたまに必要になる
\renewcommand{\kanjifamilydefault}{\gtdefault}
\usepackage{otf} %宗教上の理由でmin10が嫌いなので


\usepackage[all]{xy}
\usepackage{amsthm,amsmath,amssymb,comment}
\usepackage{amsmath}    % \UTF{00E6}\UTF{0095}°\UTF{00E5}\UTF{00AD}\UTF{00A6}\UTF{00E7}\UTF{0094}¨
\usepackage{amssymb}  
\usepackage{color}
\usepackage{amscd}
\usepackage{amsthm}  
\usepackage{wrapfig}
\usepackage{comment}	
\usepackage{graphicx}
\usepackage{setspace}
\usepackage{pxrubrica}
\usepackage{enumitem}
\usepackage{mathrsfs} 
\usepackage[dvipdfmx]{hyperref}
\setstretch{1.2}


\newcommand{\R}{\mathbb{R}}
\newcommand{\Z}{\mathbb{Z}}
\newcommand{\Q}{\mathbb{Q}} 
\newcommand{\N}{\mathbb{N}}
\newcommand{\C}{\mathbb{C}} 
\newcommand{\Sin}{\text{Sin}^{-1}} 
\newcommand{\Cos}{\text{Cos}^{-1}} 
\newcommand{\Tan}{\text{Tan}^{-1}} 
\newcommand{\invsin}{\text{Sin}^{-1}} 
\newcommand{\invcos}{\text{Cos}^{-1}} 
\newcommand{\invtan}{\text{Tan}^{-1}} 
\newcommand{\Area}{S}
\newcommand{\vol}{\text{Vol}}
\newcommand{\maru}[1]{\raise0.2ex\hbox{\textcircled{\tiny{#1}}}}
\newcommand{\sgn}{{\rm sgn}}
%\newcommand{\rank}{{\rm rank}}



   %当然のようにやる.
\allowdisplaybreaks[4]
   %もちろん.
%\title{第1回. 多変数の連続写像 (岩井雅崇, 2020/10/06)}
%\author{岩井雅崇}
%\date{2020/10/06}
%ここまで今回の記事関係ない
\usepackage{tcolorbox}
\tcbuselibrary{breakable, skins, theorems}

\theoremstyle{definition}
\newtheorem{thm}{定理}
\newtheorem{lem}[thm]{補題}
\newtheorem{prop}[thm]{命題}
\newtheorem{cor}[thm]{系}
\newtheorem{claim}[thm]{主張}
\newtheorem{dfn}[thm]{定義}
\newtheorem{rem}[thm]{注意}
\newtheorem{exa}[thm]{例}
\newtheorem{conj}[thm]{予想}
\newtheorem{prob}[thm]{問題}
\newtheorem{rema}[thm]{補足}
\newtheorem{dfnthm}[thm]{定義・定理}

\DeclareMathOperator{\Ric}{Ric}
\DeclareMathOperator{\Vol}{Vol}
 \newcommand{\pdrv}[2]{\frac{\partial #1}{\partial #2}}
 \newcommand{\drv}[2]{\frac{d #1}{d#2}}
  \newcommand{\ppdrv}[3]{\frac{\partial #1}{\partial #2 \partial #3}}


\begin{document}
\begin{center}
{\Large 基礎解析学2 -一変数と多変数の積分- 演習問題}
\end{center}

\begin{flushright}
 岩井雅崇 2023/01/10
\end{flushright}

\section{期末試験の日時・内容}

期末試験の情報は次のとおりです.
\begin{enumerate}
\setlength{\parskip}{0cm}
  \setlength{\itemsep}{2pt} 
\item 期末試験の日時は2023年2月7日(火)2限(10:30-12:00)です. 試験開始5分前には着席しておいてください.
\item 期末試験の場所は豊中総合学館302です. 
\item 教科書・ノート等の持ち込みはできません. スマートフォン・携帯も使用できません.
\end{enumerate}

\hspace{-20pt}
期末試験の問題は以下のものを考えております.

\begin{itemize}
\setlength{\parskip}{0cm}
  \setlength{\itemsep}{2pt} 
\item 重積分の計算問題(2次元). 「第5回授業の重積分2 -累次積分」「第6回授業の重積分3 -重積分の変数変換公式-」の内容.
\item 重積分の計算問題(3次元). 「第7回授業の重積分4 -3次元の積分と体積-」の内容.
\item 広義積分. 「第8回授業の広義積分1-広義積分の定義とガウス積分-」の内容.
\end{itemize}
これらの内容が期末試験のメインとなります. これらの内容で100点程度を考えております. 

またおまけの問題は「第4回授業の重積分1 -重積分の定義-」と「第2回授業の一変数の積分1 -微分積分学の基本定理-」を考えております. これは解けることを想定していない問題です. 10点くらいを考えております. 

以下110点(ぐらい)で出来に応じて何か修正をします. 基本的な計算問題を解けていれば不可になることはありません .逆を返すと「この計算問題は間違えるはずがない」という問題に関してありえない解答をしていれば, 不可になります. \footnote{正解なら0点で間違えれば-20点という問題を出す人もいますが, 今回はやめておきました.}


 \vspace{11pt}
\begin{wrapfigure}{r}[0pt]{0.2\textwidth}
  \centering
 \includegraphics[height=25mm, width=25mm]{cal2.png}
\end{wrapfigure}

この演習の問題及び授業の資料・板書内容は授業ページ(\url{https://masataka123.github.io/2022_winter_int/})にもあります. 
右下のQRコードからを読み込んでも構いません.

\newpage 
\section{演習問題}

\hspace{11pt}
{\Large 第1問.} 重積分の計算問題(2次元)
\vspace{11pt}

{\large(1). $D=\{ (x,y) \in \R^2 \,|\, 0 \leqq y, \text{\,} 0 \leqq x-y, \text{\,} x+y \leqq 2\}$とする.
重積分$\iint_{D} (x^2-y^2)dxdy$の値を求めよ.}\vspace{5pt}

{\large(2). $D= \{ (x,y)\in \R^2 \,|\, \sqrt{x^2 + y^2 } \leqq 1  \}$とする. 
重積分$\iint_{D} e^{-x^2-y^2}dxdy$の値を求めよ.}\vspace{5pt}

{\large(3). $D=\{ (x,y) \in \R^2  \,|\, x^2 + y^2 \leqq x\}$とする.
重積分$\iint_{D} \sqrt{x}dxdy$の値を求めよ.}\vspace{5pt}

{\large(4). $D=\{ (x,y) \in \R^2 \,|\, 0 \leqq x, \text{\,}  0 \leqq y,\text{\,} \sqrt{x} + \sqrt{y} \leqq 1\}$とする.
重積分$\iint_{D} x^2dxdy$の値を求めよ.}

\vspace{22pt}
{\Large 第2問.} 重積分の計算問題(3次元) 
\vspace{11pt}

{\large(1). $a$を正の実数とする. 円柱$V_1= \{ (x,y,z) \in \R^3 \,|\,x^2 + y^2 \leqq ax\}$と球$V_2 = \{ (x,y,z) \in \R^3 \,|\, x^2 + y^2 + z^2 \leqq a^2 \}$の共通部分$V_1 \cap V_2$の体積を求めよ.}\vspace{4pt}

{\large(2). $a$を正の実数とする. 円柱$V_1= \{ (x,y,z) \in \R^3 \,|\,x^2 + y^2 \leqq a^2\}$と球$V_2 = \{ (x,y,z) \in \R^3 \,|\, x^2 + y^2 + z^2 \leqq 4a^2 \}$の共通部分$V_1 \cap V_2$の体積を求めよ.}\vspace{4pt}

{\large(3). $ V = \{ (x,y,z) \in \R^3\,|\, x^2 + y^2 + z^2 \leqq 1\}$とする. 重積分
$\iiint_{V} x dxdydz$ の値を求めよ.}


\vspace{22pt}
{\Large 第3問.} 広義積分
\vspace{11pt}



{\large$p$を実数とし$f(x) = x^p \log x$とする. 次の問いに答えよ. }
\vspace{11pt}

{\large(1). $p< -1$ならば広義積分$\int_{1}^{\infty} f(x) dx$は収束することを示せ.}\vspace{4pt}

{\large(2). $p\geqq -1$ならば広義積分$\int_{1}^{\infty} f(x) dx$は発散することを示せ.}



\newpage
\section{演習問題の解答}

\hspace{11pt}
{\Large 第1問.} 重積分の計算問題(2次元)
\vspace{11pt}

{\large(1). $D=\{ (x,y) \in \R^2 \,|\, 0 \leqq y, \text{\,} 0 \leqq x-y, \text{\,} x+y \leqq 2\}$とする.
重積分$\iint_{D} (x^2-y^2)dxdy$の値を求めよ.}\vspace{5pt}

\hspace{-11pt}(解.) 
   $D= \{ (x,y) \in \R^2 \,|\, 0 \leqq y \leqq 1, 
y \leqq x  \leqq 2-y \}$
より累次積分を使用して
 
\begin{align*}
\begin{split}
\iint_{D} (x^2 -y^2) dxdy
&= \int_{0}^{1} 
 \int_{y}^{2-y} (x^2 -y^2) dxdy =\int_{0}^{1} 
 \left[ \frac{x^3}{3} - x y^2 \right]_{y}^{2-y} dy=
\int_{0}^{1} 
\left( \frac{4 y^3}{3} -4y + \frac{8}{3} \right) dy
=1.
    \end{split}
  \end{align*}
  

{\large(2). $D= \{ (x,y)\in \R^2 | \sqrt{x^2 + y^2 } \leqq 1  \}$とする. 
重積分$\iint_{D} e^{-x^2-y^2}dxdy$の値を求めよ. }

\hspace{-11pt}(解.) 
$E=[0,1]\times[0,2\pi]$とし, 
 $$
\begin{array}{ccccc}
\Phi: &E & \rightarrow & \R^2 & \\
&(r,\theta) & \longmapsto & (r \cos \theta , r \sin \theta)&
\end{array}
$$
とすると, 多重積分の変数変換の公式から
\begin{align*}
\begin{split}
\iint_{D} e^{-x^2-y^2}dxdy
&=
\iint_{E} e^{-(r \cos \theta )^2- (r \sin \theta )^2} |r|drd\theta =
\iint_{E} e^{-r^2}r drd\theta \\
&=
\int_{0}^{2\pi} \left( \int_{0}^{1}e^{-r^2}r dr\right)d\theta 
=\int_{0}^{2\pi} \left[ \frac{-e^{-r^2}}{2} \right]_{0}^{1} d\theta 
=\int_{0}^{2\pi} \frac{1-e^{-1}}{2} d\theta =\pi\left( 1-\frac{1}{e}\right).
    \end{split}
  \end{align*}
  

{\large(3). $D=\{ (x,y) \in \R^2 | x^2 + y^2 \leqq x\}$とする.
重積分$\iint_{D} \sqrt{x}dxdy$の値を求めよ.}

\hspace{-11pt}(解.) $E= \{ (r ,\theta) \in \R^2  | 0 \leqq r \leqq \cos \theta, 
 -\frac{\pi}{2}\leqq \theta  \leqq\frac{\pi}{2}\}$とし, 
 $$
\begin{array}{ccccc}
\Phi: &E & \rightarrow & \R^2 & \\
&(r,\theta) & \longmapsto & (r \cos \theta , r \sin \theta)&
\end{array}
$$
とすると, 多重積分の変数変換の公式から
\begin{align*}
\begin{split}
\iint_{D} \sqrt{x}dxdy
&=
\iint_{E}  (r\cos \theta)^{\frac{1}{2}} rdrd\theta 
= \int_{- \frac{\pi}{2}}^{\frac{\pi}{2}} 
 \int_{0}^{\cos \theta} r^{\frac{3}{2}} (\cos \theta)^{\frac{1}{2}} drd\theta \\
&=
\int_{- \frac{\pi}{2}}^{\frac{\pi}{2}} (\cos \theta)^{\frac{1}{2}} 
 \left[ \frac{2}{5}r^{\frac{5}{2}} \right]_{0}^{\cos \theta} d\theta 
=\frac{2}{5} \int_{- \frac{\pi}{2}}^{\frac{\pi}{2}} (\cos \theta)^{3} d\theta 
=\frac{8}{15}.
    \end{split}
  \end{align*}


{\large(4). $D=\{ (x,y) \in \R^2 : 0 \leqq x, \text{\,}  0 \leqq y,\text{\,} \sqrt{x} + \sqrt{y} \leqq 1\}$とする.
重積分$\iint_{D} x^2dxdy$の値を求めよ.}

\hspace{-11pt}(解.)  $E= \{ (u ,v) \in \R^2 : 0 \leqq u \leqq 1,  
0 \leqq v \leqq 1-u \}$とし, 
 $$
\begin{array}{ccccc}
\Phi: &E & \rightarrow & \R^2 & \\
&(u,v) & \longmapsto & (u^2 , v^2)&
\end{array}
$$
とすると, 多重積分の変数変換の公式から
\begin{align*}
\begin{split}
\iint_{D} x^2 dxdy
&=
\iint_{E} u^4(4uv) dudv = \int_{0}^{1} 
 \int_{0}^{1-u} 4 u^5 v \, dvdu \\
&=
\int_{0}^{1} 4u^5
 \left[ \frac{v^2}{2} \right]_{0}^{1-u} du
=2\int_{0}^{1} \left( u^5 - 2u^6 + u^7 \right) du
=\frac{1}{84}.
    \end{split}
  \end{align*}

\vspace{22pt}
{\Large 第2問.} 重積分の計算問題(3次元) 
\vspace{11pt}

{\large(1). $a$を正の実数とする. 円柱$V_1= \{ (x,y,z) \in \R^3 \,|\,x^2 + y^2 \leqq ax\}$と球$V_2 = \{ (x,y,z) \in \R^3 \,|\, x^2 + y^2 + z^2 \leqq a^2 \}$の共通部分$V_1 \cap V_2$の体積を求めよ.}

\hspace{-11pt}(解.)
$D=\{ (x,y) \in \R^2 | x^2 + y^2 \leqq ax\}$とおくと
$$
V_1 \cap V_2= \{ (x,y,z) \in \R^3 \,|\, (x,y) \in D, -\sqrt{a^2 -x^2 -y^2} \leqq z \leqq  \sqrt{a^2 -x^2 -y^2} \}
$$
である. よって$V_1 \cap V_2$の体積は
\begin{align*}
\begin{split}
\iiint_{V} dxdydz 
&=
\iint_{D} \left( \int_{-\sqrt{a^2 -x^2 -y^2}}^{\sqrt{a^2 -x^2 -y^2} } dz \right) dxdy
= \iint_{D} 2\sqrt{a^2 -x^2 -y^2}dxdy \\
&=  \int_{- \frac{\pi}{2}}^{\frac{\pi}{2}}\int_{0}^{a \cos \theta} 2\sqrt{a^2 - r^2} \,r drd\theta\quad \text{{\footnotesize(第1問(3)に同じ.)}}\\
&=\int_{- \frac{\pi}{2}}^{\frac{\pi}{2}} \left[ \frac{-2}{3} (a^2 - r^2)^{\frac{3}{2}}\right]_{0}^{a \cos \theta}d\theta \\
&=\frac{2 a^3}{3}\int_{- \frac{\pi}{2}}^{\frac{\pi}{2}} 1 - |\sin \theta|^3 d\theta 
=\frac{4 a^3}{3}\int_{0}^{\frac{\pi}{2}} 1 - (\sin \theta)^3 d\theta
= \frac{4 a^3}{3}\left( \frac{\pi}{2} - \frac{2}{3}\right).
    \end{split}
  \end{align*}




{\large(2). $a$を正の実数とする. 円柱$V_1= \{ (x,y,z) \in \R^3 \,|\,x^2 + y^2 \leqq a^2\}$と球$V_2 = \{ (x,y,z) \in \R^3 \,|\, x^2 + y^2 + z^2 \leqq 4a^2 \}$の共通部分$V_1 \cap V_2$の体積を求めよ.}

\hspace{-11pt}(解.)
$D=\{ (x,y) \in \R^2 | x^2 + y^2 \leqq a^2\}$とおくと
$$
V_1 \cap V_2= \{ (x,y,z) \in \R^3 \,|\, (x,y) \in D, -\sqrt{4a^2 -x^2 -y^2} \leqq z \leqq  \sqrt{4a^2 -x^2 -y^2} \}
$$
である. よって$V_1 \cap V_2$の体積は
\begin{align*}
\begin{split}
\iiint_{V} dxdydz 
&=
\iint_{D} \left( \int_{-\sqrt{4a^2 -x^2 -y^2}}^{\sqrt{4a^2 -x^2 -y^2} } dz \right) dxdy
= \iint_{D} 2\sqrt{4a^2 -x^2 -y^2}dxdy \\
&=  \int_{0}^{2\pi}\int_{0}^{a} 2\sqrt{4a^2 - r^2} \,r drd\theta\quad \text{{\footnotesize(第1問(2)に同じ.)}}\\
&=\int_{0}^{2\pi} \left[ \frac{-2}{3} (4a^2 - r^2)^{\frac{3}{2}}\right]_{0}^{a }d\theta  = \int_{0}^{2\pi}\frac{2}{3}(8 - 3 \sqrt{3})a^3 d\theta =\frac{4 \pi a^3}{3}(8 - 3 \sqrt{3}). 
    \end{split}
  \end{align*}

\newpage
{\large(3). $ V = \{ (x,y,z) \in \R^3 | x^2 + y^2 + z^2 \leqq 1\}$とする. 重積分
$\iiint_{V} x dxdydz$ の値を求めよ.}

\hspace{-11pt}(解.)
$D=\{ (x,y) \in \R^2 | x^2 + y^2 \leqq 1\}$とおくと
$$
V = \{ (x,y,z) \in \R^3 \,|\, (x,y) \in D, -\sqrt{1 -x^2 -y^2} \leqq z \leqq  \sqrt{1 -x^2 -y^2} \}
$$
である. よって
\begin{align*}
\begin{split}
\iiint_{V} xdxdydz 
&=
\iint_{D} \left( \int_{-\sqrt{1 -x^2 -y^2}}^{\sqrt{1 -x^2 -y^2} } xdz \right) dxdy
= \iint_{D} 2x\sqrt{1 -x^2 -y^2}dxdy \\
&=  \int_{0}^{2\pi}\int_{0}^{1} 2r \cos \theta \sqrt{1 - r^2} \,r drd\theta\quad \text{{\footnotesize(第1問(2)に同じ.)}}\\
&= \int_{0}^{1} \left(  \int_{0}^{2\pi} 2r^2\sqrt{1 - r^2} \cos \theta  d \theta  \right) dr 
= \int_{0}^{1} \left[ 2r^2\sqrt{1 - r^2} \sin\theta \right]_{0}^{2\pi} dr 
= \int_{0}^{1} 0 dr =0.
    \end{split}
  \end{align*}
  
  \vspace{11pt}
{\Large 第3問.} 広義積分
\vspace{11pt}


{\large$p$を実数とし$f(x) = x^p \log x$とする. }
\vspace{11pt}

{\large(1). $p< -1$ならば広義積分$\int_{1}^{\infty} f(x) dx$は収束することを示せ.}

\hspace{-11pt}(解.)
$p<-1$よりある$\epsilon >0$があって$p+\epsilon<-1$となる.
よって
$$
\lim_{x \rightarrow +\infty}\frac{f(x)}{x^{p+\epsilon}}=
\lim_{x \rightarrow +\infty}\frac{x^p \log x}{x^{p+\epsilon}}
=\lim_{x \rightarrow +\infty}\frac{\log x}{x^{\epsilon}}
=0
$$
であるので, $[1, + \infty)$上で$\frac{f(x)}{x^{p+\epsilon}}$は有界である.
%\footnote{ここを正確に述べると, $\lim_{x \rightarrow +\infty}\frac{f(x)}{x^{p+\epsilon}}=0$より, ある$R>0$があって, $[R, + \infty)$上で$|\frac{f(x)}{x^{p+\epsilon}}| <1$となる. 一方$[1,R]$上で$\frac{f(x)}{x^{p+\epsilon}}$は最大値・最小値を持つので(閉区間上の連続関数は最大値を持つ), ある$M>0$があって,  $[1, R]$上で$|\frac{f(x)}{x^{p+\epsilon}}| <M$となる. よって, $[1, + \infty)$上で$|\frac{f(x)}{x^{p+\epsilon}}| < 1+M$となる. よって有界である.}
つまりある$C>0$があって, $[1, + \infty)$上で$|\frac{f(x)}{x^{p+\epsilon}}| < C$となる.
以上より, $[1, + \infty)$上で$|f(x)| < C{x^{p+\epsilon}}$かつ, 広義積分$\int_{1}^{\infty}C{x^{p+\epsilon}}dx$は$p+\epsilon<-1$より収束するので, \underline{第8回授業の広義積分の収束判定法から}広義積分$\int_{1}^{\infty} f(x) dx$も収束する.

\vspace{11pt}

{\large(2). $p\geqq -1$ならば広義積分$\int_{1}^{\infty} f(x) dx$は発散することを示せ.}

\hspace{-11pt}(解.)
$3 \leqq x$ならば$1 \leqq \log x$であるので, $[3, + \infty)$上で$x^p\leqq x^p\log x$である. また広義積分$\int_{3}^{\infty} x^p dx$は$p\geqq -1$であるので$+ \infty$に発散する. よって\underline{第8回授業の広義積分の収束判定法から}広義積分$\int_{3}^{\infty} f(x) dx$も$+ \infty$に発散する. 以上より広義積分$\int_{1}^{\infty} f(x) dx$も$+ \infty$に発散する.

\vspace{11pt}
[補足1.] 上の議論に関して使っている定理は次のものである.
 \begin{thm}
$[1, +\infty)$上の連続関数$g(x)$について, ある定数$B>0$があって$\lim_{x \rightarrow +\infty}g(x)=B$ならば, ある定数$C>0$があって$g(x) < C$となる. 
 \end{thm}

(証.) $\lim_{x \rightarrow +\infty}g(x)=B$より, ある$R>0$があって, $[R, + \infty)$上で$|g(x)| < B+1$となる. 一方$[1,R]$上で$g(x)$は最大値・最小値を持つので(閉区間上の連続関数は最大値を持つ), ある$M>0$があって,  $[1, R]$上で$|g(x)| <M$となる. よって, $[1, + \infty)$上で$|g(x) |< 1+M+B =C$となる. 
  
[補足2.]  前バージョンで「どのようにして収束判定したのかわからない」というコメントがあった. 使っている収束判定法は次のものである.
  \begin{tcolorbox}[
    colback = white,
    colframe = green!35!black,
    fonttitle = \bfseries,
    breakable = true]
    \begin{thm}
$f(x)$を$[a,b)$上の連続関数とする.
$[a,b)$上の連続関数$g(x)$があって, $[a , b)$上で$|f(x)| \leqq g(x)$かつ広義積分$\int_{a}^{b} g(x) dx$が収束すると仮定する.
このとき広義積分$\int_{a}^{b} f(x) dx$もまた収束する.
 \end{thm}
 \end{tcolorbox}
 
   \begin{tcolorbox}[
    colback = white,
    colframe = green!35!black,
    fonttitle = \bfseries,
    breakable = true]
    \begin{thm}
$f(x)$を$[a,b)$上の連続関数とする.
$[a,b)$上の連続関数$g(x)$があって, $[a , b)$上で$0 \leqq g(x) \leqq f(x)$かつ広義積分$\int_{a}^{b} g(x) dx$が発散すると仮定する.
このとき広義積分$\int_{a}^{b} f(x) dx$もまた発散する.
 \end{thm}
 \end{tcolorbox}

 
 \begin{tcolorbox}[
    colback = white,
    colframe = green!35!black,
    fonttitle = \bfseries,
    breakable = true]
    \begin{thm}[広義積分の判定法]
    \label{kougi}
$f(x)$を$[a,b)$上の連続関数とする.
\begin{enumerate}
\item $b=+ \infty$のとき, ある$\lambda >1$があって, $f(x)x^{\lambda}$が
$[a, +\infty)$上で有界ならば, 広義積分$\int_{a}^{\infty} f(x)dx $は収束する.
\item $b$が実数のとき($b <+ \infty$のとき), ある$\mu <1$があって, $f(x)(x-b)^{\mu}$が
$[a, b)$上で有界ならば, 広義積分$\int_{a}^{b} f(x)dx $は収束する.
\end{enumerate}
 \end{thm}
 \end{tcolorbox}

この解答では「定理2と定理3」または「定理4」を用いて収束判定をしていることがわかる.
 
 \end{document}
 

 
