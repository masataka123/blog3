\documentclass[dvipdfmx,a4paper,11pt]{article}
\usepackage[utf8]{inputenc}
%\usepackage[dvipdfmx]{hyperref} %リンクを有効にする
\usepackage{url} %同上
\usepackage{amsmath,amssymb} %もちろん
\usepackage{amsfonts,amsthm,mathtools} %もちろん
\usepackage{braket,physics} %あると便利なやつ
\usepackage{bm} %ラプラシアンで使った
\usepackage[top=30truemm,bottom=30truemm,left=25truemm,right=25truemm]{geometry} %余白設定
\usepackage{latexsym} %ごくたまに必要になる
\renewcommand{\kanjifamilydefault}{\gtdefault}
\usepackage{otf} %宗教上の理由でmin10が嫌いなので


\usepackage[all]{xy}
\usepackage{amsthm,amsmath,amssymb,comment}
\usepackage{amsmath}    % \UTF{00E6}\UTF{0095}°\UTF{00E5}\UTF{00AD}\UTF{00A6}\UTF{00E7}\UTF{0094}¨
\usepackage{amssymb}  
\usepackage{color}
\usepackage{amscd}
\usepackage{amsthm}  
\usepackage{wrapfig}
\usepackage{comment}	
\usepackage{graphicx}
\usepackage{setspace}
\usepackage{pxrubrica}
\setstretch{1.2}


\newcommand{\R}{\mathbb{R}}
\newcommand{\Z}{\mathbb{Z}}
\newcommand{\Q}{\mathbb{Q}} 
\newcommand{\N}{\mathbb{N}}
\newcommand{\C}{\mathbb{C}} 
\newcommand{\Sin}{\text{Sin}^{-1}} 
\newcommand{\Cos}{\text{Cos}^{-1}} 
\newcommand{\Tan}{\text{Tan}^{-1}} 
\newcommand{\invsin}{\text{Sin}^{-1}} 
\newcommand{\invcos}{\text{Cos}^{-1}} 
\newcommand{\invtan}{\text{Tan}^{-1}} 
\newcommand{\Area}{\text{Area}}
\newcommand{\vol}{\text{Vol}}
\newcommand{\maru}[1]{\raise0.2ex\hbox{\textcircled{\tiny{#1}}}}
\newcommand{\sgn}{{\rm sgn}}
%\newcommand{\rank}{{\rm rank}}



   %当然のようにやる.
\allowdisplaybreaks[4]
   %もちろん.
%\title{第1回. 多変数の連続写像 (岩井雅崇, 2020/10/06)}
%\author{岩井雅崇}
%\date{2020/10/06}
%ここまで今回の記事関係ない
\usepackage{tcolorbox}
\tcbuselibrary{breakable, skins, theorems}

\theoremstyle{definition}
\newtheorem{thm}{定理}
\newtheorem{lem}[thm]{補題}
\newtheorem{prop}[thm]{命題}
\newtheorem{cor}[thm]{系}
\newtheorem{claim}[thm]{主張}
\newtheorem{dfn}[thm]{定義}
\newtheorem{rem}[thm]{注意}
\newtheorem{exa}[thm]{例}
\newtheorem{conj}[thm]{予想}
\newtheorem{prob}[thm]{問題}
\newtheorem{rema}[thm]{補足}

\DeclareMathOperator{\Ric}{Ric}
\DeclareMathOperator{\Vol}{Vol}
 \newcommand{\pdrv}[2]{\frac{\partial #1}{\partial #2}}
 \newcommand{\drv}[2]{\frac{d #1}{d#2}}
  \newcommand{\ppdrv}[3]{\frac{\partial #1}{\partial #2 \partial #3}}


%ここから本文.
\begin{document}
%\maketitle


\begin{center}
{\Large 第2回. 行列の定義 (三宅先生の本, 1.1の内容)}
\end{center}

\begin{flushright}
 岩井雅崇 2022/04/21
\end{flushright}



\section{行列の定義}

\begin{itemize}
\item $m \times n$個の数(実数または複素数) $a_{ij}$ ($i = 1, \ldots, m$, $j = 1, \ldots, n$)を
$$
\begin{bmatrix}
a_{11}& a_{12} & \cdots &a_{1n} \\
a_{21}& a_{22} & \cdots &a_{2n} \\
\vdots& \vdots	&	\ddots   &	\vdots\\
a_{m1}& a_{m2} & \cdots &a_{mn} \\
\end{bmatrix}
\textit{\,\,\ または\,\,\,}
\begin{pmatrix}
a_{11}& a_{12} & \cdots &a_{1n} \\
a_{21}& a_{22} & \cdots &a_{2n} \\
\vdots& \vdots	&	\ddots   &	\vdots \\
a_{m1}& a_{m2} & \cdots &a_{mn} \\
\end{pmatrix}
$$
のように並べたものを \underline{$m$行$n$列の行列}という.
\underline{ $m \times n$行列}, \underline{ $m \times n$型の行列}, \underline{ $(m, n)$行列}ということもある. 
\item 上の行列を$A$としたとき, $a_{ij}$を行列$A$の$(i,j)$成分という. 行列$A$を\underline{$[a_{ij}]_{m\times n}$や$(a_{ij})$}と略記することもある.
\item $\begin{pmatrix} a_{i1} & \cdots & a_{in}\end{pmatrix}$を\underline{$A$の行}といい, 上から第1行, 第2行, $\cdots$, 第$m$行という.
\item $\begin{pmatrix}a_{1j} \\ \vdots  \\ a_{mj}\end{pmatrix}$を\underline{$A$の列}といい, 上から第1列, 第2列, $\cdots$, 第$n$列という.
\item $1 \times n$行列$(a_{11} \cdots a_{1n})$を\underline{行ベクトル}と呼び, $m \times 1$行列$\begin{pmatrix}a_{11} \\ \vdots  \\ a_{m1}\end{pmatrix}$を\underline{列ベクトル}と呼ぶ(この授業や教科書での用語).
\end{itemize}

 \begin{exa}
 行列$A$を次で定める.
 $$
 A = 
 \begin{pmatrix}
 1 &2&5 \\
 3&10&4
 \end{pmatrix}
 $$
 \begin{itemize}
 \item $A$は2行3列の行列($2 \times 3$行列).
 \item (1,2)成分は2, (2,1)成分は3, (2,3)成分は4である.
 \item 第2行は$\begin{pmatrix}3 & 10 & 4\end{pmatrix}$. 第3列は$\begin{pmatrix}5 \\ 4\end{pmatrix}$である.
 \end{itemize}
 \end{exa}
  \begin{exa}
 行列$A$を次で定める.
 $$
 A = 
 \begin{pmatrix}
 13 &2&5&3 \\
 1 &4&2&5 \\
  7&8&6&1 
 \end{pmatrix}
 $$
 \begin{itemize}
 \item $A$は3行4列の行列($3 \times 4$行列).
 \item (1,1)成分は13, (2,4)成分は5, (3,2)成分は8である.
 \item 第2行は$\begin{pmatrix}1&4&2&5\end{pmatrix}$. 第3列は$\begin{pmatrix}5 \\2\\ 6\end{pmatrix}$である.
 \end{itemize}
 \end{exa}
 
 \begin{exa}行列$A = (2)$とすると, $A$は1行1列の行列($1 \times 1$行列)である.\end{exa}
  
%\begin{exa}行列$A$を次で定める.$$A = (3,5,2)$$$A$は1行3列の行列($1 \times 3$行列)である.\end{exa}

\section{特別な行列}
\begin{itemize}
\item  $
 \begin{pmatrix}
0 &0&0\\
0 &0&0
 \end{pmatrix}, 
  \begin{pmatrix}
0 &0\\
0 &0
 \end{pmatrix}, 
   \begin{pmatrix}
0 
 \end{pmatrix}, 
   \begin{pmatrix}
0 &0\\
0 &0 \\
0&0
 \end{pmatrix}
 $
 のように全ての成分が0の行列を\underline{\ruby{零}{ぜろ}行列}という.
 \item $n \times n$行列のことを\underline{$n$次正方行列}という.
 \item $n$次正方行列
 $$
 A =
 \begin{pmatrix}
a_{11}& a_{12} & \cdots &a_{1n} \\
a_{21}& a_{22} & \cdots &a_{2n} \\
\vdots& \vdots	&	\ddots   &	\vdots \\
a_{n1}& a_{n2} & \cdots &a_{nn} \\
\end{pmatrix}
 $$
 について, $a_{11}, a_{22}, \ldots, a_{nn}$を\underline{$A$の対角成分}という.
 \item 対角成分以外0の行列を\underline{対角行列}という. 例えば以下の行列は対角行列である:
  $$
 \begin{pmatrix}
5&0\\
0 &1
 \end{pmatrix}, 
  \begin{pmatrix}
3
 \end{pmatrix}, 
   \begin{pmatrix}
2&0 &0\\
0 &1 &0\\
0&0&5
 \end{pmatrix}
 $$
 \item 対角成分が全て1な$n$次対角行列を\underline{単位行列}と言い, $E_n$とかく. 例えば以下の行列は単位行列である:
   $$
   E_2 =
 \begin{pmatrix}
1&0\\
0 &1
 \end{pmatrix}, 
 E_1=
  \begin{pmatrix}
1
 \end{pmatrix}, 
 E_3=
   \begin{pmatrix}
1&0 &0\\
0 &1 &0\\
0&0&1
 \end{pmatrix}
 $$
 \item 行列$A$の行と列を入れ替えた行列を\underline{転置行列}と言い${}^{t}A$とかく.
\end{itemize}
\begin{exa}
 $
 A = 
 \begin{pmatrix}
 2 &5&8 \\
 3&2&1
 \end{pmatrix}
 \text{ についてその転置行列は}
{}^{t}A = 
 \begin{pmatrix}
 2 &3 \\
 5&2\\
 8&1
 \end{pmatrix}
  \text{であり, }
 $
 $
{}^{t}({}^{t}A) = 
 \begin{pmatrix}
 2 &5&8 \\
 3&2&1
 \end{pmatrix}
 =A
 $
 である.
 
  $
 A = 
 \begin{pmatrix}
 3 &1 \\
 2&4
 \end{pmatrix}
 \text{ についてその転置行列は}
{}^{t}A = 
 \begin{pmatrix}
 3 &2 \\
 1&4\\
 \end{pmatrix}
  \text{であり, }
 $
  $
{}^{t}({}^{t}A) = 
 \begin{pmatrix}
 3 &1 \\
 2&4
 \end{pmatrix}
 =A
 $
 である.
\end{exa}

 \begin{tcolorbox}[
    colback = white,
    colframe = green!35!black,
    fonttitle = \bfseries,
    breakable = true]
    \begin{prop}[転置行列の性質]
    \text{}
\begin{itemize}
\item $A$が$m\times n$行列なら${}^{t}A $は$n\times m$行列.
\item $A=[a_{ij}]_{m \times n}$とし, ${}^{t}A=[b_{ij}]_{n \times m}$とするとき, $b_{ij} =a_{ji}$.
\item ${}^{t}({}^{t}A) =A$.
\end{itemize}

  \end{prop}
 \end{tcolorbox}
 
 \section{クロネッカーのデルタ}
 $$
 \delta_{ij} = 
 \begin{cases}
1 & \text{$i=j$のとき}\\
0 & \text{$i \neq j$のとき}
\end{cases}
  $$
 \text{を\underline{クロネッカーのデルタ}という.}
 \begin{exa}
 $\delta_{11}=\delta_{22}=1, \delta_{12}=\delta_{21}=0$である. $n$次正方行列$E_n$は$E_n = [\delta_{ij}]_{n\times n}$と略記できる.
 \end{exa}

\section{演習問題}
演習問題の解答は授業動画にあります.

1.  行列$A$を次で定める.
 $$
 A = 
 \begin{pmatrix}
 2 &4&-3&8 \\
 3&-1&2&-5 \\
  18&0&2&12
 \end{pmatrix}
 $$
 次の問いに答えよ.
 \begin{itemize}
 \item $A$の型をいえ.
 \item $A$の$(3,2)$成分をいえ.
  \item $A$の第2行をいえ.
 \item $A$の第3列をいえ.
 \item $A$の転置行列${}^{t}A$を求めよ.
 \end{itemize}
 
 2. ${}^{t}A =-A$となる$n$次正方行列を交代行列という. 交代行列の対角成分は0であることを示せ.


\newpage

\begin{center}
{\Large 第3回. 行列の演算 (三宅先生の本, 1.2と1.3の内容)}
\end{center}

\begin{flushright}
 岩井雅崇 2022/04/28
\end{flushright}



\section{行列の和と差}

 \begin{tcolorbox}[
    colback = white,
    colframe = green!35!black,
    fonttitle = \bfseries,
    breakable = true]
    \begin{dfn}[行列の和と差]
    \text{}
 
$m \times n$行列
$
A=\begin{pmatrix}
a_{11}& a_{12} & \cdots &a_{1n} \\
a_{21}& a_{22} & \cdots &a_{2n} \\
\vdots& \vdots	&	\ddots   &	\vdots \\
a_{m1}& a_{m2} & \cdots &a_{mn} \\
\end{pmatrix}
$, 
$
B=\begin{pmatrix}
b_{11}& b_{12} & \cdots &b_{1n} \\
b_{21}& b_{22} & \cdots &b_{2n} \\
\vdots& \vdots	&	\ddots   &	\vdots \\
b_{m1}& b_{m2} & \cdots &b_{mn} \\
\end{pmatrix}
$
とする.

このとき行列の和$A+B$と差$A-B$を次で定める.
$$
A+B=
\begin{pmatrix}
a_{11}+b_{11}& a_{12}+b_{12}& \cdots &a_{1n} +b_{1n}\\
a_{21}+b_{21}& a_{22}+b_{22}& \cdots &a_{2n}+b_{2n} \\
\vdots& \vdots	&	\ddots   &	\vdots \\
a_{m1}+b_{m1}& a_{m2} +b_{m2}& \cdots &a_{mn} +b_{mn}\\
\end{pmatrix}.
$$
$$
A-B=
\begin{pmatrix}
a_{11}-b_{11}& a_{12}-b_{12}& \cdots &a_{1n} -b_{1n}\\
a_{21}-b_{21}& a_{22}-b_{22}& \cdots &a_{2n}-b_{2n} \\
\vdots& \vdots	&	\ddots   &	\vdots \\
a_{m1}-b_{m1}& a_{m2}-b_{m2}& \cdots &a_{mn}-b_{mn}\\
\end{pmatrix}.
$$
  \end{dfn}
 \end{tcolorbox}
 
 \begin{exa}
 $A = 
 \begin{pmatrix}
 1 &-2&8 \\
 2&5&-1
 \end{pmatrix}
 $, 
 $
 B = 
 \begin{pmatrix}
 -2&5&1 \\
 3&-1&2
 \end{pmatrix}
 $
 とする.
 
 このとき$
 A+B =
 \begin{pmatrix}
 -1 &3&9 \\
 5&4&1
 \end{pmatrix}
 $, 
 $
  A-B =
 \begin{pmatrix}
 3 &-7&7 \\
 -1&6&-3
 \end{pmatrix}
 $である.
 \end{exa}

\begin{exa}
 $A = 
 \begin{pmatrix}
 3&1 \\
 1&4
 \end{pmatrix}
 $, 
 $
 B = 
 \begin{pmatrix}
 2&7\\
 5&8
 \end{pmatrix}
 $
 とする.
 
 このとき$
 A+B =
 \begin{pmatrix}
 5&8 \\
6&12
 \end{pmatrix}
 $, 
 $
  A-B =
 \begin{pmatrix}
 1&-6 \\
 -4&-4
 \end{pmatrix}
 $である.
 \end{exa}
 
 \begin{exa}
 $A = 
 \begin{pmatrix}
 2&1 \\
 1&5
 \end{pmatrix}
 $,
$ 
 B = 
 \begin{pmatrix}
 1&1 &3 \\
 4&6 & 7
 \end{pmatrix}
 $
 とする.このとき$A+B$は型が違うため定義されない. 
 \end{exa}
 
 \begin{tcolorbox}[
    colback = white,
    colframe = green!35!black,
    fonttitle = \bfseries,
    breakable = true]
    \begin{prop}[行列の和と差の性質]
$A, B$を行列とする.
 \begin{itemize}
 \item $A\pm B =B\pm A$.
  \item $A\pm O =A$ (ただし$O$は零行列).
  \item $(A+B)+C =A + (B+C)$.
  \item ${}^{t}(A+B) = {}^{t}A+ {}^{t}B$.
 \end{itemize}
  \end{prop}
 \end{tcolorbox}
 
 \section{行列のスカラー倍}
 
  \begin{tcolorbox}[
    colback = white,
    colframe = green!35!black,
    fonttitle = \bfseries,
    breakable = true]
    \begin{dfn}[行列のスカラー倍]
    \text{}
    
 $m \times n$行列
 $
A=\begin{pmatrix}
a_{11}& a_{12} & \cdots &a_{1n} \\
a_{21}& a_{22} & \cdots &a_{2n} \\
\vdots& \vdots	&	\ddots   &	\vdots \\
a_{m1}& a_{m2} & \cdots &a_{mn} \\
\end{pmatrix}$
とし, $c$を数とする($c$をスカラーとも呼ぶ).

$A$の$c$倍$cA$を次で定める.
$$
cA=
\begin{pmatrix}
ca_{11}&c a_{12} & \cdots &ca_{1n} \\
ca_{21}& ca_{22} & \cdots &ca_{2n} \\
\vdots& \vdots	&	\ddots   &	\vdots \\
ca_{m1}& ca_{m2} & \cdots &ca_{mn} \\
\end{pmatrix}.
$$
  \end{dfn}
 \end{tcolorbox}

\begin{exa}
 $A = 
 \begin{pmatrix}
 1 &-2&8 \\
 2&5&-1
 \end{pmatrix}
 $,
 $
 c =3
 $
 とする.
 このとき$
 cA =
 \begin{pmatrix}
 3 &-6&24 \\
 6&15&-3
 \end{pmatrix}
 $である.
 \end{exa}
 \begin{exa}
 $A = 
 \begin{pmatrix}
 2&1 \\
 4&3
 \end{pmatrix}
 $, 
 $
 c =-1
 $
 とする.
 このとき$
 cA =
 \begin{pmatrix}
 -2 &-1 \\
-4&-3
 \end{pmatrix}
 $である.
 \end{exa}
 
 
 \begin{tcolorbox}[
    colback = white,
    colframe = green!35!black,
    fonttitle = \bfseries,
    breakable = true]
    \begin{prop}[行列のスカラー倍の性質]
$A$を行列, $a,b$を数とする.
 \begin{itemize}
 \item $0A =O$ (ただし$O$は零行列).
  \item $1A=A$. 
  \item $(-1)A$を$-A$と書くことにすると, $A + (-A) =O$. 
  \item $(ab) A = a(bA)$.
 \end{itemize}
  \end{prop}
 \end{tcolorbox}
 
 \section{行列の積}
 
  \begin{tcolorbox}[
    colback = white,
    colframe = green!35!black,
    fonttitle = \bfseries,
    breakable = true]
    \begin{dfn}[行列の積]
    
 $m \times n$行列$A = [a_{ij}]_{m \times n}$と$n \times l$行列$B= [b_{jk}]_{n \times l}$とする.
このとき$A$と$B$の積$AB$は$m \times l$行列で, 次の式で定義される.

$$
AB = [c_{ik}]_{m \times l}\text{としたとき, }
c_{ik} = a_{i1}b_{1k} + a_{i2}b_{2k} + \cdots + a_{in}b_{nk} = \sum_{j=1}^{n} a_{ij}b_{jk}.
$$
  \end{dfn}
 \end{tcolorbox}
 
 \begin{exa}
 $ A=\begin{pmatrix} 1 &2 &3 \end{pmatrix}$, 
 $ 
 B = 
 \begin{pmatrix}
5 \\7\\2
 \end{pmatrix}
 $
 とする. 
 
 $A$は$1\times 3$行列で$B$は$3 \times 1$行列なので, 行列の積$AB$が$1 \times 1$行列として定義でき, 
 $$
 AB = \begin{pmatrix}1 &2&3  \end{pmatrix}
 \begin{pmatrix}
5 \\7\\2
 \end{pmatrix}
 = \begin{pmatrix}1\times 5 + 2 \times 7 + 3 \times 2  \end{pmatrix}= 
  \begin{pmatrix}5+14+6 \end{pmatrix}= \begin{pmatrix}25 \end{pmatrix}.
 $$
 
 \end{exa}
 
  \begin{exa}
 $ A= 
 \begin{pmatrix}
2 & 2\\
4 & 3
 \end{pmatrix}
 $, $
 B = 
 \begin{pmatrix}
5 \\1
 \end{pmatrix}
 $
 とする. 
 
 $A$は$2\times 2$行列で$B$は$2 \times 1$行列なので, 行列の積$AB$が$2 \times 1$行列として定義でき, 
 $$
 AB = 
 \begin{pmatrix}
2 & 2\\
4 & 3
 \end{pmatrix}
  \begin{pmatrix}
5 \\1
 \end{pmatrix}
 =  
 \begin{pmatrix}
2\times 5 + 2\times 1 \\
4 \times 5 + 3 \times 1
 \end{pmatrix}
 = 
  \begin{pmatrix}
12 \\
23
 \end{pmatrix}.
 $$
 
 \end{exa}
 
 \begin{exa}
 $ A= 
 \begin{pmatrix}
2 & 3\\
1 & 4
 \end{pmatrix}
 $, $
 B = 
 \begin{pmatrix}
5 & 2\\
2 & 3
 \end{pmatrix}
 $
 とする. 
 
 $A$は$2\times 2$行列で$B$は$2 \times 2$行列なので, 行列の積$AB$が$2 \times 2$行列として定義でき, 
 $$
 AB = 
 \begin{pmatrix}
2 & 3\\
1 & 4
 \end{pmatrix}
 \begin{pmatrix}
5 & 2\\
2 & 3
 \end{pmatrix}
 =  
 \begin{pmatrix}
2 \times 5 + 3 \times 2& 2 \times 2 + 3 \times 3\\
1 \times 5 + 4 \times 2 & 1\times 2 + 4 \times 3
 \end{pmatrix}
 = 
 \begin{pmatrix}
16 & 13\\
13 & 14
 \end{pmatrix}.
 $$
 
また$B$は$2\times 2$行列で$A$は$2 \times 2$行列なので, 行列の積$BA$が$2 \times 2$行列として定義でき, 
 $$
 BA = 
  \begin{pmatrix}
5 & 2\\
2 & 3
 \end{pmatrix}
  \begin{pmatrix}
2 & 3\\
1 & 4
 \end{pmatrix}
 =
  \begin{pmatrix}
12 & 23\\
7 & 18
 \end{pmatrix}.
 $$

よって\underline{行列の積に関して$AB=BA$とは限らない($AB \neq BA$となることがある).}
 \end{exa}
 
  \begin{exa}
 $ A= 
 \begin{pmatrix}
2 & 1&-3\\
1 & -5 & 2
 \end{pmatrix}
 $, $
 B = 
  \begin{pmatrix}
8 & 7&5 & 2
 \end{pmatrix}
 $
 とする. 
 
 $A$は$2 \times 3$行列で$B$は$1 \times 4$行列であるので, 行列の積$AB$は定義されない.
 \end{exa}
 
 \begin{tcolorbox}[
    colback = white,
    colframe = green!35!black,
    fonttitle = \bfseries,
    breakable = true]
    \begin{prop}[行列の積の性質]
$A,B,C$を行列とする.
 \begin{itemize}
 \item $AO =O = OA$ (ただし$O$は零行列).
  \item $AE_{n}=E_{n}A =A$ (ただし$E_n$は単位行列). 
  \item $(AB)C = A(BC)$. 
  \item ${}^{t}(AB) = {}^{t}B {}^{t}A$
 \end{itemize}
  \end{prop}
 \end{tcolorbox}
\begin{itemize}
\item $A$を$n$次正方行列とするとき$A^{m} = \underbrace{A \cdots A}_{m \text{ 個}}$とする
\item $A^{m}=O$となる行列を\underline{\ruby{冪}{べき}\ruby{零}{ぜろ}行列}という.
\end{itemize}

 \begin{tcolorbox}[
    colback = white,
    colframe = green!35!black,
    fonttitle = \bfseries,
    breakable = true]
    \begin{prop}[行列の演算の性質]
$A,B,C$を行列とし, $a,b$を数とする.
 \begin{itemize}
 \item $a(AB)=(aA)B$. 
  \item $a(A+B)=aA + aB$. 
  \item $(a+b)A = aA + bA$. 
  \item $A(B+C) = AB + AC$.
  \item $(A+B)C = AC + BC$.
 \end{itemize}
  \end{prop}
 \end{tcolorbox}
 
\section{三宅先生の本1.3の内容に関して}
この授業では三宅先生の本1.3の内容「行列の分割」についての説明は割愛する(重要度が低いと思われるため).
ただし証明等で行列の分割の記法を用いるため, 各自で三宅先生の本1.3の内容を読むことをお勧めする.

\section{演習問題}
演習問題の解答は授業動画にあります.

1.  次の行列の計算を行え.
 $$
 \begin{pmatrix}
 2 &3&-1 \\
 0&5&4\\
 -1&0&-2
 \end{pmatrix}
 \left\{
 \begin{pmatrix}
 0 &5&9 \\
 3&-2&8\\
 -1&8&1
 \end{pmatrix}
 - 2
  \begin{pmatrix}
 -1 &0&1 \\
 3&2&3\\
 -4&2&-1
 \end{pmatrix}
\right\}
 $$
  
 2. 次の行列$A,B,C,D$のうち, 積が定義される全ての組み合わせを求め, その積を計算せよ.
 $$
  A=\begin{pmatrix}
 2 \\ 1\\-1
 \end{pmatrix} 
B= \begin{pmatrix}
 3 &2\\
 4&1\\
 0&1
 \end{pmatrix} 
 C=
  \begin{pmatrix}
 2 &0&1 
 \end{pmatrix}
 D= \begin{pmatrix}
 2&3\\
 -1&4
 \end{pmatrix}
 $$

\newpage

\begin{center}
{\Large 第4回. 連立1次方程式1-基本変形- (三宅先生の本, 1.4, 2.1の内容)}
\end{center}

\begin{flushright}
 岩井雅崇 2022/05/12
\end{flushright}

\section{連立1次方程式}

 \begin{tcolorbox}[
    colback = white,
    colframe = green!35!black,
    fonttitle = \bfseries,
    breakable = true]
    \begin{dfn}[係数行列, 拡大係数行列]
$m$個の式からなる$n$変数連立1次方程式
\begin{equation*}
%\label{equation}
\left\{ 
\begin{matrix}
a_{11}x_1&+& a_{12} x_2& +&\cdots &+&a_{1n}x_n &= &b_1 \\
a_{21}x_1&+& a_{22} x_2& +&\cdots &+&a_{2n}x_n &= &b_2 \\
\vdots		&& 	\vdots				 && 		& &\vdots&&\vdots	\\
a_{m1}x_1&+& a_{m2} x_2& +&\cdots &+&a_{mn}x_n &= &b_m \\
\end{matrix}
\right.
\text{に対して}
\end{equation*}
$$
A=\begin{pmatrix}
a_{11}& a_{12} & \cdots &a_{1n} \\
a_{21}& a_{22} & \cdots &a_{2n} \\
\vdots& \vdots	&	\ddots   &	\vdots \\
a_{m1}& a_{m2} & \cdots &a_{mn} \\
\end{pmatrix}
\bm{x} =\begin{pmatrix}
x_1\\x_2\\\vdots\\x_n
\end{pmatrix}
\bm{b} =\begin{pmatrix}
b_1\\b_2\\\vdots\\b_m
\end{pmatrix}
\text{とおく.}
$$
行列$A$を連立1次方程式の\underline{係数行列}といい, 
$$
[A : \bm{b}] = \begin{pmatrix}
a_{11}& a_{12} & \cdots &a_{1n} & b_1\\
a_{21}& a_{22} & \cdots &a_{2n} &b_2\\
\vdots& \vdots	&	\ddots   &	\vdots&\vdots \\
a_{m1}& a_{m2} & \cdots &a_{mn}&b_m \\
\end{pmatrix}
\text{を連立1次方程式の\underline{拡大係数行列}という.}
$$
  \end{dfn}
 \end{tcolorbox}
 これにより上の連立1次方程式は$A\bm{x}=\bm{b}$とかける.

 \begin{exa}
 連立1次方程式
 $
 \left\{ 
\begin{matrix}
2x&+&3y& = &7 \\
x&-&4y& = &9 \\
\end{matrix}
\right.
 $
 について, 係数行列は
 $A = 
 \begin{pmatrix}
 2 & 3 \\
 1 & -4
 \end{pmatrix}
 $
 で, 拡大係数行列は
 $[A : \bm{b}] = 
  \begin{pmatrix}
 2 & 3  &7\\
 1 & -4 &9
 \end{pmatrix}
 $
 である.
 \end{exa}
 
  \begin{exa}
 連立1次方程式
 $
 \left\{ 
\begin{matrix}
3x_1&-&2x_2& +& x_3 &+& 4x_4 &=& 7 \\
x_1 &  & 	   & -& 3x_3 &+& x_4 &=& 5 \\
2x_1&-& x_2& +& 9x_3 & & 	 &=& 0 \\
\end{matrix}
\right.
 $
 について, \\
 係数行列は
 $A = 
 \begin{pmatrix}
 3 & -2  & 1&4\\
 1 & 0   & -3&1\\
2 & -1  & 9&0\\
 \end{pmatrix}
 $
 で, 拡大係数行列は
 $[A : \bm{b}] = 
 \begin{pmatrix}
 3 & -2  & 1&4 & 7\\
 1 & 0   & -3&1 &5\\
2 & -1  & 9&0 & 0\\
 \end{pmatrix}
 $
 である.
 \end{exa}
 
 \section{行列の基本変形}
  \begin{tcolorbox}[
    colback = white,
    colframe = green!35!black,
    fonttitle = \bfseries,
    breakable = true]
    \begin{dfn}[行列の基本変形]
 行列の次の3つの変形を(行)基本変形という.
 \begin{enumerate}
 \item 1つの行を何倍か($\neq 0$倍)する.
 \item 2つの行を入れ替える.
 \item1つの行に他の行の何倍かを加える.
 \end{enumerate}
  \end{dfn}
  
 \end{tcolorbox} 

拡大係数行列の(行)基本変形を行うことで連立1次方程式が解ける(連立方程式の解き方に関しては, 第6回資料を見てください).

\begin{exa}
 連立1次方程式
 $
 \left\{ 
\begin{matrix}
2x&+&3y& -&z &= &-3 \\
-x&+&2y& +&2z &= &1 \\
x&+&y& -&z &= &-2 \\
\end{matrix}
\right.
 $
 を考える. 
 これを拡大係数行列の基本変形と式変形で解いてみて, その対応を表すと下の通りとなる.\footnote{途中で現れる「$\maru{1}+\maru{3}\times(-2)$」は「行列の1行目に3行目の$(-2)$倍を加える」あるいは「1行目の式に3行目の式の$(-2)$倍を加える」を意味している(一応教科書に従った記法である).}

  %まずこれを式変形で解くと次のとおりである.\footnote{途中で現れる「$\maru{1}+\maru{3}\times(-2)$」は「1行目の式に3行目の式の(-2)倍を加える」を意味している(一応教科書に従った記法である).}
%これに対する拡大係数行列の基本変形は以下の通りである. \footnote{途中で現れる「$\maru{1} + \maru{3}\times(-2)$」は「行列の1行目に3行目の(-2)倍を加える」を意味している.}
 
  \begin{align*}
 & &\begin{pmatrix}
 2& 3  & -1&-3\\
-1 & 2 & 2&1\\
1& 1  & -1&-2\\
 \end{pmatrix}
 & \overset{\text{対応}}{\longleftrightarrow}& &\left\{ 
\begin{matrix}
2x&+&3y& -&z &= &-3 \\
-x&+&2y& +&2z &= &1 \\
x&+&y& -&z &= &-2 \\
\end{matrix}
\right.
 \\ %%
 &\overset{\text{$\maru{1} + \maru{3}\times(-2)$}}{\longrightarrow} 
 & \begin{pmatrix}
 0& 1  & 1&1\\
-1 & 2 & 2&1\\
1& 1  & -1&-2\\
 \end{pmatrix}
&\overset{\text{対応}}{\longleftrightarrow} 
&\overset{\text{$\maru{1} + \maru{3}\times(-2)$}}{\longrightarrow} 
 & \left\{ 
\begin{matrix}
 & & y& + &z &= &1 \\
-x&+&2y& +&2z &= &1 \\
x&+&y& -&z &= &-2 \\
\end{matrix}
\right.
 \\ %%
  &\overset{\text{$\maru{2}+\maru{3}\times 1$ }}{\longrightarrow} 
 &\begin{pmatrix}
 0& 1  & 1&1\\
0 & 3& 1&-1\\
1& 1  & -1&-2\\
 \end{pmatrix}
&\overset{\text{対応}}{\longleftrightarrow} 
&\overset{\text{$\maru{2}+\maru{3}\times 1$ }}{\longrightarrow} 
 & \left\{ 
\begin{matrix}
 & & y& + &z &= &1\\
 & &3y& +& z &= &-1 \\
x&+&y& -&z &= &-2 \\
\end{matrix}
\right.
 \\ %%
   &\overset{\text{\maru{3}と\maru{1}を入れ替え}}{\longrightarrow} 
 &\begin{pmatrix}
1& 1  & -1&-2\\
0 & 3& 1&-1\\
 0& 1  & 1&1\\
 \end{pmatrix}
&\overset{\text{対応}}{\longleftrightarrow} 
&\overset{\text{\maru{3}と\maru{1}を入れ替え}}{\longrightarrow} 
 & \left\{ 
\begin{matrix}
x&+&y& -&z &= &-2 \\
 & &3y& +& z &= &-1 \\
 & & y& + &z &= &1\\
\end{matrix}
\right.
 \\ %%
    &\overset{\text{\maru{2}と\maru{3}を入れ替え}}{\longrightarrow} 
 &\begin{pmatrix}
1& 1  & -1&-2\\
 0& 1  & 1&1\\
 0 & 3& 1&-1\\
 \end{pmatrix}
&\overset{\text{対応}}{\longleftrightarrow} 
&\overset{\text{\maru{2}と\maru{3}を入れ替え}}{\longrightarrow} 
 & \left\{ 
\begin{matrix}
x&+&y& -&z &= &-2 \\
 & & y& + &z &= &1\\
  & &3y& +& z &= &-1 \\
\end{matrix}
\right.
 \\ %%
    &\overset{\maru{1}  + \maru{2}\times(-1)}{\underset{\maru{3} +  \maru{2}\times(-3)}{\longrightarrow}}
 &\begin{pmatrix}
1& 0  & -2&-3\\
 0& 1  & 1&1\\
 0 & 0& -2&-4\\
 \end{pmatrix}
&\overset{\text{対応}}{\longleftrightarrow} 
&\overset{\maru{1}  + \maru{2}\times(-1)}{\underset{\maru{3} +  \maru{2}\times(-3)}{\longrightarrow}}
 & \left\{ 
\begin{matrix}
x& & & -&2z &= &-3 \\
 & & y& + &z &= &1\\
  & & & -& 2z &= &-4 \\
\end{matrix}
\right.
 \\ %%
    &\overset{\text{\maru{3}$\times (-\frac{1}{2})$}}{\longrightarrow}
 &\begin{pmatrix}
1& 0  & -2&-3\\
 0& 1  & 1&1\\
 0 & 0& 1&2\\
 \end{pmatrix}
&\overset{\text{対応}}{\longleftrightarrow} 
&\overset{\text{\maru{3}$\times (-\frac{1}{2})$}}{\longrightarrow}
 & \left\{ 
\begin{matrix}
x& & & -&2z &= &-3 \\
 & & y& + &z &= &1\\
  & & &  & z &= &2 \\
\end{matrix}
\right.
 \\ %%
&\overset{\maru{1} + \maru{3}\times 2}{\underset{\maru{2} + \maru{3}\times(-1) }{\longrightarrow}}
 &\begin{pmatrix}
1& 0  & 0&1\\
 0& 1  & 0&-1\\
 0 & 0& 1&2\\
 \end{pmatrix}
&\overset{\text{対応}}{\longleftrightarrow} 
&\overset{\maru{1} + \maru{3}\times 2}{\underset{\maru{2} + \maru{3}\times(-1) }{\longrightarrow}}
 & \left\{ 
\begin{matrix}
x& & &  &  &= &1 \\
 & & y&   & &= &-1\\
  & & &  & z &= &2 \\
\end{matrix}
\right.
 \\ %%
 \end{align*}
 以上より解は$x=1, y=-1, z=2$である.
\end{exa}

%%%%%%%%%%%%%%%%%%%%%%%%%%%%%%%%%%%%%%%%%%%%%%%%%%%%%%%%%
 
 \begin{comment}
 ボツネタ
 
  
 %\overset{\text{$ \maru{1}  + \maru{2}\times(-1)$ と$\maru{3} +  \maru{2}\times(-3) $}}{\longrightarrow} 
 $\overset{\maru{1}  + \maru{2}\times(-1)}{\underset{\maru{3} +  \maru{2}\times(-3)}{\longrightarrow}}$
 
 \begin{align*}
 &\begin{pmatrix}
 2& 3  & -1&-3\\
-1 & 2 & 2&1\\
1& 1  & -1&-2\\
 \end{pmatrix}
\overset{\text{$\maru{1} + \maru{3}\times(-2)$}}{\longrightarrow} 
 \begin{pmatrix}
 0& 1  & 1&1\\
-1 & 2 & 2&1\\
1& 1  & -1&-2\\
 \end{pmatrix}
\overset{\text{$\maru{2}+\maru{3}\times 1$ }}{\longrightarrow} 
 \begin{pmatrix}
 0& 1  & 1&1\\
0 & 3& 1&-1\\
1& 1  & -1&-2\\
 \end{pmatrix}
 \\
 &\overset{\text{\maru{3}と\maru{1}を入れ替え}}{\longrightarrow} 
\begin{pmatrix}
1& 1  & -1&-2\\
0 & 3& 1&-1\\
 0& 1  & 1&1\\
 \end{pmatrix}
  \overset{\text{\maru{2}と\maru{3}を入れ替え}}{\longrightarrow} 
\begin{pmatrix}
1& 1  & -1&-2\\
 0& 1  & 1&1\\
 0 & 3& 1&-1\\
 \end{pmatrix}
 \\
 &  \overset{\text{
$ \maru{1}  + \maru{2}\times(-1)$ と$\maru{3} +  \maru{2}\times(-3) $
 }}{\longrightarrow} 
 \begin{pmatrix}
1& 0  & -2&-3\\
 0& 1  & 1&1\\
 0 & 0& -2&-4\\
 \end{pmatrix}
   \overset{\text{\maru{3}$\times (-\frac{1}{2})$}}{\longrightarrow}
\begin{pmatrix}
1& 0  & -2&-3\\
 0& 1  & 1&1\\
 0 & 0& 1&2\\
 \end{pmatrix} 
 \\
 &\overset{\text{
$\maru{1} + \maru{3}\times 2 $ と $\maru{2} + \maru{3}\times(-1) $
  }}{\longrightarrow} 
  \begin{pmatrix}
1& 0  & 0&-1\\
 0& 1  & 0&-1\\
 0 & 0& 1&2\\
 \end{pmatrix} 
\end{align*}
 
 
 
 
 \begin{align*}
&\begin{pmatrix}
 2& 3  & -1&-3\\
-1 & 2 & 2&1\\
1& 1  & -1&-2\\
 \end{pmatrix}
 &\text{対応}&
  \left\{ 
\begin{matrix}
2x&+&3y& -&z &= &-3 \\
-x&+&2y& +&2z &= &1 \\
x&+&y& -&z &= &-2 \\
\end{matrix}
\right.
 \\
&\overset{\text{$\maru{1} + \maru{3}\times(-2)$}}{\longrightarrow} 
 \begin{pmatrix}
 0& 1  & 1&1\\
-1 & 2 & 2&1\\
1& 1  & -1&-2\\
 \end{pmatrix}
 \\
&\overset{\text{$\maru{2}+\maru{3}\times 1$ }}{\longrightarrow} 
 \begin{pmatrix}
 0& 1  & 1&1\\
0 & 3& 1&-1\\
1& 1  & -1&-2\\
 \end{pmatrix}
 \\
 &\overset{\text{\maru{3}と\maru{1}を入れ替え}}{\longrightarrow} 
\begin{pmatrix}
1& 1  & -1&-2\\
0 & 3& 1&-1\\
 0& 1  & 1&1\\
 \end{pmatrix}
  \overset{\text{\maru{2}と\maru{3}を入れ替え}}{\longrightarrow} 
\begin{pmatrix}
1& 1  & -1&-2\\
 0& 1  & 1&1\\
 0 & 3& 1&-1\\
 \end{pmatrix}
 \\
 &  \overset{\text{
$ \maru{1}  + \maru{2}\times(-1)$ と$\maru{3} +  \maru{2}\times(-3) $
 }}{\longrightarrow} 
 \begin{pmatrix}
1& 0  & -2&-3\\
 0& 1  & 1&1\\
 0 & 0& -2&-4\\
 \end{pmatrix}
   \overset{\text{\maru{3}$\times (-\frac{1}{2})$}}{\longrightarrow}
\begin{pmatrix}
1& 0  & -2&-3\\
 0& 1  & 1&1\\
 0 & 0& 1&2\\
 \end{pmatrix} 
 \\
 &\overset{\text{
$\maru{1} + \maru{3}\times 2 $ と $\maru{2} + \maru{3}\times(-1) $
  }}{\longrightarrow} 
  \begin{pmatrix}
1& 0  & 0&-1\\
 0& 1  & 0&-1\\
 0 & 0& 1&2\\
 \end{pmatrix} 
\end{align*}

\begin{align*}
 \begin{pmatrix}
 2& 3  & -1&-3\\
-1 & 2 & 2&1\\
1& 1  & -1&-2\\
 \end{pmatrix}
&\overset{\text{対応}}{\leftrightarrow} 
 \left\{ 
\begin{matrix}
2x&+&3y& -&z &= &-3 \\
-x&+&2y& +&2z &= &1 \\
x&+&y& -&z &= &-2 \\
\end{matrix}
\right.
\\
\end{align*}


  \end{comment}

 
 
\section{演習問題}
演習問題の解答は授業動画にあります.

1. 連立1次方程式
 $
 \left\{ 
\begin{matrix}
x_1&+&x_2& -&x_3 &= & 1\\
2x_1&+&x_2& +&3x_3&= &4 \\
-x_1&+&2x_2& -&4x_3 &= &-2 \\
\end{matrix}
\right.
 $
 を解け.


\newpage

%\maketitle
\begin{center}
{\Large 第5回. 連立1次方程式2-行列の簡約化- (三宅先生の本, 2.2の内容)}
\end{center}

\begin{flushright}
 岩井雅崇 2022/05/19
\end{flushright}

\section{簡約な行列}

\begin{tcolorbox}[
    colback = white,
    colframe = green!35!black,
    fonttitle = \bfseries,
    breakable = true]
    \begin{dfn}[簡約な行列]
  行列$A$が次の4つの条件を満たすとき, $A$を\underline{簡約な行列}という.
  \begin{enumerate}
\item 行ベクトルのうちに零ベクトル(全ての成分が0である行)があれば, それは零ベクトルでないものよりも下にある.
\item 零ベクトルでない行ベクトルの主成分は1である.
\item 第$i$行の主成分を$a_{ij_{i}}$とすると, $j_1<j_2<j_3<\cdots$となる. すなわち各行の主成分は下の行ほど右にある.
\item 各行の主成分を含む列の他の成分は全て0である. すなわち第$i$行の主成分が$a_{ij_{i}}$であるならば, 第$j_i$列の$a_{ij_{i}}$以外の成分は全て0である.
  \end{enumerate}
  \end{dfn}
 \end{tcolorbox}
 \begin{exa}
以下の行列は全て簡約な行列である.
$$
 \begin{pmatrix}
 0& 1& 3  & 0&2\\
 0& 0& 0  & 1&1\\
 0& 0& 0 & 0&0\\
 \end{pmatrix}
  \begin{pmatrix}
 1& 0& 1  & 4&0&-1\\
 0& 1& 7 & -4&0&1\\
 0& 0& 0 & 0&1&3\\
 \end{pmatrix}
   \begin{pmatrix}
 0& 0& 0  & 1&6&0&3&0\\
 0& 0& 0 & 0&0&1&2&0\\
 0& 0& 0 & 0&0&0&0&0 \\
 \end{pmatrix}
$$
\end{exa}
 \begin{exa}
 次に簡約ではない行列の例を理由とともに挙げる.
 \begin{itemize}
\item 
$ 
\begin{pmatrix}
 1& 0& 1  & 1&0\\
 0& 0& 0  & 0&0\\
 0& 0& 0 & 0&1\\
 \end{pmatrix} 
 $
 は1番目の条件が満たされていないので簡約ではない.
 \item 
$ 
\begin{pmatrix}
 1& 0& 1  & 1&0\\
 0& 0& 0  & 0&3\\
 \end{pmatrix} 
 $
 は2番目の条件が満たされていないので簡約ではない.
 \item 
$ 
\begin{pmatrix}
 0& 0& 1  & 0&0\\
 1& 0& 0  & 0&0\\
 \end{pmatrix} 
 $
 は3番目の条件が満たされていないので簡約ではない.
 \item 
$ 
\begin{pmatrix}
 1& 0& 1  & 1&0\\
 1& 0& 0  & 0&1\\
 \end{pmatrix} 
 $
 は4番目の条件が満たされていないので簡約ではない.
 \end{itemize}
\end{exa}

\begin{tcolorbox}[
    colback = white,
    colframe = green!35!black,
    fonttitle = \bfseries,
    breakable = true]
    \begin{dfn}[簡約化]
  行列$A$に(行)基本変形
 \begin{enumerate}
 \item 1つの行を何倍か($\neq 0$倍)する.
 \item 2つの行を入れ替える.
 \item1つの行に他の行の何倍かを加える.
 \end{enumerate}
 を繰り返して簡約な行列$B$を得ることを\underline{$A$を簡約化する}といい, \underline{$B$を$A$の簡約化}という.
   \end{dfn}
 \end{tcolorbox}
 
 \begin{tcolorbox}[
    colback = white,
    colframe = green!35!black,
    fonttitle = \bfseries,
    breakable = true]
    \begin{thm}
    任意の行列は基本変形を繰り返すことによって簡約化することができ, その簡約化は一意に定まる.
   \end{thm}
 \end{tcolorbox}
 
 \begin{tcolorbox}[
    colback = white,
    colframe = green!35!black,
    fonttitle = \bfseries,
    breakable = true]
    \begin{dfn}[階数(ランク)]
$A$を行列とし, $B$を$A$の簡約化とする.
${\rm rank}(A)$を$B$の零ベクトルでない行の個数とし\underline{$A$の階数(ランク)}と呼ぶ.
   \end{dfn}
 \end{tcolorbox}
${\rm rank}(A)$は簡約化の仕方によらずに定まる数である.
また$A$を$m\times n$行列とすると${\rm rank}(A) \le \min(m,n)$である.
\begin{exa}
$A=
 \begin{pmatrix}
 0& 1& 3  & 0&2\\
 0& 0& 0  & 1&1\\
 0& 0& 0 & 0&0\\
 \end{pmatrix}
 $
 とすると, これは簡約な行列であり零ベクトルでない行の個数は2個である. よって${\rm rank}(A)=2$.
 
 $B= \begin{pmatrix}
 1& 0& 1  & 4&0&-1\\
 0& 1& 7 & -4&0&1\\
 0& 0& 0 & 0&1&3\\
 \end{pmatrix}
 $ とすると, これは簡約な行列であり零ベクトルでない行の個数は3個である. よって${\rm rank}(B)=3$.
\end{exa}

\begin{exa}
$
 \begin{pmatrix}
 1& 2& -3  \\
 1& 1& 1  \\
 \end{pmatrix}
 $
 を基本変形で簡約化すると次のとおりである.\footnote{第4回授業資料と同じで「$\maru{2} + \maru{1}\times(-1)$」は「行列の2行目に1行目の(-1)倍を加える」を意味している.}
 \begin{align*}
  \begin{pmatrix}
 1& 2& -3  \\
 1& 1& 1  \\
 \end{pmatrix}
 \overset{\text{$\maru{2} + \maru{1}\times(-1)$}}{\longrightarrow} 
   \begin{pmatrix}
 1& 2& -3  \\
 0& -1& 4  \\
 \end{pmatrix}
 \overset{\text{$\maru{2} \times(-1)$}}{\longrightarrow} 
   \begin{pmatrix}
 1& 2& -3  \\
 0& 1& -4  \\
 \end{pmatrix}
  \overset{\maru{1} + \maru{2}\times(-1)}{\longrightarrow} 
   \begin{pmatrix}
 1& 0& 5  \\
 0& 1& -4  \\
 \end{pmatrix}.
  \end{align*}
  よってこの行列の階数(ランク)は2である.
\end{exa}

\begin{exa}
$
 \begin{pmatrix}
 1& 0& 2  &1\\
 2& 1& 1  &0\\
 0& 1& 1  &0\\
 \end{pmatrix}
 $
 を基本変形で簡約化すると次のとおりである.
 
 \begin{align*}
 &\begin{pmatrix}
 1& 0& 2  &1\\
 2& 1& 1  &0\\
 0& 1& 1  &0\\
 \end{pmatrix}
 \overset{\maru{2} + \maru{1}\times(-2)}{\longrightarrow} 
\begin{pmatrix}
 1& 0& 2  &1\\
 0& 1& -3 &-2\\
 0& 1& 1  &0\\
 \end{pmatrix}
\overset{\maru{3} + \maru{2}\times(-1)}{\longrightarrow} 
\begin{pmatrix}
 1& 0& 2  &1\\
 0& 1& -3 &-2\\
 0& 0& 4  &2\\
 \end{pmatrix}
 \\ %%
 & \overset{\maru{3}\times \frac{1}{2}}{\longrightarrow} 
\begin{pmatrix}
 1& 0& 2  &1\\
 0& 1& -3 &-2\\
 0& 0& 2  &1\\
 \end{pmatrix} 
 \overset{\maru{1} + \maru{3}\times (-1)}{\underset{\maru{2} + \maru{3}\times \frac{3}{2}}{\longrightarrow}}
 \begin{pmatrix}
 1& 0& 0  &0\\
 0& 1& 0 &-\frac{1}{2}\\
 0& 0& 2  &1\\
 \end{pmatrix} 
 \overset{\maru{3}\times \frac{1}{2}}{\longrightarrow} 
  \begin{pmatrix}
 1& 0& 0  &0\\
 0& 1& 0 &-\frac{1}{2}\\
 0& 0& 1 &\frac{1}{2}\\
 \end{pmatrix}.
 \end{align*}
   よってこの行列の階数(ランク)は3である.
 \end{exa}
 
\section{演習問題}
演習問題の解答は授業動画にあります.

1.
$
 \begin{pmatrix}
 1& 0& -1  & 0&-2\\
 0& 1& 1  & 0&1\\
 -1& 0& 1 & 1&1\\
 2& 1& -1 & 0&3\\
 \end{pmatrix}
 $
 を簡約化し, その階数を求めよ.

2.
$
 \begin{pmatrix}
 1& 0& -1  & 0&-2\\
 0& 1& 1  & 0&1\\
  -1& 0& 1 & 1&1\\
 2& 1& -1 & 0&-3\\
 \end{pmatrix}
 $
 を簡約化し, その階数を求めよ.

\newpage

\begin{center}
{\Large 第6回. 連立1次方程式3-一般的な解法- (三宅先生の本, 2.3の内容)}
\end{center}

\begin{flushright}
 岩井雅崇 2022/05/26
\end{flushright}

\section{簡約化を用いた連立1次方程式の解法}
\begin{tcolorbox}[
    colback = white,
    colframe = green!35!black,
    fonttitle = \bfseries,
    breakable = true]
    \begin{thm}
連立1次方程式$A\bm{x} =\bm{b}$が解を持つ
$\Leftrightarrow$ ${\rm rank}([A:\bm{b}]) = {\rm rank}(A)$.
  \end{thm}
 \end{tcolorbox}
%連立1次方程式$A\bm{x} =\bm{b}$を解くためには, 拡大係数行列$[A:\bm{b}]$を簡約化してあげれば良い(この解き方を掃き出し法・ガウスの消去法とも言います). 

\begin{tcolorbox}[
    colback = white,
    colframe = green!35!black,
    fonttitle = \bfseries,
    breakable = true]
連立1次方程式$A\bm{x} =\bm{b}$の解きかた(掃き出し法・ガウスの消去法).
 \begin{enumerate}
 \item[手順1.] 連立方程式$A\bm{x} =\bm{b}$から拡大係数行列$[A:\bm{b}]$を作る.
 \item[手順2.] 拡大係数行列$[A:\bm{b}]$を(行)基本変形で簡約化する.
 \item[手順3.] その簡約化された行列のデータから連立方程式を書き下し, 一般解を求める.
 \end{enumerate}
 \end{tcolorbox}
 
\begin{exa}
連立1次方程式
 $
 \left\{ 
\begin{matrix}
x_1&+&2x_2& = &2 \\
x_1&+&4x_2& = &4\\
\end{matrix}
\right.
 $
 を解け.
 
 (解). 連立方程式の拡大係数行列は
 $[A:\bm{b}]=
  \begin{pmatrix}
 1& 2& 2  \\
 2& 4& 4  \\
 \end{pmatrix}
 $
 である. これを簡約化すると
 $
  \begin{pmatrix}
 1& 2& 2  \\
 0& 0& 0  \\
 \end{pmatrix} 
 $
 となる. よってこれより
 $
  \left\{ 
\begin{matrix}
x_1&+&2x_2& = &2 \\
0x_1&+&0x_2& = &0\\
\end{matrix}
\right.
$
である. 

以上より解は
$
 \left\{ 
\begin{matrix}
x_1&=& 2 -2c_2\\
x_2 &=& c_2\\
\end{matrix}
\text{\,\, ($c_2$は任意定数)}
\right.
$
となる. 

解の書き方として
$
\begin{pmatrix}
x_1\\
x_2 \\
\end{pmatrix}
=
\begin{pmatrix}
2\\
0 \\
\end{pmatrix}
+t 
\begin{pmatrix}
-2\\
1 \\
\end{pmatrix}
(t \in \R)
$
と書くこともある.
\end{exa}

\begin{exa}
連立1次方程式
 $
 \left\{ 
\begin{matrix}
x_1&+&2x_2& = &2 \\
x_1&+&4x_2& = &5\\
\end{matrix}
\right.
 $
 を解け.
 
 (解). 連立方程式の拡大係数行列は
 $[A:\bm{b}]=
  \begin{pmatrix}
 1& 2& 2  \\
 2& 4& 5  \\
 \end{pmatrix}
 $
 である. 
 これを簡約化すると
 $
  \begin{pmatrix}
 1& 2& 0  \\
 0& 0& 1 \\
 \end{pmatrix} 
 $
 となる. よってこれより
 $
  \left\{ 
\begin{matrix}
x_1&+&2x_2& = &0 \\
0x_1&+&0x_2& = &1\\
\end{matrix}
\right.
$
である. 以上より解は存在しない.
\end{exa}

\begin{exa}
連立1次方程式
 $
 \left\{ 
\begin{matrix}
x_1&-&2x_2&   &		&+&3x_4& &	&= 2 \\
x_1&-&2x_2& + &x_3&+&2x_4&+&x_5&= 2 \\
2x_1&-&4x_2& + &x_3&+&5x_4&+&2x_5&= 5 \\
\end{matrix}
\right.
 $
 を解け.
 
 (解). 拡大係数行列は 
 $[A:\bm{b}]=
  \begin{pmatrix}
 1& -2& 0 & 3& 0& 2   \\
  1& -2& 1& 2& 1& 2   \\
 2& -4& 1 & 5& 2& 5   \\
 \end{pmatrix}
 $
 である. これを基本変形で簡約化すると
 $
  \begin{pmatrix}
 1& -2& 0 & 3& 0& 2   \\
 0& 0& 1& -1& 0& 1   \\
 0& 0& 0 & 0& 1& 1   \\
 \end{pmatrix}
 $
 となる.
 これをもう一回式に書き下すと
 $$
\left \{
 \begin{matrix}
x_1&-&2x_2&   &		&+&3x_4& &	&= 2 \\
      & &		&   &x_3       &- & x_4& &       &= -1 \\
      & & &   &    & &		& & x_5&= 1 \\
\end{matrix}
\right.
\text{である.}
 $$
 
以上より解は
$
 \left\{ 
\begin{matrix}
x_1&=& 2 +2c_2 -3c_4\\
x_2&=&c_2 \\
x_3&=& -1 + c_4\\
x_4&=&c_4 \\
x_5&=& 1\\
\end{matrix}
\text{\,\, ($c_2, c_4$は任意定数)}
\right.
$
となる. 
 
解の書き方として
$
\begin{pmatrix}
x_1\\
x_2 \\
x_3 \\
x_4 \\
x_5 \\
\end{pmatrix}
=
\begin{pmatrix}
2\\
0 \\
-1 \\
0\\
1 \\
\end{pmatrix}
+ s
\begin{pmatrix}
2\\
1\\
0\\
0\\
0 \\
\end{pmatrix}
+ t
\begin{pmatrix}
-3\\
0\\
1\\
1\\
0 \\
\end{pmatrix}
(s, t \in \R)
$
と書くこともある.

 \end{exa}

\begin{rema}
実際に連立1次方程式をプログラミングで解くときも, 掃き出し法・ガウスの消去法によって解きます. 実際にc++で書いたソースコードを以下のホームページで見ることができます.\footnote{第7回授業で簡約化の証明をする際にもこのホームページを参考にさせていただきました.}
\begin{itemize}
\item Gauss-Jordan の掃き出し法と、連立一次方程式の解き方 \\
 \texttt{https://drken1215.hatenablog.com/entry/2019/03/20/202800}
\end{itemize}
\end{rema}

\section{演習問題}
演習問題の解答は授業動画にあります.

1.
連立1次方程式
 $
 \left\{ 
\begin{matrix}
x_1& &         &  +& 2x_3&- &x_4&+ & 2x_5&= 3 \\
2x_1&+&x_2& + &3x_3&-&x_4&-&x_5&= -1 \\
-x_1&+&3x_2& - &5x_3&+&4x_4&+&x_5&= -6 \\
\end{matrix}
\right.
 $
 を解け.

\newpage

\begin{center}
{\Large 第7回. 正則行列 (三宅先生の本, 2.4の内容)}
\end{center}

\begin{flushright}
 岩井雅崇 2022/06/02
\end{flushright}
\section{正則行列}

\begin{tcolorbox}[
    colback = white,
    colframe = green!35!black,
    fonttitle = \bfseries,
    breakable = true]
    \begin{dfn}
$A$を$n$次正方行列とする.
 ある行列$B$があって
 $$
 AB =BA =E_{n} %\text{(\,\,\,ただし$E_n$は単位行列)}
 $$
 となるとき\underline{$B$を$A$の逆行列}といい$B=A^{-1}$とかく.
 
 行列$A$が逆行列$A^{-1}$を持つとき, $A$は\underline{正則行列}という(\underline{$A$は正則である}ともいう).
  \end{dfn}
 \end{tcolorbox}
 
 \begin{exa}
 $A=
  \begin{pmatrix}
 1& -5  \\
 0& 1  \\
 \end{pmatrix} 
 $
 の逆行列は
  $A^{-1}=
  \begin{pmatrix}
 1& 5  \\
 0& 1  \\
 \end{pmatrix} 
 $
 である. \\ 
 実際
  $
  \begin{pmatrix}
 1& -5  \\
 0& 1  \\
 \end{pmatrix} 
  \begin{pmatrix}
 1& 5  \\
 0& 1  \\
 \end{pmatrix} 
=
  \begin{pmatrix}
 1& 5  \\
 0& 1  \\
 \end{pmatrix} 
   \begin{pmatrix}
 1& -5  \\
 0& 1  \\
 \end{pmatrix} 
 =
   \begin{pmatrix}
 1& 0 \\
 0& 1  \\
 \end{pmatrix} 
 $
 である.
 特に$A$は正則行列である. 
 \end{exa}

 \begin{exa}
2次正方行列
 $A=
  \begin{pmatrix}
 a& b  \\
 c& d  \\
 \end{pmatrix} 
 $
 について
  $ad-bc \neq 0$ならば, $A$は逆行列を持ち
 $$
 A^{-1} =   
 \frac{1}{ad-bc}
 \begin{pmatrix}
 d& -b  \\
 -c& a  \\
 \end{pmatrix} 
 \text{\,\,\,である.}
 $$
  特に$A$は正則行列である. 
 \end{exa}
 
  \begin{exa}
  $
   A=\begin{pmatrix}
 0& 1 \\
 0& 1  \\
 \end{pmatrix} 
 $
 は逆行列を持たない. 特に$A$は正則行列ではない.
  \end{exa}
  
  \begin{tcolorbox}[
    colback = white,
    colframe = green!35!black,
    fonttitle = \bfseries,
    breakable = true]
    \begin{thm}
    $A$を$n$次正方行列とするとき, 以下は同値.
\begin{enumerate}
\item $\rank (A) =n$
\item $A$の簡約化は$E_n$である.
\item 任意の$n$次列ベクトル$\bm{b}$について, $A \bm{x}=\bm{b}$はただ一つの解をもつ.
\item $A \bm{x}=0$の解は$\bm{x}=0$に限る.
\item $A$は正則行列.
\item $A$の行列式$\det(A)$は0ではない(行列式に関しては第9, 10, 11回の講義でやります).
\end{enumerate}
  \end{thm}
 \end{tcolorbox}
 
 \section{掃き出し法を使った逆行列の求め方}
 \begin{tcolorbox}[
    colback = white,
    colframe = green!35!black,
    fonttitle = \bfseries,
    breakable = true]
    \begin{thm}
    $A$を$n$次正方行列とし, $n \times 2n$行列$[A : E_n]$の簡約化が$[E_n : B]$となるとする.
    このとき$A$は正則行列で, $B$は$A$の逆行列である.
  \end{thm}
 \end{tcolorbox}
 この定理により掃き出し法を用いて逆行列を得ることができる.
 
 \begin{exa}
 $
  A=\begin{pmatrix}
 1& 2&1 \\
 2& 3 & 1 \\
 1& 2 &  2 \\
 \end{pmatrix} 
 $
 の逆行列を求めよ.
 
 (解).
 $[A:E_3] = 
 \begin{pmatrix}
 1& 2&1  &1& 0&0 \\
 2& 3 & 1 &0& 1&0 \\
 1& 2 &  2 &0& 0&1 \\
 \end{pmatrix} 
 $
 を(行)基本変形を用いて簡約化すると, \\
 $
 \begin{pmatrix}
 1& 0&0  &-4& 2&1 \\
 0& 1 & 0 &3& -1&-1 \\
 0& 0&  1 &-1& 0&1 \\
 \end{pmatrix} 
 $
 となる. よって$A$の逆行列は
 $
 \begin{pmatrix}
-4& 2&1 \\
3& -1&-1\\
1& 0&1 \\
 \end{pmatrix} 
 $
 である.
 \end{exa}

\section{演習問題}
演習問題の解答は授業動画にあります.

1.
$
\begin{pmatrix}
 2& -1& 0\\
 2& -1 & -1 \\
 1& 0 &  -1 \\
 \end{pmatrix} 
 $
 の逆行列を求めよ.

 

\newpage

\begin{center}
{\Large 第9回. 行列式1 -置換- (三宅先生の本, 3.1の内容)}
\end{center}

\begin{flushright}
 岩井雅崇 2022/06/16
\end{flushright}

\section{置換}

\begin{tcolorbox}[
    colback = white,
    colframe = green!35!black,
    fonttitle = \bfseries,
    breakable = true]
    \begin{dfn}
    \text{}
    \begin{itemize}
\item $\{ 1, \ldots, n\}$から$\{ 1, \ldots, n\}$への1対1写像を\underline{置換}と言い$\sigma$で表す.
つまり置換$\sigma$とは$k_1, \ldots, k_n$を1から$n$の並び替えとして, 
1を$k_1$に, 2を$k_2$に, $\cdots$, $n$を$k_n$にと変化させる規則のことである.
\item 上の置換$\sigma$を
$$
\sigma =
  \begin{pmatrix}
 1& 2  &\cdots &n\\
 k_1& k_2  &\cdots &k_n\\
 \end{pmatrix} 
$$
とかき, $\sigma(1) =k_1, \sigma(2) =k_2, \ldots, \sigma(n) =k_n$とする.
    \end{itemize}
  \end{dfn}
 \end{tcolorbox}
 
 \begin{exa}
 置換$\sigma$を
$
\sigma =
  \begin{pmatrix}
 1& 2  &3 &4\\
 3& 1  &4 &2\\
 \end{pmatrix} 
$
とする. 
これは「1を$3$に, 2を$1$に, 3を4に, 4を$2$にと変化させる規則」である.
 $\sigma(1) =3, \sigma(2) =1, \sigma(3) =4,  \sigma(4) =2$である.
 \end{exa}
 
 \begin{exa}
 置換$\sigma$を
$
\sigma =
  \begin{pmatrix}
 1& 2  &3 \\
 2& 1  &3 \\
 \end{pmatrix} 
$
とする. 
これは「1を$2$に, 2を$1$に, 3を3にと変化させる規則」である.
 $\sigma(1) =2, \sigma(2) =1, \sigma(3) =3$である.
 
 この置換は3に関しては何も変化させていないので
 $
\sigma =
  \begin{pmatrix}
 1& 2   \\
 2& 1   \\
 \end{pmatrix} 
$
ともかく.
 \end{exa}

\begin{tcolorbox}[
    colback = white,
    colframe = green!35!black,
    fonttitle = \bfseries,
    breakable = true]
    \begin{dfn}
置換$\sigma, \tau$について, その積$\sigma \tau$を
$\sigma(\tau(i))$で定める.
  \end{dfn}
 \end{tcolorbox}
 
\begin{exa}
 置換$\sigma, \tau$を
$
\sigma =
  \begin{pmatrix}
 1& 2  &3 & 4 \\
 4& 3  &1  &2 \\
 \end{pmatrix} 
\tau=
  \begin{pmatrix}
 1& 2  &3 & 4 \\
 2& 3  &4  &1 \\
 \end{pmatrix} 
 $
とすると, 
$$
  \begin{matrix}
 \sigma (\tau (1)) &= &  \sigma (2)  & = & 3  \\
 \sigma (\tau (2)) &= &  \sigma (3)  & = & 1 \\
 \sigma (\tau (3)) &= &  \sigma (4)  & = & 2  \\
 \sigma (\tau (4)) &= &  \sigma (1)  & = & 4  \\
 \end{matrix} 
 \text{\,\,\,であるので, }
 \sigma \tau
= 
 \begin{pmatrix}
 1& 2  &3 & 4 \\
 3& 1  &2  &4 \\
 \end{pmatrix} 
 \text{である.}
$$

\end{exa}


\begin{tcolorbox}[
    colback = white,
    colframe = green!35!black,
    fonttitle = \bfseries,
    breakable = true]
    \begin{dfn}
\text{}
\begin{itemize}
\item $
\epsilon =
  \begin{pmatrix}
 1& 2  &\cdots &n\\
 1& 2  &\cdots &n\\
 \end{pmatrix} $を\underline{単位置換}という.
 \item  
 $ \sigma =
  \begin{pmatrix}
 1& 2  &\cdots &n\\
 k_1& k_2  &\cdots &k_n\\
 \end{pmatrix} 
$について, 
$
  \begin{pmatrix}
 k_1& k_2  &\cdots &k_n\\
 1& 2  &\cdots &n\\
 \end{pmatrix} 
$を\underline{$\sigma$の逆置換}と言い$\sigma^{-1}$で表す.
\end{itemize}
  \end{dfn}
 \end{tcolorbox}

\begin{exa} 
$\sigma = 
\begin{pmatrix}
 1& 2  &3 & 4 & 5\\
 4& 5  &1  &3 &2\\
 \end{pmatrix} 
$
とするとき
$
\sigma^{-1}
=
\begin{pmatrix}
 4& 5  &1  &3 &2\\
 1& 2  &3 & 4 & 5\\
 \end{pmatrix} 
 =
 \begin{pmatrix}
 1& 2  &3 & 4 & 5\\
 3& 5  &4  &1 &2\\
 \end{pmatrix} 
 \text{である.}
$
\end{exa}

\begin{tcolorbox}[
    colback = white,
    colframe = green!35!black,
    fonttitle = \bfseries,
    breakable = true]
    \begin{dfn}
 $ \sigma =
  \begin{pmatrix}
 k_1& k_2  &\cdots &k_l\\
 k_2& k_3  &\cdots &k_1\\
 \end{pmatrix} 
$となる置換$\sigma$を\underline{巡回置換}と言い
$\sigma =
  \begin{pmatrix}
 k_1& k_2  &\cdots &k_l\\
 \end{pmatrix} 
$と表す.

特に $ \sigma =
  \begin{pmatrix}
 k_1& k_2  \\
 k_2& k_1 \\
 \end{pmatrix} 
$となる巡回置換を\underline{互換}と言い$\sigma =
  \begin{pmatrix}
 k_1& k_2 \\
 \end{pmatrix} 
$と表す.
  \end{dfn}
 \end{tcolorbox}


\begin{tcolorbox}[
    colback = white,
    colframe = green!35!black,
    fonttitle = \bfseries,
    breakable = true]
    \begin{thm}
任意の置換$\sigma$は互換の積$\tau_1 \cdots \tau_{l}$で表わすことができ, $l$の偶奇は$\sigma$によってのみ定まる.
  \end{thm}
 \end{tcolorbox}
 
 \begin{tcolorbox}[
    colback = white,
    colframe = green!35!black,
    fonttitle = \bfseries,
    breakable = true]
    \begin{dfn}
置換$\sigma$が互換の積$\tau_1 \cdots \tau_{l}$で表せられているとする.
\begin{itemize}
\item $\sgn (\sigma) = (-1)^{l}$とし, これを\underline{$\sigma$の符号}と呼ぶ.
\item $\sgn (\sigma) = 1$なる置換$\sigma$を\underline{偶置換}といい, $\sgn (\sigma) = -1$なる置換$\sigma$を\underline{奇置換}という.
\end{itemize}
  \end{dfn}
 \end{tcolorbox}
 
 \begin{exa}
 $
 \sigma 
  =
 \begin{pmatrix}
 1& 2  &3 & 4 & 5 & 6 & 7\\
 4& 1  &6  &2 &7 & 5 & 3\\
 \end{pmatrix} 
 $を互換の積で表し, その符号を求めよ.
 
 (解). 
 $1 \overset{\sigma}{\rightarrow} 4 \overset{\sigma}{\rightarrow}2 \overset{\sigma}{\rightarrow}1 $と変化し,  
  $3 \overset{\sigma}{\rightarrow} 6\overset{\sigma}{\rightarrow}5 \overset{\sigma}{\rightarrow}7 \overset{\sigma}{\rightarrow}3$と変化するので, 
  $$
  \sigma = 
   \begin{pmatrix}
 1& 4 &2 
 \end{pmatrix} 
    \begin{pmatrix}
 3& 6 &5 &7
 \end{pmatrix} 
 \text{である.}
  $$
  さらに
  $   \begin{pmatrix}
 1& 4 &2 
 \end{pmatrix} 
 = 
 \begin{pmatrix}
 1& 4  
 \end{pmatrix} 
 \begin{pmatrix}
 4 &2 
 \end{pmatrix},
\begin{pmatrix}
 3& 6 &5 &7
 \end{pmatrix} 
 =
 \begin{pmatrix}
 3& 6  
 \end{pmatrix} 
  \begin{pmatrix}
 6& 5  
 \end{pmatrix} 
  \begin{pmatrix}
 5& 7  
 \end{pmatrix} 
 $
 であるので, 
 $$
\sigma= 
\begin{pmatrix}
 1& 4  
 \end{pmatrix} 
 \begin{pmatrix}
 4 &2 
 \end{pmatrix}
 \begin{pmatrix}
 3& 6  
 \end{pmatrix} 
  \begin{pmatrix}
 6& 5  
 \end{pmatrix} 
  \begin{pmatrix}
 5& 7  
 \end{pmatrix} 
 $$
 となり, $\sgn(\sigma)= (-1)^{5}=-1$である.
 
 \end{exa}

  \begin{tcolorbox}[
    colback = white,
    colframe = green!35!black,
    fonttitle = \bfseries,
    breakable = true]
    \begin{prop}置換$\sigma, \tau$について, 
    $\sgn(\epsilon) = 1$, $\sgn(\sigma^{-1}) = \sgn(\sigma)$, 
$\sgn(\sigma \tau) = \sgn(\sigma) \sgn(\tau) $が成り立つ(ただし$\epsilon$は単位置換とする).
  \end{prop}
 \end{tcolorbox}
 
 
  \begin{tcolorbox}[
    colback = white,
    colframe = green!35!black,
    fonttitle = \bfseries,
    breakable = true]
    \begin{dfn}
$S_n$を$n$文字置換の集合とし, $A_n$を$n$文字置換の集合とする.
  \end{dfn}
 \end{tcolorbox}
 \footnote{専門用語で$S_n$は対称群と言い, $A_n$は交代群と言います. }

  \begin{tcolorbox}[
    colback = white,
    colframe = green!35!black,
    fonttitle = \bfseries,
    breakable = true]
    \begin{prop}\text{}
    \begin{itemize}
\item $S_n$の個数は$n!$個である.
\item 偶置換と奇置換の個数は同じである.
\item $A_n$の個数は$\frac{n!}{2}$個である.
\item $\sigma, \tau \in A_n$ならば$\sigma \tau \in A_n$
    \end{itemize}
  \end{prop}
 \end{tcolorbox}

\newpage

\begin{center}
{\Large 第10回. 行列式2 -行列式の計算方法- (三宅先生の本, 3.2, 3.3の内容)} 
\end{center}

\begin{flushright}
 岩井雅崇 2022/06/23
\end{flushright}

%この資料では第10回授業と第11回授業の内容を取り扱います.

\section{行列式}

\begin{tcolorbox}[
    colback = white,
    colframe = green!35!black,
    fonttitle = \bfseries,
    breakable = true]
    \begin{dfn}
$n$次正方行列$A = (a_{ij})$について
$$
\det(A) =  \sum_{\sigma \in S_n}\sgn(\sigma) 
a_{1 \sigma(1)} a_{2 \sigma(2)} \cdots a_{n \sigma(n)} 
\text{を\underline{$A$の行列式}と言う.}
$$
 $A$の行列式は$\det(A)$, $|A|$, 
$
\begin{vmatrix}
a_{11}& a_{12} & \cdots &a_{1n} \\
a_{21}& a_{22} & \cdots &a_{2n} \\
\vdots& \vdots	&	\ddots   &	\vdots \\
a_{n1}& a_{n2} & \cdots &a_{nn} \\
\end{vmatrix}
$
ともかく.
  \end{dfn}
 \end{tcolorbox}

\begin{exa}
\label{2jidet}
$A = 
  \begin{pmatrix}
a_{11}& a_{12}\\
a_{21}& a_{22}\\
 \end{pmatrix} 
$
とすると$\det(A) = a_{11}a_{22} - a_{12}a_{21}$である.

(証).
$S_2 = \left\{   \begin{pmatrix}
1& 2\\
1& 2\\
 \end{pmatrix} , 
   \begin{pmatrix}
1&2\\
2& 1\\
 \end{pmatrix} 
  \right\}$
  であるので, $A$の行列式は
  \begin{align*}
  \det(A) &= 
  \sgn \begin{pmatrix}
1& 2\\
1& 2\\
 \end{pmatrix} a_{11}a_{22}
 +
  \sgn \begin{pmatrix}
1& 2\\
2& 1\\
 \end{pmatrix} a_{12}a_{21}
 =
 a_{11}a_{22} - a_{12}a_{21}.
  \end{align*}

\end{exa}

\begin{exa}
$A = 
  \begin{pmatrix}
a_{11}& a_{12} & a_{13}\\
a_{21}& a_{22} & a_{23}\\
a_{31}& a_{32} & a_{33}\\
 \end{pmatrix} 
$
の行列式を求める.

$S_3 = \left\{   
\begin{pmatrix}
1& 2 &3\\
1& 2 &3\\
 \end{pmatrix} , 
\begin{pmatrix}
1& 2 &3\\
2& 1 &3\\
 \end{pmatrix} , 
\begin{pmatrix}
1& 2 &3\\
1& 3 &2\\
 \end{pmatrix} , 
 \begin{pmatrix}
1& 2 &3\\
3& 2 &1\\
 \end{pmatrix} , 
 \begin{pmatrix}
1& 2 &3\\
2& 3 &1\\
 \end{pmatrix} , 
 \begin{pmatrix}
1& 2 &3\\
3& 1 &2\\
 \end{pmatrix}
  \right\}$
  であるので, $A$の行列式は
  
  \begin{align*}
  \det(A) 
  &= 
  \sgn \begin{pmatrix}
1& 2 &3\\
1& 2 &3\\
 \end{pmatrix}
 a_{11}a_{22}a_{33}
 +
   \sgn \begin{pmatrix}
1& 2 &3\\
2& 1 &3\\
 \end{pmatrix} 
 a_{12}a_{21}a_{33}
 +
\sgn \begin{pmatrix}
1& 2 &3\\
1& 3 &2\\
 \end{pmatrix} 
 a_{11}a_{23}a_{32} 
  \\%%%
  &+
\sgn  \begin{pmatrix}
1& 2 &3\\
3& 2 &1\\
 \end{pmatrix}
 a_{13}a_{22}a_{31} 
 +
\sgn  \begin{pmatrix}
1& 2 &3\\
2& 3 &1\\
 \end{pmatrix}
 a_{12}a_{23}a_{31} 
  +
\sgn   \begin{pmatrix}
1& 2 &3\\
3& 1 &2\\
 \end{pmatrix}
 a_{13}a_{21}a_{32} 
 \\%%
   &= a_{11}a_{22}a_{33}- a_{12}a_{21}a_{33}- a_{11}a_{23}a_{32} 
   - a_{13}a_{22}a_{31}  + a_{12}a_{23}a_{31}  +  a_{13}a_{21}a_{32} 
  \end{align*}
  以上より
  $
 \det(A)= 
 a_{11}a_{22}a_{33}+ a_{12}a_{23}a_{31}  +  a_{13}a_{21}a_{32} 
- a_{11}a_{23}a_{32}     - a_{13}a_{22}a_{31}  - a_{12}a_{21}a_{33}
  $
  である.
\end{exa}
\begin{rema}
2次正方行列や3次正方行列の行列式は視覚的に綺麗に表わすことができる(サラスの公式と呼ばれる).
\end{rema}

\section{行列式の基本性質}

\begin{tcolorbox}[
    colback = white,
    colframe = green!35!black,
    fonttitle = \bfseries,
    breakable = true]
    \begin{thm}
   \label{determinant}
$A,B$を$n$次正方行列とする.
\begin{enumerate}
\item $\det({}^{t} A) =\det(A)$.
\item $\det(AB)=(\det(A))(\det(B)) = \det(BA)$.
\item $\det(A) \neq 0$であることと$A$が正則であることは同値.
\item 
$
\begin{vmatrix}
a_{11}& a_{12} & \cdots &a_{1n} \\
0 	   & a_{22} & \cdots &a_{2n} \\
\vdots& \vdots	&	\ddots   &	\vdots \\
0	& a_{n2} & \cdots &a_{nn} \\
\end{vmatrix}
=a_{11}
\begin{vmatrix}
 a_{22} & \cdots &a_{2n} \\
 \vdots	&	\ddots   &	\vdots \\
 a_{n2} & \cdots &a_{nn} \\
\end{vmatrix}
$.
\item 
$
\begin{vmatrix}
a_{11}& a_{12} & a_{13} &\cdots &a_{1n-1}&a_{1n} \\
0 	   & a_{22} & a_{23} &\cdots&a_{2n-1} &a_{2n} \\
0 	   & 0  		& a_{33} &\cdots &a_{3n-1}&a_{3n} \\
\vdots& \vdots	&  \ddots &\ddots&	\vdots  &	\vdots \\
0	& 0      		&      	&\ddots	&a_{n-1n-1} &a_{n-1n} \\
0	& 0      		&     \cdots	&	\cdots&0		&a_{nn} \\
\end{vmatrix}
=a_{11}a_{22}\cdots a_{nn}
$.
%以下行ベクトル$a_1, \cdots, a_n$を用いて$A = \begin{pmatrix}a_1 \\a_2 \\\vdots\\a_n\end{pmatrix}$とかけているとする.
\item  1つの行を$c$倍すると行列式は$c$倍される:
$
\begin{vmatrix}
a_{11}&  \cdots &a_{1n} \\
\vdots&	 	  &	\vdots \\
ca_{i1} & \cdots &ca_{in} \\
\vdots& 		   &	\vdots \\
a_{n1}	& \cdots &a_{nn} \\
\end{vmatrix}
=
c
\begin{vmatrix}
a_{11}&  \cdots &a_{1n} \\
\vdots&	 	  &	\vdots \\
a_{i1} & \cdots & a_{in} \\
\vdots& 		   &	\vdots \\
a_{n1}	& \cdots &a_{nn} \\
\end{vmatrix}
$.
\item  
$
\begin{vmatrix}
a_{11}&  \cdots &a_{1n} \\
\vdots&	 	  &	\vdots \\
b_{i1} + c_{i1} & \cdots &b_{in} + c_{in} \\
\vdots& 		   &	\vdots \\
a_{n1}	& \cdots &a_{nn} \\
\end{vmatrix}
=
\begin{vmatrix}
a_{11}&  \cdots &a_{1n} \\
\vdots&	 	  &	\vdots \\
b_{i1}  & \cdots &b_{in}  \\
\vdots& 		   &	\vdots \\
a_{n1}	& \cdots &a_{nn} \\
\end{vmatrix}
+
\begin{vmatrix}
a_{11}&  \cdots &a_{1n} \\
\vdots&	 	  &	\vdots \\
c_{i1} & \cdots & c_{in} \\
\vdots& 		   &	\vdots \\
a_{n1}	& \cdots &a_{nn} \\
\end{vmatrix}
$.
\item 2つの行を入れ替えたら, 行列式は$-1$倍される:
$$
\begin{vmatrix}
a_{11}&  \cdots &a_{1n} \\
\vdots& 		   &	\vdots \\
a_{j1} & \cdots & a_{jn} \\
\vdots&	 	  &	\vdots \\
a_{i1} & \cdots & a_{in} \\
\vdots& 		   &	\vdots \\
a_{n1}	& \cdots &a_{nn} \\
\end{vmatrix}
= (-1)
\begin{vmatrix}
a_{11}&  \cdots &a_{1n} \\
\vdots&	 	  &	\vdots \\
a_{i1} & \cdots & a_{in} \\
\vdots& 		   &	\vdots \\
a_{j1} & \cdots & a_{jn} \\
\vdots& 		   &	\vdots \\
a_{n1}	& \cdots &a_{nn} \\
\end{vmatrix}.
$$
\item 第$i$行の$c$倍を第$j$行に加えても行列式は変わらない:
$$
\begin{vmatrix}
a_{11}&  \cdots &a_{1n} \\
\vdots& 		   &	\vdots \\
a_{j1} +ca_{i1}& \cdots & a_{jn} +ca_{in}\\
\vdots& 		   &	\vdots \\
a_{n1}	& \cdots &a_{nn} \\
\end{vmatrix}
= 
\begin{vmatrix}
a_{11}&  \cdots &a_{1n} \\
\vdots& 		   &	\vdots \\
a_{j1} & \cdots & a_{jn} \\
\vdots& 		   &	\vdots \\
a_{n1}	& \cdots &a_{nn} \\
\end{vmatrix}.
$$
\item 列ベクトルに関して上の6から9と同様のことが成り立つ.
\end{enumerate}
  \end{thm}
 \end{tcolorbox}
 
 \begin{tcolorbox}[
    colback = white,
    colframe = green!35!black,
    fonttitle = \bfseries,
    breakable = true]
    \begin{cor}
$A,B$を$n$次正方行列とする. $AB=E_n$ならば, $A$は正則で$B$は$A$の逆行列.
  \end{cor}
 \end{tcolorbox}
 
 \section{行列式の計算方法}
 定理\ref{determinant}を用いると行列式を比較的簡単に計算できる.
 
  \begin{exa}
 $
 \begin{pmatrix}
 1&3&4\\
 -2&-5&7\\
 -3&2&-1\\
 \end{pmatrix}
$
の行列式を定理\ref{determinant}を用いて計算すると次の通りになる.

\begin{align*}
 &\begin{vmatrix}
 1&3&4\\
 -2&-5&7\\
 -3&2&-1\\
 \end{vmatrix}
 \overset{\text{定理\ref{determinant}.(9)} } {=}
 \begin{vmatrix}
 1&3&4\\
 0&1&15\\
 0&11&11\\
 \end{vmatrix}
 \overset{\text{定理\ref{determinant}.(4)} } {=}
 1
 \begin{vmatrix}
1&15\\
11&11\\
 \end{vmatrix}
  \overset{\text{定理\ref{determinant}.(6)} } {=}
 11
 \begin{vmatrix}
1&15\\
1&1\\
 \end{vmatrix}\overset{\text{例 \ref{2jidet}} } {=}
 11 \left\{ 1 \times 1 - 15 \times 1 \right\} 
 =
 -154.
 \end{align*}
 
\end{exa}



 \begin{exa}
 $
 \begin{pmatrix}
 2&-4&-5&3\\
 -6&13&14&1\\
 1&-2&-2&-8\\
 2&-5&0&5\\
 \end{pmatrix}
$
の行列式を定理\ref{determinant}を用いて計算すると次の通りになる.
\begin{align*}
 &\begin{vmatrix}
 2&-4&-5&3\\
 -6&13&14&1\\
 1&-2&-2&-8\\
 2&-5&0&5\\
 \end{vmatrix}
 \overset{\text{定理\ref{determinant}.(8)} } {=}
 (-1)
  \begin{vmatrix}
   1&-2&-2&-8\\
 -6&13&14&1\\
 2&-4&-5&3\\
 2&-5&0&5\\
 \end{vmatrix}
  \overset{\text{定理\ref{determinant}.(9)} }  {=}
 (-1)
  \begin{vmatrix}
   1&-2&-2&-8\\
 0&1 &2  &-47\\
 0& 0&-1&19\\
 0&-1&4&21\\
 \end{vmatrix}
\\ %%%
& \overset{\text{定理\ref{determinant}.(4)} } {=}
 (-1)
  \begin{vmatrix}
1 &2  &-47\\
 0&-1&19\\
-1&4&21\\
 \end{vmatrix}
  \overset{\text{定理\ref{determinant}.(9)} } {=}
   (-1)
  \begin{vmatrix}
1 &2  &-47\\
 0&-1&19\\
 0&6&-26\\
 \end{vmatrix}
 \overset{\text{定理\ref{determinant}.(4)} } {=}
  (-1)
    \begin{vmatrix}
-1&19\\
6&-26\\
 \end{vmatrix}
 \\ %%
 & \overset{\text{例 \ref{2jidet}} } {=}
 (-1)\left\{(-1)\times (-26) - 6\times 19\right\} = 88.
\end{align*}

 \end{exa}

\section{演習問題}
演習問題の解答は授業動画にあります.

1. 行列式
$
\begin{vmatrix}
 0& -3& -6 &15 \\
 -2& 5& 14 &4 \\
 1& -3& -2 &5 \\
 15 & 10& 10 &-5 \\
 \end{vmatrix} 
 $
 を計算せよ.


\newpage

\begin{center}
{\Large 第11回. 行列式3 -行列式の基本性質- (三宅先生の本, 3.2, 3.3の内容)} 
\end{center}

\begin{flushright}
 岩井雅崇 2022/06/30
\end{flushright}

一部の内容について, 齋藤正彦著 線型代数学 (東京図書)の第3章を参考にした.

%\section{行列式の基本性質}

\begin{tcolorbox}[
    colback = white,
    colframe = green!35!black,
    fonttitle = \bfseries,
    breakable = true]
    \begin{prop}
$a_1, \ldots, a_{n}$を行ベクトルとし, $n$次正方行列
$A = 
\begin{pmatrix}
a_1 \\ \vdots \\ a_{n}
\end{pmatrix}
$とする.
\begin{enumerate}
\item $\tau$を$n$次の置換とすると
$$
\det \begin{pmatrix}
a_{\tau(1)} \\ \vdots \\ a_{\tau(n)} 
\end{pmatrix}
= 
\sgn(\tau) \det \begin{pmatrix}
a_1 \\ \vdots \\ a_{n}
\end{pmatrix}
= \sgn(\tau) \det(A).
\text{\,\,\,(交代性)}
$$
\item $b_i, c_i$を行ベクトルとし, $\alpha, \beta$を数とすると,
$$
\det \begin{pmatrix}
a_1 \\ \vdots \\ \alpha b_i + \beta c_i \\ \vdots  \\ a_{n}
\end{pmatrix}
= 
\alpha\det \begin{pmatrix}
a_1 \\ \vdots \\  b_i \\ \vdots  \\ a_{n}
\end{pmatrix}
+
\beta
\det \begin{pmatrix}
a_1 \\ \vdots \\ c_i \\ \vdots  \\ a_{n}
\end{pmatrix}
\text{\,\,\,(多重線型性)}
$$
\end{enumerate}

  \end{prop}
 \end{tcolorbox}
 

\begin{tcolorbox}[
    colback = white,
    colframe = green!35!black,
    fonttitle = \bfseries,
    breakable = true]
    \begin{thm}
行ベクトル$x_1, \ldots, x_n$について, 数
$F\begin{pmatrix}
x_1 \\ \vdots \\ x_{n}
\end{pmatrix}$
を対応させる関数$F$を考える.
この$F$が交代性と多重線型性を満たすとき, 
$$
F\begin{pmatrix}
x_1 \\ \vdots \\ x_{n}
\end{pmatrix}
=
F\begin{pmatrix}
f_1 \\ \vdots \\ f_{n}
\end{pmatrix}
\det
\begin{pmatrix}
x_1 \\ \vdots \\ x_{n}
\end{pmatrix}
\text{となる.}
$$
ここで$f_i =
\begin{pmatrix}
0 & \cdots&\overset{i}{\hat{1}}&\cdots &0
\end{pmatrix}
$
という行ベクトルとする.

特に行列$A$に対して数$F(A)$を対応させる関数が, 行に関して交代性と多重線型性を満たすとき
$F(A) =F(E_n) \det(A)$となる.
  \end{thm}
 \end{tcolorbox}
 
\newpage

\begin{center}
{\Large 第12回. 余因子行列と余因子展開 (三宅先生の本, 3.4の内容)} 
\end{center}

\begin{flushright}
 岩井雅崇 2022/07/07
\end{flushright}


\section{余因子行列}

\begin{tcolorbox}[
    colback = white,
    colframe = green!35!black,
    fonttitle = \bfseries,
    breakable = true]
    \begin{dfn}
    $n$次正方行列$A=(a_{ij})$の$i$行と$j$列を取り除いた$n-1$次正方行列を$\tilde{A}_{ij}$とかく(この授業だけの記法). つまり
  $$
  \tilde{A}_{ij}
  =
    \begin{pmatrix}
a_{11}&   \cdots &a_{1j-1}&a_{1j+1}&\cdots&a_{1n} \\
\vdots&   		& \vdots &\vdots &   		&\vdots  \\
a_{i-11}&   \cdots &a_{i-1j-1}&a_{i-1j+1}&\cdots&a_{i-1n} \\
a_{i+11}&   \cdots &a_{i+1j-1}&a_{i+1j+1}&\cdots&a_{i+1n} \\
\vdots&   		& \vdots &\vdots &   		&\vdots  \\
a_{n1}&   \cdots &a_{nj-1}&a_{nj+1}&\cdots&a_{nn} \\
\end{pmatrix}
\text{とする.}
$$
    \end{dfn}
 \end{tcolorbox}
\begin{exa}
$A=
\begin{pmatrix}
a_{11} & a_{12} \\
a_{21} & a_{22}
\end{pmatrix}
$
のとき, 
$  \tilde{A}_{11} =(a_{22})$, $  \tilde{A}_{12} =(a_{21})$, $  \tilde{A}_{21} =(a_{12})$, $  \tilde{A}_{22} =(a_{11})$.
\end{exa}
\begin{exa}
$
A=
\begin{pmatrix}
a_{11} & a_{12}&a_{13} \\
a_{21} & a_{22}&a_{23} \\
a_{31} & a_{32}&a_{33} \\
\end{pmatrix}
$
のとき, 
$  \tilde{A}_{12} =
\begin{pmatrix}
a_{21} & a_{23} \\
a_{31} & a_{33}
\end{pmatrix}
$, 
$  \tilde{A}_{22} =
\begin{pmatrix}
a_{11} & a_{13} \\
a_{31} & a_{33}
\end{pmatrix}
$, 
$  \tilde{A}_{31} =
\begin{pmatrix}
a_{12} & a_{13} \\
a_{22} & a_{23}
\end{pmatrix}
$.
\end{exa}

\begin{tcolorbox}[
    colback = white,
    colframe = green!35!black,
    fonttitle = \bfseries,
    breakable = true]
    \begin{dfn}
    $n$次正方行列$A=(a_{ij})$について, $\tilde{A} =(b_{ij})$を
    \underline{$
    b_{ij} = (-1)^{i+j} \det(\tilde{A}_{ji})$}
    で定める.
 \underline{$\tilde{A}$を$A$の余因子行列}という.
    \end{dfn}
 \end{tcolorbox}
\begin{exa}
\label{inverse_2}
$
\begin{pmatrix}
a_{11} & a_{12} \\
a_{21} & a_{22}
\end{pmatrix}
$
のときの余因子行列$\tilde{A} $を求める.
$  \tilde{A}_{11} =(a_{22})$, $  \tilde{A}_{12} =(a_{21})$, $  \tilde{A}_{21} =(a_{12})$, $  \tilde{A}_{22} =(a_{11})$より次が成り立つ.
\begin{itemize}
\item $\tilde{A} $の$(1,1)$成分は$(-1)^{1+1}\det( \tilde{A}_{11}) = a_{22}$.
\item $\tilde{A} $の$(1,2)$成分は$(-1)^{1+2}\det( \tilde{A}_{21}) = -a_{12}$.
\item $\tilde{A} $の$(2,1)$成分は$(-1)^{2+1}\det( \tilde{A}_{12}) = -a_{21}$.
\item $\tilde{A} $の$(2,2)$成分は$(-1)^{2+2}\det( \tilde{A}_{22}) = a_{11}$.
\end{itemize}
以上より余因子行列$\tilde{A} = 
\begin{pmatrix}
a_{22} &- a_{12} \\
-a_{21} & a_{11}
\end{pmatrix}$となる.
\end{exa}


\begin{tcolorbox}[
    colback = white,
    colframe = green!35!black,
    fonttitle = \bfseries,
    breakable = true]
    \begin{thm}
 $A$を$n$次正方行列とする.
 \begin{enumerate}
\item %$i=1, \ldots, n, j=1, \ldots, n$について
任意の$1 \leqq i \leqq n, 1 \leqq j\leqq n$なる$i,j$について, 次が成り立つ.
 \begin{align*}
 \det(A) & =(-1)^{1+j}a_{1j}\det(\tilde{A}_{1j}) + \cdots +(-1)^{n+j}a_{nj}\det(\tilde{A}_{nj}) 
 \\
 &=(-1)^{i+1}a_{i1}\det(\tilde{A}_{i1}) + \cdots +(-1)^{i+n}a_{in}\det(\tilde{A}_{in}).
  \end{align*}
  これを\underline{余因子展開}という.
 \item $A\tilde{A} = \tilde{A}A =(\det A)E_n$. 特に$\det(A)\neq0$ならば$A^{-1} = \frac{1}{\det A} \tilde{A}$.
 \end{enumerate}
     \end{thm}
 \end{tcolorbox}

\begin{exa}行列
$A=
\begin{pmatrix}
2 & 7&13 & 5\\
5 & 3&8 & 2\\
0 & 0 & 9  & 4\\
0 & 0&-2 & 1\\
\end{pmatrix}
$の行列式$\det(A)$を余因子展開で求める.

\begin{align*}
\det(A) 
&= 
(-1)^{1+1}a_{11}\det(\tilde{A}_{11}) + (-1)^{2+1}a_{21}\det(\tilde{A}_{21}) + (-1)^{3+1}a_{31}\det(\tilde{A}_{31}) + (-1)^{4+1}a_{41}\det(\tilde{A}_{41}) 
\\ %%
&=
2 
\begin{vmatrix}
 3&8 & 2\\
0 & 9  & 4\\
0&-2 & 1\\
\end{vmatrix}
- 5 
\begin{vmatrix}
 7&13 & 5\\
0 & 9  & 4\\
0&-2 & 1\\
\end{vmatrix}
+0
\begin{vmatrix}
 7&13 & 5\\
 3&8 & 2\\
0&-2 & 1\\
\end{vmatrix}
-0
\begin{vmatrix}
 7&13 & 5\\
 3&8 & 2\\
0 & 9  & 4\\
\end{vmatrix}
\\ %%
&=2 
\begin{vmatrix}
 3&8 & 2\\
0 & 9  & 4\\
0&-2 & 1\\
\end{vmatrix}
- 5 
\begin{vmatrix}
 7&13 & 5\\
0 & 9  & 4\\
0&-2 & 1\\
\end{vmatrix}
\\%%
&=2 \times 3
\begin{vmatrix}
9  & 4\\
-2 & 1\\
\end{vmatrix}
-5 \times 7
\begin{vmatrix}
9  & 4\\
-2 & 1\\
\end{vmatrix}
=(2 \times 3 - 5 \times 7) \times (9 \times 1 - 4 \times (-2)) = -493.
\end{align*}
\end{exa}


\begin{exa}
2次正方行列
$
A = 
\begin{pmatrix}
a & b \\
c & d
\end{pmatrix}
$について, $\det A =ad-bc \neq0$ならば$A$は正則であり, 
例\ref{inverse_2}から
$$
A^{-1} = \frac{1}{\det A} \tilde{A}
=\frac{1}{ad-bc}
\begin{pmatrix}
d & -b \\
-c & a
\end{pmatrix}.
$$
\end{exa}

\section{演習問題}
演習問題の解答は授業動画にあります.

1. 行列式
$
\begin{vmatrix}
3 & 5&1 & 2&-1\\
2 & 6&0 & 9&1\\
0 & 0& 7& 1&2\\
0 & 0& 3& 2&5\\
0 & 0& 0& 0&-6\\
\end{vmatrix}
$を計算せよ.


\newpage

\begin{center}
{\Large 第13回. クラメルの公式と特殊な行列式 (三宅先生の本, 3.4, 3.5の内容)} 
\end{center}

\begin{flushright}
 岩井雅崇 2022/07/14
\end{flushright}

この授業で行う内容は理解しなくても構いません(結構マニアックな話題を扱います). 
また覚える必要もございません. 
\section{クラメルの公式}

\begin{tcolorbox}[
    colback = white,
    colframe = green!35!black,
    fonttitle = \bfseries,
    breakable = true]
    \begin{thm}
$A$を正則な$n$次正方行列とし, 列ベクトル$a_1, \ldots, a_{n}$を用いて
$
A = 
\begin{pmatrix}
a_1 & \cdots & a_{n}
\end{pmatrix}
$
と表されているとする.
このとき連立1次方程式$A \bm{x} =\bm{b}$の解は次のようになる.
$$
\bm{x}= \begin{pmatrix}
x_1 \\ \vdots \\ x_{n}
\end{pmatrix}, 
x_i = \frac{\det
\begin{pmatrix}
a_1 & \cdots& \bm{b}&\cdots & a_{n}
\end{pmatrix}
}{\det A}.
$$
    \end{thm}
 \end{tcolorbox}
\begin{exa}
$
A = 
\begin{pmatrix}
5 &1\\
3&2 \\ 
\end{pmatrix}
$, $
\bm{b} = 
\begin{pmatrix}
3\\
2 \\ 
\end{pmatrix}
$
とする. 
連立1次方程式$A \bm{x} =\bm{b}$の解を
$
\bm{x}= \begin{pmatrix}
x_1 \\x_2
\end{pmatrix} 
$
とすると,
$$
x_1 = \frac{\det
\begin{pmatrix}
 \bm{b}& a_{2}
\end{pmatrix}
}{\det A}
= 
\frac{ 
\begin{vmatrix}
3&1\\
2&2 \\
\end{vmatrix}
}
{
\begin{vmatrix}
5&1\\
3&2 \\
\end{vmatrix}
}
=\frac{4}{7} 
\text{, }
x_2 = \frac{\det
\begin{pmatrix}
a_{1}& \bm{b}
\end{pmatrix}
}{\det A}
= 
\frac{ 
\begin{vmatrix}
5&3\\
3&2 \\
\end{vmatrix}
}
{
\begin{vmatrix}
5&1\\
3&2 \\
\end{vmatrix}
}
=\frac{1}{7}
\text{となる.}
$$
\end{exa}

\section{特殊な行列式}
\begin{tcolorbox}[
    colback = white,
    colframe = green!35!black,
    fonttitle = \bfseries,
    breakable = true]
    \begin{thm}
  \begin{enumerate}
\item (ヴァンデルモンドの行列式)
\begin{align*}
 &
 \begin{vmatrix}
1   &   1     & \cdots &  1  \\
x_1& x_2 & \cdots &x_n \\
{x_1}^{2}& {x_2}^{2} & \cdots &{x_n}^{2} \\
\vdots & \vdots    &  &\vdots \\
{x_1}^{n-1}& {x_2}^{n-1} & \cdots &{x_n}^{n-1} \\
\end{vmatrix}
= \prod_{1 \leqq i < j \leqq n} (x_j - x_i).
\\%%
%&=(x_2 - x_1)  (x_3 - x_1)  \cdots  (x_n - x_1)  (x_3- x_2)    \cdots  (x_n - x_2)  (x_4 - x_3)\cdots (x_n - x_{n-1})
%\begin{matrix}=(x_2 - x_1)  (x_3 - x_1)  \cdots  (x_n - x_1) \\\times (x_3- x_2)  (x_4 - x_2)  \cdots  (x_n - x_2) \\\times \cdots \\\times (x_n - x_{n-1}) \\ \end{matrix}
%\begin{matrix}(x_2 - x_1) & \times&  (x_3 - x_1) & \times&  \cdots &\times & (x_n - x_1) \\\end{matrix}
\end{align*}
\item (ヴァンデルモンドの行列式の応用) $b_1, \ldots, b_n, c_1, \ldots, c_n$を実数とし, $b_1, \ldots, b_n$は相異なると仮定する.
 このとき実数係数の$n$次式$f(x) = x^{n} +  a_1 x^{n-1} +  \cdots +  a_{n-1} x + a_{n}$があって, 任意の$i = 1, \ldots, n$について$f(b_i) =c_i $となる.

  \end{enumerate}
    \end{thm}
 \end{tcolorbox}
$\prod_{1 \leqq i < j \leqq n}$は積の記号で, $\prod_{1 \leqq i < j \leqq n} (x_j - x_i)$は「$1 \leqq i < j \leqq n$を満たす$(i,j)$について$ (x_j - x_i)$を全てかけた数」を表している.



\begin{tcolorbox}[
    colback = white,
    colframe = green!35!black,
    fonttitle = \bfseries,
    breakable = true]
    \begin{thm}
$$
 \begin{vmatrix}
a_0   &   -1   &0 &0 & \cdots &  0 \\
a_1   &   x    &-1  &0 & \cdots &  0 \\
a_2   &   0    & x   &-1 & \cdots &  0 \\
\vdots  &    \vdots   &   &\ddots & \ddots &  0 \\
a_{n-1}&   0    & 0   &                  & x&  -1\\
a_n   &   0   & 0   &0 & \cdots &  x\\
\end{vmatrix}
=a_0 x^{n} +  a_1 x^{n-1} +  \cdots +  a_{n-1} x + a_{n}.
$$
    \end{thm}
 \end{tcolorbox}
 
\section{終結式と判別式}
以下の内容は「永田雅宜著 理系のための線型代数の基礎 (紀伊國屋書店)」の第3章に基づく.

\begin{tcolorbox}[
    colback = white,
    colframe = green!35!black,
    fonttitle = \bfseries,
    breakable = true]
    \begin{dfn}
$a_0, a_1, \ldots, a_n$を複素数とし, $f(x) = a_0 x^{n} +  a_1 x^{n-1} +  \cdots +  a_{n-1} x + a_{n}$とする(ただし$a_0 \neq 0$とする). $f(x)=0$の解を$\alpha_{1}, \ldots, \alpha_{n}$とするとき, 
$$
D = a_{0}^{2n -2} \prod_{1 \leqq i < j \leqq n} (\alpha_j - \alpha_i)^2
\text{ を\underline{$f(x)$の判別式}という.}
$$
    \end{dfn}
 \end{tcolorbox}
 簡単にわかることとして, 「$D \neq 0$ $\Leftrightarrow$ $f(x)=0$の解が相異なる」である.
\begin{exa}
$f(x) = a_0 x^2 + a_1 x + a_2$の判別式$D$を求める(ただし$a_0 \neq 0$とする). 
$\alpha_1, \alpha_2$を$f(x)=0$の解とすると, 解と係数の関係から
$$
a_0 x^2 + a_1 x + a_2 = a_0 (x - \alpha_1) (x- \alpha_2)
$$
であるので, $-a_ 1 = a_0 (\alpha_1 + \alpha_2), a_2 = a_0\alpha_1 \alpha_2$となる. よって
$$
D = a_{0}^{2} (\alpha_2 - \alpha_1)^2 = a_{0}^{2}\{ (\alpha_1 + \alpha_2)^2 - 4 \alpha_1 \alpha_2 \} = a_{1}^{2} - 4 a_0 a_{2}.
 $$
\end{exa}

\begin{tcolorbox}[
    colback = white,
    colframe = green!35!black,
    fonttitle = \bfseries,
    breakable = true]
    \begin{dfn}
複素係数多項式$f(x) = a_0 x^{n} +  a_1 x^{n-1} +  \cdots +  a_{n-1} x + a_{n}$, 
$g(x) = b_0 x^{m} +  b_1 x^{m-1} +  \cdots +  b_{m-1} x + b_{m}$
(ただし$a_0 \neq 0, b_0 \neq 0$)について, $m+n$次正方行列を次で定める.
$$
 \begin{pmatrix}
 a_0 	& a_1& \cdots &a_n 	&0		&0 	     & \cdots&0 \\
0   		& a_0 & a_1		& \cdots &a_n 	&0 	      & \cdots&0 \\
 0 		& 0      & a_0 		&    a_1	& \cdots &a_n  & \cdots&0 \\
\vdots  & \vdots  &      \ddots 	&     \ddots &  \ddots     &   & \ddots&\vdots \\
   0		 & 0         &	 \cdots	&    0	& a_0 &a_1 & \cdots &a_n \\
 b_0 	& b_1& b_2 &  \cdots  	& \cdots		&b_m	     & \cdots&0 \\
%0		& b_0 	& b_1& b_2 &  \cdots  	&b_m 	    &      &0 \\
\vdots 	&    \ddots 	&   \ddots    &   \ddots     & 		&   	    & \ddots&\vdots \\
0		&     \cdots 	&      b_0 & b_1 &  b_2  	&  \cdots& \cdots&b_m \\
 \end{pmatrix}
$$
この行列の行列式を\underline{$f,g$の終結式と言い, $R(f,g)$と表す.}
    \end{dfn}
 \end{tcolorbox}


\begin{tcolorbox}[
    colback = white,
    colframe = green!35!black,
    fonttitle = \bfseries,
    breakable = true]
    \begin{thm}
    \begin{enumerate}
\item $f(x) =0$の解を$\alpha_1, \ldots, \alpha_n$とし, $g(x) =0$の解を$\beta_1, \ldots, \beta_m$とすると
$$
R(f,g) = a_{0}^{m}b_{0}^{n} \prod_{1 \leqq i \leqq n, 1\leqq j \leqq m} (\alpha_i - \beta_j)
=a_{0}^{m} \prod_{1 \leqq i \leqq n} g(\alpha_{i})\text{である.}
$$
特に$R(f,g)=0$は$f(x)=g(x)=0$が共通解を持つことと同値である.
\item $f'$を$f$の微分とすると, 
$$
R(f, f') = (-1)^{\frac{n(n-1)}{2}} a_0 D.
$$
特に$f(x)$の判別式$D$は$a_0, \ldots, a_n$の式でかける. また$R(f,f')=0$は$f(x)=0$が重根を持つことと同値である.
    \end{enumerate}

    \end{thm}
 \end{tcolorbox}
 
 \newpage


\begin{center}
{\Large 第14回. 内積と外積} 
\end{center}

\begin{flushright}
 岩井雅崇 2022/07/21
\end{flushright}

以下の内容は「基礎数学研究会 新版基礎線形代数 (東海大学出版会)」の第8章を参考にした. 
これも覚える必要はない(\underline{ただしベクトル解析などで役に立つ内容である}).

\section{内積}

$\R$を実数の集合とし, $n \geqq1$なる自然数について
$$
\R^n  = \{ (x_1, \ldots, x_n) | x_1, \ldots, x_n \in \R\} \text{とする.}
$$
\begin{exa}
$\R^2$は平面をあらわし, $\R^3$は空間を表す.
\end{exa}

\begin{tcolorbox}[
    colback = white,
    colframe = green!35!black,
    fonttitle = \bfseries,
    breakable = true]
    \begin{dfn}
$\bm{a}=(a_1, \ldots, a_n), \bm{b}=(b_1, \ldots, b_n)\in \R^n$, $\alpha \in \R$について和, 差, スカラー倍, 内積, 長さ(ノルム)を次で定める.
\begin{itemize}
\item 和 $\bm{a} + \bm{b} = (a_1 + b_1, \ldots, a_n + b_n)$.
\item 差 $\bm{a} - \bm{b} = (a_1 - b_1, \ldots, a_n - b_n)$.
\item スカラー倍 $\alpha \bm{a} = (\alpha a_1, \ldots, \alpha a_n)$.
\item 内積 $\bm{a} \cdot\bm{b} = a_1 b_1 + \cdots + a_n b_n $.
\item 長さ(ノルム) $||\bm{a}||= \sqrt{\bm{a} \cdot\bm{a}} = \sqrt{a_{1}^{2}+ \cdots + a_{n}^{2}}$.
\end{itemize}
    \end{dfn}
 \end{tcolorbox}

\begin{exa}
$\bm{a}=(3,5), \bm{b} = (6,1), \alpha=2$とすると
$\bm{a} + \bm{b} =(9,6)$, $\bm{a} - \bm{b} =(-3,4)$, $\alpha \bm{a}= (6,10)$, 
$\bm{a} \cdot\bm{b} = 3 \times 6 + 5 \times 1 =23$, $||\bm{a}||=\sqrt{3^2 + 5^2}= \sqrt{34}$となる.
\end{exa}

\begin{tcolorbox}[
    colback = white,
    colframe = green!35!black,
    fonttitle = \bfseries,
    breakable = true]
    \begin{prop}
$\bm{a}, \bm{b} \in \R^n$とする.
\begin{enumerate}
\item (中線定理) $||\bm{a} + \bm{b}||^2 + ||\bm{a} - \bm{b}||^2 = 2(||\bm{a}||^2 + ||\bm{b}||^2)$.
\item $\bm{a} \cdot\bm{b} = \frac{1}{4}(||\bm{a} + \bm{b}||^2 - ||\bm{a} - \bm{b}||^2)
= \frac{1}{2}(||\bm{a} + \bm{b}||^2 - ||\bm{a} ||^2- || \bm{b}||^2)
= \frac{1}{2}(||\bm{a} ||^2 + || \bm{b}||^2 - ||\bm{a} - \bm{b}||^2)$.
\item (Cauchy-Schwarzの不等式) $(\bm{a} \cdot\bm{b})^2 \leqq ||\bm{a} ||^2 ||\bm{b} ||^2 $.
\item (三角不等式) $ ||\bm{a} + \bm{b} ||   \leqq ||\bm{a} || +  ||\bm{b} ||  $.
\item $n=3$とし$\bm{a} =(a_1, a_2, a_3), \bm{b}=(b_1, b_2, b_3)$とする.
$\R^3$上の点Pを$(a_1, a_2, a_3)$, $\R^3$上の点Qを$(b_1, b_2, b_3)$, $\R^3$上の原点を点Oとする.
このとき線分OPとOQがなす角を$\theta$とすると
$$
\bm{a} \cdot\bm{b} = ||\bm{a} || || \bm{b}|| \cos \theta \text{となる.}
$$
特に$||\bm{a} ||  \neq 0$かつ$|| \bm{b}|| \neq0$のとき, $\bm{a} \cdot\bm{b} =0$は直線OPとOQが直交していることと同値である.
\end{enumerate}
    \end{prop}
 \end{tcolorbox}
\begin{exa}
$\bm{a} = (a_1, a_2, a_3)$に直交し点$\bm{c} = (c_1, c_2, c_3)$を通る平面$S$を求めよ.

(解). $\bm{x}=(x_1, x_2, x_3)$が平面$S$の点であるとき, $\bm{x} - \bm{c}$と$\bm{a}$は直交する.
よって
$(\bm{x} - \bm{c}) \cdot \bm{a} =0$である.
$$
(\bm{x} - \bm{c}) \cdot \bm{a} = 
a_1(x_1 - c_1) +a_2(x_2 - c_2) +a_3(x_3 - c_3)
$$
であるので, $S = \{ (x_1, x_2, x_3) \in \R^3 | a_1(x_1 - c_1) +a_2(x_2 - c_2) +a_3(x_3 - c_3)=0\} $となる.
\end{exa}

\section{外積}
\begin{tcolorbox}[
    colback = white,
    colframe = green!35!black,
    fonttitle = \bfseries,
    breakable = true]
    \begin{dfn}
$\bm{a}=(a_1, a_2, a_3), \bm{b}=(b_1, b_2, b_3)\in \R^3$について, 外積$\bm{a} \times \bm{b}$を次で定める. 
\begin{align*}
\bm{a} \times \bm{b} 
&=\left(\begin{vmatrix}
a_2&a_3\\
b_2&b_3 \\
\end{vmatrix},
\begin{vmatrix}
a_3&a_1\\
b_3&b_1 \\
\end{vmatrix},
\begin{vmatrix}
a_1&a_2\\
b_1&b_2 \\
\end{vmatrix}
\right)
\\
&=
( a_2b_3 - a_3b_2, a_3b_1-a_1b_3, a_1b_2-a_2b_1)  
\end{align*}
\end{dfn}
 \end{tcolorbox}
 \begin{exa}
 $\bm{a}=(3, 5, 0), \bm{b}=(6, 1, 0)$とすると
 $$
 \bm{a} \times \bm{b}
 =\left(\begin{vmatrix}
5&0\\
1&0 \\
\end{vmatrix},
\begin{vmatrix}
0&3\\
0&6 \\
\end{vmatrix},
\begin{vmatrix}
3&5\\
6&1 \\
\end{vmatrix}
\right)
= (0,0,-27)
\text{, } 
\bm{b} \times  \bm{a} 
 =\left(\begin{vmatrix}
1&0 \\
5&0\\
\end{vmatrix},
\begin{vmatrix}
0&6 \\
0&3\\
\end{vmatrix},
\begin{vmatrix}
6&1 \\
3&5\\
\end{vmatrix}
\right)
= (0,0,27).
 $$
 \end{exa}
 
 \begin{tcolorbox}[
    colback = white,
    colframe = green!35!black,
    fonttitle = \bfseries,
    breakable = true]
    \begin{prop}
$\bm{a}, \bm{b} \in \R^3$とする.
\begin{enumerate}
\item $\bm{b} \times  \bm{a}  = - \bm{a} \times  \bm{b}$. 特に$\bm{a} \times  \bm{a} =0$.
\item $\bm{a} \times  \bm{b}$は$\bm{a} $や$\bm{b}$に直交する.
\item $\bm{a} \times  \bm{b} =0$であることは$\bm{a} $と$\bm{b}$が平行であることと同値.
\item $|| \bm{a} \times  \bm{b}||$は$\bm{a} $と$\bm{b}$を2辺とする平行四辺形の面積に等しい.
\end{enumerate}
    \end{prop}
 \end{tcolorbox}
 
 
\begin{exa}
$a_1, a_2, b_1,b_2$を実数とする. 
このとき$\begin{vmatrix}
a_1&a_2 \\
b_1&b_2\\
\end{vmatrix}$
の行列式の絶対値$|a_1b_2 - a_2b_1|$は$(a_1, a_2)$と$(b_1, b_2)$を2辺とする平行四辺形の面積に等しい.

\end{exa}
\section{3次の行列式と内積外積}

\begin{tcolorbox}[
    colback = white,
    colframe = green!35!black,
    fonttitle = \bfseries,
    breakable = true]
    \begin{thm}
$\bm{a}=(a_1, a_2, a_3), \bm{b}=(b_1, b_2, b_3), \bm{c}=(c_1, c_2, c_3)\in \R^3$について, 
$$
\det
\begin{pmatrix}
a_1& a_2 & a_3\\
b_1& b_2 & b_3\\
c_1& c_2 & c_3\\
\end{pmatrix}
=\bm{a} \cdot (\bm{b} \times \bm{c}).
$$
特に
$\bm{a} \cdot (\bm{b} \times \bm{c}) = \bm{b} \cdot (\bm{c} \times \bm{a})=\bm{c} \cdot (\bm{a} \times \bm{b})$である(スカラー3重積とも呼ばれる).
\end{thm}
 \end{tcolorbox}
 
 \begin{tcolorbox}[
    colback = white,
    colframe = green!35!black,
    fonttitle = \bfseries,
    breakable = true]
    \begin{thm}
$\bm{a}=(a_1, a_2, a_3), \bm{b}=(b_1, b_2, b_3), \bm{c}=(c_1, c_2, c_3)\in \R^3$とすると次の値は等しい.
\begin{itemize}
\item $\det
\begin{pmatrix}
a_1& a_2 & a_3\\
b_1& b_2 & b_3\\
c_1& c_2 & c_3\\
\end{pmatrix}$の絶対値.
\item $\bm{a} \cdot (\bm{b} \times \bm{c})$の絶対値.
\item $\bm{a}, \bm{b}, \bm{c}$によって生成される平行6面体の体積.
\end{itemize}
\end{thm}
 \end{tcolorbox}
 
 \end{document}
