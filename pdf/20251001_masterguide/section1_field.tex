
\section{岩井の専門分野}
\label{sec-1}

\subsection{専門分野}\vspace{-6pt}
\label{subsec-senmon}
専門分野は「代数幾何学・複素幾何学・多変数複素解析」です. 代数・幾何・解析の全部をやっているのか?と思われそうですが, そうではありません.この分野がやっていることは次のことです. 

\begin{tcolorbox}[mybox]
%[mybox, title=例]
複素代数多様体(=複素射影多様体$\C\mathbb{P}^N$の部分多様体)をあらゆる手法(代数幾何学・複素幾何学・多変数複素解析など)を使って調べて, 面白い定理を示す.
 \end{tcolorbox}
 
あらゆる手法に分野は問いません. 使えるなら数論を使っても良いです. 
面白い定理に関しては以下のものがあります.(人名に関しては\ref{subsec-nendai}も参照.)
\begin{itemize}[left=0pt]
  \setlength{\parskip}{0cm} % 段落間
  \setlength{\itemsep}{5pt} % 項目間
  \item \textbf{標準環の有限生成性(Finite generation of canonical ring)} \\ 標準環$R(X, K_X)= \oplus_{m \in \N} H^0 (X, mK_X)$が有限生成である. Birkar-Cascini-Hacon-Mckernan, Siuによる.  
  %\cite{BCHM10, Siu08}
  \item \textbf{一般型代数多様体の多重標準写像 \\ (Pluricanonical system of varieties of general type)} \\
  任意の$n \in \N$について, ある$c(n) >0$があって, $m \ge c(n)$ならば, 任意の$n$次元一般代数多様体$X$に関して$m K_{X}$が与える写像は双有理になる. Hacon-Mckernan, Takayama, Tsujiによる. 
  %\cite{HM06, Tak06, Tsuji06}
  \item \textbf{多重種数の変形不変性(Invariance of plurigenera)} \\
  $h^0(X,mK_{X})$は変形で不変である. Siu, Păunによる. 
  %\cite{Siu98, Siu03, Pau08}
%\item  \textbf{アバンダンス予想の解析的アプローチ(Analytic approach to Abundance Conjecture)} Demailly-Hacon-Păun, Gongyo-Matsumuta 
%\cite{DHP13}\cite{GM}
\item \textbf{相対標準束の順像層の正値性定理 \\ (Semipositivity of direct image sheaves of relative pluricanonical bundles)} \\
ファイブレーション$f : X \to Y$について, 相対標準束の順像層$f_{*}(mK_{X/Y})$が0以上の特異曲率を持つという定理. 古典的な結果はTakao Fujita, Kawamata, Viehwegなど. これに関して, ベクトル束の特異エルミート計量を用いて定式化したのがBerntsson-P\u{a}un, P\u{a}un-Takayamaである. %Fujino.
\item \textbf{底空間が複素トーラスの飯高予想\\ (Iitaka Conjecture over an Abelian variety base)} \\
複素トーラスへのファイブレーション$f : X \to A$について, 小平次元に関する不等式$\kappa(K_X) \ge \kappa(K_A) + \kappa(K_F)$が成り立つ. Cao-Păun, Hacon-Popa-Schnellによる. この証明には上の順像層の正値性定理を使う. 
%\cite{CP17, HPS18} 
\item \textbf{特異点をもった多様体のBeauville-Bogomolov分解 \\ (Singular Beauville-Bogomolov decomposition)}\\
 Beauville-Bogomolov分解とは$c_1(X)=0$な多様体はトーラス, Calabi-Yau, 既約シンプレクティックに分解される定理である. 
\begin{itemize}
  \setlength{\parskip}{0cm} % 段落間
  \setlength{\itemsep}{0cm} % 項目間
\item 滑らかな多様体の場合は, Bogomolov, Beauvilleによる. 
\item 射影特異多様体(KLT多様体)の場合は, Druel, Greb-Guenancia-Kebekus, H\"oring-Peternell, Campanaによる. 
\item ケーラ特異多様体の場合は, Graf-Schwald, Claudon-Graf-Guenancia-Naumann, Bakker-Guenancia-Lehnによる. 
\end{itemize}
%Beauvilleによる.これが射影特異多様体(KLT多様体)でも成り立つことがDruel, Greb-Guenancia-Kebekus, H\"oring-Peternell, Campanaによってわかった. その後ケーラ特異多様体でも成り立つことがGraf-Schwald, Claudon-Graf-Guenancia-Naumann, Bakker-Guenancia-Lehnによってわかった. 
\item \textbf{接ベクトル束$T_X$や反標準束$-K_X$が正値性を持つ多様体の構造定理 \\ (Structure Theorem for  projective varieties with semipositve curvature)} \\
接ベクトル束$T_X$や反標準束$-K_X$の曲率が正や0以上の場合は, 多様体の構造が決定される. これは私も関わっている. 
\begin{itemize}
  \setlength{\parskip}{0cm} % 段落間
  \setlength{\itemsep}{0cm} % 項目間
\item $T_X$が正の曲率を持つ場合はMori, Siu-Yauによる. $X$は$ \C \mathbb{P}^n$と同型になる.   
\item $T_X$が0以上の曲率を持つ場合はHoward-Smyth-Wu, Mok, Campana-Peternell, Demailly-Peternell-Schneiderによる. $X$は複素トーラスとFanoに分解される.  
\item $T_X$が0以上の特異曲率を持つ場合はHosono-Iwai-Matsumuraによる. 特異多様体の場合はIwai-Matsumura-Zhong. $X$は複素トーラスと有理連結多様体に分解される.  
\item $-K_X$が0以上の曲率を持つ場合はCao-H\"oringによる. 特異多様体の場合はMatsumura-Wang.  $X$は複素トーラス, Calabi-Yau, 既約シンプレクティック, 有理連結多様体に分解される.  
\end{itemize}
\end{itemize}

この分野の特徴は他分野とのインタラクションが非常に多いことです. 代数幾何学の人と共同研究することもあれば, 複素(微分)幾何学の人とも共同研究したりすることもあります. 代数幾何学の人の前で講演することもあれば, 多変数複素解析のセミナーの世話人をすることもあります.
%代数幾何学, とりわけ双有理幾何学との関わりが強いと思います.
%個人的には"複素代数幾何学"をやっていると思っています. が, 

大阪大学での所属分野は"幾何学"になります. 日本数学会では"函数論"(1変数や多変数の複素関数論を扱う分野)に所属してます. ArXivでは論文をAG(代数幾何学)・CV(多変数関数論)・DG(微分幾何)に出しています. 

\subsection{岩井の研究紹介}
\url{http://www.math.sci.osaka-u.ac.jp/staff/iwai.html}に詳しく書きました.
ここにも書いておきます. 

\begin{tcolorbox}[mybox]

\hspace{12pt}射影多様体(複素射影空間の部分多様体)について、複素幾何学や代数幾何学、多変数複素解析などあらゆる手法を用いて研究しております。より具体的には「射影多様体はリッチ曲率正・0・負の3つタイプの多様体に分解される」という射影多様体の分類に関する予想を、接ベクトル束や余接ベクトル束、第2チャーン類の視点から研究を行なっております。
\vspace{5pt}

(1).初期の研究において「接ベクトル束が0以上の(特異な)曲率を持つ射影多様体は、リッチ曲率正の多様体とトーラス(リッチ曲率0)に分解される」という構造定理を確立しました。この定理には多変数複素解析由来の滑らかとは限らない計量(特異エルミート計量)の手法を用います。またこの研究は複素幾何学的な葉層構造に拡張することができました。
\vspace{5pt}

(2). 葉層構造を用いることで「余接ベクトル束が0以上の曲率を持つ射影多様体は、リッチ曲率負の多様体とトーラスに分解される」ことを、第2チャーン類が消えている状況下で示しました。この構造定理を示す際に、代数幾何学の予想であるアバンダンス予想を第2チャーン類が消えている状況下で示しました。
\vspace{5pt}

(3).現在は第2チャーン類と射影多様体の構造の関係を調べております。
\end{tcolorbox}

もう少し付け加えると次のとおりです. 
\begin{enumerate}[label=$(\arabic*)$]
  \setlength{\parskip}{0cm} % 段落間
  \setlength{\itemsep}{0cm} % 項目間
\item \ref{subsec-senmon}節でも言いましたが, Hosono-Iwai-Matsumura \cite{HIM22}で「$T_X$が0以上の特異曲率を持つなら, $X$は複素トーラスと有理連結多様体に分解される.」ことを示しました. Iwai-Matsumura-Zhong \cite{IMZ23}はこれを特異多様体に拡張し, \cite{Iwa22}ではこれを葉層に拡張し, \cite{Iwa21}ではこれをログ対に拡張しました. 
\item 「余接ベクトル束$\Omega_X$が0以上の曲率を持ち, 第2チャーン類$c_2(X)=0$ならば, $X$は複素トーラスと種数2以上のリーマン面に分解される.」ことをIwai-Matsumura \cite{IM22}で示しました. 
\item (2)の研究をした後に「第2チャーン類は構造定理と関係あるのでは?」と思い, 第2チャーン類の研究を続けました. そしてIwai-Matsumura-M\"uller \cite{IMM24} において「$K_X$の曲率が0以上であり. $3c_2(X) = c_1 (X)^{2}$という等式が成り立つ特異代数多様体は3パターンの構造しかありえない」ことを示しました.  

他にも第2チャーン類に関して研究しています. Iwai-Jiang-Liu \cite{IJL23}では「任意の$n \in \N$について, ある$b_{n}>0$があって, 任意の$n$次元弱Fano多様体に対して$b_n c_2(X) c_1(X)^{n-2} \ge c_1(X)^{n}$」が成り立つことを示しました. 最近はIwai-Jinnnouchi-Zhang \cite{IJZ25}で「$K_X$が巨大の場合のMiyaoka-Yau不等式」を確立しました. 
\end{enumerate}

%\subsection{キーワード}

%特異空間・ケーラー多様体・Hodge理論・ベクトルバンドルの安定性・Bogomolov--Gieseker不等式・non-pluripolar product・プルリポテンシャル理論・層のコホモロジー・代数的葉層・K安定性・極小モデル理論

\subsection{関連する人たちの年代図}
\label{subsec-nendai}
私の分野で関係する人を年代別に並べてみました. ほとんどのソースはMathematics Genealogy Project \url{https://www.genealogy.math.ndsu.nodak.edu} に基づきます.

「岩井(2020)」は「岩井が2020年に博士をとった」ことを意味し,  「(A) $\to $ (B)」は「AはBの(博士をとった時の)指導教官」ことを意味します. 赤枠は同じ研究室出身という意味です. もちろん全員は書いてません. 

こう見るとこの分野は"多変数複素解析"に属すると思います. 源泉が多変数複素解析からきているからです.
そのため歴史的な理由から「日本・フランス・ドイツ」が盛んです. 
またSiu先生やChern先生・Yau先生の影響もあって中国勢も多いです. 
本当は他の中国勢, 特にZhou先生を入れたかったのですが, 調べても出てきませんでした.  

%あと調べてみると「Greb先生もKebekus先生も多変数複素の人だったのか」と気づいた. あとSiu先生ってめちゃくちゃ年上なことにも気づいた. 

なお高山先生・辻先生の博士の年代はWikipedia情報になります. 
確定情報ではないので?マークをつけました. 高山先生の指導教官に関しては私が直接聞きました. \footnote{高山先生に「高山先生の指導教官って誰ですか?」と聞いたら, 偶然横にいた辻先生が「私ですね.」と言った.} 辻先生の指導教官は誰になるのだろうか? Siu先生な気もするが, これはわかりません. 

\newpage

\begin{tikzpicture}[
  every node/.style={font=\small},
  person/.style={draw, rounded corners=2pt, fill=blue!10, inner sep=4pt},
  team/.style={draw=red, thick, inner sep=6pt, rounded corners=4pt},
  arrow/.style={->, thick},
  country/.style={font=\small\bfseries, anchor=base, align=center} % 修正: anchorをbase east → base に
]

% 年代ラベル(縦方向)
\node at (0,0) (y1960){$\sim$};
\node at (0,-2.5) (y1970){$\sim$} ;
\node at (0,-5.5) (y1980){$\sim$};
\node at (0,-8.5) (y1990){$\sim$};
\node at (0,-11.5) (y2000){$\sim$};
\node at (0,-14.5) (y2010){$\sim$};
\node at (0,-18) (y2020){$\sim$};

\node at (0,-1.25){1970 以前};
\node at (0,-4) {1980};
\node at (0,-7)  {1990};
\node at (0,-10)  {2000};
\node at (0,-13) {2010};
\node at (0,-16) {2020};

% X座標基準点
\coordinate (xjapan) at (4.25,0);
\coordinate (xfrance) at (12.0,0);

% 国別ラベル(中央揃えに変更)
\node[country] at ($(xjapan) + (0,3)$) {日本};
\node[country] at ($(xfrance) + (0,3)$) {フランス};

% 1960以前
%Ohsawa
\node[person, align=center] at ($(xjapan) + (y1960) + (2.25, -0.8)$) (ihara) {伊原 \\ (1967)};
\node[person] at ($(xjapan) + (y1960) +(-2.5, 1)$) (akizuki) {秋月 (1939)};
\node[person] at ($(xjapan) + (y1960) +(-2.5, 0)$) (nakano) {中野 (1956)};
\draw[arrow] (akizuki) -- (nakano);
%Demailly
\node[person, align=center] at ($(xfrance) + (y1960) +(0, 1.5)$) (montel) {Montel(1907)};
\node[person, align=center] at ($(xfrance) + (y1960) +(0, 0.5)$) (lelong) {Lelong(1941)};
\draw[arrow] (montel) -- (lelong) ;
%Campana
\node[person, align=center] at ($(xfrance) + (y1960) +(-3,1)$) (hcartan) {Henri Cartan \\ (1928)};%montelに行く
\draw[arrow] (montel) -- (hcartan) ;
\node[person, align=center] at ($(xfrance) + (y1960) +(-3,-0.1)$) (douady) {Douady (1966)};
\draw[arrow] (hcartan) --  (douady);
\node[person, align=center] at ($(xfrance) + (y1960) +(-3,-1)$) (barlet) {Barlet (1969)};
\draw[arrow] (douady) -- (barlet);
%Beauville
\node[person, align=center] at ($(xfrance) + (y1960) +(3, 1.4)$) (Schwartz) {Schwartz \\(1943)};
\node[person, align=center] at ($(xfrance) + (y1960) +(3, 0.15)$) (grothendieck) {Grothendieck \\ (1953)};
\draw[arrow](Schwartz) -- (grothendieck);
\node[person, align=center] at ($(xfrance) + (y1960) +(3, -0.9)$) (verdier) {Verdier(1967)};
\draw[arrow](grothendieck) -- (verdier);

% 1970
%Ohsawa
%Demailly
\node[person] at ($(xfrance) + (y1970) +(0, 0.75)$) (skoda) {Skoda(1972)};
\draw[arrow] (lelong) -- (skoda) ;
%Campana
%Beauville
\node[person, align=center] at ($(xfrance) + (y1970) + (3,-0.5)$) (beauville) {Beauville \\ (1977)};
\draw[arrow] (verdier) --  (beauville)  ;

% 1980
%Ohsawa
\node[person, align=center] at ($(xjapan) + (y1980) + (-2.5,1.25)$) (osawa) {大沢\\ (1981)};
\draw[arrow] (nakano) -- (osawa);
\node[person, align=center] at ($(xjapan) + (y1980) + (0,-0.5)$) (tsuji) {辻 \\ (1988?)};
%Demailly
\node[person, align=center] at ($(xfrance) + (y1980) + (0,0.75)$) (demailly) {Demailly \\ (1982)};
\draw[arrow] (skoda) -- (demailly) ;
%Campana
\node[person, align=center] at ($(xfrance) + (y1980) + (-3,1.5)$) (campana) {Campana\\(1980)};
\draw[arrow] (barlet) --  (campana);
%Beauville

% 1990
%Ohsawa
\node[person, align=center] at ($(xjapan) + (y1990) + (-2.5,0)$) (yoshikawa) {吉川 \\ (1994)};
\draw[arrow] (osawa) -- (yoshikawa);
\node[person, align=center] at ($(xjapan) + (y1990) + (0,0)$) (takayama) {高山 \\ (1995?)};
\draw[arrow] (tsuji) -- (takayama);
%Demailly
\node[person, align=center] at ($(xfrance) + (y1990) + (0, 0.4)$) (mourougane) {Mourougane \\(1997)};
\draw[arrow] (demailly) -- (mourougane);
\node[person] at ($(xfrance) + (y1990) + (0,-0.6)$) (paun) {P\u{a}un(1998)};
%Campana
%Beauville

% 2000 
%Ohsawa
\node[person, align=center] at ($(xjapan) + (y2000) + (2.25, 1.5)$) (yamanoi) {山ノ井 \\ (2000)};
\draw[arrow] (ihara) -- (yamanoi);
%Demailly
\node[person, align=center] at ($(xfrance) + (y2000) + (0,1.4)$) (boucksom) {Boucksom \\ (2002)};
\node[person, align=center] at ($(xfrance) + (y2000) + (0,0.2)$) (popovici) {Popovici \\ (2003)};
\begin{pgfonlayer}{foreground}
\node[person, align=center] at ($(xfrance) + (y2000) + (0,-1)$) (diverio) {Diverio (2008)};
\end{pgfonlayer}
%Campana
\begin{pgfonlayer}{foreground}
\node[person, align=center] at ($(xfrance) + (y2000) + (-3,-0.4)$) (claudon) {Claudon \\ (2007)};
\end{pgfonlayer}
\draw[arrow] (campana) -- (claudon);
%Beauville
\node[person, align=center] at ($(xfrance) + (y2000) + (3, 1.5)$) (druel) {Druel  \\ (2000)};
\draw[arrow] (beauville) -- (druel) ;

% 2010
%Ohsawa
\node[person, align=center] at ($(xjapan) + (y2010) + (-2.5,0.5)$) (adachi) {足立\\(2013)};
\node[team, fit=(yoshikawa)(adachi)] (teamohsawa) {};
\node[person] at ($(xjapan) + (y2010) + (-0,1)$) (matsumura) {松村(2012)};
\draw[arrow] (takayama) -- (matsumura);
\node[person] at ($(xjapan) + (y2010) + (-0,0.25)$) (hisamoto) {久本(2013)};
\node[person] at ($(xjapan) + (y2010) + (-0,-0.5)$) (koike) {小池(2015)};
%Demailly
\node[person, align=center] at ($(xfrance) + (y2010) + (0,0.6)$) (cao) {Junyan Cao \\ (2013)};
\node[person, align=center] at ($(xfrance) + (y2010) + (-3,0.5)$) (guenancia) {Guenancia \\ (2013)};
\draw[arrow] (boucksom.south west) -- (guenancia.north);
\draw[arrow] (paun.south west) -- (guenancia.north);
\node[person, align=center] at ($(xfrance) + (y2010) + (0,-0.7)$) (deng) {Ya Deng \\ (2017)};
%Campana
%Beauville
\node[person, align=center] at ($(xfrance) + (y2010) + (3,0)$) (wenhao) {Wenhao Ou \\(2015)};
\draw[arrow] (druel) -- (wenhao);

% 2020
%Ohsawa
\node[person] at ($(xjapan) + (y2020) + (-0, 2)$) (iwai) {岩井(2020)};
\node[person] at ($(xjapan) + (y2020) + (-0,1.25)$) (inoue) {井上(2020)};
\node[person] at ($(xjapan) + (y2020) + (-0,0.5)$) (inayama) {稲山(2020)};
\node[person] at ($(xjapan) + (y2020) + (-0,-0.5)$) (watanabe) {渡邊 (2024)};
\node[team, fit=(matsumura)(hisamoto)(koike)(iwai)(inoue)(inayama)(watanabe)] (teamtakayama) {};
\node[person, align=center] at ($(xjapan) + (y2020) + (2.25, 1.25)$) (kusakabe) {日下部 \\ (2020)};
\draw[arrow] (yamanoi) -- (kusakabe);
%Demailly
\node[person, align=center] at ($(xfrance) + (y2020) + (0,1.25)$) (wu) {Xiaojun Wu \\ (2020)};
\begin{pgfonlayer}{foreground}
	\node[team, fit=(mourougane)(paun)(boucksom)(popovici)(diverio)(cao)(deng)(wu)] (teamdemailly) {};
\end{pgfonlayer}	
\node[person, align=center] at ($(xfrance) + (y2020) + (-3,1.25)$) (wang) {Juanyong \\  Wang \\ (2020)};
\draw[arrow] (boucksom.south west) -- (wang.north east);
\draw[arrow] (cao.south west) -- (wang.north east);
%\node[team, fit=(guenancia)(wang)] (teamdemailly) {};
%Campana
%Beauville

\end{tikzpicture}



\newpage
\begin{tikzpicture}[
  every node/.style={font=\small},
  person/.style={draw, rounded corners=2pt, fill=blue!10, inner sep=4pt},
  team/.style={draw=red, thick, inner sep=6pt, rounded corners=4pt},
  arrow/.style={->, thick},
  country/.style={font=\small\bfseries, anchor=base, align=center} % 修正: anchorをbase east → base に
]

% 年代ラベル(縦方向)
\node at (0,0) (y1960){$\sim$};
\node at (0,-2.5) (y1970){$\sim$} ;
\node at (0,-5.5) (y1980){$\sim$};
\node at (0,-8.5) (y1990){$\sim$};
\node at (0,-11.5) (y2000){$\sim$};
\node at (0,-14.5) (y2010){$\sim$};
\node at (0,-18) (y2020){$\sim$};

\node at (0,-1.25){1970 以前};
\node at (0,-4) {1980};
\node at (0,-7)  {1990};
\node at (0,-10)  {2000};
\node at (0,-13) {2010};
\node at (0,-16) {2020};

% X座標基準点
\coordinate (xgermany) at (4.25,0);
\coordinate (xchina) at (12.0,0);

% 国別ラベル(中央揃えに変更)
\node[country] at ($(xgermany) + (0,3)$) {ドイツ};
\node[country] at ($(xchina) + (-0.6,3)$) {その他(アメリカ・中国など)};

% 1960以前
%Peternell
\node[person, align=center] at ($(xgermany) + (y1960) + (-2,0.8)$) (behnke) {Behnke \\ (1923)};
\node[person, align=center] at ($(xgermany) + (y1960) + (-2,-0.5)$) (grauert) {Grauert \\(1956)};
\draw[arrow](behnke) -- (grauert) ;
%Huckleberry
\node[person] at ($(xgermany) + (y1960) + (4.75, 0.7)$) (ahlfors) {Ahlfors(1932)};
\node[person] at ($(xgermany) + (y1960) + (4.75,-0.2)$) (royden) {Royden(1951)};
\draw[arrow] (ahlfors) -- (royden);
%Hucleberryに繋ぐ
%Stein
\node[person, align=center] at ($(xgermany) + (y1960) + (-0.2,0.4)$) (stein) {Stein \\ (1937)};
\draw[arrow](behnke) -- (stein) ;
%Siu
\node[person] at ($(xchina) + (y1960) + (0,0.8)$) (bochner) {Bochner(1921)};
\node[person] at ($(xchina) + (y1960) + (0,-0.3)$) (gunning) {Gunning(1955)};
\draw[arrow] (bochner) -- (gunning);
\node[person] at ($(xchina) + (y1970) + (0,1.3)$) (siu) {Siu(1966)};
\draw[arrow] (gunning) -- (siu) ;
% 1970
%Other
\node[person, align=center] at ($(xgermany) + (y1970) + (2.3,1)$) (huckleberry) {Huckleberry \\ (1970)};%greb, Kebekusに繋ぐ
\draw[arrow] (royden)--(huckleberry);
%Stein
\node[person, align=center] at ($(xgermany) + (y1970) + (-0.2,1)$) (schneider) {Schneider \\ (1969)};
\draw[arrow] (stein)--(schneider);
% 1980
%Peternell
\node[person, align=center] at ($(xgermany) + (y1980) + (-2,1)$) (peternell) {Peternell \\ (1981)};
\draw[arrow](grauert) -- (peternell) ;
\node[person, align=center] at ($(xgermany) + (y1980) + (1,0)$) (heinzner) {Heinzner\\ (1986)};
\draw[arrow] (huckleberry)--(heinzner);
%Siu
\node[person] at ($(xchina) + (y1980) + (0,1)$) (mok) {Mok(1980)};
\draw[arrow] (siu) -- (mok) ;
\node[person] at ($(xchina) + (y1980) + (0,-1)$) (nadel) {Nadel(1988)};
% 1990
%Peternell
%Others
\node[person, align=center] at ($(xgermany) + (y1990) + (3,0)$) (kebekus) {Kebekus \\ (1996)};
\draw[arrow] (huckleberry) -- (kebekus);
%Stein
\node[person, align=center] at ($(xgermany) + (y1990) + (-0.2,0.25)$) (toma) {Matei \\ Toma \\ (1992)};
\draw[arrow](schneider)--(toma);
%Siu
\node[person, align=center] at ($(xchina) + (y1990) + (3,0)$) (eyssidieux) {Eyssidieux \\ (1994)}; %Mokに繋ぐ
\draw[arrow] (mok.south east) -- (eyssidieux.north);
\node[person, align=center] at ($(xchina) + (y1990) + (0,0)$) (hwang) {Jun-Muk \\ Hwang \\ (1993)}; %Siuに繋ぐ

% 2000 
%Peternell
\node[person, align=center] at ($(xgermany) + (y2000) + (-2,0)$) (hoering) {H\"oring   \\ (2006)};
\draw[arrow](peternell) --  (hoering);
%Others
\node[person, align=center] at ($(xgermany) + (y2000) + (1,-0.25)$) (greb) {Greb \\ (2008)};
\draw[arrow] (huckleberry.south) -- (greb.north east);
\draw[arrow] (heinzner) -- (greb);
%SIu
\node[person, align=center] at ($(xchina) + (y2000) + (0,1)$) (Heier) {Gordon Heier  \\ (2002)}; 
%\draw[arrow] (huckleberry) -- (Heier.north west);
\node[team, fit=(mok)(nadel)(hwang)(Heier)] (teamsiu) {};

% 2010
%peternel
\node[person, align=center] at ($(xgermany) + (y2010) + (-2,0)$) (liu) {Jie Liu \\ (2018)};
\draw[arrow] (hoering) -- (liu);
%Others
\node[person, align=center] at ($(xgermany) + (y2010) + (3,0.5)$) (patrick) {Patrick Graf \\ (2013)};
\draw[arrow] (kebekus) -- (patrick);
%MOk

% 2020

%Peternell
\node[person , align=center] at ($(xgermany) + (y2020) + (-2,1)$) (cecile) {C\'ecile Gachet \\ (2023)};
\node[team, fit=(liu)(cecile)] (teamhoering) {};
\node[person, align=center] at ($(xgermany) + (y2020) + (1,0)$) (niklas) {Niklas M\"uller \\ (2025)};
\draw[arrow] (greb) -- (niklas);


\end{tikzpicture}
\newpage
\bibliographystyle{alpha}
\bibliography{ref_MY.bib}


