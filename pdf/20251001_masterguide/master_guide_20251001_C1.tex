%\documentclass[dvipdfmx,a4paper,12pt]{amsart}
\documentclass[dvipdfmx,a4paper,12pt]{article} %titleとe-mailをコメントアウトする.


%%% Packages %%%
\setlength{\lineskip}{0pt}

% --- 和文フォント設定 ---amsartを使う時はコメントアウト
%\renewcommand{\kanjifamilydefault}{\gtdefault}
%\usepackage{otf}          % min10を避けるため
%\usepackage{pxrubrica}    % 和文ルビ

% --- 基本パッケージ ---
\usepackage{graphicx}
\usepackage[all]{xy}
\usepackage{wrapfig}
\usepackage{pgfplots}
\usepackage{color}
\usepackage[dvipsnames]{xcolor}

% --- 数学関連 ---
\usepackage{amsmath,amssymb,amsthm,amsfonts,mathtools}
\usepackage{amscd,dsfont,bigdelim,braket,physics,mathrsfs,bm}

% --- 書式・リスト関連 ---
\usepackage{latexsym}
\usepackage{setspace}
\usepackage{multirow}
\usepackage{enumerate}
\usepackage{enumitem}

% --- コメント・取消線など ---
\usepackage{comment}
\usepackage[normalem]{ulem} % \emph の下線化を抑止(cancelと共存)

% --- URL・文字コード ---
\usepackage{url}
\usepackage[utf8]{inputenc}  % ← 使用しているエンジンがuplatexなら不要、pdflatexなら有効に
\usepackage{xurl} 

% --- showkeys(常に表示) ---
%\usepackage{showkeys}
%\renewcommand*{\showkeyslabelformat}[1]{%
%  \fbox{\parbox{1.6cm}{\normalfont\tiny\sffamily#1\vspace{6mm}}}%}
% --- hyperrefは最後に読み込む ---
\usepackage[dvipdfmx,breaklinks, colorlinks,linkcolor=blue,anchorcolor=blue,citecolor=blue]{hyperref}

%%% レイアウト調整 %%%
%%% レイアウト調整(geometryに統一) %%%
\usepackage[
  top=30mm,        % 上余白
  bottom=30mm,     % 下余白
  left=25mm,       % 左余白
  right=25mm,      % 右余白
  headheight=12pt, % ヘッダー高さ(必要なら)
  headsep=10mm,    % ヘッダーと本文の間
  footskip=32pt,   % 本文とフッターの距離
  includehead,     % ヘッダー分も高さに含める
  includefoot      % フッター分も高さに含める
]{geometry}

%%% 行間調整(適宜 1.2 などに変更) %%%
\usepackage{setspace}
\setstretch{1.2}

% --- 段落設定 ---
\setlength{\parskip}{0.8em}   % 段落間のスペース
\setlength{\parindent}{0pt}   % 段落先頭の字下げをなくす

%%% 追加(重複なし)パッケージ・設定 %%%

% --- 目次の体裁調整 ---
%\usepackage{tocloft}
%\renewcommand{\contentsname}{目次} % 日本語化
%\setlength{\cftbeforesecskip}{0pt}
%\setlength{\cftbeforesubsecskip}{0pt}
%\setlength{\cftbeforesubsubsecskip}{0pt}

% --- セクション見出しの体裁調整 ---
\usepackage{titlesec}
\titleformat*{\section}{\Large\bfseries}
\titleformat*{\subsection}{\large\bfseries}
\titlespacing*{\section}{0pt}{1.5ex plus .2ex minus .2ex}{0.8ex plus .1ex}
\titlespacing*{\subsection}{0pt}{1.0ex plus .2ex minus .2ex}{0.5ex plus .1ex}

% --- ヘッダー/フッター設定 ---
\usepackage{fancyhdr}
\pagestyle{fancy}
\fancyhf{}
\rhead{岩井 雅崇}
\lhead{大阪大学 数学専攻}
\cfoot{\thepage}

% --- 段落設定 ---
%\setlength{\parskip}{0.8em}  % 段落間スペース
%\setlength{\parindent}{0pt}  % 字下げなし

% --- tcolorbox 設定 ---%\begin{tcolorbox}[mybox]と使う
\usepackage[most]{tcolorbox}
\tcbuselibrary{breakable, skins, theorems}
\tcbset{
  mybox/.style={
    colback = white,
    colframe = green!35!black,
    fonttitle = \bfseries,
    breakable = true
  }
}

% --- TikZ 設定 ---
\usepackage{tikz}
\usetikzlibrary{positioning, arrows.meta, fit, calc, backgrounds}
\pgfdeclarelayer{background}
\pgfdeclarelayer{foreground}
\pgfsetlayers{background,main,foreground}

% --- footnote がページをまたがない設定 ---
\interfootnotelinepenalty=10000

% --- 目次に表示する階層の深さ ---
\setcounter{tocdepth}{2}

% --- 日本語目次---
\usepackage{pxjahyper}

%--newtheorem--

\newtheorem{thm}{Theorem}[section] 
\newtheorem{theo}[thm]{Theorem}
\newtheorem{cor}[thm]{Corollary}
\newtheorem{prop}[thm]{Proposition}
\newtheorem{conj}[thm]{Conjecture}
\newtheorem*{mainthm}{Theorem}
\newtheorem{deflem}[thm]{Definition-Lemma}
\newtheorem{lem}[thm]{Lemma}
\theoremstyle{definition} 
\newtheorem{defn}[thm]{Definition}
\newtheorem{propdefn}[thm]{Proposition-Definition} 
\newtheorem{lemdefn}[thm]{Lemma-Definition} 
\newtheorem{thmdefn}[thm]{Theorem-Definition} 
\newtheorem{eg}[thm]{Example} 
\newtheorem{ex}[thm]{Example} 
\newtheorem{ques}[thm]{Question}
\newtheorem{remin}[thm]{Reminder}
\theoremstyle{remark}
\newtheorem{rem}[thm]{Remark}
\newtheorem{setup}[thm]{Setup}
\newtheorem{obs}[thm]{Observation}
\newtheorem{notation}[thm]{Notation}
\newtheorem{cl}{Claim}
\newtheorem{claim}[thm]{Claim}
\newtheorem{assup}[thm]{Assumption}
\newtheorem{step}{Step}
\newtheorem*{clproof}{Proof of Claim}
\newtheorem{cln}[thm]{Claim}
\newtheorem*{ack}{Acknowledgements} 
\numberwithin{equation}{section}
\newtheorem{case}{Case}

%--newcommand--

\newcommand{\rk}[0]{\operatorname{rk}}
\newcommand{\supp}[0]{\operatorname{Supp}}
\newcommand{\Rad}[0]{\operatorname{Rad}}
\newcommand{\Sha}[0]{\operatorname{Sha}}
\newcommand{\sha}[0]{\operatorname{sha}}
\newcommand{\eend}[0]{\operatorname{End}}
\newcommand{\codim}[0]{\operatorname{codim}}
\newcommand{\nd}[0]{\operatorname{nd}}
\renewcommand{\rank}[0]{\operatorname{rank}}
\newcommand{\degree}[0]{\operatorname{deg}}
\newcommand{\Exc}[0]{\operatorname{Exc}}
\newcommand{\pr}{{\rm pr}}
\newcommand{\id}{{\rm id}}
\newcommand{\Sym}{{\rm Sym}}
\newcommand{\End}[0]{\operatorname{End}}
\newcommand{\Coker}[0]{\operatorname{Coker}}

\newcommand{\Supp}{{\rm Supp}}
\newcommand{\Hom}[0]{\mathscr{H}\!\textit{om}}
\newcommand{\Ext}[0]{\mathscr{E}\!\textit{xt}}
\newcommand{\GL}[0]{\operatorname{GL}}
\newcommand{\SheafHom[1]}{\mathscr{H}\!\!\!\text{\calligra om}_{\,{#1}}}
\newcommand{\PGL}[0]{\mathbb{P}\GL(r,\C)}

\newcommand{\Alb}{{\rm Alb}}
\newcommand{\verti}{{\rm vert}}
\newcommand{\hor}{{\rm hor}}
\newcommand{\univ}{{\rm univ}}
\newcommand{\Tor}{{\rm tor}}
\newcommand{\shaf}{\mathrm{sha}}
\newcommand{\Shaf}{\mathrm{Sha}}
\newcommand{\reg}{{\rm{reg}}}
\newcommand{\sing}{{\rm{sing}}}
\newcommand{\qt}{{\rm{qt}}}
\newcommand\sO{{\mathcal O}}
\newcommand{\Div}[0]{\operatorname{div}}
\newcommand{\ddbar}{dd^c}
\newcommand{\cV}{\mathcal{V}}
\newcommand{\deldel}{\sqrt{-1}\partial \overline{\partial}}
\newcommand{\dbar}{\overline{\partial}}
\newcommand{\I}[1]{\mathcal{I}(#1)}
\newcommand{\Aut}[1]{\mathrm{Aut}(#1)}
\newcommand{\Ker}[1]{\mathrm{Ker}(#1)}
\newcommand{\Image}[1]{\mathrm{Im}(#1)}
\DeclareMathOperator{\Ric}{Ric}
\DeclareMathOperator{\Vol}{Vol}
 \newcommand{\pdrv}[2]{\frac{\partial #1}{\partial #2}}
 \newcommand{\drv}[2]{\frac{d #1}{d#2}}
  \newcommand{\ppdrv}[3]{\frac{\partial #1}{\partial #2 \partial #3}}
\newcommand{\underalign}[2]{\quad \underset{\mathclap{\strut #1}}{#2}\quad}
\newcommand{\polar}{\beta}
  
\newcommand{\R}{\mathbb{R}}
\newcommand{\Z}{\mathbb{Z}}
\newcommand{\N}{\mathbb{Z}_+}
\newcommand{\C}{\mathbb{C}}
\newcommand{\Q}{\mathbb{Q}}
\newcommand{\D}{\mathbb{D}}
\newcommand{\mP}{\mathbb{P}}
\newcommand{\mO}{\mathcal{O}}
\newcommand{\B}{\mathds{B}}
\newcommand{\tl}{\hspace{-0.8ex}<\hspace{-0.8ex}}
\renewcommand{\tr}{\hspace{-0.8ex}>}

\newcommand{\xb}[1]{\textcolor{blue}{#1}}
\newcommand{\xr}[1]{\textcolor{red}{#1}}
\newcommand{\xm}[1]{\textcolor{magenta}{#1}}

\newcommand{\illegible}[1]{\textcolor{red}{[ILLEGIBLE: #1]}}


\title{岩井の分野紹介} % タイトルとサブタイトル(改行+スペースあり)
\author{岩井 雅崇 (いわい まさたか)} % 著者名は空白
\date{\today} % 今日の日付を表示

\begin{document} % ドキュメントの開始

%\maketitle % タイトル・著者・日付を出力




\section{岩井の専門分野}
\label{sec-1}

\subsection{専門分野}\vspace{-6pt}
\label{subsec-senmon}
専門分野は「代数幾何学・複素幾何学・多変数複素解析」です. 代数・幾何・解析の全部をやっているのか?と思われそうですが, そうではありません.この分野がやっていることは次のことです. 

\begin{tcolorbox}[mybox]
%[mybox, title=例]
複素代数多様体(=複素射影多様体$\C\mathbb{P}^N$の部分多様体)をあらゆる手法(代数幾何学・複素幾何学・多変数複素解析など)を使って調べて, 面白い定理を示す.
 \end{tcolorbox}
 
あらゆる手法に分野は問いません. 使えるなら数論を使っても良いです. 
面白い定理に関しては以下のものがあります.(人名に関しては\ref{subsec-nendai}も参照.)
\begin{itemize}[left=0pt]
  \setlength{\parskip}{0cm} % 段落間
  \setlength{\itemsep}{5pt} % 項目間
  \item \textbf{標準環の有限生成性(Finite generation of canonical ring)} \\ 標準環$R(X, K_X)= \oplus_{m \in \N} H^0 (X, mK_X)$が有限生成である. Birkar-Cascini-Hacon-Mckernan, Siuによる.  
  %\cite{BCHM10, Siu08}
  \item \textbf{一般型代数多様体の多重標準写像 \\ (Pluricanonical system of varieties of general type)} \\
  任意の$n \in \N$について, ある$c(n) >0$があって, $m \ge c(n)$ならば, 任意の$n$次元一般代数多様体$X$に関して$m K_{X}$が与える写像は双有理になる. Hacon-Mckernan, Takayama, Tsujiによる. 
  %\cite{HM06, Tak06, Tsuji06}
  \item \textbf{多重種数の変形不変性(Invariance of plurigenera)} \\
  $h^0(X,mK_{X})$は変形で不変である. Siu, Păunによる. 
  %\cite{Siu98, Siu03, Pau08}
%\item  \textbf{アバンダンス予想の解析的アプローチ(Analytic approach to Abundance Conjecture)} Demailly-Hacon-Păun, Gongyo-Matsumuta 
%\cite{DHP13}\cite{GM}
\item \textbf{相対標準束の順像層の正値性定理 \\ (Semipositivity of direct image sheaves of relative pluricanonical bundles)} \\
ファイブレーション$f : X \to Y$について, 相対標準束の順像層$f_{*}(mK_{X/Y})$が0以上の特異曲率を持つという定理. 古典的な結果はTakao Fujita, Kawamata, Viehwegなど. これに関して, ベクトル束の特異エルミート計量を用いて定式化したのがBerntsson-P\u{a}un, P\u{a}un-Takayamaである. %Fujino.
\item \textbf{底空間が複素トーラスの飯高予想\\ (Iitaka Conjecture over an Abelian variety base)} \\
複素トーラスへのファイブレーション$f : X \to A$について, 小平次元に関する不等式$\kappa(K_X) \ge \kappa(K_A) + \kappa(K_F)$が成り立つ. Cao-Păun, Hacon-Popa-Schnellによる. この証明には上の順像層の正値性定理を使う. 
%\cite{CP17, HPS18} 
\item \textbf{特異点をもった多様体のBeauville-Bogomolov分解 \\ (Singular Beauville-Bogomolov decomposition)}\\
 Beauville-Bogomolov分解とは$c_1(X)=0$な多様体はトーラス, Calabi-Yau, 既約シンプレクティックに分解される定理である. 
\begin{itemize}
  \setlength{\parskip}{0cm} % 段落間
  \setlength{\itemsep}{0cm} % 項目間
\item 滑らかな多様体の場合は, Bogomolov, Beauvilleによる. 
\item 射影特異多様体(KLT多様体)の場合は, Druel, Greb-Guenancia-Kebekus, H\"oring-Peternell, Campanaによる. 
\item ケーラ特異多様体の場合は, Graf-Schwald, Claudon-Graf-Guenancia-Naumann, Bakker-Guenancia-Lehnによる. 
\end{itemize}
%Beauvilleによる.これが射影特異多様体(KLT多様体)でも成り立つことがDruel, Greb-Guenancia-Kebekus, H\"oring-Peternell, Campanaによってわかった. その後ケーラ特異多様体でも成り立つことがGraf-Schwald, Claudon-Graf-Guenancia-Naumann, Bakker-Guenancia-Lehnによってわかった. 
\item \textbf{接ベクトル束$T_X$や反標準束$-K_X$が正値性を持つ多様体の構造定理 \\ (Structure Theorem for  projective varieties with semipositve curvature)} \\
接ベクトル束$T_X$や反標準束$-K_X$の曲率が正や0以上の場合は, 多様体の構造が決定される. これは私も関わっている. 
\begin{itemize}
  \setlength{\parskip}{0cm} % 段落間
  \setlength{\itemsep}{0cm} % 項目間
\item $T_X$が正の曲率を持つ場合はMori, Siu-Yauによる. $X$は$ \C \mathbb{P}^n$と同型になる.   
\item $T_X$が0以上の曲率を持つ場合はHoward-Smyth-Wu, Mok, Campana-Peternell, Demailly-Peternell-Schneiderによる. $X$は複素トーラスとFanoに分解される.  
\item $T_X$が0以上の特異曲率を持つ場合はHosono-Iwai-Matsumuraによる. 特異多様体の場合はIwai-Matsumura-Zhong. $X$は複素トーラスと有理連結多様体に分解される.  
\item $-K_X$が0以上の曲率を持つ場合はCao-H\"oringによる. 特異多様体の場合はMatsumura-Wang.  $X$は複素トーラス, Calabi-Yau, 既約シンプレクティック, 有理連結多様体に分解される.  
\end{itemize}
\end{itemize}

この分野の特徴は他分野とのインタラクションが非常に多いことです. 代数幾何学の人と共同研究することもあれば, 複素(微分)幾何学の人とも共同研究したりすることもあります. 代数幾何学の人の前で講演することもあれば, 多変数複素解析のセミナーの世話人をすることもあります.
%代数幾何学, とりわけ双有理幾何学との関わりが強いと思います.
%個人的には"複素代数幾何学"をやっていると思っています. が, 

大阪大学での所属分野は"幾何学"になります. 日本数学会では"函数論"(1変数や多変数の複素関数論を扱う分野)に所属してます. ArXivでは論文をAG(代数幾何学)・CV(多変数関数論)・DG(微分幾何)に出しています. 

\subsection{岩井の研究紹介}
\url{http://www.math.sci.osaka-u.ac.jp/staff/iwai.html}に詳しく書きました.
ここにも書いておきます. 

\begin{tcolorbox}[mybox]

\hspace{12pt}射影多様体(複素射影空間の部分多様体)について、複素幾何学や代数幾何学、多変数複素解析などあらゆる手法を用いて研究しております。より具体的には「射影多様体はリッチ曲率正・0・負の3つタイプの多様体に分解される」という射影多様体の分類に関する予想を、接ベクトル束や余接ベクトル束、第2チャーン類の視点から研究を行なっております。
\vspace{5pt}

(1).初期の研究において「接ベクトル束が0以上の(特異な)曲率を持つ射影多様体は、リッチ曲率正の多様体とトーラス(リッチ曲率0)に分解される」という構造定理を確立しました。この定理には多変数複素解析由来の滑らかとは限らない計量(特異エルミート計量)の手法を用います。またこの研究は複素幾何学的な葉層構造に拡張することができました。
\vspace{5pt}

(2). 葉層構造を用いることで「余接ベクトル束が0以上の曲率を持つ射影多様体は、リッチ曲率負の多様体とトーラスに分解される」ことを、第2チャーン類が消えている状況下で示しました。この構造定理を示す際に、代数幾何学の予想であるアバンダンス予想を第2チャーン類が消えている状況下で示しました。
\vspace{5pt}

(3).現在は第2チャーン類と射影多様体の構造の関係を調べております。
\end{tcolorbox}

もう少し付け加えると次のとおりです. 
\begin{enumerate}[label=$(\arabic*)$]
  \setlength{\parskip}{0cm} % 段落間
  \setlength{\itemsep}{0cm} % 項目間
\item \ref{subsec-senmon}節でも言いましたが, Hosono-Iwai-Matsumura \cite{HIM22}で「$T_X$が0以上の特異曲率を持つなら, $X$は複素トーラスと有理連結多様体に分解される.」ことを示しました. Iwai-Matsumura-Zhong \cite{IMZ23}はこれを特異多様体に拡張し, \cite{Iwa22}ではこれを葉層に拡張し, \cite{Iwa21}ではこれをログ対に拡張しました. 
\item 「余接ベクトル束$\Omega_X$が0以上の曲率を持ち, 第2チャーン類$c_2(X)=0$ならば, $X$は複素トーラスと種数2以上のリーマン面に分解される.」ことをIwai-Matsumura \cite{IM22}で示しました. 
\item (2)の研究をした後に「第2チャーン類は構造定理と関係あるのでは?」と思い, 第2チャーン類の研究を続けました. そしてIwai-Matsumura-M\"uller \cite{IMM24} において「$K_X$の曲率が0以上であり. $3c_2(X) = c_1 (X)^{2}$という等式が成り立つ特異代数多様体は3パターンの構造しかありえない」ことを示しました.  

他にも第2チャーン類に関して研究しています. Iwai-Jiang-Liu \cite{IJL23}では「任意の$n \in \N$について, ある$b_{n}>0$があって, 任意の$n$次元弱Fano多様体に対して$b_n c_2(X) c_1(X)^{n-2} \ge c_1(X)^{n}$」が成り立つことを示しました. 最近はIwai-Jinnnouchi-Zhang \cite{IJZ25}で「$K_X$が巨大の場合のMiyaoka-Yau不等式」を確立しました. 
\end{enumerate}

%\subsection{キーワード}

%特異空間・ケーラー多様体・Hodge理論・ベクトルバンドルの安定性・Bogomolov--Gieseker不等式・non-pluripolar product・プルリポテンシャル理論・層のコホモロジー・代数的葉層・K安定性・極小モデル理論

\subsection{関連する人たちの年代図}
\label{subsec-nendai}
私の分野で関係する人を年代別に並べてみました. ほとんどのソースはMathematics Genealogy Project \url{https://www.genealogy.math.ndsu.nodak.edu} に基づきます.

「岩井(2020)」は「岩井が2020年に博士をとった」ことを意味し,  「(A) $\to $ (B)」は「AはBの(博士をとった時の)指導教官」ことを意味します. 赤枠は同じ研究室出身という意味です. もちろん全員は書いてません. 

こう見るとこの分野は"多変数複素解析"に属すると思います. 源泉が多変数複素解析からきているからです.
そのため歴史的な理由から「日本・フランス・ドイツ」が盛んです. 
またSiu先生やChern先生・Yau先生の影響もあって中国勢も多いです. 
本当は他の中国勢, 特にZhou先生を入れたかったが, 調べても出てきませんでした.  

%あと調べてみると「Greb先生もKebekus先生も多変数複素の人だったのか」と気づいた. あとSiu先生ってめちゃくちゃ年上なことにも気づいた. 

なお高山先生・辻先生の博士の年代はWikipedia情報になります. 
確定情報ではないので?マークをつけました. 高山先生の指導教官に関しては私が直接聞きました. \footnote{高山先生に「高山先生の指導教官って誰ですか?」と聞いたら, 偶然横にいた辻先生が「私ですね.」と言った.} 辻先生の指導教官は誰になるのだろうか? Siu先生な気もするが, これはわかりません. 

\newpage

\begin{tikzpicture}[
  every node/.style={font=\small},
  person/.style={draw, rounded corners=2pt, fill=blue!10, inner sep=4pt},
  team/.style={draw=red, thick, inner sep=6pt, rounded corners=4pt},
  arrow/.style={->, thick},
  country/.style={font=\small\bfseries, anchor=base, align=center} % 修正: anchorをbase east → base に
]

% 年代ラベル(縦方向)
\node at (0,0) (y1960){$\sim$};
\node at (0,-2.5) (y1970){$\sim$} ;
\node at (0,-5.5) (y1980){$\sim$};
\node at (0,-8.5) (y1990){$\sim$};
\node at (0,-11.5) (y2000){$\sim$};
\node at (0,-14.5) (y2010){$\sim$};
\node at (0,-18) (y2020){$\sim$};

\node at (0,-1.25){1970 以前};
\node at (0,-4) {1980};
\node at (0,-7)  {1990};
\node at (0,-10)  {2000};
\node at (0,-13) {2010};
\node at (0,-16) {2020};

% X座標基準点
\coordinate (xjapan) at (4.25,0);
\coordinate (xfrance) at (12.0,0);

% 国別ラベル(中央揃えに変更)
\node[country] at ($(xjapan) + (0,3)$) {日本};
\node[country] at ($(xfrance) + (0,3)$) {フランス};

% 1960以前
%Ohsawa
\node[person, align=center] at ($(xjapan) + (y1960) + (2.25, -0.8)$) (ihara) {伊原 \\ (1967)};
\node[person] at ($(xjapan) + (y1960) +(-2.5, 1)$) (akizuki) {秋月 (1939)};
\node[person] at ($(xjapan) + (y1960) +(-2.5, 0)$) (nakano) {中野 (1956)};
\draw[arrow] (akizuki) -- (nakano);
%Demailly
\node[person, align=center] at ($(xfrance) + (y1960) +(0, 1.5)$) (montel) {Montel(1907)};
\node[person, align=center] at ($(xfrance) + (y1960) +(0, 0.5)$) (lelong) {Lelong(1941)};
\draw[arrow] (montel) -- (lelong) ;
%Campana
\node[person, align=center] at ($(xfrance) + (y1960) +(-3,1)$) (hcartan) {Henri Cartan \\ (1928)};%montelに行く
\draw[arrow] (montel) -- (hcartan) ;
\node[person, align=center] at ($(xfrance) + (y1960) +(-3,-0.1)$) (douady) {Douady (1966)};
\draw[arrow] (hcartan) --  (douady);
\node[person, align=center] at ($(xfrance) + (y1960) +(-3,-1)$) (barlet) {Barlet (1969)};
\draw[arrow] (douady) -- (barlet);
%Beauville
\node[person, align=center] at ($(xfrance) + (y1960) +(3, 1.4)$) (Schwartz) {Schwartz \\(1943)};
\node[person, align=center] at ($(xfrance) + (y1960) +(3, 0.15)$) (grothendieck) {Grothendieck \\ (1953)};
\draw[arrow](Schwartz) -- (grothendieck);
\node[person, align=center] at ($(xfrance) + (y1960) +(3, -0.9)$) (verdier) {Verdier(1967)};
\draw[arrow](grothendieck) -- (verdier);

% 1970
%Ohsawa
%Demailly
\node[person] at ($(xfrance) + (y1970) +(0, 0.75)$) (skoda) {Skoda(1972)};
\draw[arrow] (lelong) -- (skoda) ;
%Campana
%Beauville
\node[person, align=center] at ($(xfrance) + (y1970) + (3,-0.5)$) (beauville) {Beauville \\ (1977)};
\draw[arrow] (verdier) --  (beauville)  ;

% 1980
%Ohsawa
\node[person, align=center] at ($(xjapan) + (y1980) + (-2.5,1.25)$) (osawa) {大沢\\ (1981)};
\draw[arrow] (nakano) -- (osawa);
\node[person, align=center] at ($(xjapan) + (y1980) + (0,-0.5)$) (tsuji) {辻 \\ (1988?)};
%Demailly
\node[person, align=center] at ($(xfrance) + (y1980) + (0,0.75)$) (demailly) {Demailly \\ (1982)};
\draw[arrow] (skoda) -- (demailly) ;
%Campana
\node[person, align=center] at ($(xfrance) + (y1980) + (-3,1.5)$) (campana) {Campana\\(1980)};
\draw[arrow] (barlet) --  (campana);
%Beauville

% 1990
%Ohsawa
\node[person, align=center] at ($(xjapan) + (y1990) + (-2.5,0)$) (yoshikawa) {吉川 \\ (1994)};
\draw[arrow] (osawa) -- (yoshikawa);
\node[person, align=center] at ($(xjapan) + (y1990) + (0,0)$) (takayama) {高山 \\ (1995?)};
\draw[arrow] (tsuji) -- (takayama);
%Demailly
\node[person, align=center] at ($(xfrance) + (y1990) + (0, 0.4)$) (mourougane) {Mourougane \\(1997)};
\draw[arrow] (demailly) -- (mourougane);
\node[person] at ($(xfrance) + (y1990) + (0,-0.6)$) (paun) {P\u{a}un(1998)};
%Campana
%Beauville

% 2000 
%Ohsawa
\node[person, align=center] at ($(xjapan) + (y2000) + (2.25, 1.5)$) (yamanoi) {山ノ井 \\ (2000)};
\draw[arrow] (ihara) -- (yamanoi);
%Demailly
\node[person, align=center] at ($(xfrance) + (y2000) + (0,1.4)$) (boucksom) {Boucksom \\ (2002)};
\node[person, align=center] at ($(xfrance) + (y2000) + (0,0.2)$) (popovici) {Popovici \\ (2003)};
\begin{pgfonlayer}{foreground}
\node[person, align=center] at ($(xfrance) + (y2000) + (0,-1)$) (diverio) {Diverio (2008)};
\end{pgfonlayer}
%Campana
\begin{pgfonlayer}{foreground}
\node[person, align=center] at ($(xfrance) + (y2000) + (-3,-0.4)$) (claudon) {Claudon \\ (2007)};
\end{pgfonlayer}
\draw[arrow] (campana) -- (claudon);
%Beauville
\node[person, align=center] at ($(xfrance) + (y2000) + (3, 1.5)$) (druel) {Druel  \\ (2000)};
\draw[arrow] (beauville) -- (druel) ;

% 2010
%Ohsawa
\node[person, align=center] at ($(xjapan) + (y2010) + (-2.5,0.5)$) (adachi) {足立\\(2013)};
\node[team, fit=(yoshikawa)(adachi)] (teamohsawa) {};
\node[person] at ($(xjapan) + (y2010) + (-0,1)$) (matsumura) {松村(2012)};
\draw[arrow] (takayama) -- (matsumura);
\node[person] at ($(xjapan) + (y2010) + (-0,0.25)$) (hisamoto) {久本(2013)};
\node[person] at ($(xjapan) + (y2010) + (-0,-0.5)$) (koike) {小池(2015)};
%Demailly
\node[person, align=center] at ($(xfrance) + (y2010) + (0,0.6)$) (cao) {Junyan Cao \\ (2013)};
\node[person, align=center] at ($(xfrance) + (y2010) + (-3,0.5)$) (guenancia) {Guenancia \\ (2013)};
\draw[arrow] (boucksom.south west) -- (guenancia.north);
\draw[arrow] (paun.south west) -- (guenancia.north);
\node[person, align=center] at ($(xfrance) + (y2010) + (0,-0.7)$) (deng) {Ya Deng \\ (2017)};
%Campana
%Beauville
\node[person, align=center] at ($(xfrance) + (y2010) + (3,0)$) (wenhao) {Wenhao Ou \\(2015)};
\draw[arrow] (druel) -- (wenhao);

% 2020
%Ohsawa
\node[person] at ($(xjapan) + (y2020) + (-0, 2)$) (iwai) {岩井(2020)};
\node[person] at ($(xjapan) + (y2020) + (-0,1.25)$) (inoue) {井上(2020)};
\node[person] at ($(xjapan) + (y2020) + (-0,0.5)$) (inayama) {稲山(2020)};
\node[person] at ($(xjapan) + (y2020) + (-0,-0.5)$) (watanabe) {渡邊 (2024)};
\node[team, fit=(matsumura)(hisamoto)(koike)(iwai)(inoue)(inayama)(watanabe)] (teamtakayama) {};
\node[person, align=center] at ($(xjapan) + (y2020) + (2.25, 1.25)$) (kusakabe) {日下部 \\ (2020)};
\draw[arrow] (yamanoi) -- (kusakabe);
%Demailly
\node[person, align=center] at ($(xfrance) + (y2020) + (0,1.25)$) (wu) {Xiaojun Wu \\ (2020)};
\begin{pgfonlayer}{foreground}
	\node[team, fit=(mourougane)(paun)(boucksom)(popovici)(diverio)(cao)(deng)(wu)] (teamdemailly) {};
\end{pgfonlayer}	
\node[person, align=center] at ($(xfrance) + (y2020) + (-3,1.25)$) (wang) {Juanyong \\  Wang \\ (2020)};
\draw[arrow] (boucksom.south west) -- (wang.north east);
\draw[arrow] (cao.south west) -- (wang.north east);
%\node[team, fit=(guenancia)(wang)] (teamdemailly) {};
%Campana
%Beauville

\end{tikzpicture}



\newpage
\begin{tikzpicture}[
  every node/.style={font=\small},
  person/.style={draw, rounded corners=2pt, fill=blue!10, inner sep=4pt},
  team/.style={draw=red, thick, inner sep=6pt, rounded corners=4pt},
  arrow/.style={->, thick},
  country/.style={font=\small\bfseries, anchor=base, align=center} % 修正: anchorをbase east → base に
]

% 年代ラベル(縦方向)
\node at (0,0) (y1960){$\sim$};
\node at (0,-2.5) (y1970){$\sim$} ;
\node at (0,-5.5) (y1980){$\sim$};
\node at (0,-8.5) (y1990){$\sim$};
\node at (0,-11.5) (y2000){$\sim$};
\node at (0,-14.5) (y2010){$\sim$};
\node at (0,-18) (y2020){$\sim$};

\node at (0,-1.25){1970 以前};
\node at (0,-4) {1980};
\node at (0,-7)  {1990};
\node at (0,-10)  {2000};
\node at (0,-13) {2010};
\node at (0,-16) {2020};

% X座標基準点
\coordinate (xgermany) at (4.25,0);
\coordinate (xchina) at (12.0,0);

% 国別ラベル(中央揃えに変更)
\node[country] at ($(xgermany) + (0,3)$) {ドイツ};
\node[country] at ($(xchina) + (-0.6,3)$) {その他(アメリカ・中国など)};

% 1960以前
%Peternell
\node[person, align=center] at ($(xgermany) + (y1960) + (-2,0.8)$) (behnke) {Behnke \\ (1923)};
\node[person, align=center] at ($(xgermany) + (y1960) + (-2,-0.5)$) (grauert) {Grauert \\(1956)};
\draw[arrow](behnke) -- (grauert) ;
%Huckleberry
\node[person] at ($(xgermany) + (y1960) + (4.75, 0.7)$) (ahlfors) {Ahlfors(1932)};
\node[person] at ($(xgermany) + (y1960) + (4.75,-0.2)$) (royden) {Royden(1951)};
\draw[arrow] (ahlfors) -- (royden);
%Hucleberryに繋ぐ
%Stein
\node[person, align=center] at ($(xgermany) + (y1960) + (-0.2,0.4)$) (stein) {Stein \\ (1937)};
\draw[arrow](behnke) -- (stein) ;
%Siu
\node[person] at ($(xchina) + (y1960) + (0,0.8)$) (bochner) {Bochner(1921)};
\node[person] at ($(xchina) + (y1960) + (0,-0.3)$) (gunning) {Gunning(1955)};
\draw[arrow] (bochner) -- (gunning);
\node[person] at ($(xchina) + (y1970) + (0,1.3)$) (siu) {Siu(1966)};
\draw[arrow] (gunning) -- (siu) ;
% 1970
%Other
\node[person, align=center] at ($(xgermany) + (y1970) + (2.3,1)$) (huckleberry) {Huckleberry \\ (1970)};%greb, Kebekusに繋ぐ
\draw[arrow] (royden)--(huckleberry);
%Stein
\node[person, align=center] at ($(xgermany) + (y1970) + (-0.2,1)$) (schneider) {Schneider \\ (1969)};
\draw[arrow] (stein)--(schneider);
% 1980
%Peternell
\node[person, align=center] at ($(xgermany) + (y1980) + (-2,1)$) (peternell) {Peternell \\ (1981)};
\draw[arrow](grauert) -- (peternell) ;
\node[person, align=center] at ($(xgermany) + (y1980) + (1,0)$) (heinzner) {Heinzner\\ (1986)};
\draw[arrow] (huckleberry)--(heinzner);
%Siu
\node[person] at ($(xchina) + (y1980) + (0,1)$) (mok) {Mok(1980)};
\draw[arrow] (siu) -- (mok) ;
\node[person] at ($(xchina) + (y1980) + (0,-1)$) (nadel) {Nadel(1988)};
% 1990
%Peternell
%Others
\node[person, align=center] at ($(xgermany) + (y1990) + (3,0)$) (kebekus) {Kebekus \\ (1996)};
\draw[arrow] (huckleberry) -- (kebekus);
%Stein
\node[person, align=center] at ($(xgermany) + (y1990) + (-0.2,0.25)$) (toma) {Matei \\ Toma \\ (1992)};
\draw[arrow](schneider)--(toma);
%Siu
\node[person, align=center] at ($(xchina) + (y1990) + (3,0)$) (eyssidieux) {Eyssidieux \\ (1994)}; %Mokに繋ぐ
\draw[arrow] (mok.south east) -- (eyssidieux.north);
\node[person, align=center] at ($(xchina) + (y1990) + (0,0)$) (hwang) {Jun-Muk \\ Hwang \\ (1993)}; %Siuに繋ぐ

% 2000 
%Peternell
\node[person, align=center] at ($(xgermany) + (y2000) + (-2,0)$) (hoering) {H\"oring   \\ (2006)};
\draw[arrow](peternell) --  (hoering);
%Others
\node[person, align=center] at ($(xgermany) + (y2000) + (1,-0.25)$) (greb) {Greb \\ (2008)};
\draw[arrow] (huckleberry.south) -- (greb.north east);
\draw[arrow] (heinzner) -- (greb);
%SIu
\node[person, align=center] at ($(xchina) + (y2000) + (0,1)$) (Heier) {Gordon Heier  \\ (2002)}; 
%\draw[arrow] (huckleberry) -- (Heier.north west);
\node[team, fit=(mok)(nadel)(hwang)(Heier)] (teamsiu) {};

% 2010
%peternel
\node[person, align=center] at ($(xgermany) + (y2010) + (-2,0)$) (liu) {Jie Liu \\ (2018)};
\draw[arrow] (hoering) -- (liu);
%Others
\node[person, align=center] at ($(xgermany) + (y2010) + (3,0.5)$) (patrick) {Patrick Graf \\ (2013)};
\draw[arrow] (kebekus) -- (patrick);
%MOk

% 2020

%Peternell
\node[person , align=center] at ($(xgermany) + (y2020) + (-2,1)$) (cecile) {C\'ecile Gachet \\ (2023)};
\node[team, fit=(liu)(cecile)] (teamhoering) {};
\node[person, align=center] at ($(xgermany) + (y2020) + (1,0)$) (niklas) {Niklas M\"uller \\ (2025)};
\draw[arrow] (greb) -- (niklas);


\end{tikzpicture}
\newpage
\bibliographystyle{alpha}
\bibliography{ref_MY.bib}



%
\section{教科書一覧・セミナーの仕方・必要な知識}
\label{sec-2}

\subsection{学部セミナー・修士セミナーで読む本}
学部4年セミナーや修士セミナーで読むことを前提に書いています.
%ただせっかくなので, 学部2$\sim$3年生向けの本もリストアップした. 
セミナーで読む本は状況において異なります. 概ね以下の3つのケースに分けました. 
\begin{enumerate}[label=$(\arabic*)$]
  \setlength{\parskip}{0cm} % 段落間
  \setlength{\itemsep}{0cm} % 項目間
 \item 修士に行く予定が"ない"人が学部4年セミナー読む本.
\item 修士に行く予定が"ある"人が学部4年セミナー読む本.
\item 修士セミナーで修士1年の人が読む本.
\end{enumerate}

(1). 数学の修士に行かないのであれば, 学部4年セミナーで読む本は何も気にする必要はないです. 
下の(2)のような本を読んでもいいし,  (2)が難しければ難易度を下げた本を読んでも良いです. 
さっぱりわからない本で1年過ごすのは面白くないと思います. なので少し専門的な数学を学んで卒業していただければと思います. 
学部3年の内容に相当する本を読んでも良いし, 私の専門外の教科書をセミナーで読んでも良いです. 本の例に関して\ref{subsec-seminar-23}節を参照してください. 

(2). 学部4年セミナーにおいて, 修士に行く予定がある人には下の本をお勧めします.

\begin{tcolorbox}[mybox]
\begin{itemize}[left=0pt]
  \setlength{\parskip}{0cm} % 段落間
  \setlength{\itemsep}{8pt} % 項目間
\item 代数幾何学を専門にするなら, Hartshorne. \\
理由はこの本を読めば「スキーム, 層,層係数コホモロジー論」を理解できるから.
\item 複素幾何学・多変数複素解析を専門にするなら, Wells, Huybrechts, または堀川. \\ 
理由はこの本を読めば「層・コホモロジー・ホッジ理論(調和積分論)・小平消滅定理・小平埋め込み定理」を理解できるから. (ただし堀川だけは例外)
\item 多変数複素解析にいくなら中野またはForstneric. \\
中野は半分くらいは複素幾何の内容で, もう半分は$L^2$理論の内容. 
Forstnericでは岡多様体を扱ってます. 岡多様体は新たな多変数複素解析の分野として定着するのではと期待してます. 
\end{itemize}
\end{tcolorbox}
もちろんこの本以外の本でも良いが, 上の教科書は"無難"です (ただしForstnericが無難かどうかは不明).
ここで"無難"とは, 「修士で私と違う先生のところに行った場合でも, 難なくやっていける」と言う意味です. 

(3). 修士セミナーは専門的な内容を学ぶことになります. 私がカバーできるのは次の三つです. 
\begin{enumerate}
  \setlength{\parskip}{0cm} % 段落間
  \setlength{\itemsep}{0pt} % 項目間
  \item 代数幾何学や複素解析空間から派生した複素代数幾何学. 
\item ケーラーアインシュタイン計量の存在など, (阪大では由緒正しい)複素微分幾何学.
\item 多変数複素解析の領域理論や$L^2$理論から派生した複素解析幾何学.
\end{enumerate}
%修士・博士が解けそうな問題が少ない分野である.\footnote{私が博士の時, 有理連結多様体の論文を読んで発表したとき, 高山先生から「指導教官と同じ分野を選んではいけない」と言われた. 私はその言葉をガン無視した. 結果的にはその選択は正しかったが, 後にそのことを高山先生に言うと「まあ松村がいたからなんとかなったが, 君一人では何もできなかっただろう」と返された. 正論すぎてぐうの音も出なかった.}
私の専門は1です. ただ現在は修士・博士が解けそうな問題が少ない分野です.  
2や3は専門ではないので詳しいことは教えられません. 
が, 博士に行くなら指導教官を超える必要があるので, 2や3を私のところで研究して, 私を翻弄してください.\footnote{私が博士の時, 高山先生から「指導教官と同じ分野を選んではいけない」と言われました. これは正しいと思います. 指導教官の真似事をして博士を出れると思わない方がいいです. 博士を出るにはなんでもいいから指導教官を越える必要があります. (まあ私はその言葉をガン無視して, 複素代数幾何学を専門にしましたが.ただ高山先生の方向性とはかなり違います.)} 
%まあ私はその言葉をガン無視して, 複素代数幾何学を専門にした. が, 方向性は違うし, } %が, 高山先生の方向性とはかなり違う. 博論を出した時も「構造定理の部分では高山先生を超えた」と思っていた. (もちろん業績的な部分ではボロ負けだが.)}
私としてもそちらの方が嬉しいです. 

修士セミナーでは以下の本をお勧めします.
\begin{tcolorbox}[mybox]
\begin{itemize}[left=0pt]
  \setlength{\parskip}{0cm} % 段落間
  \setlength{\itemsep}{8pt} % 項目間
\item 1の代数幾何, 3の複素解析に行くなら, DemaillyのAnaytic methodの本. この本は読むのに1年かかります. (ただし複素解析を望む場合, 少し難易度を下げても良い. Berndtssonの本など.) これを読んだ後はHacon-Popa-Schnellの論文を読むことになります.  \\ 
複素解析っぽい方面に進むなら, 早めに松村先生・稲山先生・渡邉先生にあって, 最近の研究などの情報を仕入れることになります. 
代数幾何っぽい方面に進むなら, 高山先生・松村先生・岩井から情報を仕入れることになります. 
%\footnote{ただ岩井が考えた問題は岩井が解いてしまうので, 岩井とは異なる代数の方面を学ぶことになる. SchnellのMixed Hodge Moduleとか?} 
\item 2の微分幾何にいくなら, Szekelyhidiの本. この本を全部読んでもいいし, 後半を早めに切り上げてそのあとは論文を読んでも良いです. \\ 
この分野に進むなら, 早めに久本先生・橋本先生・井上先生などがいる日本の複素微分幾何勢の大きなグループに所属し, 勉強会などに参加して行って, その人たちに最近の研究などを聞くことになると思います.  
\item 1$\sim$3以外の多変数複素解析にいくなら, 岡多様体. Forstnericの教科書か関連論文を読むことになります. これは私はわからないので, 私が日下部先生から仕入れた論文を読んでもらうことになります. \\ この分野に進むなら早いうちに日下部先生やその研究室の学生に会うべきです. 
\end{itemize}
\end{tcolorbox}
 %が, 上の教科書は"無難"である. 
もちろん上の本以外の分野の本を読むのは全然良いです. 例えばゴリゴリの代数幾何学・双有理幾何学の話を聞くことはできます. が, 私のところで勉強するより, 他の先生(藤野先生・藤田先生・大川先生など)のところで読む方が絶対に良いと思います. 

\vspace{8pt}
また面白い本と教科書は別です. Hartshorneは教科書としては良い本だが, 面白いかと言えば微妙です. Grauert-Remmertは今読むとめちゃくちゃ面白い本だが, 教科書としてはおすすめしません.(理由は後述.) 
また「メイン」で読むと「サブ」で読むと言う表現をしているがこれは次のとおりです.
\begin{itemize}
  \setlength{\parskip}{0cm} % 段落間
  \setlength{\itemsep}{0pt} % 項目間
\item 「メインで読む」とは, セミナーで読むことを想定している本. 基本的に修士など研究するために必要な知識を網羅している本が該当します.
\item 「サブで読む」とは, 一人で読んだり同期の自主ゼミで読んだりする本. 尖った内容や, 研究に必要だがメインほどではない内容を扱っている本が該当します.
\end{itemize}
おそらく1年で真剣に読める本は, メインの1本とサブの0$\sim$2本の計1$\sim$3本くらいです.なので真剣に読む本は無難な方が良いです. が, 尖った内容は後に役に立つ可能性もあり, これは運です.

最後に伊原先生の本から次の言葉を引用します. 
\begin{tcolorbox}[mybox]
「自分は一ヶ月に一冊読んでいる」などという先輩や仲間に惑わされないように。本質的なところを感じとれない人の方が、すぐ数値的な評価をしたがる。数学者になる、ということは、より深い価値がわかる人をめざす、ということでもあると思います。
\end{tcolorbox}
%適当に読む本は何本でもいいが, 適当に読んでも後に全く残らない. (一時期, 舟木先生の確率論の本を読んだことがあるが, 適当に読みすぎてもうさっぱり忘れてしまった.)
%そしてこのような適当な読み方は年を取ってもできる. 
%だが, 一つの本を長い時間かけて読むという行為は年々できなくなる. 
%真剣に読める本は1年で1$\sim$3本読めれば十分だと思う
%\footnote{したくなくなると言った方がいい.} 
%一方で1,2年に読んだAtiyah-Macdonaldの本は今でも内容をうっすらながら覚えている.  
%伊原先生も「本質的なところを感じ取れない人の方が, すぐ数値的な評価をしたがる.」と言っている. 真剣に読める本は1年で1$\sim$3本読めれば十分だと思う


\subsection{本の難易度}

★マークに関しては難易度です. 難易度は以下のようになっております.\footnote{難易度名に関しては東方projectやファイアーエムブレムシリーズを参照しました.} 
\vspace{-8pt}
\begin{itemize}
  \setlength{\parskip}{0cm} % 段落間
  \setlength{\itemsep}{4pt} % 項目間
\item[★] 難易度Easy. 学部2$\sim$3年生向け. 学部4年のセミナーで読む場合は, 修士に行かない人のみが選択できます. 
\item[★★] 難易度Normal. 学部4年生向け. 修士に行く予定のある人はこの難易度以上の本を選ぶことになります.
\item[★★★] 難易度Hard. 修士1年生向け. おおよそ修士1年にこの難易度の本を読めれば, そのあと論文を読むことができると思います. 
\item[★★★★] 難易度Lunatic 修士2年生以上. 岩井も読めない教科書・論文もあります.
\end{itemize}

コメントに関しては, 私が学生の時に読んだ記憶を思い出して書きました. 本の内容も改訂して変わっている可能性もあります. また記憶違いもありますし, 個人的な意見が多いです. ご了承ください. 

なおリストを書くにあたって次の文献を参考にしました. (以下, 敬称略)
\vspace{-8pt}
\begin{itemize}[left=0pt]
  \setlength{\parskip}{0cm} % 段落間
  \setlength{\itemsep}{4pt} % 項目間
\item 藤野修. 文献案内 \url{https://www.math.kyoto-u.ac.jp/~fujino/ag-ref.pdf}
\item 桂利行, 太田快人 山本修身. 実用家向きの代数幾何学文献案内(定評ある教科書・古典的書籍) \url{https://www.jstage.jst.go.jp/article/bjsiam/14/3/14_KJ00003509979/_article/-char/ja/}
\item 松尾信一郎. 推薦図書 \\ \url{https://www.math.nagoya-u.ac.jp/~shinichiroh/2018/01/20/recommended-texts.html}
\item 本多宣博. 4年生・大学院生のセミナーで使用するテキスト・参考書 \\ \url{https://www.honda.math.sci.titech.ac.jp}
\item 東京大学数学科 各教員の研究分野 紹介PDF \\ \url{https://www.ms.u-tokyo.ac.jp/kyoumu/kenkyubunya.html}
\end{itemize}

%また以下の方々にも教えてもらった. 深く感謝申し上げます. 

次から本を紹介するが, その前にお得な情報. 
Springer Link \url{https://link.springer.com}と言って, 所属している大学によっては, Springerの本のpdfが無料で入手できます.(もちろん合法.) 阪大はSpringerと契約としていたはずです. ぜひ活用してほしいです. 

\subsection{必ず読むべき本}

\textbf{伊原 康隆. "志学数学 -研究の諸段階 発表の工夫" 丸善出版} \vspace{-6pt}

難易度:なし \vspace{-6pt}

コメント: どの分野に行くにしても必ず読むべき本です. 読んでいない人は今すぐ読むことをおすすめします. 
河東先生も書評にて「今すぐこの本を買ってきて読みなさい」と言ってます. その通りです. 
「いやお金がなくて...」という人は, 阪大の総合図書館に3冊あるので借りて読もう. 

ちなみに河東先生の"数学者の思案"(岩波書店)も面白いです. 私はKindle版で買いました.
河東先生のホームページも面白いです. 以下のページを読むことをおすすめします. 
\begin{itemize}[left=0pt]
  \setlength{\parskip}{0cm} % 段落間
  \setlength{\itemsep}{0cm} % 項目間
\item セミナーの準備の仕方について \url{https://www.ms.u-tokyo.ac.jp/~yasuyuki/sem.htm}
\item どうでもよい記事 \url{https://www.ms.u-tokyo.ac.jp/~yasuyuki/misc.htm}
\end{itemize}
「セミナーの準備の仕方について」については一回読むべき.(\ref{subsec-seminar}節参照).  
「どうでもよい記事」は時たま読みたくなります. 有益な情報も多いです. これが無料で公開されているのが不思議なくらいです. 

\subsection{代数幾何学の教科書}

代数幾何学を勉強するなら, 学部4年セミナーを通して「スキーム, 層,層係数コホモロジー論」を理解する必要があります. そのスキームを理解するのに可換環論を少しを知っておく必要があります. 
なので学部4年のセミナーが始まるまでにはある程度の可換環論(Atiyah-Macdonaldの内容)くらいは学んでおくことになります. 

修士1年以降の代数幾何学の内容で私がカバーできるのは, 双有理幾何学ぐらいです.
ただそれもちょっとしか理解してないので, あまりためにはなりません. 
なので私のところで代数幾何学を学ぶより, 藤野先生・藤田先生・大川先生のところで専門的なことを学んだ方が絶対いいと思います. 

\subsubsection{学部4年セミナー向け}

\textbf{Robin Hartshorne. "Algebraic Geometry" Graduate Texts in Mathematics (GTM, volume 52), Springer} \vspace{-6pt}

難易度:★★$\sim$★★★  \vspace{-6pt}

コメント: この本をおすすめする理由はただ一つ, 「多くの論文でこの本を引用している」から.  もちろんこれより読みやすい代数幾何学の教科書はあります. が, こだわりがなければこの本を読んだ方が良いと思います. 

1年で読むなら2$\sim$3章. 2章のスキームの評判は悪いが, これはどうしようもないです. 3章のコホモロジーが一番面白い. 
1章の古典的代数幾何は事前に読むか, 後から読んでも事足ります. 
4章の曲線論は読むとためになるので, 3章終わって時間余ったら読む感じで. 
5章の曲面論はこの本じゃなくてもいいです. (が, 読んでもいいと思う.)
\vspace{8pt}

\textbf{宮西 正宜. "代数幾何学" 裳華房}  \vspace{-6pt}

難易度:★★$\sim$★★★  \vspace{-6pt}
 
コメント: 1章の層の理論はかなり詳しく書かれているのが良いです(ちょっと読みづらいけど). スキームの部分は普通. 最後の曲面論は宮西先生の分野によっている気がします. 気のせいですかね. 
\vspace{8pt}

\textbf{上野 健爾. "代数幾何" 岩波書店}\vspace{-6pt}

難易度:★★$\sim$★★★  \vspace{-6pt}

コメント:  読んだことはないです. 目次を見る限り丁寧そうな本な気がします.  
\vspace{8pt}

\textbf{David Mumford. "Algebraic Geometry. I. Complex projective varieties" Springer.}\vspace{-6pt}

難易度:★★$\sim$★★★  \vspace{-6pt}

コメント:  藤野先生や桂先生たちのおすすめ.  読んだことはないです. 藤野先生曰く, "この本には「スキー
ム論」も「複素多様体論」も「層係数コホモロジー論」も出てこないが、代数幾何学の神髄に早い段階で触れることの出来る良書である。"とのこと. 

私は「代数幾何学やるなら学部4年でスキーム, 層, コホモロジーに慣れる必要がある」と思うので, 学部4年のセミナーで読むのはあまりおすすめしません. ちなみに2巻があるらしく, 2巻ではスキームをやってます. 
%いい点はスキーム論や層係数コホモロジー論が出てこないところ. 悪いところはスキーム論や層係数コホモロジー論が出てこないところ. 学部4年でスキーム, 層, コホモロジーに慣れる必要があるので, 学部4年のセミナーで読むのは私はおすすめしない. サブで読むのがいいと思う.
\vspace{8pt}

\textbf{D. マンフォード. "代数幾何学講義" シュプリンガー数学クラシックス}\vspace{-6pt}

\textbf{David Mumford. The Red Book of Varieties and Schemes}\vspace{-6pt}

難易度:★ $\sim$ ★★ \vspace{-6pt}

コメント:  通称"Red Book". 私は学部3年で自主ゼミで半分くらいまで読みました. 結構適当だが雰囲気を味わうにはいい本だと思います. これも上と同じ理由で学部4年のセミナーで読むのはあまりおすすめしません. 
\vspace{8pt}


\textbf{Qing Liu. "Algebraic Geometry and Arithmetic Curves" Oxford Graduate Texts in Mathematics. Oxford University Press}

難易度:★★$\sim$★★★  \vspace{-6pt}

コメント: どちらかというと数論系の人が読んでいるイメージ. 
%これが気に食わない人はEGA(1800ページ)を読むのだろうか. 私が学部4年の時に同期にEGAを読んでいる人がいたが, 私は「」
\vspace{8pt}

\textbf{Ulrich G\"ortz , Torsten Wedhorn. "Algebraic Geometry I: Schemes" Springer}

難易度:★★$\sim$★★★  \vspace{-6pt}

コメント: 長い. 圏論好きな人はこの本で代数幾何学を学んでもいいかもしれないです. 最近2巻が出ました. 
\vspace{8pt}

\textbf{飯高 茂. "代数幾何学" 岩波講座 基礎数学. 岩波書店} \vspace{-6pt}

\textbf{Sigeru Iitaka. "Algebraic Geometry
An Introduction to Birational Geometry of Algebraic Varieties"  (GTM, volume 76) Springer} \vspace{-6pt}

難易度:★★$\sim$★★★  \vspace{-6pt} 

コメント: 飯高先生が岩波講座 基礎数学で書いたもの. 私は学部4年に高木俊輔先生のもとでこの本の英語版を読みました. 日本語版は難しすぎて読めないので, 英語版を読むのをおすすめします.

タイトル通り, 双有理幾何学につながるような内容が多いです. %が, ちょくちょく議論が間違っていたり適当な部分がある. 
前半は基礎的な内容であるが, 後半はかなりマニアックな内容(Grothendieck's devissage theorem, 藤田$\delta$種数, 飯高予想)が多いです. おそらくこの本で勉強するよりは, 勉強した後にこの本を流し見する方が良いと思います. なお2021年の研究の際にこの本にお世話になりました.
\vspace{8pt}


\textbf{"Elements de geometrie algebrique"} \vspace{-6pt}

\textbf{"Seminaire de Geometrie Algebrique du Bois Marie"} \vspace{-6pt}

難易度:★★$\sim$★★★★  \vspace{-6pt} 

コメント:  EGAとSGA.EGAは1500ページ, SGAは6500ページ. 長すぎる. しかもフランス語. 
学部生の時から「こんなん誰が読むんや?」と思っていたが, 今でも多くの人が読んでいる気がします. 数論系の人が読んでるイメージ. 代数幾何勢はHartshorneな気がします. 

有名なGrothendieck-Riemann-RochがあるのはSGA6. GRRは使ったことあるが, 証明見たことないです. 
またつい最近Grothendieck宇宙に関して調べたら, これもSGAにのっていました. ゴリゴリの集合論みたいなこともSGAに乗っているので, SGAはなんでものってるんだなあ.

というかHartshorneが出る前の代数幾何学の研究者はEGAを読んでたのですかね? 
大沢先生は「私はEGAはわからなかったが中野先生の代数幾何学の本(中野茂男 著 復刊 代数幾何学入門)は読めた」とおっしゃっていました. 
%ちなみに別の機会に「永田 可換体論(永田雅宜 著 可換体論 )を読んだんだけど...」と大沢先生が言っててびっくり 永田 可換体論はかなり難しかった印象です. 

学部の時に「Hartshorneの3章はGrothendieckが関わったからよくできている」みたいな話を聞いた気もします. また「HartshorneがResidue and Dualityの本を書いた時, Grothendieckからめちゃくちゃ訂正を喰らった」みたいなの聞いた気もします. (まあ噂程度ですので, 真に受けとめないでください.)

\vspace{8pt}



\textbf{向井 茂. "モジュライ理論" 岩波書店}

難易度:★★$\sim$★★★  \vspace{-6pt}

コメント: 気になるから読んでほしい. 

\subsubsection{修士セミナー向け}

\textbf{Janos Kollar, 森 重文. "双有理幾何学" 岩波書店} \vspace{-6pt}

\textbf{Janos Kollar, Shigefumi Mori. "Birational Geometry of Algebraic Varieties" Cambridge Tracts in Mathematics, Series Number 134. Cambridge University Press} \vspace{-6pt}

難易度:★★★$\sim$★★★★\vspace{-6pt} 

コメント: 双有理幾何学をやるならこの本を修士1年に勉強することになります. なぜなら多くの論文で"We employ the standard notation and conventions in [Har77, KM98]."というふうに, HartshorneとKollar-Moriは引用されるからです. 私も書きました. 

私のところに来て双有理幾何学を勉強する人はいない(と思う)ので, この本はサブで読むことになります. 
 この分野は双有理幾何学・極小モデル理論(MMP)からは逃れられないので, (適当でもいいので)読むことをおすすめします. 私は適当にしか読んでないので, いまだにこの本を開くことになってます. 
 \vspace{8pt}
 
 \textbf{Yujiro Kawamata, Katsumi Matsuda, Kenji Matsuki. "Introduction to the Minimal Model Problem" Adv. Stud. Pure Math., 1987. } \vspace{-6pt}

難易度:★★★$\sim$★★★★\vspace{-6pt} 

オンライン入手可能: 
\url{https://projecteuclid.org/ebooks/advanced-studies-in-pure-mathematics/Algebraic-Geometry-Sendai-1985/chapter/Introduction-to-the-Minimal-Model-Problem/10.2969/aspm/01010283} 
\vspace{-6pt}

コメント: 個人的にはKollar-Moriよりこっちの方が好きです. 松木先生が書いたであろうプログラミングっぽい図が好きです. 通称KMM. ちなみに代数幾何学・双有理幾何学で"KMM"と呼ばれる論文は少なくとも3つあります. 
%私は修士の時の教育実習のの通勤時の電車で適当によんだ. 
\vspace{8pt}

\textbf{Robert Lazarsfeld. "Positivity in Algebraic Geometry I, II" Springer} \vspace{-6pt}

難易度:★★★$\sim$★★★★\vspace{-6pt} 

コメント: DemaillyのAnalytic methodの本を代数的に理解するならこの本. この本は正標数の研究室, とりわけ高木俊輔先生の研究室の人が読んでいるイメージです. 乗数イデアル層を代数的に理解するためには, この本の第3部をサブで読むことになります. なお第2部のベクトル束の話は研究でお世話になりました. 

高木俊輔先生の書評も見てください. (\url{https://www.ms.u-tokyo.ac.jp/~stakagi/academic/review_Lazarsfeld.pdf})
\vspace{8pt}


\textbf{Arnaud Beauville. "Complex Algebraic Surfaces" London Mathematical Society Student Texts, Series Number 34. Cambridge University Press} \vspace{-6pt}

難易度:★★★$\sim$★★★★\vspace{-6pt} 

コメント: 曲面論を手っ取り早く習うならこの本. この本を修士セミナーで読みたいのであれば, 私ではなく藤田先生のところで読むべきです. 私は修士1年の時に長岡大先生の修士セミナーに聴講して聞いたぐらいです. 
\vspace{8pt}

\textbf{Daniel Huybrechts, Manfred Lehn. "The Geometry of Moduli Spaces of Sheaves" Cambridge University Press} \vspace{-6pt}

難易度:★★★$\sim$★★★★\vspace{-6pt} 

コメント: semistable sheafなどのモジュライを学びたいならこの本. この本を修士セミナーで読みたいのであれば, 私ではなく大川先生や高橋先生のところで読むべきです. ただ気になるから読んでほしい.
 \vspace{8pt}
 
 \textbf{Caucher Birkar, Paolo Cascini, Christopher D. Hacon, James McKernan. "Existence of minimal models for varieties of log general type" Journal of American Mathematical Society, 2010.} \vspace{-6pt}

難易度:★★★★\vspace{-6pt} 

コメント: 通称BCHM. 数学の論文は通称ではヒットしないのが普通ですが, あまりにも有名すぎて"BCHM"で検索したらこの論文がトップに出てきます. 気になるから読んでほしい. 結果しか引用してないので. 
\vspace{8pt}

\textbf{川又 雄二郎. "高次元代数多様体論" 岩波書店} \vspace{-6pt}

\textbf{Yujiro Kawamata. "Algebraic Varieties: Minimal Models and Finite Generation" Cambridge University Press} \vspace{-6pt}

難易度: ★★★★\vspace{-6pt} 

コメント: 川又先生によるBCHMの解説書. 買ったはいいものの読んでません. 読んでほしい.
なお英語版はChen Jiang先生が翻訳したものです. 
\vspace{8pt}


\textbf{Osamu Fujino. "Foundations of the Minimal Model Program" Mathematical Society of Japan Memoirs. Mathematical Society of Japan} \vspace{-6pt}

難易度: ★★★★\vspace{-6pt} 

コメント: 私たち幾何・解析の人はKLT(Kawamata log terminal)まで扱えません. KLTは$L^2$可積分を意味するからです. しかし代数の人はLC(Log Canonical)まで扱えます. この本はLC, さらにもっと一般的なquasi log schemeについて消滅定理などを示し, そしてMMPの基本的な定理も示しています. 
なので消滅定理を代数的に理解するなら, この本を読むことになります. 
研究で藤野先生の論文を引用する際にこの本にお世話になりました. 

%確かに解析的にLCを扱うという研究テーマもある気がしてきた
%タイトル通り"Foundations(基礎)"である. 
%昔これらの理論を解析的にやろうと思って藤野先生の挫折してしまった. 

\vspace{8pt}


\textbf{Osamu Fujino. "Iitaka Conjecture An Introduction" Springer} \vspace{-6pt}

難易度: ★★★★\vspace{-6pt} 

コメント: 飯高予想でのViehweg先生や藤野先生の結果などを網羅しています. P\u{a}un-Takayama, Hacon-Popa-Schnellなど順像層の正値性の話を代数的に理解するならこの本です.  
一番初めの研究で大変お世話になりました. Viehwegのテクニックは本当によく出てきます. 

藤野先生の2022年名古屋大学集中講義ノート \url{https://www.math.kyoto-u.ac.jp/~fujino/sonota.html} も参照. 
\vspace{8pt}


\textbf{Kenji Ueno. "Classification Theory of Algebraic Varieties and Compact Complex Spaces" Springer}   \vspace{-6pt} 

難易度:★★ $\sim$ ★★★★   \vspace{-6pt} 

コメント: 代数幾何の本でもあり, 多変数複素解析の本でもあります. おそらくMMPの枠組みができる前の分類理論の話です. (詳しくは藤野先生の2022年名古屋大学集中講義ノート \url{https://www.math.kyoto-u.ac.jp/~fujino/sonota.html} も参照.) 
飯高予想の論文を読むと, この教科書への引用が本当に多いです. 
よく参考にするので紹介しました. 気になる人はサブで適当に読めばいいと思います. 
\vspace{8pt}

\textbf{Yoichi Miyaoka , Thomas Peternell. "Geometry of Higher Dimensional Algebraic Varieties" Springer} \vspace{-6pt}

難易度: ★★★★\vspace{-6pt} 

コメント: 半分宣伝. この本は宮岡先生の研究のチャーン類の話が面白いのですが, 知ってる人が少ないです. いい本なのになあ. 第2チャーン類の研究でお世話になりました. 

\textbf{Christian Schnell. "A graduate course on the Generic vanishing theorem" }  \vspace{-6pt} 

難易度:★★★★\vspace{-6pt} 

オンライン入手可能: \url{https://www.math.stonybrook.edu/~cschnell/pdf/notes/generic-vanishing.pdf} \vspace{-6pt} 

コメント: Schnell先生によるレクチャーノート. 導来圏やFourier-Mukai変換を"具体的に"双有理幾何学に応用した内容を取り扱っています. 結構強力な道具なので, これ読んでほしいです. 
\vspace{8pt}

\textbf{Janos Kollar. "Rational Curves on Algebraic Varieties" Springer} \vspace{-6pt}

\textbf{Olivier Debarre. "Higher-Dimensional Algebraic Geometry" Springer} \vspace{-6pt}

難易度:★★★$\sim$★★★★\vspace{-6pt} 

コメント: Fanoなど有理曲線の使い手になるのであれば, この本を読むことになります. この本を修士セミナーで読みたいのであれば, 私ではなく藤田先生のところで読むべきです. 
Debarreの方が読みやすくて, Kollarの方が難しいです. ただKollarの方がHilbert スキームやChowスキームの構成をきちんとしています. Kollarの方が引用されている気がします.

私はRC(Rationally connected)の基本的なことをちょっと見たくらいです. 全然身についていないので毎回論文読むたびに"$\mathrm{RatCurve}^{n}$ってなんだ?"となりKollarの本を開くことになってます. 有理曲線独特の議論はFanoやRC系の論文を読むと本当によく出てきます. 
なので高木寛通先生のサーベイ\url{https://repository.kulib.kyoto-u.ac.jp/items/4e7bdf98-f0fe-4c8b-bb1b-2602841ea5af}は本当に助かります. 
 \vspace{8pt}
 
 \textbf{Chenyang Xu. "K-stability of Fano Varieties" Cambridge University Press} \vspace{-6pt}

難易度:★★★★\vspace{-6pt} 

コメント: これも半分宣伝. $K$-安定性の代数的な事柄をまとめた本. この本を修士セミナーで読みたいのであれば, 私ではなく藤田先生のところで読むべきです. 私は読んでないです. 
 \vspace{8pt}
 

\subsection{複素幾何学の教科書}
複素幾何学をカバーする範囲は広いので代数幾何学と違って, "必ず理解すべき内容"は分野によって異なります.  

個人的には学部4年セミナーで「層・コホモロジー・ホッジ理論(調和積分論)・小平消滅定理・小平埋め込み定理」まで理解できれば十分だと思います. 
おそらく一つ教科書を決めてその内容を理解し, そっから理解すべき内容を増やしていく方法になると思ります. 

あと個人的な見解として, 「代数幾何学を学んでから複素幾何学を学ぶ」ことは割と楽にできるが, 「複素幾何学を学んでから代数幾何学を学ぶ」のは苦労がいると思います. 代数幾何学のスキーム論などの抽象的な理論はじっくり時間をかけて読まないと理解するのが難しいからです.\footnote{なおスキーム論を知らなくても複素解析空間を学べば事足りるが, 複素解析空間を学ぶのはスキーム論を学ぶより難しいです. 議論がかなりややこしいから. } 
なので「複素幾何学を学んでから代数幾何学を学ぶ」には実践的な方法で学ぶことになります. 
具体的には, 代数幾何学・双有理幾何学の道具を使っていって学んでいくことになります. 
例えば代数幾何学の用語(ample, psefなど)を複素幾何学の用語(positive, with semipositive curvature current)に翻訳するなどです. 

なお複素幾何学を専門にするのであれば, (どんな形であれ)代数幾何学・双有理幾何学からは逃れられないと思った方がいいです. 論文読んでていきなり極小モデル理論が出てくることはよくあります. 仕方ない.  

\subsubsection{学部4年セミナー向け}


\textbf{Raymond O. Wells. "Differential Analysis on Complex Manifolds" Springer}  \vspace{-6pt}

難易度:★★ \vspace{-6pt}

コメント: 私が学部生の時はこの本が複素幾何の鉄板の教科書でした. 学ぶべきだと思う内容を網羅して良い本だと思います. 

ただし4章の擬微分作用素の評判は本当に悪い. その部分はガン無視するか他の教科書で学ぶことになります. (他の教科書に関しては松尾信一郎先生のページを参照.)
前者を選んだ場合は代数っぽいな複素幾何が好きと言うことになり, 後者を選んだ場合は解析っぽいな複素幾何が好きと言うことになります. 私は前者でしたがどちらでも良いです. 
\vspace{8pt}

\textbf{Daniel Huybrechts. "Complex Geometry: An Introduction" Universitext, Springer} \vspace{-6pt}
 
難易度:★★ \vspace{-6pt}

コメント: 読んだことはないです. がページをペラペラめくったら割といい本だと思います. 意外だったのが層の理論をAppendixに回しているところ. 複素幾何をやる上で層は必須なので, そのAppendixも含めて勉強するべき. 
\vspace{8pt}

\textbf{堀川 穎二. "複素代数幾何学入門" 岩波書店}

難易度:★★$\sim$★★★ \vspace{-6pt} 

コメント: 複素数体上の代数幾何学を複素解析的に扱った本. この本はかなりおすすめ. スキーム論ガン無視で複素代数幾何学をやるのであれば, この本で勉強してもいいかも. この本を読んで複素代数幾何学の道に進むのは可能です. 

リーマン面の部分までは(因子の部分を除いて)従来の本と同じ. 最後の楕円曲線の退化の部分を取り扱っているのは, 日本語だとこの本くらいしかないと思います. 最後まで読むことができれば, 代数勢にも負けないくらいの代数幾何学の力はつくと思います. 残念なのはホッジ理論(調和積分論)・小平消滅定理・埋め込み定理がない点.  この部分だけはWellsやHuybrechtsで補う必要があります. 
\vspace{8pt}

\textbf{小林 昭七. "複素幾何" 岩波書店}  \vspace{-6pt} 

難易度:★★$\sim$★★★ \vspace{-6pt} 

コメント: 日本語で複素幾何を学ぶならこれになるのでしょうか? 
ベクトル束の第2基本形式の部分は研究で本当にお世話になりました. 
もちろんこの本で勉強してもいいです.  が, せっかくセミナーで読むなら英語の本を読んだ方がいいと思います. 
\vspace{8pt}

\textbf{Christian Schnell. "A graduate course on Complex manifolds" }  \vspace{-6pt} 

難易度:★★$\sim$★★★ \vspace{-6pt} 

オンライン入手可能: \url{https://www.math.stonybrook.edu/~cschnell/pdf/notes/complex-manifolds.pdf} \vspace{-6pt} 

コメント: Schnell先生によるレクチャーノート. ちらっと見ましたがこれもいい本です. 学びたい内容が簡潔(136ページ)にまとまっています. そして代数っぽい例が多いです. Schnell先生の論文や本はわかりやすいものが多い印象です.
個人的にはベクトル束のことが少ないかなとは思いますが, この本で勉強しても良いと思います.  
 
\vspace{8pt}


\textbf{Claire Voisin. "Hodge Theory and Complex Algebraic Geometry I, II" Cambridge University Press}  \vspace{-6pt} 

難易度:★★$\sim$★★★ \vspace{-6pt} 

コメント: 最初の方は従来と同じですが, 途中からVHS(Variation of Hodge Structure)などの話になります. これを幾何学的な側面から学べる本はこれくらいだと思います. なのでその部分でもこの本は読む価値はあります. 欠点は二冊に分かれていて長い点と議論がまあまあ適当な点. %その点はVoisonらしい. 
\vspace{8pt}

\textbf{小平 邦彦. "複素多様体論" 岩波書店}\vspace{-6pt}

\textbf{Kunihiko Kodaira "Complex Manifolds and Deformation of Complex Structures" Springer} \vspace{-6pt}

難易度:★★$\sim$★★★ \vspace{-6pt} 

コメント:  前半は従来の本と同じですが, 後半は小平・スペンサーの変形理論の話です. 英語版の方が変形理論の証明が丁寧でした. 後半の内容は面白いが, この本で学ぶのが適切かがわかりません. 幾何学的な視点でモジュライを研究したいなら, どう勉強すればいいかを井上先生に詳しく聞いた方がいいかも. 

複素幾何の創始者でもある小平先生の本で学ぶと言うことは推奨されるべきですが, ホッジ理論(調和積分論)の部分の記法が独特で読みづらいです.丁寧ですが読むのに疲れます. また小平消滅定理・埋め込み定理がないのも残念です.  なお小平先生自身が小平消滅定理のことを書いた本はないです(セミナーノートならあります). 最後の藤原先生の補足は解析系の人が読んでいるイメージ. 
\vspace{8pt}

\textbf{小木曽 啓示. "代数曲線論" 朝倉書店}\vspace{-6pt}

難易度:★★ \vspace{-6pt} 

コメント: 代数幾何的に代数曲線やリーマン面を学びたいならこの本はおすすめです. わかりやすくてかなり良い本です. 読むべき. 
なお私はこの本は代数幾何ではなく複素幾何の本だと思います. コホモロジーに関する証明がかなり解析的です. 修士に行かない人はこの本を学部4年セミナーで読んでも良いです. 修士に行く予定があるなら, サブでいいので読むことを"強く"おすすめします. 
\vspace{8pt}

\textbf{Jean-Pierre Demailly. "Complex Analytic and Differential Geometry"} \vspace{-6pt}

難易度:★★$\sim$★★★★  \vspace{-6pt}

オンライン入手可能: \url{https://www-fourier.ujf-grenoble.fr/~demailly/manuscripts/agbook.pdf}\vspace{-6pt}

コメント: 長い. セミナーで読むにはおすすめしません. 理由は長いのと割とマニアックな内容が多いから. 1章はいいとして, 2書のanalytic space, 3章のcurrentの部分は半分くらいの内容が専門家向けです. この辺りは難しすぎます. ところどころは読めますが, 間に専門的な内容が挟まっている印象です. これは専門家が勉強・引用する際に使う本です. 
\vspace{8pt}

\textbf{Phillip Griffiths, Joseph Harris. "Principles of Algebraic Geometry" Wiley Classics Library.}  \vspace{-6pt}

難易度:★★$\sim$★★★ \vspace{-6pt} 

コメント: 長い. セミナーで読むにはおすすめしません. 長いから. 800ページを1年で読めると思わない方がいいです. これは辞典のように使うべき. 
\vspace{8pt}

\textbf{今野 宏. "微分幾何学" 東京大学出版会}  \vspace{-6pt}

難易度:★★$\sim$★★★★ \vspace{-6pt} 

コメント: 微分幾何学に基礎的なことがほぼ全て網羅されていると思います. ページ数が普通の本と変わらないのにこんなにも内容が多い本はこの本くらいだと思います. 非常に良い本です. 

ただセミナーで読むにはおすすめしません. 理由は「証明・議論が簡潔すぎて全く頭に入らない」から.  議論・証明が本当に簡潔で無駄が全くないです. ただなぜか何回読んでも全く頭に残らないです. おそらく他の本で勉強したのちにこの本を読むのが良いと思います. 護身用として持っておくと良いと思います. 

\subsubsection{修士セミナー向け}

\textbf{Jean-Pierre Demailly. "Analytic Methods in Algebraic Geometry"  International Press of Boston Inc} \vspace{-6pt}

難易度:★★★$\sim$★★★★ \vspace{-6pt} 

オンライン入手可能: \url{https://www-fourier.ujf-grenoble.fr/~demailly/manuscripts/analmeth_book.pdf}\vspace{-6pt}

コメント: 私含め高山研(東京大学)の修士の学生は一年でこれを読むことになっています. 改めて見れば「いい本とは言い難いが, この分野を網羅していてまとまっている本はこれしかない」と言う感じです. 読みやすくはないし, 難しいです. ただこれを読めば, 修士2年から論文を読むことができます. 

なお高山先生の最適化コースは"1$\sim$6章→11$\sim$19章". 私もそれがいいと思います. 大まかに説明すると以下の通り.
\vspace{-8pt}
\begin{itemize}[left=0pt]
  \setlength{\parskip}{0cm} % 段落間
  \setlength{\itemsep}{0cm} % 項目間
\item 1, 2章は定義が続くので軽めに読めばいいと思います.  ここはサッと終わらしたいです. 2章のカレントの部分は深入りすると沼なので, 適度に切り上げましょう. 
\item 3, 4章は複素幾何の復習. ここでこの分野のメインである特異エルミート計量が出てきます. ちなみにDemailly先生は接続や曲率で独自の記号を使っているので注意. 
\item 5章は$L^2$-estimateの内容. わからなければDemailly先生の$L^2$-estimateのレクチャーノートを参考にした方が良いかと. ここで重要な道具である乗数イデアル層が出てきます. 
\item 6章が前半の一区切り. この部分は「代数幾何の用語が特異エルミート計量で定義される」という一番面白いところ. 
\item 11,12章はベクトル束の話. この辺りは多変数複素解析の$L^2$理論を専門にするなら熟読すべき.
\item 13章はOhsawa-Takegoshi 拡張定理. この分野のブレイクスルー. 読むべき.
\item 14章はOhsawa-Takegoshi 拡張定理の応用で, Demailly近似定理とDemailly-Kollar 01の内容. Demailly近似はよく使うので, 読むべき. 
\item 15章はDemailly-Ein-Lazarsfeld 00の内容. これは19章の布石. 
\item 16章はDemailly-Peternell-Schneider 01の内容. 消滅定理の専門になるなら読んでもいいかも.
\item 17章はSiu98, Siu03, P\u{a}un08によるInvariance of plurigenera. ベクトル束の特異エルミート計量の話に関わるので読むべき.
\item 18章は Demailly-P\u{a}un 04の内容. 当時は「Nakai-Moishezonのケーラー版なだけか」と思っていたが, 今になってこの定理の良さに気付きました. ケーラーは部分多様体が全くないこともあるので, これが成り立つのはにわかには信じ難いです. 
\item 19章は Boucksom-Demailly-P\u{a}un-Peternell 13の内容. Boucksom先生の博論の内容に関わります. これは結果が面白いです. 「$K_X$がpsefは$X$がnonuniruledと同値」であることは, 2025年にKahlerでも成り立つことがWenhao Ou先生の結果によりわかりました. 
\end{itemize}
もしかしたら11$\sim$19章は適度に省いてもいいかもです. 例えば代数的ことをしたいのなら「14章の後半から16章」を省略し, 解析的なことをしたいのならば「18,19章」を飛ばしてもいいかもしれません. 
あと個人的にはDemailly-Peternell-Schneider 94がないのは残念です. あれこそ教科書に入れてもいいのになあ. 

その他の章の感想は以下の通り.
\vspace{-8pt}
\begin{itemize}[left=0pt]
  \setlength{\parskip}{0cm} % 段落間
  \setlength{\itemsep}{0cm} % 項目間
\item 7章の藤田予想, 10章の松坂の巨大定理の部分は面白いが証明がテクニカル. 一回読んでも「???」となるので, あとで読めばいいと思う. 
\item 8章のHolomorphic Morseに関しては適当に読んでいいと思います. 
もし真剣に読みたいのであれば, Ma-Marinescuの本を読めばいいかと. なおDemailly先生はHolomorpshic Morseがお好きのようです. 
\item 9章のGreen-Griffith-Langは合っているかがわかりません. 
\item 20, 21章は読まなくていいです. ただ20章の辻先生のcanonical metric自体はSong-Tianで使われてたはずで応用はあり, 21章は何か光るものがありそうです. 
\end{itemize}
\vspace{8pt}

\textbf{Christopher Hacon, Mihnea Popa, Christian Schnell. "Algebraic fiber spaces over abelian varieties: arounda recent theorem by Cao and P\u{a}un"  Amer. Math. Soc., Providence, RI, 2018.}  \vspace{-6pt} 

難易度:★★★$\sim$★★★★ \vspace{-6pt} 

コメント: 私以降の高山研, とりわけ稲山先生や渡邉先生はDemailyの本の後にこれを読んでベクトル束の特異エルミート計量を勉強したと思います. 私もこの論文が一番わかりやすいと思います. 
読むなら後半Section Dからの, Schnell先生が書いたであろう特異エルミート計量の内容. 
前半はGV-sheafやFourier-Mukaiの話で私もわかりません. 読まなくて良いです.

Hacon-Popa-Schnell 18はCao-P\u{a}un 17とP\u{a}un-Takayama 18の解説という位置付けです. 
実はP\u{a}un-Takayama 18 の論文の方が強いことを言っていますが, P\u{a}un-Takayama 18 は本当に読めないです.
ちなみにP\u{a}un先生のサーベイ(\url{https://arxiv.org/abs/1606.00174})は読むのをおすすめします. 
%でもやっぱりSchnellのベクトル束の特異エルミート計量の定義の方がわかりやすくて良いと思う.
\vspace{8pt}

\textbf{Gabor Szekelyhidi. "An Introduction to Extremal Kahler Metrics" Graduate Studies in Mathematics, 152. American Mathematical Society}  \vspace{-6pt} 

難易度:★★★$\sim$★★★★ \vspace{-6pt} 

コメント: Aubin-Yau, Yauの定理を手っ取り早く学ぶならこの本になると思います. 後半はExtremal計量やK-stabilityの話. 多分test configulationの定義がちょっと古く, 洗練された今の形の定義と違うように見えます.(多分同じだと思いますが.) でもこういった定義や, 4章のFutaki不変量の定義は一回味わってもいいかも. 
%後半に関しては完全に著者の研究の話である.

私は適当に読んだだけです. なので後半の内容は必読かどうかがちょっと不明です. K-stability関連は急速に発展しているので, これは古い内容になるかもしれません. 
よってセミナーで読むなら前半だけ読んで, 後半は別の本を読んでもいいと思います. 
あとSzekelyhidiはどう読めばいいんですかね?(セキリヒディでいい?)
\vspace{8pt}

\textbf{中島 啓. "非線形問題と複素幾何学" 岩波書店}  \vspace{-6pt} 

難易度:★★★$\sim$★★★★ \vspace{-6pt} 

コメント: いい本. 1999年までのKahler Einstein計量の存在問題を扱った本で, Aubin-Yau, Yauの定理の内容から, Nadelの乗数イデアル層の話まで書いており, 当時の内容を存分に書いたいい本です.

ですが, この分野はこの20年で急速発展したため, この本の内容はもはや古典になってしまいました. 
%(Test Configurationが出ていないK安定性の定義は今はほぼ使われていないと思う)
なのでサブで読んだ方が無難な気もします. Aubin-Yau, Yauの定理までならこの本を読んでもいいかもです.
\vspace{8pt}

\textbf{S. Boucksom, T. Hisamoto, M. Jonsson. "Uniform K-stability, Duistermaat-Heckman measures and singularities of pairs." Ann. Inst. Fourier (Grenoble)}  \vspace{-6pt} 

難易度:★★★$\sim$★★★★ \vspace{-6pt} 

オンライン入手可能: \url{http://sebastien.boucksom.perso.math.cnrs.fr/publis.html}

コメント: K安定性を微分幾何学的に勉強するなら, 最初にこの論文を読むのですかね? 私には分からないので他にもいい本や論文・サーベイがあったら教えてください. 
この論文自体はかなり読みやすかった気がします. 
ちなみにこの分野の講演を聞いてると, 「初めに$I$とか$J$とかいろんな汎函数を定義しないといけなくて大変だなあ」と思います. 

Boucksom先生の論文でnonArchimedeanでないものは読みやすくて面白いものが多い印象です. 
例えば以下のものがあります.
\vspace{-8pt}
\begin{itemize}[left=0pt]
  \setlength{\parskip}{0cm} % 段落間
  \setlength{\itemsep}{0cm} % 項目間
  \item "Divisorial Zariski decompositions on compact complex manifolds." 中山先生のDivisorial Zariski decompositionをcompact manifoldに拡張したもの. かなり細かく丁寧に書かれてる. 
  \item "The pseudoeffective cone of a compact Kähler manifold and varieties of negative Kodaira dimension" 通称BDPP. 「$K_X$ psefが$X$ non-uniruledと同値」という結果が面白いです. しかもこの証明に使われるテクニックはBoucksom先生の博論の内容になります. 読むなら5章まででいいです. が, 6章以降はPeternell先生が書いているっぽくて面白いです.
  \item "Monge-Ampère equations in big cohomology classes" 通称BEGZ. 内容としては, nonpluripolar product使って巨大cohomologyクラスでMonge-Ampere方程式を解くといったもの. 結構解析の部分が辛いが, 証明のコアはFujita近似定理で代数的なものなので面白い. 
  \item  "A variational approach to complex Monge-Ampère equations" 通称BBGZ. これ解析あんまり知らなくても結構読める. 当時読んでて「こんな方法でMonge-Ampere方程式を解けるのか」と思った. これと上のBoucksom-Hisamoto-Jonssonが結構リンクしてて面白い. 
  \item "Uniruledness of stable base loci of adjoint linear systems via Mori Theory" 高山先生の結果をMMP使って簡単に示したもの. "MMP使ってMinimal modelまで持っていって, 最後はKawamata 91使って終わり"という気付けば簡単だが, 面白いテクニックである.
  \item "Kahler-Einstein metrics and the Kähler-Ricci flow on log Fano varieties" 通称BBEGZ. 当時読んでも分からなかった. 
  \item "Uniform K-stability and asymptotics of energy functionals in Kähler geometry" これも当時読んでも分からなかった. 後のBBJの布石になったんですかね?
  \item "Appendix to D. Witt Nystrom's paper Duality between the pseudoeffective and the movable cone on a projective manifold." BDPPの予想をProjectiveの場合にWitt Nystrom先生がといた. その際のAppendixである. Appendix単体でも面白い. この論文も最近(2025年)お世話になった(Witt Nystrom先生にもお世話になった).
  \item " A variational approach to the Yau-Tian-Donaldson conjecture" 通称BBJ. ver1だけ読んだことある. その時は数ページの論文でDemailly近似っぽいことしていてなるほどなあと思った. が, 今見たらめちゃくちゃ内容が増えててびっくりした. 
  \item "A note on Lang's conjecture for quotients of bounded domains" もうこの時期から「Boucksom先生の論文はnonArchimedeanすぎて読めない」と思ってたらいきなり複素解析の論文が出てびっくりした. 本当になんでもできるんだなあ.
    \end{itemize}
学生の時に適当に読み漁ってましたが, まさか10年後にBEGZなどを論文で使うとは思っていなかったです.
Boucksom先生の博論も引用することになったし, 何があるか分からないですね. 
  \vspace{8pt}


\textbf{Jian Song, Ben Weinkove. "Lecture notes on the Kahler-Ricci flow"}  \vspace{-6pt} 

難易度:★★★$\sim$★★★★ \vspace{-6pt} 

オンライン入手可能: \url{https://arxiv.org/abs/1212.3653}\vspace{-6pt}

コメント: Kahler Ricci Flowを勉強するならこの本になるのでしょうか?一回読んだことあるが, ほぼほぼ内容を忘れてしまいました. 私が修士・博士の時はKahler Ricci Flowが盛んでしたが, 最近はあんまりその話を聞きません. Tianが言っていたKahler Ricci Flowを使ったMMPは一体どうなったのでしょうか? 知っている人は教えてください. 
\vspace{8pt}

\textbf{Shoshichi Kobayashi. "Differential Geometry of Complex Vector Bundles"  Mathematical Society of Japan}  \vspace{-6pt} 

難易度:★★$\sim$★★★★ \vspace{-6pt} 

オンライン入手可能: \url{https://www.mathsoc.jp/publications/pubmsj/}\vspace{-6pt}

コメント: 最初の1$\sim$3章は学部4年生でも読めます. 問題は4章以後.  4$\sim$5章は完全にKobayashi Hittchin対応の内容. この内容は私は研究していてお世話になりました. 6章以降はもうModuliの話になって私は全くわかりませんでした. 4章以後の内容は専門家向けだと思います. 本自体としては今でも引用するくらい良い本です. %教科書として良いかは微妙である. 
\vspace{8pt}

\textbf{辻 元. "複素多様体論講義" SGCライブラリ 94 サイエンス社}  \vspace{-6pt} 

難易度:★★★$\sim$★★★★ \vspace{-6pt} 

コメント: "多変数複素解析を用いた代数幾何学"の内容を網羅してます. 素晴らしい. タイトル通り"広範な基礎"を取り扱ってます. この本は教科書として読むより, サーベイとして読むべき. 
\vspace{8pt}

\textbf{Sebastien Boucksom, Philippe Eyssidieux, Vincent Guedj. "An Introduction to the Kähler-Ricci Flow" Springer}  \vspace{-6pt} 

難易度:★★★$\sim$★★★★ \vspace{-6pt} 

コメント: 読んだことはないです. 特異多様体のBott-Chern類の研究で4章を参照したぐらいです. 特異多様体でKähler-Ricci Flowをやっているらしく, 「これは誰が読めるんかなあ?」と思いました. 滑らかな多様体の上でもKähler-Ricci Flowは難しい. 

\textbf{Vincent Guedj, Ahmed Zeriahi “Degenerate Complex Monge-Ampere Equations”  Eur. Math. Soc.}
 \vspace{-6pt} 

難易度:★★★$\sim$★★★★ \vspace{-6pt} 

コメント: これも読んだことはないです. 見た感じ解析が辛そうな感じがします. この本の書評を参照にしてください. 
\vspace{-8pt}
\begin{itemize}[left=0pt]
  \setlength{\parskip}{0cm} % 段落間
  \setlength{\itemsep}{0cm} % 項目間
\item 細野元気氏の書評 \url{https://www.jstage.jst.go.jp/article/sugaku/74/2/74_0742204/_article/-char/ja}
\item Kolodziej先生の書評 \url{https://link.springer.com/article/10.1365/s13291-018-0182-0}
\end{itemize}
\subsection{多変数複素解析の教科書}

私がカバーできる多変数複素解析は$L^2$理論( コホモロジーの消滅定理, Ohsawa-Takegoshi $L^2$ 拡張定理, ベクトル束の特異エルミート計量など)ぐらいです. 
それもユーザーとして使っているだけです. 
この内容を専門にするなら, 計算ゴリ押し力が問われると思います. 
%つまり, コホモロジーの消滅定理やOhsawa-Takegoshi $L^2$ 拡張定理, ベクトル束の特異エルミート計量は, 利用するだけである. 
%この道に進むなら, かなり計算ゴリ押し力が問われる. その点ではこの分野は"解析"に近い. 
%また悲しい話ですが, $\C^n$の領域(正則領域・正則凸・擬凸)を調べる多変数複素解析は今(2025年)の流行りではないです. 

$L^2$理論に進むなら, 複素幾何, とりわけベクトル束や特異エルミート計量からは逃れられないと思います. 
%小林昭七先生が複素幾何の本のあとがきで「今後, 多変数関数論はますます幾何的になっていくと思われる」と言っていました.本当にその通りです.
%現代において多変数複素解析を専門にするのには, 複素幾何を学ぶ必要がある. 
%現代において$\C^n$の領域の論文のほとんどが$L^2$の論文である. 
なので$L^2$理論を勉強・専門にしたい場合でも, 学部4年・修士セミナーにおいてはWells, Demaillyなどの複素幾何の本を読むことをおすすめします. (中野の本でも良いです. )
Wells, Demaillyを読んだ後でも多変数複素解析の専門家になることはできます. 稲山先生や渡邉先生などがまさにそうです. 

そのほかの多変数複素解析に関して次の通りです. 
\vspace{-8pt}
\begin{tcolorbox}[mybox]
\begin{itemize}[left=0pt]
  \setlength{\parskip}{0cm} % 段落間
  \setlength{\itemsep}{5pt} % 項目間
\item 岡理論から派生した岡多様体を学びたい人はForstnevicの本を読めばいいと思います.  \\ これを読む場合, 修士に日下部先生のところに行くことも視野に入れた方がいいです. 修士に私のところで岡多様体を学びたい場合は, 私も岡多様体を学びながら指導することになります. 
\item $\C^n$の領域(正則領域・正則凸・擬凸)を調べる多変数複素解析の勉強・研究は私はあまりお勧めしません. 今(2025年)でこの内容の論文をあまり見ないからです.(流行りではないと言ってもいいです.) もし勉強したい場合は$L^2$理論か岡多様体を軸にして勉強した方がいいと思います. 
\item CR多様体など解析をゴリゴリしたい人はH\"ormanderあたりを読めば良いと思います. ただそのような人は私より松本先生のところで本を読んだ方が絶対に良いです. 私は解析は全くわからないです. 
\end{itemize}
\end{tcolorbox}




\subsubsection{学部4年向け}

\textbf{中野 茂男. "多変数函数論 - 微分幾何学的アプローチ" 朝倉書店}   \vspace{-6pt} 

難易度:★★ $\sim$★★★  \vspace{-6pt} 

コメント: 私が修士1年になる前に高山先生から「多変数複素解析の本を読んでこい」と言われて, 慌てて読んだ本. この本は複素幾何と多変数複素解析の$L^2$理論を一緒に学べるのでお勧めです. $L^2$理論に進むならこの本を学部4年セミナーで読んでも良いです. ただ現在入手困難なのが悔やまれます. いい本なのになあ. 

%個人的にはCastelnuovoのcontraction theoremの高次元版の話題が書いてあったのが面白かったです. この時代は代数幾何と多変数複素解析が混ざっていた時代です. 特異点解消で有名な広中先生も多変数複素解析の論文を書いていました. 
%なのでこの時代を生き抜いた多変数複素解析の大御所は軒並み代数幾何, とりわけ特異点に強いイメージがある. ()
\vspace{8pt}

\textbf{Franc Forstneric. "Stein Manifolds and Holomorphic Mappings
The Homotopy Principle in Complex Analysis" Springer}   \vspace{-6pt} 

難易度:★★ $\sim$★★★★ \vspace{-6pt} 

コメント: 岡多様体の本. 個人的には多変数複素解析の新たな分野として位置付けられるのでは?と思います. 日本には日下部先生という岡多様体のトップクラスの研究者がいますし, この分野は今後来るのではと思います. 
是非とも読んでほしい. 学部4年セミナーで読んでも修士セミナーで読んでも良いです.  ただ「岡多様体は解析というよりは幾何に近い」イメージがあるので, 結局のところ複素幾何は必要だと思います.

日下部先生に聞いたところ, 「1$\sim$2章は複素幾何のことが(証明なしで)載っていて, 3$\sim$7章が読めれば良い」とのことです. なので標準的なコースとしては, 複素幾何をあらかた勉強した後に3$\sim$7章を読むのがいいかもしれません. 
\vspace{8pt}

\textbf{Lars Hörmander. "Introduction to Complex Analysis in Several Variables" North-Holland Mathematical Library}   \vspace{-6pt} 

難易度:★★ $\sim$★★★ \vspace{-6pt} 

コメント: 多変数複素解析の古典的名著. どちらかというと平地研(東大)のようなCR多様体など解析ゴリゴリの人が学部4年に読んでいる印象です. 私は読んだことないです. 読んだ人は「ヘルマンダーが偉すぎて, 行間がひどい」と言っている気がします. 
この本を学部4年セミナーで読みたいのならば, 私ではなく松本先生のところで読むべき. 
%私の前で読んでもいいが, 私は「計算が大変ですねー」くらいのコメントしかできないds. 
\vspace{8pt}

\textbf{大沢 健夫. "多変数複素解析" 岩波書店}   \vspace{-6pt} 

難易度:★★ $\sim$★★★★  \vspace{-6pt} 

コメント: ちょっと読んだ程度です. $L^2$理論を用いて$\C^n$の領域(正則領域・正則凸・擬凸)を学ぶなら, 日本語だとこの本になるかもしれません. 
$L^2$理論を研究する予定なら, 複素幾何をメインで学んで, サブでこれを読んだ方がいいと思います. 
増補版も出ていて, Guan-Zhou・BlockiのOhsawa-Takegoshiの最良評価に関する事柄も増えています. 
大沢先生の本は随所にユーモアがあって読んでいて面白いです. 
\vspace{8pt}

\textbf{大沢 健夫. "複素解析幾何とディーバー方程式 " 培風館}   \vspace{-6pt} 

難易度:★★ $\sim$★★★★  \vspace{-6pt} 

コメント: これもちょっと読んだ程度です. これも面白い本ですが, 内容が多岐に渡っているので初学者には難しいです.
どちらかというと一通り学んでから読んだ方がいいと思います. 今読むと「この内容もあるのか」と驚くばかりです. 
\vspace{8pt}


\textbf{落合 卓四郎, 野口 潤次郎. "幾何学的関数論" 岩波書店}   \vspace{-6pt} 

難易度:★★ $\sim$★★★  \vspace{-6pt} 

コメント: 3章のカレントの部分だけを読みました. カレントについて詳しく書いている本はこの本ぐらいなので, 3章だけでも読む価値があります. 目次を見る限りは双曲性とNevanlinna理論の話らしい. 私としてはNevanlinna理論は気になるので読んでほしいです. ただこの本を学部4年セミナーで読みたいのならば, 私ではなく山ノ井先生のところで読むべき.
\vspace{8pt}

\textbf{Kunihiko Kodaira. "Nevanlinna Theory" Springer}   \vspace{-6pt} 

難易度:★★  \vspace{-6pt} 

コメント: 小平先生がNevanlinna理論を講義した時の内容(セミナーノート)を大沢先生が英語に訳したもの. この本も気になっているので読んでほしい. 多様体と複素解析の知識があれば読めると思う. 
\vspace{8pt}

\textbf{野口 潤次郎. "多変数解析関数論 -学部生へおくる岡の連接定理-" 朝倉書店}   \vspace{-6pt} 

\textbf{野口 潤次郎. "岡理論新入門 -多変数関数論の基礎- "  裳華房}   \vspace{-6pt} 

難易度:★★ $\sim$★★★★  \vspace{-6pt} 

コメント: あまり読んでません. $\C^n$の領域, 特に正則領域・正則凸・擬凸などを調べる多変数複素解析を取り扱っている本だと思います. 手法が古典すぎるのでセミナーで読むのはお勧めしません. サブで読んだ方がいいと思います.

もちろんこのような古典理論($\C^n$の領域(正則領域・正則凸・擬凸)の理論)は$L^2$理論や岡多様体で再解釈されているので, その理論自体がダメというわけではないです. 
むしろ特異点をつけた場合では古典理論の方が良い場合もあります. 
ですが$L^2$理論や岡多様体の方が研究しやすいので, 研究者になる人に向けては後者を薦めざるを得ないです. 

逆に研究者になる予定がないのであればこの本やGrauert-Remmertなどをセミナーで読んでも構いません. むしろ私の方が詳しくないので読んで欲しいです. (議論がかなり辛くてあまり読めていないです.) 
%ただ岡多様体をメインで読むなら, サブでこれを読んでもいいと思います. $L^2$理論に進む予定なら, これではなく上の大沢先生の本をサブで読んだほうがいいと思います. 
 \vspace{8pt}

\textbf{Hans Grauert, Reinhold Remmert. "シュタイン空間論" シュプリンガー数学クラシックス}   \vspace{-6pt} 

\textbf{Hans Grauert, Reinhold Remmert. "Theory of Stein Spaces" Springer }   \vspace{-6pt} 

難易度:★★ $\sim$★★★★  \vspace{-6pt} 

コメント: Stein空間に関する本. 内容自体はかなり面白いし, 今見ると「normalもreducedもirreducibleも仮定しない解析空間でここまでわかるのか」と驚くばかりです. 今私もちょくちょく読んでます. 
ただこれも手法が古典すぎるので, セミナーで読むことをおすすめしません. サブで読んだ方がいいと思います.
%ただ岡多様体をメインで読むなら, サブでこれを読んでもいいと思います. $L^2$理論に進む予定なら, これではなく上の大沢先生の本をサブで読んだほうがいいと思います. 
\vspace{8pt}

\textbf{一松 信. "多変数解析函数論" 培風館}   \vspace{-6pt} 

難易度:★★ $\sim$★★★★  \vspace{-6pt} 

コメント: これもかなり古典の多変数複素解析を扱っています. 手法が古典すぎるのでセミナーで読むのはお勧めしません. サブで読んだ方がいいと思います.
ただ最後の付録"多変数解析函数論の小史と展望"は読むべき. ここだけでもこの本の価値はあります. 
\vspace{8pt}


\textbf{Gerd Fischer. "Complex Analytic Geometry" Springer}   \vspace{-6pt} 

難易度:★★ $\sim$★★★★  \vspace{-6pt} 

コメント: 解析空間の本. この本は研究で本当にお世話になります. 代数幾何の用語が複素解析の用語に翻訳されているから. こっちの方がスキーム論よりイメージしやすいと思います. きっちり書かれているし良い本です. 
ただこれもセミナーで読むことをおすすめしません. スキーム論を学んだ後に読めばいいから. 
 
ただこの本を読めばスキーム論を理解しなくても複素解析空間を理解できます. つまり複素数体上で代数幾何学をするにはこの本で十分です. なのでこの本で複素代数幾何学を目指すのもいいかもしれません. ただしスキーム論と違って, 解析空間での証明はかなり混みいるので, その点は覚悟した方がいいです. 途中で「スキームの方が楽では」と思うかもしれません. 
\vspace{8pt}


\textbf{猪狩 惺. "実解析入門" 岩波書店}   \vspace{-6pt} 

難易度:★ $\sim$★★  \vspace{-6pt} 

コメント: 「超関数がさっぱりわからん」人はこれを読んでほしい. 
私は中野先生の本を読んだ際にこの本を並行して読んでました. 
サブで参考書のように読んでほしい. 

\subsubsection{修士向け}

\textbf{Hans Grauert, Thomas Peternell, Reinhold Remmert. "Several Complex Variables VII -Sheaf Theoretical Methods in Complex Analysis-" Springer}   \vspace{-6pt} 

難易度:★★ $\sim$★★★★  \vspace{-6pt} 

コメント: 素晴らしい. 古典的な多変数複素解析から$L^2$理論まで1990年までの多変数複素解析を手っ取り早く学ぶならこの本になります. 今私も読んでいます. 教科書として読むより, サーベイとして読むべきです. 数ある章でも, Peternell先生が書いた部分が本当に面白いです. 
%ちなみにPeternell先生のサーベイはどれも面白いものが多い. しかも論文のネタになる問題が多いので助かる. 
\vspace{8pt}

\textbf{Takeo Ohsawa.  "$L^2$ Approaches in Several Complex Variables
Towards the Oka–Cartan Theory with Precise Bounds" Springer}   \vspace{-6pt} 

難易度:★★★ $\sim$ ★★★★   \vspace{-6pt} 

コメント: これを読めば2015年までの$L^2$理論, とりわけGuan-Zhou, BlockiのOhsawa-Takegoshiの最良評価までの$L^2$理論は網羅できると思います. これも教科書として読むより, サーベイとして読むべきです.
松村先生の書評( \url{https://www.jstage.jst.go.jp/article/sugaku/72/3/72_0723310/_article/-char/ja/} )も参照. 
\vspace{8pt}

\textbf{Sebastien Boucksom. "Singularities of plurisubharmonic functions and multiplier ideals"}   \vspace{-6pt} 

難易度:★★★ $\sim$ ★★★★   \vspace{-6pt} 

オンライン入手可能: \url{http://sebastien.boucksom.perso.math.cnrs.fr/notes/L2.pdf}\vspace{-6pt}

コメント: Boucksom先生のレクチャーノート. これはサブでいいので是非とも読んでほしい! 最後のvaluationの部分の話が本当に面白い. 
\vspace{8pt}

\textbf{Jean-Pierre Demailly. "$L^2$ estimates for the $\overline{\partial}$-operator on complex manifolds"}   \vspace{-6pt} 

難易度:★★★ $\sim$ ★★★★   \vspace{-6pt} 

オンライン入手可能: \url{https://www-fourier.ujf-grenoble.fr/~demailly/manuscripts/estimations_l2.pdf}\vspace{-6pt}

コメント: Demailly先生のレクチャーノート. これはAnalytic methodの前半(1$\sim$6)を読む際に参考にすればいいと思います. 
\vspace{8pt}

\textbf{Bo Berndtsson. "An Introduction to things $\bar\partial$."}   \vspace{-6pt} 

難易度:★★★ $\sim$ ★★★★   \vspace{-6pt} 

オンライン入手可能: \url{https://www.math.chalmers.se/~bob/7nynot.pdf}\vspace{-6pt}

コメント: Berndtsson先生のレクチャーノート. DemaillyのAnalytic methodが辛ければこれを読んでもいいと思います. ただ結局Demaillyの本に戻ってくる気もしますが...

\subsection{その他. 学部4年で卒業する人が選択できる本}
\label{subsec-seminar-23}

以下の本は学部4年で卒業し, 数学の修士にいく予定がない人のみが学部セミナー4年で読む際に選択できます. 
%基本的には「読んだら面白そうな本」か「私が読んでほしい本」を挙げている. 
この本以外で読みたい本があれば遠慮なく言ってほしいです. 

\subsubsection{代数分野}

\textbf{Michael F. Atiyah, I. G. MacDonald. "可換代数入門" 共立出版}   \vspace{-6pt} 

\textbf{Michael F. Atiyah, I. G. MacDonald. "Introduction to commutative algebra" CRC Press}   \vspace{-6pt} 

難易度:★$\sim$★★  \vspace{-6pt} 

コメント: 代数幾何に必要な可換環論を手っ取り早く学ぶならこの本. 演習問題が面白い. 
演習問題を含めて読むのであれば1年は余裕でかかると思います. 

\vspace{8pt}

\textbf{川又 雄二郎. "射影空間の幾何学" 朝倉書店}   \vspace{-6pt} 

難易度:★$\sim$★★  \vspace{-6pt} 

コメント:  Hartoshorneの1章相当の内容が書かれている印象です. 多分この本では19世紀-20世紀初頭の古典的な代数幾何学(Grothendieckのスキームが出る前の代数幾何学)の内容を扱っていると思います.
平面幾何学で有名な定理が, 射影幾何学の言葉で書かれています. 

内容はかなり面白いです. 代数幾何学の研究をするならスキームや層の抽象的議論は避けられないですが, 代数幾何学の雰囲気を味わうだけならこの本でも十分楽しめると思います. 
\vspace{8pt}

\textbf{平井 武. "線形代数と群の表現I, II" 朝倉書店}   \vspace{-6pt} 


難易度:★$\sim$★★    \vspace{-6pt} 

コメント: チラッっとみたが, 群論の初歩から表現論までわかりやすく載っています. 学部3年で代数わからなくなった人はこれで勉強してもいいと思います. コラム的な内容も面白かったです. 

\vspace{8pt}




\textbf{David A Cox, John Little, Donal O'Shea. "グレブナー基底と代数多様体入門" 原書4版. 丸善出版}   \vspace{-6pt} 

\textbf{David A Cox, John Little, Donal O'Shea. "Ideals, Varieties, and Algorithms: An Introduction to Computational Algebraic Geometry and Commutative Algebra" Springer}   \vspace{-6pt} 

難易度:★★$\sim$★★★\vspace{-6pt} 

コメント: NECのインターンに行った時に「岩井さんの専門は代数ですよね?グレブナー基底とかってご存知ですか?」
と聞かれました. 全くわからないと答えました. 
グレブナー基底は気になるが結局勉強せずにここまで来たので, ここで勉強したいです. 
%別にこの本でなくても「グレブナー基底」でも良い
\vspace{8pt}

\textbf{Reinhard Diestel. "グラフ理論" シュプリンガージャパン}   \vspace{-6pt} 

難易度:★★\vspace{-6pt} 

コメント: グラフ理論の定理は面白いが毎回主張や証明忘れてしまいます. 
\vspace{8pt}

\textbf{Jean-Pierre Serre. "数論講義" 岩波書店}   \vspace{-6pt} 

\textbf{Jean-Pierre Serre. "A Course in Arithmetic" GTM, volume 7. Springer}   \vspace{-6pt} 

難易度:★★ $\sim$★★★\vspace{-6pt} 

コメント: 気になるから読んでほしいです. 私は数論は得意ではないので教えてほしいです. 
修士に行く予定のある人は私のところではなく数論の先生のところで読んだ方が良いです. 
 %私は何も教えられなく, 「数論ってすごいんだなあ」としか言えない. 
%が, 学部4年で卒業する予定の人なら, 私の前で読んでもいいのでは?と思う. 有益なコメントは私はできないが. 
\vspace{8pt}

\textbf{井ノ口 順一 ."はじめて学ぶリー環 線型代数から始めよう" 現代数学社} \vspace{-6pt} 

\textbf{佐武 一郎. "リー環の話" 日本評論社} \vspace{-6pt} 

\textbf{James E. Humphreys ."Introduction to Lie Algebras and Representation Theory" Springer} \vspace{-6pt} 

\textbf{Jean-Pierre Serre. "Complex Semisimple Lie Algebras" Springer}   \vspace{-6pt} 

難易度:★★ $\sim$★★★\vspace{-6pt} 

コメント: リー代数は「いつか勉強しよう」と思って今に至ります. 読んでほしいです. 
難易度順に並べています. 井ノ口の本は線形代数を学んでいれば読める(らしい?)です. 佐武とHumphreysは同じくらいの難易度だと思います. Humphreysは学部4年セミナーの鉄板です. Serreは簡潔に書かれていて100ページもないです.(詳しく見てないですが, こういう本は難しい傾向にあります. ) 
\vspace{8pt}





\subsubsection{幾何分野}


\textbf{小林 昭七. "曲線と曲面の微分幾何" 裳華房}   \vspace{-6pt} 

\textbf{宮岡 礼子. "曲線と曲面の現代幾何学" 岩波書店}   \vspace{-6pt} 

\textbf{田崎 博之. "曲線・曲面の微分幾何" 共立出版}   \vspace{-6pt} 

難易度:★$\sim$★★  \vspace{-6pt} 

コメント: 初頭的な微分幾何を忘れたので, ここで思い出したいです. 
%似た系統の本としては「宮岡 曲線と曲面の現代幾何学――入門から発展へ」とか「田崎の曲線と曲面」もある.
上の三つならどれでも良いが, 小林の本しか持っていないし見たことないので, 小林の本の方が嬉しいです. 
\vspace{8pt}

\textbf{井ノ口順一. "はじめて学ぶリー群 線型代数から始めよう" 現代数学社}   \vspace{-6pt} 

難易度:★$\sim$★★  \vspace{-6pt} 

コメント: これは読んだことないですが, 線形代数を学んでいれば読める(らしい?)です. リー群の本でいい本があれば教えてください. 
\vspace{8pt}

\textbf{村上 信吾. "連続群論の基礎 " 朝倉書店}   \vspace{-6pt} 

難易度:★★  \vspace{-6pt} 

コメント: 上のように「リー群の本でいい本があれば教えてください」と書いていたら, この本がいいですよと教えてもらいました. 確かにわかりやすいです. 他にもリー群の教科書として"小林 俊行・大島 利雄 リー群と表現論 (岩波書店)"もよく挙げられます.
\vspace{8pt}

\textbf{J. Matousek. "Using the Borsuk-Ulam Theorem: Lectures on Topological Methods in Combinatorics and Geometry" Springer}   \vspace{-6pt} 

難易度:★ $\sim$★★\vspace{-6pt} 

コメント: ハムサンドイッチの定理をご存知でしょうか? ハムサンドイッチの定理とは「(実3次元空間に)与えられたハムチーズサンドイッチに対し、そのハム, チーズ, パンの量がそれぞれ半分となるようなサンドイッチの切り方が必ず存在する」という定理です. それはBorsuk-Ulam(ボルスク ウラム)の定理から示せます. 
このようにBorsuk-Ulamの定理に焦点を当てて, その応用などを取り扱ったのがこの本です. 

1$\sim$3章は学部3年くらいの内容です. 特に3章のBorsuk-Ulamの定理の応用が面白いです. 学部4年で卒業するのであれば, こういった"身近な内容の背景には実は抽象的な数学が潜んでいる"みたいな数学を学んで卒業するのも良いかと思います. 
\vspace{8pt}

\textbf{J. W. Milnor. "Topology from the differentiable viewpoint" Princeton University Press}   \vspace{-6pt} 

\textbf{J. W. Milnor. "微分トポロジー講義" 丸善出版}   \vspace{-6pt} 

難易度:★★ \vspace{-6pt} 

コメント: 地球上には必ず風が吹いていない場所があるのはご存知でしょうか? このことはPoincare-Hopf(ポアンカレ・ホップ)の定理から従います. この本ではこのような微分幾何的な物(ベクトル場)とトポロジー的な物(オイラー数)との関係を扱っています. 多様体の知識がないとこの本は読めないので, 多様体の本を前期で読んで, 後期でこの本を読んでもいいです. 
\vspace{8pt}


\textbf{河澄 響矢. "トポロジーの基礎 上・下" 東京大学出版会}   \vspace{-6pt} 

難易度:★ $\sim$★★★\vspace{-6pt} 

コメント: おそらく河澄先生が東大で学部3年(後期)向けにやっていたホモロジーの講義の内容を本にしたものです. 目次を見る限り, 上巻は学部3年の内容, 下巻が学部4年の内容だと思います. 本としての評価は良いです(松尾信一郎先生のページも参照.) トポロジーの新たな教科書になると思う. ちなみに私はその講義を受けていた.(\ref{subsec-class}節参照. )

個人的には是非とも読んで教えてほしいです. 理由はいよいよ研究でトポロジーを避けられない段階にきたから. 「カップ積・切除定理・Borel-Moore homology」が研究で出てきたし, Condenced mathの勉強会でもトポロジー関連(CW複体・Simplicial Set)の用語が出てきました. 
%そろそろ私が復習する段階だと思います.
\vspace{8pt}

\textbf{枡田 幹也. "代数的トポロジー" 朝倉書店}   \vspace{-6pt} 

難易度: ★ $\sim$★★★\vspace{-6pt} 

コメント: 学部3年の代数トポロジーの内容(+$\alpha$)をまとめた本. 上のBorsuk-Ulamの定理, Poincare-Hopfの定理, Browerの不動点定理なども載っています. かなり簡潔にまとまっています. 代数っぽく展開しているので私としてはこの方が好きです.  ですが, 幾何が好きな人には抽象的すぎて読みづらいかもしれないです. 
\vspace{8pt}


\textbf{Colin C. Adams. "結び目の数学―結び目理論への初等的入門" 丸善出版}   \vspace{-6pt} 

\textbf{Colin C. Adams. "The Knot Book: An Elementary Introduction to the Mathematical Theory of Knots" American Mathematical Society}   \vspace{-6pt} 

難易度:★★ $\sim$★★★\vspace{-6pt} 

コメント: 全然専門ではないです. ちょっと見て「うわー図が綺麗だなー」と思った. 修士に行く予定のある人は私のところではなく鎌田先生のところで読むべき. 
%私は何も教えられなく, 「結び目の図が綺麗ですねー」としか言えない. 
\vspace{8pt}

\textbf{加須栄 篤. "リーマン幾何学" 培風館}   \vspace{-6pt} 

難易度:★★ $\sim$★★★\vspace{-6pt} 

コメント: リーマン幾何学も「いつか勉強しよう」と思って今に至ります. なので読んでほしいです. 
なお修士に行く予定のある人は私のところではなく太田慎一先生のところで読むべき.
%である. 私は何も教えられなく, 「計算大変ですねー」としか言えない. 


\subsubsection{その他}

\textbf{Martin Aigner, G\"unter M. Ziegler. "天書の証明" 丸善出版}   \vspace{-6pt} 

\textbf{Martin Aigner, G\"unter M. Ziegler. "Proofs from THE BOOK" Springer}   \vspace{-6pt} 

難易度:★$\sim$★★★  \vspace{-6pt} 

コメント: エルデシュの綺麗な証明などを集めた本. 読むと頭が偉くなった気がします. 
2年までの内容$+\alpha$を知っていれば読めると思います.  
\vspace{8pt}

\textbf{池 祐一, E.G. エスカラ, 大林 一平, 鍛冶静雄. "位相的データ解析から構造発見へ パーシステントホモロジーを中心に" サイエンス社}   \vspace{-6pt} 

難易度:★$\sim$★★★\vspace{-6pt} 

コメント: トポロジカルデータアナリシス(TDA)が気になっています. どの本にしようか悩みましたが, 「数学の研究者である池先生の書いた本なら, 数学的に間違いはないだろう」と思うのでこの本を推薦します. 数学的な定義はなんとなくわかったが, 「どうやってアルゴリズムを作るのだろう?」という部分が気になります. 読んでほしいです. 

なお池先生が2025年に阪大に集中講義してくれました. 池先生のホームページに集中講義のノートがあります. (\url{https://sites.google.com/view/yuichi-ike/teaching}) これを参考にしてもいいと思います. 
\vspace{8pt}

\textbf{S. マックレーン "圏論の基礎" 丸善出版}   \vspace{-6pt} 

\textbf{T. レンスター "ベーシック圏論" 丸善出版}   \vspace{-6pt} 

難易度:★$\sim$★★★ \vspace{-6pt} 

コメント: 圏論は知らないうちに仮定されていることが多いです. 情報系に進む予定のある人は学部4年セミナーで読んでも良いです.  そしてその後にHaskellにハマってください(私の同期が喜ぶので). 数学科に進む予定で, 基礎論分野やトポス理論など高度に抽象的な数学をしたい場合は学部4年セミナーで読んでも良いです. ただ日本だとこの分野は情報系に含まれることも多いので, 修士に行く場合は情報系や海外に行くことも視野に入れた方がいいです. 

教科書に関して, マックレーンの方が詳しく, レンスターの方はちょっと簡単めです. セミナーで読むならどちらでも良いです. ちなみにレンスターの方を購入すれば東京科学大学の土岡先生が喜びます. 
\vspace{8pt}

\textbf{J.A. ブーフマン "暗号理論入門" 丸善出版}   \vspace{-6pt} 

難易度:★ $\sim$★★\vspace{-6pt} 

コメント: 「暗号について知りたい!」と思ってはや10年. いまだに勉強してません. 気になります. 
\vspace{8pt}


\textbf{C.M. ビショップ. "パターン認識と機械学習 上・下" 丸善出版}   \vspace{-6pt} 

難易度:★ $\sim$★★\vspace{-6pt} 

コメント: 機械学習を習得したい人が読んでいる気がします. でもこの本が出てもう20年たってるのになぜいまだに読まれているのだろう? 気になるので読んでほしいです. 
\vspace{8pt}

\textbf{舟木 直久. "確率論" 朝倉書店}   \vspace{-6pt} 

難易度:★★\vspace{-6pt} 

コメント: この本を一気に読んだことがあります. めちゃくちゃ面白かったです. 今になって内容を全て忘れてしまいました. なのでもう一度思い出したいです. 
\vspace{8pt}

\textbf{R. デュレット. "確率過程の基礎" 丸善出版}   \vspace{-6pt} 

難易度:★★ $\sim$★★★\vspace{-6pt} 

コメント: これも確率論が気になるから読んでほしい. 
\vspace{8pt}

\textbf{L. Barreira. "Ergodic Theory, Hyperbolic Dynamics and Dimension Theory." Springer} \vspace{-6pt} 

\textbf{M. Einsiedler and T. Ward. " Ergodic Theory." Springer }   \vspace{-6pt} 

難易度:★★ $\sim$★★★★\vspace{-6pt} 

コメント: 平良先生のルベーグ積分の講義ノート\url{https://sites.google.com/view/the-home-page-of-kouichi-taira/teaching}を見ていた時のこと. 
そこに「$n \in \mathbb{N}$について$l_n$を$2^{n}$の最高位の数と定める. (例えば$2^9=512$なので$l_9=5$となる.) $l_n=1$となる$n$はどの頻度で現れるか?」みたいなことが書いてました.
これはエルゴード理論でわかるらしく, 上の本が引用されてました. 
Barreiraの本のExample 3.4に答えがあるとのことです. 
多分Barreiraの本は修士の時に平良先生とか含めたセミナーで読んだ気がします. でも結局よくわかりませんでした. なんか力学系の部分が特に. 

ちなみにとある動画でルベーグ積分の構成を見た時「あれ?私が学部の時に習ったこと($\pi-\lambda$定理やカラテオドリの定理)などがないな」と思って, 上の平良先生のルベーグ積分の講義ノートを見たら全て解決しました. いやー同期や先輩後輩が偉い立場になって講義ノートのpdfを作ってくれて助かりますね. 

\vspace{8pt}

%\textbf{Barreira}   \vspace{-6pt} 難易度:★★\vspace{-6pt} コメント: 修士の自主ゼミでに読んだことがある. 数論と関係ある部分の話がかなり面白い. 

\subsection{セミナーの仕方について}
\label{subsec-seminar}
\subsubsection{河東先生のセミナーの準備の仕方について}
まずは河東先生の有名なページ「セミナーの準備の仕方について \url{https://www.ms.u-tokyo.ac.jp/~yasuyuki/sem.htm}」を読んでほしいです.

読んだ上で私はこう言いたいです. 
\begin{tcolorbox}[mybox]
\begin{center}
これは無理だ. 難しすぎる!
\end{center}
\end{tcolorbox}
河東先生の言っていることは全て正しいです. 
が, これを実行するのは本当に難しいです. 

まず初めの段落に関しては実行できると思います.むしろこれは実行してほしいです. 
大体の教科書は行間・ギャップがあるので, 準備の段階でそれを埋める必要があります. 
特に河東先生の発言
\begin{tcolorbox}[mybox]
「完全に正しいと断言できる」ということと「自分にはわかっていない」ということの違いが自分ではっきりとつけられるようにならなくては何も始まりません.
\end{tcolorbox}
は本当にその通りだと思います.

次の段落「さてそうして, ...」からは本当に難しい. この読み方を常習化できれば研究者になれると思います. 
なぜなら研究では「今現在知られている結果の仮定を弱めていく」からです. 
続いて3段落目「さらにそれができるようになったとしましょう...」は私には全てはできません. 「証明のポイントはどこか」は良くしますが, 「証明を正しく再現」はできません. 私がやったら「まあこれからこんな感じでできます」と雰囲気しか伝えられません.  
%私はできない. 頭の中でこんがらがんになってしまうからである. TeXを使わないとできない. 

最後の段落「これでもまだ準備は終わりではありません.セミナーの時間配分も考える必要があります.」は私含め全研究者へのアドバイスです. 
特に河東先生の
\begin{tcolorbox}[mybox]
日本では, 教員でもそういうこと(セミナーの時間配分)のできない人が少なからずいますが, プロとして恥ずかしいことです.そういうのをまねしてはいけません.
\end{tcolorbox}
に関しては, 「本当に申し訳ない」としか言えません. 
なお, セミナーの時間配分ができていない研究者などいくらでもいます. 
%研究者の講演はひどいもんである. (もちろん良い講演もあるが. )
講演で時間配分ができない人はよくいますし, スライド連打する人もよくいます. 
「これ聞かせる気があるんか」と思うこともしばしばです. (これは私へ向けての特大ブーメランかも?)

なのでできる人はこの方法でやっても良いが, 努力目標だと思います.
私も学生の時に一回この方法を試したことがありますが, 結局丸暗記する行為になってしまい, このアドバイス通りのことができませんでした. 
河東先生も初めに言っている通り, 
\begin{tcolorbox}[mybox]
セミナーの準備のしかたは個人ごとに自分にあったやり方でやればいいので,別に特定のやり方を押し付けるつもりはありませんが,一つの例としてやり方を説明します.
\end{tcolorbox}
とあるので, 自分に適したやり方を見つけてやってほしいです. なお河東先生自身も"「セミナーの準備のしかたについて」について\url{https://www.ms.u-tokyo.ac.jp/~yasuyuki/sem2.htm}"と言う文章を書いてます. これも面白いのでぜひ読んでほしいです. 

%%%%%%%%%%%%%%%%
\begin{comment}
最後になぜこんなことを言うかといいますと, 無責任に「河東先生のセミナーの準備の仕方について」のページを教える人が多いと思ったからです.
この方法を教える人には
\begin{tcolorbox}[mybox]
\begin{center}
そう言うあなたは講演や授業で, 河東先生のように準備ができていますか?
\end{center}
\end{tcolorbox}
と問いたいです. 

河東先生はできるんです. YouTubeで河東先生の授業を見たが本当にできています. すごいとしか言えません. 
私には到底できないので, 上の方法(河東先生のセミナーの準備の方法)を押し付ける気は全くないです. 
\end{comment}
%%%%%%%%%%%%%%%%%%%%%%%
\subsubsection{セミナーの仕方について}

要するに, セミナーは
\begin{tcolorbox}[mybox]
\begin{center}
学生が教官に向けて授業をする.
\end{center}
\end{tcolorbox}
と思ったら良いです.  
教官は学生のように質問するので, それに答えられるように準備しないといけないです(これも私へ向けての特大ブーメランです.)
最近の教官は教科書の内容をまとめて授業をするので, 皆さんも教科書の内容をまとめてセミナーをしてほしいです. 

またメモは見ても見なくてもいいです. 私は授業でも講演でもメモを見ています. 英語での講演の時は, 話すべき英語もメモに書いています. 
理由は黒板の前に立つと話すべき内容が頭から飛んでしまうからです. メモを見て話す内容を思い出しています. 

\subsection{本を読むために必要な学部3年までの前提知識}
\label{subsec-class}

\subsubsection{学部4年で卒業する人}
まず学部4年のセミナーにおいて, 以下の内容は必ず理解しないといけません. 
%(なお右に書いているものは阪大での講義名である.)
\vspace{-8pt}
\begin{itemize}[left=0pt]
  \setlength{\parskip}{0cm} % 段落間
  \setlength{\itemsep}{0cm} % 項目間
  \item 微積分
  \item 線形代数
 \item 集合と位相
\end{itemize}
とりあえず上の単位は取ってほしいです. (まあどれも必修なので, 取らないといけない.)

なお上の内容を全部理解している必要はないです.
私も「これなんだったっけ?」となることはよくあります. が, 定義とか定理を見直して, 「あ, これは昔やったことあるぞ. こうやればいける.」と思い出せるくらいにはなってほしいです. 
特に集合と位相は必須です. これができないと教科書は読めないと思っていいです. 

学部で卒業する予定の人はこれで十分です. 
よってこの後の内容は修士に行く予定のある人向けです. 

\subsubsection{修士に代数幾何学を目指す人}
必要だと思うのは以下の内容.
\vspace{-8pt}
\begin{itemize}[left=0pt]
  \setlength{\parskip}{0cm} % 段落間
  \setlength{\itemsep}{5pt} % 項目間
  \item 多様体. 代数"多様体"を扱うから. 
  \item 群・環・体. 下の「環と加群, ガロア理論」を学ぶために必要.
  \item 環と加群. スキームで使うから. 
  \item ガロア理論. 代数的に被覆を考えるなら必要.
  \end{itemize}
授業を取っておくとためになるのは以下の内容. 
\vspace{-8pt}
\begin{itemize}[left=0pt]
  \setlength{\parskip}{0cm} % 段落間
  \setlength{\itemsep}{5pt} % 項目間
  \item ホモロジー・基本群. \\ ホモロジー代数を勉強することになり, その具体例として学べるから. 
    \item 微分形式・ ドラーム コホモロジー. \\ 上に同じ. あとケーラー微分で微分形式っぽいことをやるので. 
  \item 複素関数論続論. \\ ペー関数の内容が数論, 代数幾何に関わるので. 小木曽先生の代数曲線論を並行して読むと面白みが増す. ただ上の二つよりは優先度は低いと思います. 
  \end{itemize}
  
これにプラスして学部4年に入るまでに"可換環論(Atiyah-Macdonaldの内容)"くらいは学んでおくことになります. 

\subsubsection{修士に複素幾何学・多変数複素解析を目指す人}

必要だと思うのは以下の内容.
\vspace{-8pt}
\begin{itemize}[left=0pt]
  \setlength{\parskip}{0cm} % 段落間
  \setlength{\itemsep}{5pt} % 項目間
  \item 多様体. 複素"多様体"を扱うから. 
   \item 微分形式・ ドラーム コホモロジー. \\ この複素版がドルボー コホモロジーであるから.
  \item 複素関数論続論. \\ 複素幾何の初めの方は多変数複素関数論の話で, 1変数の内容よく使うから.
  \end{itemize}
取っておくとためになるのは以下の内容. 
\vspace{-8pt}
\begin{itemize}[left=0pt]
  \setlength{\parskip}{0cm} % 段落間
  \setlength{\itemsep}{5pt} % 項目間
  \item ホモロジー・基本群.  \\ 取った方がいいと思うが, マストではないです. 岡多様体をやるなら必要かも. 
  \item 代数形の科目(群・環・体, 環と加群, ガロア理論). \\ 代数っぽい複素幾何をやるなら. ガロア理論はホモロジー・基本群を学ぶなら不要かも. 幾何学的に被覆を考えればいいから. %有限被覆取る時しかガロア理論を使わないので. 
 \item 解析形の科目(フーリエ変換, 偏微分方程式). \\  幾何解析するなら偏微分方程式を解くことになるので必要かも. なお多変数複素解析の$L^2$理論は偏微分方程式っぽい感じで方程式を解かないのでどちらでも良いです. 
  \item ルベーク積分.  \\ 多変数複素解析の$L^2$理論を学ぶなら. ルベーグの優収束定理をよく使うので. 幾何解析をする場合にも必要かも. 
    \end{itemize}


\subsubsection{岩井の場合 (昔話)}
2年次までの科目(微積分・線形代数・集合と位相・複素解析)は阪大と同じです. 
3年次以降の内容で私が受けたのは以下の科目です. 
\vspace{-8pt}
\begin{itemize}[left=0pt]
  \setlength{\parskip}{0cm} % 段落間
  \setlength{\itemsep}{0cm} 
 \item 多様体. (3年次春夏必修)
 \item 群・環・体. (3年次春夏必修)
 \item ルベーク積分. (3年次春夏必修)
 \item 複素関数論続論. (3年次春夏必修)
 \item ホモロジー・基本群. (3年次秋冬選択)
 \item 微分形式・ドラーム コホモロジー. (3年次秋冬選択)
 \item 環と加群. (3年次秋冬選択)
 \item ガロア理論. (3年次秋冬選択)
  \end{itemize}
  
Atiyah-Macdonaldの可換代数入門は演習問題も含めて読んでいました. 結局演習問題は解き終わりませんでしたが...
  %東大では「多様体, 群・環・体, ルベーグ積分, 複素関数論続論」が必修である. これはかなりきつかった. 

学部4年のセミナーでは代数幾何学の本(飯高先生の本)を読ました. 小平先生を崇拝していたので, 複素幾何はサブで勉強してました.  Wells先生か小林先生の本をちょくちょく読んだ気がします. なおホッジ・ドラーム・小平分解の解析的な証明は興味がなかったので飛ばました(小林先生の本でも飛ばしてたと思います).

院試が終わってから, なんやかんやがあって, 高山先生のところに行くことにしました. 高山先生から「修士までに多変数複素解析の本を読んでこい」と言われ, 慌てて中野先生の本を読みました. 
その際に超関数・カレントがわからなかったので, 猪狩先生の本を読みました. 解析は独学になります.  

\subsubsection{思い出話} 
以下は受けた授業の思い出話. 

\textbf{複素関数論続論 }   \vspace{-6pt} 

全く聞いてなかったです. 当時は解析に興味がなかったから.
授業は劣調和函数から始まりましたが, 当時は「こんな関数考えて何になるんだ」と思ってました.
なので授業の板書をノートに書くだけで, 手元ではその日にあった自主ゼミの勉強をしていました. 
なおこの授業の担当は高山先生でした.  

そんな奴が修士で高山研に行き, 多重劣調和函数を使いまくることになるとは思わなかったです. みんなも授業はきちんと聞きましょう.
\vspace{8pt}  

\textbf{群・環・体}   \vspace{-6pt} 

志甫先生の担当でした. 演習問題が全部で160問. 試験はその中から8問. 
「160問もある演習問題なんか誰が全部解くんや」と思ってたら, 同期の森脇湧登先生が全部解いてました. そのノートを共有してくれたので, それを見ながら勉強しました. 
\vspace{8pt} 

\textbf{ホモロジー・基本群 }   \vspace{-6pt} 

河澄先生の担当でした. この授業の内容をまとめた本が「トポロジーの基礎」だとと思います. 
最初は特異ホモロジー\footnote{東大はなぜか「特異ホモロジー」でホモロジーを定義します. 東大以外はホモロジーにおいては, 「単体的ホモロジー」を教えるはずです. (阪大もそう). 私はトポロジーの専門ではないので, 「特異ホモロジーの方が代数的に定義できるし, 計算するだけなら完全系列追えばいいだけやから楽やん」と思うが, どうもそれは少数派らしい. 単体的ホモロジーの方がわかりやすいのですかね?. }を定義して, その後基本群やホモトピー, セールファイブレーションをやってたはずです. そこまでは追えました. この辺りは完全系列が計算できればよく, 代数の話なので. 

ただ途中からはCW複体, 特性類, 分類空間の話をしていたはずです. 難しすぎてさっぱりわかりませんでした.
この辺りの内容は本当に覚えてません. 
ためになったことといえば「分類空間は本地垂迹説のようなもので...」と言っていたことです.
本地垂迹説とは「仏こそが神の真の姿"本地"であり, 日本の神々は人々を救済するために仮に神という姿で日本で現れた"垂迹"である」という神仏習合が行われた平安時代に流行った説です. これは神社検定3級の試験に出てきました. 
分類空間の話は分かりませんでしたが, のちの神社検定3級の問題には答えることができました. 

%今になってこの河澄先生の授業はなかなかカオスであった. 
演義もぶっ飛んでました.
「他大学の聴講生が発表する」こともあれば, 「河澄先生が発表する」こともありました. また演義の問題に激ムズ問題もありました. 例えばHirzebruch曲面のホモロジーを計算する問題です. 河澄先生が「小平スペンサーの変形理論使えば簡単に解けますが, その際は変形理論を説明してください」と言っていました. 当時はわからなかったが, 後になってその意味がわかりました. 
%(mod 2で変わる). 
\vspace{8pt} 

\textbf{環と加群}   \vspace{-6pt} 

川又先生の担当でした. 神の授業でした. 
初めの2, 3回くらいの授業で, シラバスの内容の8割が終わりました. 
%まず1回目の授業で非可換環が出てきた. その後2, 3回くらいの授業で, シラバスの内容の9割が終わった. 
%証明が本当に簡潔で, 「あとはこれでわかりますね」という感じであった. 
そのあとは圏論の話になり, 森田同値の話をしていたはずです. この辺りから全くわからなくなりました. 
10回目くらいでシラバスに戻ってきて, シラバスの内容の最後の2割をやりました. そのあとは休講でした. 

あまりにわからなかったので先輩に「川又先生の授業がわからないです」と言って内容を言いました. するとその先輩から「難易度が下がったね. 昔はもっと難しかったよ」という返答が返ってきました. 聞けば, ガロア理論で無限次元ガロア理論やKawamata coveringをしていたらしいです. 2年の複素解析でリーマン・ロッホをやったという話も聞きました.

演義も大変で, 演習問題には当時TAだった中村先生とChen Jiang先生の作った問題がありました. 中村先生の作った問題は, 非常に難しくて解けない. 一方Chen Jiang先生の作った問題は, 教育的でいい問題ばかりだが, 英語で解答しないといけない. 「え, 我々詰んでないか?」と思いました. 

当時は大変だったが, すぎてしまえばいい思い出です. Kawamata log terminalで有名な川又先生の授業を受けられたのは貴重な経験だったと思います. 宮岡先生の授業も受けたかったなあと思います. 
\vspace{8pt} 

\textbf{その他 (後日談)}   \vspace{-6pt} 

このような演義を受けていたので, 「なるほど, 数学科の演義は私も解けないような激ムズ問題をいれればいいのか」と思い, 阪大着任1年目の位相の演義において, 激ムズ問題を入れまくりました. 多くの学生からは不評でした. (ごくごく一部の生徒からのみ好評でしたが.) 2年目からは反省して簡単な問題を入れることにしました. 

%\section{修士に入ったらやること・お役立ち情報}
\label{sec-3}
以下の内容は修士の学生向けです. 
私が修士で高山研に入った時に「修士に入ったらやること」みたいな紙を渡された気がします. 
そのため私は「修士に入ったら以下のことは教員が教えるもの」と思い込んでました.  
が, どうもそれは教員によってまちまちらしいです. 

以下の内容は, 修士に入った時点でやった方がいいと個人的に思うことです. 
 私の研究室に来ない場合でも数学の修士に入った人は参考にしてほしいです. 

\subsection{修士に入ったらやること}

\textbf{ArXivを毎日チェックする}   \vspace{-6pt} 

Arxiv \url{https://arxiv.org/archive/math} とは論文(プレプリント)サーバーです. 
数学分野においては, 論文(プレプリント)が書けた後はまずArxivに投稿して, その後にジャーナルに投稿します. これは数学独自の文化と言ってもいいです.
なのでArxivをチェックすれば自ずと最近の研究情報を入手することができます. 

ArXivは平日毎日チェックすることをおすすめします. なぜなら毎日見ないと論文のReplacementの情報を得られないからです. 私が登録しているのはAG(Algebraic Geometry), CV(Complex Variables), DG(Differential Geometry)です. 

なおメールでも送ってくれる機能もあります. 
Subscribe \url{https://info.arxiv.org/help/subscribe.html} のページを頑張って翻訳して登録してほしいです(面倒ならこの画面をスクショしてChat GPTに「翻訳して」と聞けばやってくれます). 
Xを使っている人はXの非公式アカウントがあるので, それをフォローすれば毎日ポストしてます. 

修士になりたてのうちは, 論文のタイトルとアブストラクトで興味あるもの見つけた, イントロ"だけ"を読むのがいいです. 修士の初めは「主定理を証明を含めて全て理解しよう」とは思わない方がいいです. 
修士1年後半$\sim$2年あたりから論文を読むことになります. 
ArXivで見つけた論文を, 証明を含めて読むことになります. なお論文には当たり外れがあるので, ガチャだと思った方がいいです. 

私はArXivは"YouTube"だと思っていて, 論文のタイトル・アブストは"サムネイル"だと思ってます. この思い込みは有用です. 論文をArXivに投稿する際は, 「いかにタイトル・アブストで興味を惹きつけられるか」と言うことにこだわった方がいいです.(まあサムネ詐欺と言われないくらいの結果を出すべきですが.)
そして論文のイントロも何回も推敲すべきです. ほとんどの人はイントロしか読まないからです. 
\vspace{8pt} 

\textbf{分野のメーリスに入る}   \vspace{-6pt} 

ほとんどの分野でメーリス(メーリングリスト)があるはずです. 
絶対に登録するべきです. 集会の情報や公募の情報が流れてくるので. 
私が入っているメーリスは以下の三つです.

\begin{itemize}[left=0pt]
  \setlength{\parskip}{0cm} % 段落間
  \setlength{\itemsep}{0cm} % 項目間
\item 代数幾何 \url{https://mail.math.nagoya-u.ac.jp/mailman/listinfo/agmail}
\item 幾何学 \url{https://www.math.nagoya-u.ac.jp/~naito/geometry/mailing-list.html}
\item 函数論 \url{https://www.cajpn.org/complex/ml.html}
\end{itemize}

もし数学を続けていくのであれば, メーリスで登録するメールアドレスは大学のものを使わ"ない"方がいいです. 
この段階で数学用gmail(私ならmasataka.math[at]gmail.com)を作って, そちらでメーリスなどを登録した方がいいです. (これはArXivでも同様.)
理由としては大学を異動したらその大学のメールアドレスは使えなくなるからです. 
あとこの数学用gmailは後々のことを考えて"実名"の方が望ましい. あと決してふざけることのないように.
\vspace{8pt} 

\textbf{数学分野の交友関係を広げる}   \vspace{-6pt} 

東大など数学の院生が多い大学ならこのことは気にしなくてもいいですが, 東大以外の院生は博士に進む人が少ないので外部の交友関係を広げる必要が出てきます. \footnote{私が東大にいたときは荒野先生のおかげで, 自ら交友関係を広げなくても, 分野に関わらずに話す相手が常にいました. 今思えばこれはかなり運がよかったと思います.} 

手っ取り早いのは当該分野で若い人が集まる集会に参加すればいいと思います. 
私の分野であれば以下の集会に行けば良いと思います. 
\begin{itemize}[left=0pt]
  \setlength{\parskip}{0cm} % 段落間
  \setlength{\itemsep}{0cm} % 項目間
\item 葉山シンポジウム(7月). 若手に関わらず多くの函数論の人が多く集まる.
\item 専門家向けケーラー勉強会(8月). 参加したことがないが, 講演者を見ている限り若い人が多い気がする. 修士の学生が参加して良いかは不明. 
\item 函数論サマーセミナー(9月). 1変数・多変数関わらず複素の学生が多い印象. 修士の人が勉強したことを話して良い.
\item 金沢複素幾何シンポジウム (11月). ちょっと年配の人が多い印象. 
\item 多変数関数論冬セミナー (12月). 世話人が若手に代替わりしたので, 若手が参加しやすいと思う.
\item 数理新人セミナー(2月). 私は参加したことがない. ただ当時の同期の雰囲気を見る限り参加していいと思う. 
\item 代数学若手研究会 (3月). 私は一回しか参加してない. 講演した思い出がある. 
\item 日本数学会(3月・9月). これは自分の結果が出てから. 講演するなら函数論で講演するのがおすすめです. 幾何と代数は人が多すぎる.  
\end{itemize}
問題はお金である. これは指導教官が科研費をどれくらい持っているかによります. 
全部は行くのは無理だし, 指導教官の科研費の補助でいけるとして, 多くて1つか2つだと思います. 
確か葉山とサマーセミナーは旅費の補助があった気がします.
そのあたりはメーリスで情報を仕入れる必要があります. 

博士に行く予定があるなら, 早い目にその分野のコミュニティーに入って友達を作った方がいいと思います. 
私に近い分野だと
\begin{itemize}
  \setlength{\parskip}{0cm} % 段落間
  \setlength{\itemsep}{0cm} % 項目間
\item 松村先生・足立先生・小池先生・日下部先生・稲山先生・渡邉先生がいる函数論のグループ. 多変数関数論由来のグループだが, 代数幾何学から複素幾何・幾何解析など実際の分野はかなり幅広いと思う. 
\item 久本先生・橋本先生・井上先生などがいる日本の複素微分幾何勢のグループ. 専門家向けケーラー勉強会など活発に活動している印象. 
%\item 高山先生および元高山研のメンバー(松村先生・久本先生・小池先生・岩井・稲山先生・井上先生・渡邉先生)
\end{itemize}
にいる人たちに顔を覚えてもらえればいいかと思います.
もちろん両方入ってもいいと思います. 
%高山先生および元高山研のメンバー(松村先生・久本先生・小池先生・岩井・稲山先生・井上先生・渡邉先生)
%足立先生・小池先生・日下部先生がいる函数論のグループ」 や 「久本先生・橋本先生・井上先生などがいる日本の複素微分幾何勢のグループ」に顔を覚えてもらえればいいかと思います.
%\footnote{ちなみに私は「函数論のグループに入っている(と思っている)が, 家出しまくっている」と言う状況である. これは研究者として自立する必要があり, 大変苦労する(した)のでおすすめしない. まあ協調性がなく群れて行動するのが苦手なだけなんですが.}
\vspace{8pt} 

\subsection{修士に入ったら気をつけること}

\textbf{メールの返信をする}   \vspace{-6pt} 

最近の学生はメールの返信が丁寧で早いと思います. 本当に素晴らしいです. 
基本的にメールの返信は翌日(翌日が土曜の場合は, 次の月曜)までに返信するべきです. 遅くても2, 3日後までには返信した方がいいです. 

ちなみに私が学生の時はどうだったのかというと, 全然守れてなかったです. 博士を出た後にメールの返信を即座に返すように心がけました. 
%時たま「数学者は数学以外は適当だ」と思われるが, それは昔の話と思ってい 

%じゃあ私が学生の時はどうだったのかというと, 全然守れてなかったです. 1週間後に返信することもよくありました.  が, 博士を出た後にメールの返信を即座に返すように心がけました. 色々理由はありますが, 一番は「めちゃくちゃ忙しい藤野先生・松村先生がメールを即座に返すのに, めちゃくちゃ暇な岩井のメールの返信が遅いのはよくないだろう」と思ったからです「メールの返信が遅くて許されるのは偉い人だけ」と思った方がいいです. 
\vspace{8pt} 

\textbf{締切を守る}   \vspace{-6pt} 

これもルーズな人が多い印象. 例えば学振や修論に関して締め切り前に慌てる人が多いと思います. 
厳しいことを言うと, 学振・修論に関しては締め切り前に慌てている時点でダメです. 
二つともは何回も何回も推敲するべきものであるからです.  
%なので本来の締め切りの1週間前に仕上がっている必要がある. 

%一番簡単な方法は本来の締切よりも前に"自分の締切"を設けることである. 具体的には本来の締切の1$\sim$2週間前に"自分の締切"を設ければ良い. 結局のところほとんど多くの人が締切直前にしか作業をしない. 締め切りがいつでもいいのなら, "自分の締切"前に作業すればいいのである. 
一番簡単な方法は本来の締切の1$\sim$2週間前に"自分の締切"を設ける方法. 
結局のところほとんど多くの人が締切直前にしか作業をしません. 締め切りがいつでもいいのなら, "自分の締切"前に作業すればいいと思います. 

%なおこれは私が研究集会の講演者に講演タイトルを聞く際にも同様のことをやっています. 講演タイトルが揃ってほしい日の1週間前に締切を設けて, その締切までに講演者に講演タイトルを返信するようにお願いします.
%ほとんどの人は締切前までに返信しますが, 締切を過ぎても返信しない人も多いです. その人には個別に連絡するとすぐに返信が返ってきます. 結果として, 講演タイトルが揃ってほしい日までには講演者から返信が返ってくるので, 何の問題もないです. 
\vspace{8pt} 

\textbf{休日は自由である}   \vspace{-6pt} 

佐藤幹夫先生の有名な言葉に
\begin{tcolorbox}[mybox]
朝起きた時に、きょうも一日数学をやるぞと思っているようでは、とてもものにならない。数学を考えながら、いつのまにか眠り、朝、目が覚めたときは既に数学の世界に入っていなければならない。
\end{tcolorbox}
 があります. 私はこれに反対です. 休日は自由です. 数学をしてもいいし, せずに休んでもいいです. 

理由としては「普通の人がこれをやっても長く続かない」から. 
数学の道に進むなら今後30年は数学をすることになります. なので, 数学を長く続けられる生活リズムを作っていくことになります. 
休日含め毎日数学をする生活は修士までならできると思うが, 博士やそれ以後も続けられるかは微妙です. 
無論, 長く続けられる人は毎日数学をやっていただいて構わないです. 実際そのような人はいます.
が, できない人は休む日を設けた方がいいと思います. 休んだ方が返って効率が良いこともあります. 

なお「休日は自由である」は教員に関しても適応されます. 修論を年末に私に送りつけ「修論の添削をしてください」と学生が言ってきても, 私はガン無視する予定です. 年末年始はゆっくりしたいから. 学振に関しても, 学生がゴールデンウィーク前に「添削してください」と言ってきても, ガン無視する予定です. ゴールデンウィークは遊びたいから. 学生が「修論の締切直前なんです」みたいなことを言っても, ガン無視する予定です. その場合は潔く留年していただきます. これは上の締切を守ると言うことにもつながります. 

なぜこんなことを言っているのかと言いますと, このような愚痴は本当によく聞くからです. 「教員も人間であり休日は休む」と言うことを頭の隅でもいいので覚えていてほしです.  

なお私はよく休日にメールなどを返信しますが, これは「メールをその時に返信しないと忘れてしまう」から.
「これを送っても返事はすぐ来ないだろう」と思っています. なので私からのメールが来ても休日であれば返信しなくて良いです. 次の週の月曜に返信していただければと思います. 
\vspace{8pt} 

\textbf{英語に慣れる}   \vspace{-6pt} 

このガイドはChat GPTに原本を書いてもらいました.
その際にこの項目が出てきました. 
Chat GPT曰く「論文・教科書・プレゼンなど、あらゆる場面で英語が必要になる。少しずつ読んでいくことが大切。」とのこと. 
これはその通りです. 数学の論文は基本的に英語で書かれているので英語を避けて数学を研究するのは無理です. 特にこの分野は「フランス・ドイツ・中国」あたりの人と交流することが多いので, 英語は避けられないです. 

基本的に"読む"のと"書く"のは中学校の英語ができていればできる(はず). 数学英語はビジネス英語よりはるかに簡単です. 
問題は"話す"と"聞く". これは私も全く自信がないです. なので英会話を上手くする方法は英語の上手い先生に教わってください. 
私はいまだに会話してて「何を言ってるんだろう」となることがよくあります. でも英語で会話せざるを得ないです. 
その際のコツを何点かお伝えします. 
\begin{itemize}[left=0pt]
  \setlength{\parskip}{0cm} % 段落間
  \setlength{\itemsep}{8pt} % 項目間
\item \underline{iPadなど書くものを用意する.} 一番簡単な方法. 会話で伝わらないなら, 書けば良いです. 数式や質問内容を書いて, "これどうなるんや"と聞けばいいです. 
数学者に質問したいならこの方法が一番楽です. 返答の意味がわからない時も, 相手にiPadを渡して書いてもらったらいいです. 講演後の質問なら黒板を使えばいいです. 
\item \underline{ジェスチャーを入れる.}  私はいつも"He told me that (主語)..."の部分で(主語)に"私"を意味することを言いたい時に, "I"を使えばいいのか"You"を使えばいいのかを悩みます. 
これを解決する方法として, (主語)の部分を言ったときに自分を指差します. 間違っていても「こいつは文法間違っているけど, 彼がこいつに言ったんやな」となりこれで通じます. ジェスチャーを入れると, 英語が下手でも通じることが多いと思います. 
\item \underline{相手に自分の英語が通じてなさそうな場合は, 黙るのではなく, 言い換えて伝える.} 自分の英語が通じていなくて相手が???となっている状況がよくあります. この状況を打破するには, 「つまり...」と言い換えて喋りまくることです. 実はこの状況は, 自分の英語が悪いのではなく相手側が単に理解してないことが原因でおこることもあります.\footnote{ごく最近このような状況に遭遇しました. 中国の研究者と日常会話していて, 彼が「〇〇さん(日本人研究者)のお子さんが生まれた時に, あなた(岩井)は招待されましたか?」と言ってきました. 私には意味がわかりませんでした. 彼は"英語が通じなかったのか?"という感じでいましたが, 日本人の私としては"何で子供が生まれたら私を呼ぶんだ?"と思いました. よくよく話をしてみると, 中国では子供が生まれてしばらくした後に, 仲のいい人たちをよんでパーティーをするらしいです. } 
とりあえずこっち側が言いまくればいいです. 有限回でいずれ伝わります. 伝わらなければ, 相手側に「どこがわからんのや?」と言えばいいです. もしくは日常会話のようなどうでもいい話なら, この話を早々に切り上げて別の話をしてもいいです. 
\item \underline{スマホを使う.} これは日常会話に使えます. 日本文化など会話で伝わりづらいものは画像を見せれば案外伝わります.  
\end{itemize}

あとChat GPTに課金しているならば, 英語の読み書きをChat GPTに任せるのも良いです. 多くの日本人よりもChat GPTの方が英語は上手いです. 私は英語でメールをするときに, Chat GPTに原案を書かせてそれを修正したものを使っています. Chat GPTの英語がうますぎて不自然なので, あえて下手にしています. 
 \vspace{8pt} 

\textbf{指導教員とよく相談する}  \vspace{-6pt} 

これもChat GPTに原本を書いてもらったら出てきました 
Chat GPT曰く「悩んでいること・関心のあるテーマなどは、なるべく言葉にして伝えることが重要。」とのこと. 

私としては「指導教官を利用する」と言うのが正しいです. 皆さんは学費を払っているので, 指導教官を使う権利があります.  
数学的な内容に関する質問はもちろんのこと, 学振の書類の添削・論文の添削・修論など講演の練習など指導教官を利用するべきです. 

もちろん私もしたくないことやできないことは断ります. 「論文がすぐ書けるような問題をくれ」と言われたら「私の方が欲しいわ!」と言って断ると思います. 
が, とりあえず指導教官に言ってみた方がいいです. 教員側としては「何も言ってこないってことは大丈夫なんだな」と思っていることが多いからです. 将来の進路など数学以外でも悩んでいるなら相談した方がいいです. 
指導教官によっては返答してくれかもしれないですが, 何も言わないよりはマシです. 
\vspace{8pt} 

\textbf{その他.}

Chat GPTは修士に入ったらやることとして他に「授業に出る, セミナーに出る, 教科書を1 冊読み切る, 演習を大事にする」ということを提案してきました. 授業はどちらでもいいが出た方が無難. セミナーには出てほしいです. 進捗があろうとなかろうと, 人に会うことと外出することが大事なので. 
教科書を1冊読み切ったほうがいいです. 演習に関しては意味がわかりませんでした. 
でも意外とChat GPTもいいこと言うもんだなあ.

私個人としては「事務員さんにも礼儀正しく」も追加したいです. 
大学の事務員さんは本当に優秀です. 
こんなにも多くの仕事をどうやって捌いているのか毎回不思議に思います. 
ちなみに学生が見ているよりも, 事務員さんが仕事する範囲は広いです. 
が, 学生側はその仕事は見えないこともあって, 事務員さんへの態度が良くない傾向があります. 
%学生の時には気づかなかったが, 学生と教員に対する
%「事務員さんは仕事でお金もらってるのだからやって当然」と言う人は, 一回旅費申請を一人でやればいいと思います. それでありがたみがわかるはずです. あのシステムを一回見てみたが, 複雑すぎて全くわかりませんでした. 
%線形代数の方が簡単である. なお他大学では事務員さんがやってくれないこともあります. 
また事務員さんともめた時の原因の9.9割は「自分が大学・国のルールを熟知していない」または「大学・国のルールがそもそも悪い」です. つまり自分か大学・国が悪いです. 事務員さんに非がある場合はほとんどないです. 
%その時の怒りを事務員さんにぶつけても無駄である. 自分か大学か国にぶつけた方がいい.

\subsection{お役立ち情報}
これ以外にも役立つ情報を私や森 迪也先生がホームページでまとめています. 

\vspace{-8pt} 
\begin{itemize}[left=0pt]
  \setlength{\parskip}{0cm} % 段落間
  \setlength{\itemsep}{0cm} % 項目間
\item 岩井 雅崇 役に立つかもしれないリンク集 \\ 
\url{https://masataka123.github.io/blog3/sub5/}
\item 森 迪也 研究に役立つかもしれないリンク \\ \url{https://www.ms.u-tokyo.ac.jp/~mmori/jlink.html}
\end{itemize}


%%%%%%%%%%%%%%%%%%
\begin{comment}
\subsection{名言集}

\begin{itemize}
\item とにかくやれ!! 【仕事の姿勢】 \url{https://www.youtube.com/watch?v=JV3KOJ_Z4Vs}
\item  丸山ゴンザレス : 精神状態を保つための秘訣 \url{https://www.youtube.com/watch?v=2jceYEcTn44}
\item  【魂の授業】パチンコをやらない君達へ向けて岡野さんが”特別授業”を開講!!【鬼越トマホーク】 \url{https://www.youtube.com/watch?v=Ug54PDqRRA0}
\end{itemize}

最後に関して, 私はパチンコを進めているわけではないです(パチンコや競馬をやったことないです). が何かどうも研究に通ずるところがあると思います. 気のせいですかね
\end{comment}
%%%%%%%%%%%%%%%%%%




\end{document}
