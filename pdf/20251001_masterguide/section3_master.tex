\section{修士に入ったらやること・お役立ち情報}
\label{sec-3}
以下の内容は修士の学生向けです. 
私が修士で高山研に入った時に「修士に入ったらやること」みたいな紙を渡された気がします. 
そのため私は「修士に入ったら以下のことは教員が教えるもの」と思い込んでました.  
が, どうもそれは教員によってまちまちらしいです. 

以下の内容は, 修士に入った時点でやった方がいいと個人的に思うことです. 
 私の研究室に来ない場合でも数学の修士に入った人は参考にしてほしいです. 

\subsection{修士に入ったらやること}

\textbf{ArXivを毎日チェックする}   \vspace{-6pt} 

Arxiv \url{https://arxiv.org/archive/math} とは論文(プレプリント)サーバーです. 
数学分野においては, 論文(プレプリント)が書けた後はまずArxivに投稿して, その後にジャーナルに投稿します. これは数学独自の文化と言ってもいいです.
なのでArxivをチェックすれば自ずと最近の研究情報を入手することができます. 

ArXivは平日毎日チェックすることをおすすめします. なぜなら毎日見ないと論文のReplacementの情報を得られないからです. 私が登録しているのはAG(Algebraic Geometry), CV(Complex Variables), DG(Differential Geometry)です. 

なおメールでも送ってくれる機能もあります. 
Subscribe \url{https://info.arxiv.org/help/subscribe.html} のページを頑張って翻訳して登録してほしいです(面倒ならこの画面をスクショしてChat GPTに「翻訳して」と聞けばやってくれます). 
Xを使っている人はXの非公式アカウントがあるので, それをフォローすれば毎日ポストしてます. 

修士になりたてのうちは, 論文のタイトルとアブストラクトで興味あるもの見つけた, イントロ"だけ"を読むのがいいです. 修士の初めは「主定理を証明を含めて全て理解しよう」とは思わない方がいいです. 
修士1年後半$\sim$2年あたりから論文を読むことになります. 
ArXivで見つけた論文を, 証明を含めて読むことになります. なお論文には当たり外れがあるので, ガチャだと思った方がいいです. 

私はArXivは"YouTube"だと思っていて, 論文のタイトル・アブストは"サムネイル"だと思ってます. この思い込みは有用です. 論文をArXivに投稿する際は, 「いかにタイトル・アブストで興味を惹きつけられるか」と言うことにこだわった方がいいです.(まあサムネ詐欺と言われないくらいの結果を出すべきですが.)
そして論文のイントロも何回も推敲すべきです. ほとんどの人はイントロしか読まないからです. 
\vspace{8pt} 

\textbf{分野のメーリスに入る}   \vspace{-6pt} 

ほとんどの分野でメーリス(メーリングリスト)があるはずです. 
絶対に登録するべきです. 集会の情報や公募の情報が流れてくるので. 
私が入っているメーリスは以下の三つです.

\begin{itemize}[left=0pt]
  \setlength{\parskip}{0cm} % 段落間
  \setlength{\itemsep}{0cm} % 項目間
\item 代数幾何 \url{https://mail.math.nagoya-u.ac.jp/mailman/listinfo/agmail}
\item 幾何学 \url{https://www.math.nagoya-u.ac.jp/~naito/geometry/mailing-list.html}
\item 函数論 \url{https://www.cajpn.org/complex/ml.html}
\end{itemize}

もし数学を続けていくのであれば, メーリスで登録するメールアドレスは大学のものを使わ"ない"方がいいです. 
この段階で数学用gmail(私ならmasataka.math[at]gmail.com)を作って, そちらでメーリスなどを登録した方がいいです. (これはArXivでも同様.)
理由としては大学を異動したらその大学のメールアドレスは使えなくなるからです. 
あとこの数学用gmailは後々のことを考えて"実名"の方が望ましい. あと決してふざけることのないように.
\vspace{8pt} 

\textbf{数学分野の交友関係を広げる}   \vspace{-6pt} 

東大など数学の院生が多い大学ならこのことは気にしなくてもいいですが, 東大以外の院生は博士に進む人が少ないので外部の交友関係を広げる必要が出てきます. \footnote{私が東大にいたときは荒野先生のおかげで, 自ら交友関係を広げなくても, 分野に関わらずに話す相手が常にいました. 今思えばこれはかなり運がよかったと思います.} 

手っ取り早いのは当該分野で若い人が集まる集会に参加すればいいと思います. 
私の分野であれば以下の集会に行けば良いと思います. 
\begin{itemize}[left=0pt]
  \setlength{\parskip}{0cm} % 段落間
  \setlength{\itemsep}{0cm} % 項目間
\item 葉山シンポジウム(7月). 若手に関わらず多くの函数論の人が多く集まる.
\item 専門家向けケーラー勉強会(8月). 参加したことがないが, 講演者を見ている限り若い人が多い気がする. 修士の学生が参加して良いかは不明. 
\item 函数論サマーセミナー(9月). 1変数・多変数関わらず複素の学生が多い印象. 修士の人が勉強したことを話して良い.
\item 金沢複素幾何シンポジウム (11月). ちょっと年配の人が多い印象. 
\item 多変数関数論冬セミナー (12月). 世話人が若手に代替わりしたので, 若手が参加しやすいと思う.
\item 数理新人セミナー(2月). 私は参加したことがない. ただ当時の同期の雰囲気を見る限り参加していいと思う. 
\item 代数学若手研究会 (3月). 私は一回しか参加してない. 講演した思い出がある. 
\item 日本数学会(3月・9月). これは自分の結果が出てから. 講演するなら函数論で講演するのがおすすめです. 幾何と代数は人が多すぎる.  
\end{itemize}
問題はお金である. これは指導教官が科研費をどれくらい持っているかによります. 
全部は行くのは無理だし, 指導教官の科研費の補助でいけるとして, 多くて1つか2つだと思います. 
確か葉山とサマーセミナーは旅費の補助があった気がします.
そのあたりはメーリスで情報を仕入れる必要があります. 

博士に行く予定があるなら, 早い目にその分野のコミュニティーに入って友達を作った方がいいと思います. 
私に近い分野だと
\begin{itemize}
  \setlength{\parskip}{0cm} % 段落間
  \setlength{\itemsep}{0cm} % 項目間
\item 松村先生・足立先生・小池先生・日下部先生・稲山先生・渡邉先生がいる函数論のグループ. 多変数関数論由来のグループだが, 代数幾何学から複素幾何・幾何解析など実際の分野はかなり幅広いと思う. 
\item 久本先生・橋本先生・井上先生などがいる日本の複素微分幾何勢のグループ. 専門家向けケーラー勉強会など活発に活動している印象. 
%\item 高山先生および元高山研のメンバー(松村先生・久本先生・小池先生・岩井・稲山先生・井上先生・渡邉先生)
\end{itemize}
にいる人たちに顔を覚えてもらえればいいかと思います.
もちろん両方入ってもいいと思います. 
%高山先生および元高山研のメンバー(松村先生・久本先生・小池先生・岩井・稲山先生・井上先生・渡邉先生)
%足立先生・小池先生・日下部先生がいる函数論のグループ」 や 「久本先生・橋本先生・井上先生などがいる日本の複素微分幾何勢のグループ」に顔を覚えてもらえればいいかと思います.
%\footnote{ちなみに私は「函数論のグループに入っている(と思っている)が, 家出しまくっている」と言う状況である. これは研究者として自立する必要があり, 大変苦労する(した)のでおすすめしない. まあ協調性がなく群れて行動するのが苦手なだけなんですが.}
\vspace{8pt} 

\subsection{修士に入ったら気をつけること}

\textbf{メールの返信をする}   \vspace{-6pt} 

最近の学生はメールの返信が丁寧で早いと思います. 本当に素晴らしいです. 
基本的にメールの返信は翌日(翌日が土曜の場合は, 次の月曜)までに返信するべきです. 遅くても2, 3日後までには返信した方がいいです. 

ちなみに私が学生の時はどうだったのかというと, 全然守れてなかったです. 博士を出た後にメールの返信を即座に返すように心がけました. 
%時たま「数学者は数学以外は適当だ」と思われるが, それは昔の話と思ってい 

%じゃあ私が学生の時はどうだったのかというと, 全然守れてなかったです. 1週間後に返信することもよくありました.  が, 博士を出た後にメールの返信を即座に返すように心がけました. 色々理由はありますが, 一番は「めちゃくちゃ忙しい藤野先生・松村先生がメールを即座に返すのに, めちゃくちゃ暇な岩井のメールの返信が遅いのはよくないだろう」と思ったからです「メールの返信が遅くて許されるのは偉い人だけ」と思った方がいいです. 
\vspace{8pt} 

\textbf{締切を守る}   \vspace{-6pt} 

これもルーズな人が多い印象. 例えば学振や修論に関して締め切り前に慌てる人が多いと思います. 
厳しいことを言うと, 学振・修論に関しては締め切り前に慌てている時点でダメです. 
二つともは何回も何回も推敲するべきものであるからです.  
%なので本来の締め切りの1週間前に仕上がっている必要がある. 

%一番簡単な方法は本来の締切よりも前に"自分の締切"を設けることである. 具体的には本来の締切の1$\sim$2週間前に"自分の締切"を設ければ良い. 結局のところほとんど多くの人が締切直前にしか作業をしない. 締め切りがいつでもいいのなら, "自分の締切"前に作業すればいいのである. 
一番簡単な方法は本来の締切の1$\sim$2週間前に"自分の締切"を設ける方法. 
結局のところほとんど多くの人が締切直前にしか作業をしません. 締め切りがいつでもいいのなら, "自分の締切"前に作業すればいいと思います. 

%なおこれは私が研究集会の講演者に講演タイトルを聞く際にも同様のことをやっています. 講演タイトルが揃ってほしい日の1週間前に締切を設けて, その締切までに講演者に講演タイトルを返信するようにお願いします.
%ほとんどの人は締切前までに返信しますが, 締切を過ぎても返信しない人も多いです. その人には個別に連絡するとすぐに返信が返ってきます. 結果として, 講演タイトルが揃ってほしい日までには講演者から返信が返ってくるので, 何の問題もないです. 
\vspace{8pt} 

\textbf{休日は自由である}   \vspace{-6pt} 

佐藤幹夫先生の有名な言葉に
\begin{tcolorbox}[mybox]
朝起きた時に、きょうも一日数学をやるぞと思っているようでは、とてもものにならない。数学を考えながら、いつのまにか眠り、朝、目が覚めたときは既に数学の世界に入っていなければならない。
\end{tcolorbox}
 があります. 私はこれに反対です. 休日は自由です. 数学をしてもいいし, せずに休んでもいいです. 

理由としては「普通の人がこれをやっても長く続かない」から. 
数学の道に進むなら今後30年は数学をすることになります. なので, 数学を長く続けられる生活リズムを作っていくことになります. 
休日含め毎日数学をする生活は修士までならできると思うが, 博士やそれ以後も続けられるかは微妙です. 
無論, 長く続けられる人は毎日数学をやっていただいて構わないです. 実際そのような人はいます.
が, できない人は休む日を設けた方がいいと思います. 休んだ方が返って効率が良いこともあります. 

なお「休日は自由である」は教員に関しても適応されます. 修論を年末に私に送りつけ「修論の添削をしてください」と学生が言ってきても, 私はガン無視する予定です. 年末年始はゆっくりしたいから. 学振に関しても, 学生がゴールデンウィーク前に「添削してください」と言ってきても, ガン無視する予定です. ゴールデンウィークは遊びたいから. 学生が「修論の締切直前なんです」みたいなことを言っても, ガン無視する予定です. その場合は潔く留年していただきます. これは上の締切を守ると言うことにもつながります. 

なぜこんなことを言っているのかと言いますと, このような愚痴は本当によく聞くからです. 「教員も人間であり休日は休む」と言うことを頭の隅でもいいので覚えていてほしです.  

なお私はよく休日にメールなどを返信しますが, これは「メールをその時に返信しないと忘れてしまう」から.
「これを送っても返事はすぐ来ないだろう」と思っています. なので私からのメールが来ても休日であれば返信しなくて良いです. 次の週の月曜に返信していただければと思います. 
\vspace{8pt} 

\textbf{英語に慣れる}   \vspace{-6pt} 

このガイドはChat GPTに原本を書いてもらいました.
その際にこの項目が出てきました. 
Chat GPT曰く「論文・教科書・プレゼンなど、あらゆる場面で英語が必要になる。少しずつ読んでいくことが大切。」とのこと. 
これはその通りです. 数学の論文は基本的に英語で書かれているので英語を避けて数学を研究するのは無理です. 特にこの分野は「フランス・ドイツ・中国」あたりの人と交流することが多いので, 英語は避けられないです. 

基本的に"読む"のと"書く"のは中学校の英語ができていればできる(はず). 数学英語はビジネス英語よりはるかに簡単です. 
問題は"話す"と"聞く". これは私も全く自信がないです. なので英会話を上手くする方法は英語の上手い先生に教わってください. 
私はいまだに会話してて「何を言ってるんだろう」となることがよくあります. でも英語で会話せざるを得ないです. 
その際のコツを何点かお伝えします. 
\begin{itemize}[left=0pt]
  \setlength{\parskip}{0cm} % 段落間
  \setlength{\itemsep}{8pt} % 項目間
\item \underline{iPadなど書くものを用意する.} 一番簡単な方法. 会話で伝わらないなら, 書けば良いです. 数式や質問内容を書いて, "これどうなるんや"と聞けばいいです. 
数学者に質問したいならこの方法が一番楽です. 返答の意味がわからない時も, 相手にiPadを渡して書いてもらったらいいです. 講演後の質問なら黒板を使えばいいです. 
\item \underline{ジェスチャーを入れる.}  私はいつも"He told me that (主語)..."の部分で(主語)に"私"を意味することを言いたい時に, "I"を使えばいいのか"You"を使えばいいのかを悩みます. 
これを解決する方法として, (主語)の部分を言ったときに自分を指差します. 間違っていても「こいつは文法間違っているけど, 彼がこいつに言ったんやな」となりこれで通じます. ジェスチャーを入れると, 英語が下手でも通じることが多いと思います. 
\item \underline{相手に自分の英語が通じてなさそうな場合は, 黙るのではなく, 言い換えて伝える.} 自分の英語が通じていなくて相手が???となっている状況がよくあります. この状況を打破するには, 「つまり...」と言い換えて喋りまくることです. 実はこの状況は, 自分の英語が悪いのではなく相手側が単に理解してないことが原因でおこることもあります.\footnote{ごく最近このような状況に遭遇しました. 中国の研究者と日常会話していて, 彼が「〇〇さん(日本人研究者)のお子さんが生まれた時に, あなた(岩井)は招待されましたか?」と言ってきました. 私には意味がわかりませんでした. 彼は"英語が通じなかったのか?"という感じでいましたが, 日本人の私としては"何で子供が生まれたら私を呼ぶんだ?"と思いました. よくよく話をしてみると, 中国では子供が生まれてしばらくした後に, 仲のいい人たちをよんでパーティーをするらしいです. } 
とりあえずこっち側が言いまくればいいです. 有限回でいずれ伝わります. 伝わらなければ, 相手側に「どこがわからんのや?」と言えばいいです. もしくは日常会話のようなどうでもいい話なら, この話を早々に切り上げて別の話をしてもいいです. 
\item \underline{スマホを使う.} これは日常会話に使えます. 日本文化など会話で伝わりづらいものは画像を見せれば案外伝わります.  
\end{itemize}

あとChat GPTに課金しているならば, 英語の読み書きをChat GPTに任せるのも良いです. 多くの日本人よりもChat GPTの方が英語は上手いです. 私は英語でメールをするときに, Chat GPTに原案を書かせてそれを修正したものを使っています. Chat GPTの英語がうますぎて不自然なので, あえて下手にしています. 
 \vspace{8pt} 

\textbf{指導教員とよく相談する}  \vspace{-6pt} 

これもChat GPTに原本を書いてもらったら出てきました 
Chat GPT曰く「悩んでいること・関心のあるテーマなどは、なるべく言葉にして伝えることが重要。」とのこと. 

私としては「指導教官を利用する」と言うのが正しいです. 皆さんは学費を払っているので, 指導教官を使う権利があります.  
数学的な内容に関する質問はもちろんのこと, 学振の書類の添削・論文の添削・修論など講演の練習など指導教官を利用するべきです. 

もちろん私もしたくないことやできないことは断ります. 「論文がすぐ書けるような問題をくれ」と言われたら「私の方が欲しいわ!」と言って断ると思います. 
が, とりあえず指導教官に言ってみた方がいいです. 教員側としては「何も言ってこないってことは大丈夫なんだな」と思っていることが多いからです. 将来の進路など数学以外でも悩んでいるなら相談した方がいいです. 
指導教官によっては返答してくれかもしれないですが, 何も言わないよりはマシです. 
\vspace{8pt} 

\textbf{その他.}

Chat GPTは修士に入ったらやることとして他に「授業に出る, セミナーに出る, 教科書を1 冊読み切る, 演習を大事にする」ということを提案してきました. 授業はどちらでもいいが出た方が無難. セミナーには出てほしいです. 進捗があろうとなかろうと, 人に会うことと外出することが大事なので. 
教科書を1冊読み切ったほうがいいです. 演習に関しては意味がわかりませんでした. 
でも意外とChat GPTもいいこと言うもんだなあ.

私個人としては「事務員さんにも礼儀正しく」も追加したいです. 
大学の事務員さんは本当に優秀です. 
こんなにも多くの仕事をどうやって捌いているのか毎回不思議に思います. 
ちなみに学生が見ているよりも, 事務員さんが仕事する範囲は広いです. 
が, 学生側はその仕事は見えないこともあって, 事務員さんへの態度が良くない傾向があります. 
%学生の時には気づかなかったが, 学生と教員に対する
%「事務員さんは仕事でお金もらってるのだからやって当然」と言う人は, 一回旅費申請を一人でやればいいと思います. それでありがたみがわかるはずです. あのシステムを一回見てみたが, 複雑すぎて全くわかりませんでした. 
%線形代数の方が簡単である. なお他大学では事務員さんがやってくれないこともあります. 
また事務員さんともめた時の原因の9.9割は「自分が大学・国のルールを熟知していない」または「大学・国のルールがそもそも悪い」です. つまり自分か大学・国が悪いです. 事務員さんに非がある場合はほとんどないです. 
%その時の怒りを事務員さんにぶつけても無駄である. 自分か大学か国にぶつけた方がいい.

\subsection{お役立ち情報}
これ以外にも役立つ情報を私や森 迪也先生がホームページでまとめています. 

\vspace{-8pt} 
\begin{itemize}[left=0pt]
  \setlength{\parskip}{0cm} % 段落間
  \setlength{\itemsep}{0cm} % 項目間
\item 岩井 雅崇 役に立つかもしれないリンク集 \\ 
\url{https://masataka123.github.io/blog3/sub5/}
\item 森 迪也 研究に役立つかもしれないリンク \\ \url{https://www.ms.u-tokyo.ac.jp/~mmori/jlink.html}
\end{itemize}


%%%%%%%%%%%%%%%%%%
\begin{comment}
\subsection{名言集}

\begin{itemize}
\item とにかくやれ!! 【仕事の姿勢】 \url{https://www.youtube.com/watch?v=JV3KOJ_Z4Vs}
\item  丸山ゴンザレス : 精神状態を保つための秘訣 \url{https://www.youtube.com/watch?v=2jceYEcTn44}
\item  【魂の授業】パチンコをやらない君達へ向けて岡野さんが”特別授業”を開講!!【鬼越トマホーク】 \url{https://www.youtube.com/watch?v=Ug54PDqRRA0}
\end{itemize}

最後に関して, 私はパチンコを進めているわけではないです(パチンコや競馬をやったことないです). が何かどうも研究に通ずるところがあると思います. 気のせいですかね
\end{comment}
%%%%%%%%%%%%%%%%%%


