\documentclass[12pt]{amsart}



\setlength{\topmargin}{-0.5cm}
\setlength{\textwidth}{15cm}
\setlength{\textheight}{22.6cm}
\setlength{\topmargin}{-0.25cm}
\setlength{\headheight}{1em}
\setlength{\headsep}{0.5cm}
\setlength{\oddsidemargin}{0.40cm}
\setlength{\evensidemargin}{0.40cm}

\usepackage{amscd}
\usepackage{amsmath}
\usepackage{amssymb}
\usepackage{amsthm}
\usepackage{bigdelim}
\usepackage{color}
\usepackage{enumerate}
\usepackage{graphicx}
\usepackage{mathrsfs}
\usepackage{multirow}
\usepackage[all]{xy}

\usepackage{amsmath,calligra,mathrsfs}


\usepackage[dvipdfmx,colorlinks,linkcolor=red,anchorcolor=blue,citecolor=blue]{hyperref} 
\usepackage{comment}
\newtheorem{thm}{Theorem}[section]
\newtheorem{theo}[thm]{Theorem}
\newtheorem{cor}[thm]{Corollary}
\newtheorem{prop}[thm]{Proposition}
\newtheorem{conj}[thm]{Conjecture}
\newtheorem*{mainthm}{Theorem}
\newtheorem{deflem}[thm]{Definition-Lemma}
\newtheorem{lem}[thm]{Lemma}
\theoremstyle{definition}
\newtheorem{defn}[thm]{Definition}
\newtheorem{claim}[thm]{Claim}
\newtheorem{propdefn}[thm]{Proposition-Definition}
\newtheorem{lemdefn}[thm]{Lemma-Definition}
\newtheorem{thmdefn}[thm]{Theorem-Definition}
\newtheorem{setup}[thm]{Setting}
\newtheorem{e.g.}[thm]{Example}
\newtheorem{ex}[thm]{Example}
\newtheorem{ques}[thm]{Question}
\theoremstyle{remark}
\newtheorem{rem}[thm]{Remark}

\newtheorem{obs}[thm]{Observation}
\newtheorem{notation}[thm]{Notations}
\newtheorem{assup}[thm]{Assumption}
\newtheorem{step}{Step}
\newtheorem*{clproof}{Proof of Claim}
\newtheorem{cln}[thm]{Claim}
\newtheorem*{ack}{Acknowledgments}

\numberwithin{equation}{section}

\newcommand{\supp}[0]{\operatorname{Supp}}
\newcommand{\Rad}[0]{\operatorname{Rad}}
\newcommand{\Sha}[0]{\operatorname{Sha}}
\newcommand{\sha}[0]{\operatorname{sha}}
\newcommand{\eend}[0]{\operatorname{End}}
\newcommand{\codim}[0]{\operatorname{codim}}
\newcommand{\nd}[0]{\operatorname{nd}}
\newcommand{\rank}[0]{\operatorname{rank}}
\newcommand{\degree}[0]{\operatorname{deg}}
\newcommand{\Exc}[0]{\operatorname{Exc}}
\newcommand{\pr}{{\rm pr}}
\newcommand{\id}{{\rm id}}
\newcommand{\Sym}{{\rm Sym}}
\newcommand{\rk}{{\rm rk}}
\newcommand{\Supp}{{\rm Supp}}




\newcommand{\Hom}{{\mathscr{H}\text{\kern -3pt {\calligra\large om}}\,}}

\newcommand{\verti}{{\rm vert}}
\newcommand{\hor}{{\rm hor}}
\newcommand{\univ}{{\rm univ}}

\newcommand{\shaf}{\mathrm{sha}}
\newcommand{\Shaf}{\mathrm{Sha}}

\newcommand\sO{{\mathcal O}}
\newcommand{\Div}[0]{\operatorname{div}}
\newcommand{\ddbar}{dd^c}
\newcommand{\cV}{\mathcal{V}}
\newcommand{\deldel}{\sqrt{-1}\partial \overline{\partial}}
\newcommand{\dbar}{\overline{\partial}}
\newcommand{\I}[1]{\mathcal{I}(#1)}
\newcommand{\Aut}[1]{\mathrm{Aut}(#1)}
\newcommand{\Ker}[1]{\mathrm{Ker}(#1)}
\newcommand{\Image}[1]{\mathrm{Im}(#1)}


\DeclareMathOperator{\Ric}{Ric}
\DeclareMathOperator{\Vol}{Vol}
 \newcommand{\pdrv}[2]{\frac{\partial #1}{\partial #2}}
 \newcommand{\drv}[2]{\frac{d #1}{d#2}}
  \newcommand{\ppdrv}[3]{\frac{\partial #1}{\partial #2 \partial #3}}
\newcommand{\R}{\mathbb{R}}
\newcommand{\Z}{\mathbb{Z}}
\newcommand{\N}{\mathbb{N}}
\newcommand{\C}{\mathbb{C}}
\newcommand{\Q}{\mathbb{Q}}
\newcommand{\D}{\mathbb{D}}
\newcommand{\mP}{\mathbb{P}}
\newcommand{\mO}{\mathcal{O}}
\newcommand{\tl}{\hspace{-0.8ex}<\hspace{-0.8ex}}
\newcommand{\tr}{\hspace{-0.8ex}>}

\newcommand{\xb}[1]{\textcolor{blue}{#1}}
\newcommand{\xr}[1]{\textcolor{red}{#1}}
\newcommand{\xm}[1]{\textcolor{magenta}{#1}}

\title[Remarks on minimal compact K\"ahler manifolds with vanishing second Chern class]
{Remarks on minimal compact K\"ahler manifolds with vanishing second Chern class}


\author{Masataka IWAI}
\address{Department of Mathematics, Graduate School of Science, Osaka University,
1-1, Machikaneyama-cho, Toyonaka, Osaka 560-0043, Japan.}
\email{{\tt masataka@math.sci.osaka-u.ac.jp}}
\email{{\tt masataka.math@gmail.com}}

%\author{Shin-ichi MATSUMURA}
%\address{Mathematical Institute, Tohoku University,6-3, Aramaki Aza-Aoba, Aoba-ku, Sendai 980-8578, Japan.}
%\email{{\tt mshinichi-math@tohoku.ac.jp}}
%\email{{\tt mshinichi0@gmail.com}}


\date{\today, version 0.01}


\baselineskip = 15pt
\footskip = 32pt


\begin{document}

\maketitle

%\begin{abstract}
%このメモは2024年11月に開かれた集会で指摘された問題"いつどの多様体が$c_2=0$になるか"について考えたものです. 
%\end{abstract}

%\tableofcontents


\section{Remarks on minimal compact K\"ahler manifolds with vanishing second Chern class}

\begin{thm}
\label{thm-Torus-fibration}
Let $X \to C $ be a smooth torus fibration onto a curve $C$ of genus $\ge 2$.
Then, the numerical Kodaira dimension $\nu(K_{X})$ is equal to one and  the cotangent bundle $\Omega_{X}$ is nef. In particular, $c_2(\Omega_{X})=0$.
\end{thm}

The following proof is based on an idea by Xiaojun Wu.
\begin{proof}
Since $f$ is a smooth torus fibration, the curvature of $K_{X/C}$ is equal to the pull-back of the Weil-Peterson form as follows (c.f. \cite[Equation 1.1]{BCS20}):
$$
2 \pi c_1(K_{X/C}) = \frac{1}{{\rm vol} X_{c}} f^{*}\omega_{{\rm WP}}.
$$
The dimension of $C$ is one, so we conclude that $c_1(\Omega_{X})^2 =0$.

Since $f^{*}\Omega_{C}$ is nef, it is enough to show that $\Omega_{X/C}$ is also nef.
This is directly follows from \cite[Theorem 5.2 and Corollary 7.8]{Gri70} and \cite[Proposition 5]{Kra97}.
Indeed, by \cite{Gri70}, $R^{1}f_{*}\mathcal{O}_{X}$ is Griffith seminegative.\footnote{In Griffith's paper, the  fibration is assumed to be projective. However, this result holds even in the case when $f$ is proper K\"ahler submersion. For more details, please refer to \cite[Section 2.3]{MT07}. Using the same notation as in \cite[Section 2.3]{MT07}, setting $d=1$ and $p=0$, it follows that $R^{1}f_{*}\mathcal{O}_{X}$ is Griffith seminegative.} 
Thus $f_{*}\Omega_{X/C}$ is Griffith semipositive, so is $\Omega_{X/C}=f^{*}f_{*}\Omega_{X/C}$.
\end{proof}

Since the condition "$c_2(\Omega_{X})=0$" is invariant under finite étale covers, we obtain the following corollary from \cite{IM22}.

\begin{cor}[cf. \cite{IM22}]
Let $X$ be a compact K\"ahler manifold  with nef canonical divisor $K_{X}$. 
Then $c_2(\Omega_{X})=0$ in $H^{2,2}(X,\R)$ \xr{if and only if}
there exists a finite \'etale cover $X'\rightarrow X$ such that one of the following holds depending on the Kodaira dimension$:$
\begin{itemize}
\item[$(i)$] In the case where $\nu(K_{X})=\kappa(K_{X})=0$, 
the variety $X'$ is isomorphic to a complex torus. 
\item[$(ii)$] In the case where $\nu(K_{X})=\kappa(K_{X})=1$, 
the variety $X'$ admits a smooth torus fibration $X \to C$ onto a curve of genus $\ge 2$.
\end{itemize}
\end{cor}
Similarly, we obtain the following corollary.
\begin{cor}[cf. \cite{IMM24}]
Let $X$ be a projective klt variety of dimension $n$ with nef canonical divisor $K_{X}$. Then the Miyaoka's equality holds for some ample divisors $H_i$ on $X$$:$
\begin{equation*}
\label{eq-miyaoka}
 \left( 3 \widehat{c}_2(\Omega_{X}^{[1]}) - \widehat{c}_1(\Omega_{X}^{[1]})^2 \right)
H_1\cdots H_{n-2}=0
\end{equation*}
\xr{if and only if} there exists a finite quasi-\'etale cover $X'\rightarrow X$ such that one of the following holds depending on the Kodaira dimension$:$
\begin{itemize}
\item[$(i)$] In the case where $\nu(K_{X})=\kappa(K_{X})=0$, 
the variety $X'$ is isomorphic to an abelian variety. 

\item[$(ii)$] In the case where $\nu(K_{X})=\kappa(K_{X})=1$, 
the variety $X'$ admits the structure of an abelian group scheme $X' \rightarrow C$ over a curve $C$ of general type.


\item[$(iii)$] In the case where $\nu(K_{X})=\kappa(K_{X})=2$, the variety $X'$ is isomorphic to the product $A \times S$ of an abelian variety $A$ 
and a smooth surface $S$ whose universal cover is an open ball in $\mathbb{C}^{2}$. 
\end{itemize}
\end{cor}

\subsection*{Acknowledgments}
The author would like to thank Xiaojun Wu for kindly considering this problem and sharing his ideas.  
He also thank Prof.\, Ngaiming Mok for suggesting the problem that inspired Theorem \ref{thm-Torus-fibration}.  
Finally, he thanks the organizers of \textit{"SCV, CR geometry and Dynamics"} at RIMS in Kyoto for providing the opportunity to explore this problem.
\bibliographystyle{alpha}
\bibliography{ref_minimal.bib}


\end{document}


