\chapter{Currentの定義}

\begin{thebibliography}{n}
\bibitem[Rud]{Rud}
W. Rudin. \textit{Functional analysis.} 2nd edn. International Series in Pure and Applied Mathematics. McGraw-Hill, Inc., New York. (1991.)
\bibitem[NO]{NO}
J. Noguchi, T.Ochiai \textit{Geometric Function Theory in Several Complex Variables} Translations of Mathematical Monographs
Volume: 80; 1990; 282 pp
\end{thebibliography}
 
\cite[Chapter 3]{NO}を参考にしている. 

\section{Current}

$M$ 2nd countable$m$次元$C^\infty$級多様体とする. 

\textbf{Recall}
$M \subset \R^m$の時$C^{\infty}(M)$には次の位相を次で入れていた. 

$K_1 \subset K_2 \subset \cdots \subset M$というコンパクト集合であって
$$
K_i \subset K_{i+1}^{\circ} \quad \text{and} \quad M = \bigcup K_{i}^{\circ} 
$$ 
となるものを一つ固定し, $f \in C^{\infty}(M)$について, $N \in \N$として
$$
P_{N}(f) := \max\{|D^{\alpha} f(x)| \mid |\alpha| \le N \& x \in K_{N} \}
$$
$$
V_{N} := \max\{ f \in C^{\infty}(M) \mid P_N (f) < \frac{1}{N}\}
$$
を$0$のopen baseとするような位相を入れていた.

$K \subset M$コンパクトに対して
$$
\mathcal{D}_{K}(M):=
\{ \phi \in C^{\infty}(M) \mid \Supp(\phi) \subset K\} \subset C^{\infty}(M)
$$
に対して相対位相を入れて
$$
\mathcal{D}(M):= \bigcup_{K \subset M \mid \text{$K$ cpt}} \mathcal{D}_{K}(M)
\cong 
\varinjlim_{K} \mathcal{D}_{K}(M)
$$
このcolimitはlocally convex vector spaceでのcolimitである. 

\begin{tcolorbox}[mybox]
\begin{lem}
\label{lem-E-1}
 \begin{enumerate}
\item  $U, U'\subset \R^m$ を開集合とする.  $U \cong U'$を微分同相とする時, $C^{\infty}(U) \cong C^{\infty}(U')$である. ここでこの同型は位相$\C$ベクトル空間の同型である
\item $M \subset \R^m$ 開集合とし, $M = \bigcup U_i$を可算個の開被覆とする時
$$
C^{\infty}(M) \hookrightarrow \prod C^{\infty}(U_i)
\quad
f \mapsto \{ f|_{U_i}\}  
$$
とすると, $C^{\infty}(M)$の位相はこの直積位相$\prod C^{\infty}(U_i)$によって引き起こされる位相となる. 
特に
$$
C^{\infty}(M)  \cong \mathrm{Eq}(\prod C^{\infty}(U_i) \rightrightarrows \prod C^{\infty}(U_i \cap U_j) )
$$
という位相$\C$ベクトル空間の同型を得る. 
\end{enumerate}
\end{lem}
\end{tcolorbox}
\begin{proof}

(1). 以下$K_1 \subset K_2 \subset \cdots \subset U$というコンパクト集合で$C^{\infty}(U)$の位相を誘導するものを一つ固定する. 
$\Phi : U' \to U, (x_1, \ldots, x_m)\mapsto (y, \ldots, y_m)$を微分同相写像とする. 
すると$K'_{i}:=\Phi(K_i)$によって, $C^{\infty}(U')$の位相を誘導する.
$V_{N}, V'_{N}$を上の通りとする. 

さてその引き戻し
$$
\Phi^{*} : C^{\infty}(U) \rightarrow C^{\infty}(U')
\quad
f(y) \mapsto f \circ \Phi(x)=f (y (x))
$$
が位相$\C$ベクトル空間の同型を誘導することを示す. 

$f \in C^{\infty}(U)$についてchain ruleより, $|\alpha | \le N'$について
\begin{equation}
\label{eq-lem-E-1}
D_{x}^{\alpha}f \circ \Phi(x)
:=
\sum_{|\beta| \le N'} (D_{y}^{\beta} f) (y(x)) \cdot \Phi_{\alpha \beta}(x)
\end{equation}
である. 

$\Phi$が$0$の近傍で連続であることを示せば良い. 
任意の$N'$について, $K_{N'}$コンパクトなので, $K_{N'}$上では$|\Phi_{\alpha \beta}(x)| \le C_{N'}$となる$C_{N'}$が取れる. 
よって$N' \cdot \frac{1}{N} \cdots C_{N'} < \frac{1}{N'}$となる$N$をとれば, 
$f \circ \Phi \in \Phi^{*}(V_N)$について
$$
|D_{x}^{\alpha}f \circ \Phi(x)|
\underset{(\ref{eq-lem-E-1})}{\le}
\sum_{|\beta| \le N} \underbrace{|(D_{y}^{\beta} f) (y(x))|}_{<\frac{1}{N}} \cdot |\Phi_{\alpha \beta}(x)|
\le N' \cdot \frac{1}{N} \cdot C_{N'} < \frac{1}{N}
$$
よって「任意の$N'>0$について, ある$N>0$があって, $\Phi^{*}(V_N)  \subset V'_{N'}$である」ため$\Phi$は$0$の近傍で連続である. 

(2) $M = \bigcup_{i=1}^{\infty} U_i$とする. 
各$U_i$で
$$
K_{i1} \subset K_{i2} \subset \cdots \subset U_i
$$
で$U_i = \bigcup_{j} K_{ij}^{\circ} $となるコンパクト列をとる.
そこで$K_N := \bigcup_{i=1}^{N}K_{i, N}$とすると
$$
K_{1} \subset K_{2} \subset \cdots \subset M
$$
であって, $K_i \subset K_{i+1}^{\circ}$ かつ$M=\bigcup K_{i}^{\circ}$となる. 

さて$\prod C^{\infty}(U_i)$の$0$でのlocal baseは
$$
\{ 
V_{\bm{i}, \bm{N}}:=V_{i_1, N_{i_1}} \times V_{i_2, N_{i_2}} \times \cdots \times V_{i_l, N_{i_l}} \times 
\prod_{i \neq i_k} C^{\infty}(U_i) \mid i_1 < i_2 < \cdots <i_l, N_{i_k} \in \N
\}
$$
となる形のものである. 
ここで
\begin{itemize}
\item $\bm{i}:=(i_1, i_2, \ldots, i_{l}), \bm{N}:=(N_{i_1}, N_{i_2},\ldots, N_{i_l})$と定める. 
\item $V_{i, N}:= \{ f \in C^{\infty}(U_i) \mid P_{N}(f) < \frac{1}{N}\}$である(ここの定義での$ P_{N}(f) $には$K_{i, N}$をつかう.)
\end{itemize}

一方で$V_{N}:= \{ f \in C^{\infty}(M) \mid P_{N}(f) < \frac{1}{N}\}$(ここの定義での$ P_{N}(f) $には$K_{iN}$をつかう.)であり, これは$C^{\infty}(M)$の0でのlocal baseである. 
以上より, 次の二つを示せば良い.
\begin{itemize}
\item 任意の$\bm{i}, \bm{N}$について, ある$N$があって, $V_N \subset V_{\bm{i}, \bm{N}} \cap  C^{\infty}(M)$が成り立つ. 
\item 任意の$N$について, ある$\bm{i}, \bm{N}$があって, $V_N \supset V_{\bm{i}, \bm{N}} \cap  C^{\infty}(M)$が成り立つ. 
\end{itemize}


\begin{align*}
V_N &:= 
\left\{ f \in C^{\infty}(M) \mid x \in K_{N}:=\bigcup_{i=1}^{N}K_{i, N}, |\alpha| \le N, |D_\alpha f(x)|<\frac{1}{N}\right\} \\
&\underset{\text{def}}{=}
\left( V_{1, N} \times V_{2, N} \times \cdots \times V_{N, N} \times 
\prod_{i > N} C^{\infty}(U_i)  \right) \cap C^{\infty}(M) \\
&\underset{\text{def}}{=} V_{(1,2, \ldots, N), (N, \ldots, N)} \cap C^{\infty}(M) \\
\end{align*}
である. これより二つ目の主張が正しいことがいえる.
一つ目の主張は任意の$\bm{i}, \bm{N}$について, 
$$
V_{\max\{\bm{i}, \bm{N}\} }\subset V_{\bm{i}, \bm{N}} \cap C^{\infty}(M)
$$
なので言える. 
\end{proof}


\begin{tcolorbox}[mybox]
\begin{defn}
\label{defn-E-2}
$M$ 2nd countable$m$次元$C^\infty$級多様体とする. 
$C^\infty(M)$に位相を次のように入れる.

まず$M = \bigcup_{l=1}^{\infty} U_i $で$U_i \subset \mathbb{R}^m $となるcountable open coverを一つ固定する. 
そして, 
\[
C^\infty(M) \hookrightarrow \prod_i C^\infty(U_i) \quad 
f \longmapsto (f|_{U_i})
\]
による部分位相を$C^\infty(M)$に入れる. 
つまり,
\[
C^{\infty}(M)  \cong \mathrm{Eq}(\prod C^{\infty}(U_i) \rightrightarrows \prod C^{\infty}(U_i \cap U_j) )
\]
となるように入れる. 
\end{defn}
\end{tcolorbox}

\begin{rem}
\ref{defn-E-2}による位相の定義において, $U_i$の取り方によらない.
\end{rem}

\begin{proof}
別の$U'_j$をとると細分$U_i \cap U'_j$が取れる.
よって次の図式が考えられる. 
  \[
\xymatrix@C=50pt@R=20pt{
C^{\infty}(M) \ar@{^{(}-_>}[r] \ar@{^{(}-_>}[rd]
&\prod C^{\infty}(U_i) \ar@{^{(}-_>}[rd]
&   \\
 &\prod C^{\infty}(U'_j)  \ar@{^{(}-_>}[r] & \prod C^{\infty}(U_i \cap U'_j)  \\
}
\]
そこで"相対位相の直積は相対位相になる"ので,   
\ref{lem-E-1}より言える.
\end{proof}

これにより
$$
C^{\infty} : (U \underset{open}{\subset} M) \mapsto C^{\infty}(U)
$$
は$\C$ベクトル空間のsheafとなる. 
よって$K \subset M$に対して
$$
\mathcal{D}_{K}(M) := \{ \phi \in C^{\infty}(M) \mid \Supp(\phi) \subset K\}
$$
として$C^{\infty}(M)$の部分位相を入れる. 
$$
\mathcal{D}(M):= \bigcup_{K \subset M \mid \text{$K$ cpt}} \mathcal{D}_{K}(M)
\cong 
\varinjlim_{K} \mathcal{D}_{K}(M)
$$
このcolimitはlocally convex vector spaceでのcolimitである. 

[指摘]
このcolimは存在する?
やっぱりSection 1でやったような位相の入れ方にもう一回戻る?
$\varinjlim_{K} \mathcal{D}_{K}(M)$については次回再考. 


\section{current続き}
以下\cite{NO}の記法に合わせる. ($M$ 2nd countable $m$次元$C^\infty$級多様体, $k \in \N$とする.)\footnote{おそらく通常の記法とかなり違う記法であると思われる. }

\begin{itemize}
\item $C(M) := \left\{\, f : M \to \mathbb C \,;\ \text{連続} \,\right\}$
\item $\mathcal E(M) := \left\{\, f : M \to \mathbb C \,;\ C^\infty\text{-関数} \,\right\}$  
普通は"$C^\infty(M)$"である.
\item $C^k(M) := \left\{\, \varphi : k\text{-forms on } M\ \text{with coeff} \in C \,\right\}$
つまりlocallyに
$$\varphi = \displaystyle \sum_J \varphi_J\, d\bar z^J
\quad 
(d\bar z^J = d\bar z^{j_1} \wedge \cdots \wedge d\bar z^{j_k})
$$
とかけるものとする. ここで$(U; z^1, \dots, z^n)$を局所座標とし, $J=(j_1, \ldots, j_k)$, $\varphi_J \in C(U)$とする. 
特に$C^0(M) = C(M)$.
\item $\mathcal{E}^k(M) := \left\{\, \varphi : k\text{-forms on } M\ \text{with coeff} \in \mathcal E \,\right\}$ 普通は"$\mathcal{A}^k(M)$"である.
\item $\mathcal{K}^k(M) := \left\{\, \varphi \in C^k(M)\ \Big|\ \operatorname{supp}\varphi\ \text{cpt} \,\right\}$
ここで$\Supp\varphi :=\overline{ \{ x \in M,\ \varphi(x) \neq 0 \}}$
である. $\varphi(x) \neq 0 $とはある$J$があって, $\varphi_J(x) \neq 0$であることを意味する. 
\item $\mathcal{D}^k(M) := \left\{\, \varphi \in \mathcal E^k(M)\ \Big|\ \operatorname{supp}\varphi\ \text{cpt} \,\right\}$
\item $A \subset M$について以下のようにおく
$$
\mathcal{K}_{A}^k(M) := \left\{\, \varphi \in \mathcal{K}^k(M)\ \Big|\ \operatorname{supp}\varphi \subset A \,\right\}
\quad 
\mathcal{D}_{A}^k(M) := \left\{\, \varphi \in \mathcal{D}^k(M)\ \Big|\ \operatorname{supp}\varphi \subset A \,\right\}
$$
\end{itemize}


\[
\begin{array}{c|c|c}

& \text{連続}
& \text{$C^{\infty}$}
\\ \hline
\text{関数} 
&  C(M)
& \mathcal E(M) 
\\ \hline
\text{form} 
&  C^k(M)
& \mathcal E^k(M) 
\\ \hline
\text{form with  support compact}
&  \mathcal K^k(M)
& \mathcal D^k(M) 
\\ \hline
\text{form with  support $\subset A$} 
&  \mathcal K_{A}^k(M)
& \mathcal D_{A}^k(M) 
\\ \hline
\end{array}
\]


\underline{Recall}
% [page 1]

$U \subset \mathbb R^m$ open のとき、$C(U),\ \mathcal E(U)$ には  
次のような 位相をいれていた.

コンパクト集合の列:$K_1 \subset K_2 \subset \cdots \subset U$  
で$K_j \subset K_{j+1}^\circ$ かつ $U = \bigcup_j K_{j}^{\circ}$となるものを取り, $f \in \mathcal E(U)$ に対し

\[
P_N(f) := \max 
\Bigl\{\, |D^\alpha f(x)|\ \Big|\ x \in K_N,\ |\alpha| \le N,\ \omega \in K_N \Bigr\}
\]

\[
V_N := \Bigl\{\, f \in \mathcal E(U)\ \Big|\ P_N(f) < \tfrac1N \Bigr\}
\]
を open base at $0$(0での開基) とする位相を入れていた. 

\ref{lem-E-1}により, これがコンパクト集合の列や座標近傍$U \subset \mathbb R^m$ によらない  


一般の多様体に関しては$M$ に対しては、$M = \bigcup_j U_j$という座標近傍の被覆を使って
% [page 1]
$$
\mathcal E(M) \subset \prod \mathcal E(U_j)
$$
部分位相を入れた.
\begin{tcolorbox}[mybox]
\begin{defn}
\label{defn-E-3}
 $U \subset \mathbb R^m$ open に対し, $C^k(U),\ \mathcal E^k(U)$ に対して位相を
$$\mathcal E^k(U) \simeq \prod\limits_{J=(j_1<\cdots<j_k)} \mathcal E(U)\, d\bar z^J
$$ による  
直積位相を入れる. 
\end{defn}
\end{tcolorbox}
これは次のノルム
\[
P_N(\varphi) := \max 
\Bigl\{\, |D^\alpha \varphi_{J}(x)|\ \Big|\ x \in K_N,\ |\alpha| \le N,\ \omega \in K_N \Bigr\}
\]
について, $\mathcal E (U)$と同じように位相を入れたものに一致する.

\begin{tcolorbox}[mybox]
\begin{lem}
\label{lem-E-4}
\begin{enumerate}
\item$U \subset \mathbb R^m,\ U' \subset \mathbb R^m$ openについて, 
$\Phi : U' \to U$がdiffeoのとき
\[
\Phi ^{*} : \mathcal E^k(U) \xrightarrow{\ \sim\ } \mathcal E^k(U') 
\]
が$\C$上の位相ベクトル空間の同型を誘導する.
\item $U \subset \mathbb R^m$ open, $U = \bigcup_j U_j$:countable open coverについて, 
\[
\mathcal E^k(U) \hookrightarrow \prod_j \mathcal E^k(U_j)
\]
は部分位相空間となる. 
\end{enumerate}
\end{lem}
\end{tcolorbox}
\begin{proof}
(1) $\Phi : U'(x^1,\dots,x^m) \to U(y^1,\dots,y^m)$ diffeoとして

\[
\Phi^{*} : \mathcal E^k(U) := \prod_{J}\mathcal{E}(U)dy^{J} \ \longrightarrow\ \mathcal E^k(U'):=\mathcal{E}(U')dx^{J} 
\]
は以下のような形になる. 
\[
\varphi = \sum_J \varphi_J\, d y^J
\ \longmapsto\ 
\sum_I \Bigl( \sum_J  \Phi^{*}\varphi_J\, \frac{\partial y^J}{\partial x^I} \Bigr)\, d x^I
\]
となる. ここで
\[
\frac{\partial y^J}{\partial x^I} 
= \det \Bigl( \frac{\partial y^{j_k}}{\partial x^{i_l}} \Bigr)
\]
とする. 
よって\ref{lem-E-1}で$\varphi_J \mapsto \Phi^{*}\varphi_J$が連続であることを言っており, 和をとるのも連続なので, 連続性が言える.

(2):
$$\prod_I \mathcal E(U) dx^{I} \cong
\mathcal E^k(U) \ \hookrightarrow\
\prod_{i} \mathcal{E}^k (U_i) 
\underset{(1)}{\cong}
\prod_{i} \left( \prod_I \mathcal E(U)\, d\bar z^I \right)
\cong
\prod_I \Bigl( \prod_j \mathcal E(U_j)\, d\bar z^I \Bigr)
$$
となる. \ref{lem-E-1}でから, $\mathcal E(U)\subset  \prod_j \mathcal E(U_j)$が部分位相になっているのでいえた. 
% [page 2]
\end{proof}

\begin{tcolorbox}[mybox]
\begin{defn}
\label{defn-E-5}
$M$ 2nd countable $m$次元$C^\infty$級多様体について
$C^k(M),\ \mathcal E^k(M)$ に次のように位相を入れる:

$M = \bigcup U_j$を座標近傍$U_j$のcountable coveringとし, 
\[
\mathcal E^k(M) \hookrightarrow \prod_j \mathcal E^k(U_j)
\]
による部分位相を入れる. 

% [page 1]
$A \subset M$ compact subsetについて, 
$\mathcal{D}_A^k(M)  \subset \mathcal{E}^k(M) $はclosed subspaceである.(下のremark参照)
$\mathcal{D}_A^k(M)$には$\mathcal{E}^k(M)$のsubtopologyををいれる  



$\mathcal{K}^k(M),\mathcal{D}^k(M)$ には、
\begin{align*}
\mathcal B := 
&\Bigl\{\, W \subset \mathcal{D}^k(M)\ ;\ W\text{ = non-empty convex balanced} \\
 &\text{s.t.}\ \forall A \subset M\ \text{cpt},\
W \cap \mathcal{D}_A^k(M)\ \text{is open in }\mathcal{D}_A^k(M) \Bigr\}
\end{align*}
を0の local baseとなる位相を入れる.
\end{defn}

\end{tcolorbox}
\begin{rem}
\ref{lem-E-1}から$\mathcal E^k(M)$の位相は座標近傍の$\{U_j\}$ の取り方に依らない.
さらにこの位相は
\[
\mathcal E^k(M) \to \mathcal E^k(U) \quad\Big|\quad U \subset M\ \text{loc. coord.}
\]
に関する weak top. である(weak topologyに関しては\ref{lem-H-11}参照)

$\mathcal{D}_A^k(M)  \subset \mathcal{E}^k(M) $はclosed subspaceなのは
$x \in U \subset M$ loc coord. に対し, 代入写像
\[
ev_x \mathcal E^k(M) \to \C^{\oplus J}
\quad
\varphi \longmapsto (\varphi(x))_J
\]
が連続で, 
$\mathcal{D}_A^k(M) = \bigcap_{x \in M \setminus A} \Ker ev_x
$
とかけるのでclosed subspaceになる. 

$\mathcal{K}^k(M),\mathcal{D}^k(M)$の位相を詳しくいうと, \ref{defn-M-8.4}のように, 
集合族$\tau$を, "$\bigcup_{i \in I} (\varphi_i + W_i )$とかけるもの"の集まりとする. 
ただし$i \in I$について, $\varphi_i \in \mathcal{D}^k(M), W_i \in \beta$とする. 
\end{rem}

% [page 1]

\[
\begin{array}{c|c|c}
\text{dom} 
& \text{local }(U\subset \mathbb R^n) 
& \text{global }(M\text{ mfd.}) 
\\ \hline
\text{form} 
& C^k(U),\ \mathcal E^k(U) 
& C^k(M),\ \mathcal E^k(M) 
\\ \hline
\text{form support $\subset A$} 
& K_A^k(U),\ D_A^k(U) 
& K_A^k(M),\ D_A^k(M) 
\\ \hline
\text{form support compact} 
& K^k(U),\ D^k(U) 
& K^k(M),\ D^k(M)
\end{array}
\]

という対応がある. 

\begin{tcolorbox}[mybox]
\begin{prop}
\label{prop-E-6}
% [page 1]
$M$ 2nd countable$m$次元$C^\infty$級多様体について以下が成り立つ.
\begin{itemize}
\item $\mathcal E^k(M),\ D_{A}^k(M)$ locally convex, complete metrizable  
位相$\C$ ベクトル空間 Heine-Borel property.
\item  $C^k(M),\ K_{A}^k(M)$  locally convex, complete, metrizable  
位相$\C$ ベクトル空間.
\end{itemize}
\end{prop}
\end{tcolorbox}
\begin{proof}
$k=0$, $M = \cup_{i =1}^{\infty} U_{i}\subset \mathbb R^m$ などに関してはすでに示している

% [page 1]

\ref{Prop-M-7.3}によって$\mathcal E^0(U) = C^\infty(U)$ について, 
コンパクト集合の列$K_1 \subset K_2 \subset \cdots $であって, 
seminorm の separating family 
\[
P_N(f) = \max \{\, |D^\alpha f(x)|\ ;\ x \in K_N,\ |\alpha| \le N \}
\]
によって位相を入れると, metrizable 位相$\C$ ベクトル空間であることがわかる. 
(ここにopen coverがcountableがいる)

また\ref{Prop-M-7.4}でcomplete と Heine-Borel propertyを示した.
(ここに平均値の定理を使う. つまり微分可能性がいる. )
よって$C(U)$ しても同じで, Heine-Borel property以外 は 同様にできる。

% [page 1]

k>0,$\mathcal E^k(U),\ C^k(U)$ に関しては, seminormを
\[
P_N(\varphi) := \max \Bigl\{\, |D^\alpha \varphi_J(x)|\ \Big|\ 
x \in K_N,\ J=(j_1<\cdots<j_k),\ |\alpha| \le N \Bigr\}
\]
に変えれば同様に言える


$\mathcal E^k(M),\ C^k(M)$ に関しては 前回 \ref{lem-E-1} (2) で与えられた  
コンパクト集合の列をとる. 具体的に
$(M = \bigcup U_j $とし
$K_{i1} \subset K_{i2} \subset \cdots \subset U_i$となるコンパクト集合列について
$$
K_N := \bigcup_{i=1}^{N}K_{i, N}
$$
\[
P_N(\varphi) := \max \Bigl\{\, |D^\beta \varphi_J(x)|\ \Big|\ 
x \in K_{iN} \subset U_i,
J = (j_1<\cdots<j_k),\ |\omega| \le N \Bigr\}
\]
とすると$K_{1} \subset K_{2} \subset \cdots \subset M$であり, 同様の議論が回る. 

% [page 1]
$K_A^k(M),\ D_A^k(M) $は それぞれ$ C^k(M),\ \mathcal E^k(M) $の閉部分空間なので, 諸性質は保たれる.

\end{proof}


% [page 1]
\begin{tcolorbox}[mybox]
\begin{prop}(cf. \ref{thm-M-8.7})
\label{prop-E-7}
\begin{enumerate}[label=$(\alph*)$]
\item $V \subset D^k(M)$ convex balanced について, $V$がopenであることは, 任意のcompact $A \subset M$について, $V \cap D_A^k(M) \subset D_A^k(M)$であることと同値.
\item $A \subset M$ compactとしたとき,  
$D_A^k(M) \subset D^k(M)$部分位相空間.
\item $E \subset D^k(M)$ boundedならば, ある
compact $A \subset M$であって, $E \subset D_A^k(M)$となる. 
\item $D^k(M)$は Heine-Borel property を持つ.
\item $D^k(M) \cong \operatorname{colim}_{A\subset M\ \text{cpt}} D_A^k(M)$. ここでこの同型はlocally convex 位相$\C$ ベクトル空間としての同型である. 
\end{enumerate}
また上の主張は$D_A$を$K_A$に変えても成り立つ. 
\end{prop}
\end{tcolorbox}
\begin{proof}

[(a)] \ref{thm-M-8.6}が$D^k(M)$でも言える(\ref{thm-M-8.7} (a)の議論が回る)

[(b)] $D_A^k(M) \hookrightarrow D^k(M)$ は, (a)より連続である.

一方,$D^k(M) \hookrightarrow \mathcal E^k(M)$ も 連続である.
なぜならば$\mathcal E^k(M)$ は locally convex より, convex balanced な open base at 0 となるものがある.
それらをひとつ取っても convex balanced であり, 各 $D_A^k(M)$ に制限しても open なので, $D^k(M)$ の上で openとなる. 

[(c)] 対偶を示す. $E \subset D^k(M)$は
任意のcompact $A \subset M$について, $E \not \subset D_{A}^{k}(M)$を満たすとする.\footnote{ある$\varphi \in E$と$x \in A$であって, $\varphi(x) \neq 0$ということと同じ.}
この時$E$がboundedでないことを示す. 

$K_1 \subset K_2 \subset \cdots \subset M$というコンパクト集合であって
$$
K_i \subset K_{i+1}^{\circ} \quad \text{and} \quad M = \bigcup K_{i}^{\circ} 
$$ 
となるものをとる. 
$E$の仮定から, ある$\varphi_n  \in E$と$x_n \in K_n$であって, 
$\varphi_n (x_n) \neq 0$かつ$\{ x_{n} \}_{n \in \N}$は集積点を持たないものが取れる. 

そこで
$x_ n \in U_n \subset M$という局所座標をとって, 
$$
W :=  \{ \psi \mid \max_{J} | \psi_J (x_n)|  < \frac{1}{n}\max_{J}|\phi_{n, J}(c)|  \text{ for any $n$} \}
$$
とおく. ($\phi_n = \sum_{J} \phi_{n, J} dx^J$と分解する.)
$W$は convex balanced open in $\mathcal{D}^*(M)$である. 
なぜならば, convex balanced は $|\cdot|$ の性質から. 
任意のcompact $ A \subset M$ に対し,  $U_n \subset A$となる$U_n$ は
有限個である. そして
$W \cap \mathcal{D}^{k}_A(M)$ は open in $\mathcal{D}^*_A(M)$である. (開集合の有限この共通部分なので. )

しかし任意の$n \in \N$について, $\varphi_n \not \in nW$となるので, $E \not \subset nW$であり, $E$はboundedではない. 

[(d)]  
bounded  closed $E \subset \mathcal{D}^{k}(M)$ はcompactであることを言う. 
(c) より, あるcompact $A \subset M$で, $E \subset \mathcal{D}^{k}_A(M)$.
(b) より, $\mathcal{D}^{k}_A(M) \subset \mathcal{D}^k(M)$は部分位相が入るので, $E \subset \mathcal{D}^{k}_A(M)$ の中では bounded かつclosed.
よって\ref{prop-E-6}の$\mathcal{D}^{k}_A(M)$のHeine-Borel Propertyより言える. 

[(e)]
$\mathcal{D}^{k}(M) \xrightarrow{f} V$を
locally convex 位相$\C$ ベクトル空間の$\C$ 線型写像とする. 
すると次の同値変形ができる. 
\begin{align*}
& \text{$f$が連続である} \\
&\iff \text{$\forall U \subset V$ : convex balanced open
に対し, $f^{-1}(U)$ open} \\
&\iff \text{$\forall U \subset V$ : convex balanced open.
compact $A \subset M$に対し
$f^{-1}(U) \cap \mathcal{D}^{k}_A(M)$ open in $\mathcal{D}^{k}_A(M)$} \\
&\iff \text{$f|_{\mathcal{D}^{k}_A(M)} : \mathcal{D}^{k}_A(M) \to V$ 連続} \\
\end{align*}

map が $\mathbb{C}$-linearなことについても同様に言えるので, 
\[
\mathrm{Hom}\bigl(\mathcal{D}^{k}(M), V\bigr)
\simeq
\varprojlim_{A \subset M\ \mathrm{compact}}
\mathrm{Hom}\bigl(\mathcal{D}^{k}_A(M), V\bigr).
\]
\end{proof}

\begin{tcolorbox}[mybox]
\begin{defn}
\label{defn-E-8}
$\{V_\lambda \subset U_\lambda\}_{\lambda \in \Lambda}$ と書いた時, 次を意味する. 
\begin{itemize}
\item $U_\lambda = (U_\lambda; x_{\lambda}^1, \dots, x_{\lambda}^n)$: local coordinate of $M$
\item $V_\lambda \subset U_\lambda$: 相対コンパクトな開集合
(つまり $\overline{V_\lambda} \subset U_\lambda$ コンパクt)
\item $M = \bigcup_\lambda V_\lambda$かつ, $\{U_\lambda\}_\Lambda$ は locally finite open covering of $M$. (つまり, 任意の$x \in M$について,  ある$x$の近傍$V$で, $U \cap U_\lambda \neq \emptyset$となる$\lambda$は有限個)
\end{itemize}
\end{defn}
\end{tcolorbox}
上のような$V_\lambda$の存在に関しては, 多様体が2nd countableであることから. ( 2nd countable $\Rightarrow$ para-compact)
また定義から$\Lambda$は可算集合となる. 

また局所有限性から任意のcompact $A \subset M$に対して, 
$A \cap U_\lambda \neq \varnothing$ なる $\lambda$ は 有限である. 

\begin{proof}
もし無限になるなら, ある $x_i \in A \cap U_{\lambda_i}$($i=1,2,\dots$)
が取れる. 
$A$ compactなので, ある$x$に収束する部分列 $\{x_{i_k}\}$が取れる. 
すると$x$の任意の近傍$V$について, $V \cap U_{\lambda_{i_k}} \neq \varnothing$が言えて$\{V_\lambda\}$の仮定に矛盾する
\end{proof}

以下
% [page 7]
$U_\lambda$上の座標$(x_{\lambda}^1, \dots, x_{\lambda}^n)$と
$\alpha \in \mathbb{Z}_{\ge 0}^n$ について, 
$$
D_\lambda^\alpha
:= \left(\frac{\partial}{\partial x^{1}_{\lambda}}\right)^{\alpha_1}
\cdots
\left(\frac{\partial}{\partial x^{n}_{\lambda}}\right)^{\alpha_n}
$$

また$\varphi \in \mathcal{E}^k(M),\ \lambda \in \Lambda,\ \ell \in \mathbb{Z}_{\ge 0}$について, 
\[
\|\varphi\|_{\lambda}^{\ell}
:= \max \left\{
\left| D_\lambda^\alpha \varphi_{\lambda, J}(x) \right|
\ \middle|\
x \in \overline{V_\lambda},\ J,\ |\alpha| \le \ell
\right\}.
\]

最後に$\varphi \in \mathcal{D}^{k}(M),\ \ell \in \mathbb{Z}_{\ge 0}$について, 
\[
\|\varphi\|^{\ell}
:= \max_\lambda \|\varphi\|_{\lambda}^{\ell}
= \max \left\{
\left| D_\lambda^\alpha \varphi_{\lambda, J}(x) \right|
\ \middle|\
x \in \overline{V_\lambda},\ J, \ \lambda, \ |\alpha| \le \ell
\right\}.
\]
とする .
これらはseminormである. 

\begin{tcolorbox}[mybox]
\begin{lem}
\label{lem-E-9}
 $\{V_\lambda \subset U_\lambda\}$ を固定する. 
 この時
$$
\{\|\cdot\|_\lambda^\ell \mid \lambda \in \Lambda,\ \ell \ge 0\}
$$
はseparating family of seminorm
on $\mathcal{E}^k(M)$ であり, 
前に定めた位相(\ref{defn-E-5})と同じ位相を定める. 
\end{lem}
\end{tcolorbox}
\begin{proof}
Seminormであることはすぐにわかる. 
$$
V ( \lambda,\ell, \varepsilon)
:= \{\varphi \in \mathcal{E}^k(M) \mid \|\varphi\|_\lambda^\ell < \varepsilon\}
$$
とする. 
示すことは
$$\{\ \bigcap_{\text{finite}} 
V  ( \lambda_i,\ell_i, \varepsilon_i)
\mid \lambda_i \in \Lambda,\ \ell_i \ge 0,\ \varepsilon_i > 0 \,\}
$$
がlocal base at $0$であることを示せば良い. 

$V ( \lambda,\ell, \varepsilon)$がopenであることは, 
制限写像$\mathrm{res} : \mathcal{E}^k(M) \to \mathcal{E}^k(U_\lambda)$によって, 
$$\{\varphi \in \mathcal{E}^k(U_\lambda) \mid \|\varphi\|_\lambda^\ell < \varepsilon\}
\subset \prod_{J}\mathcal{E}^0(U_\lambda)
$$
のpullbackになるので良い. 

任意の$0 \in V \subset \mathcal{E}^k(M)$となるopenについて, 
ある$N \gg 0$と
compact subset
$K_N := K_{N, 1} \cup K_{N, 2} \cup \cdots \cup K_{N, N}$であって, 
\[
p_N(\varphi)
:= \max \left\{
\left| D^\alpha \varphi_{J}(x) \right|
\ \middle|\
x \in K_N,\ J,\ |\alpha| \le N
\right\}.
\]
としたとき, $V$は
\(
\left\{
\varphi \in \mathcal{E}^k(M)
\ \middle|\
p_N(\varphi) < \frac{1}{N}
\right\}
\)を含む. 
よって$K_N \subset \bigcup_{i \text{finite}} V_{\lambda_i}$をとり, 
$\ell_i \le N$ とし,
$0<\varepsilon_i \ll \frac{1}{N}$ をとると
\[
\bigcap_i V(\lambda_i,\ell_i,\varepsilon_i)
\subset
\left\{
\varphi \ \middle| \ p_N(\varphi) < \frac{1}{N}
\right\}
\subset V
\]
となりいえた. 
\end{proof}

\begin{tcolorbox}[mybox]
\begin{cor}
\label{cor-E-10}
compact $A \subset M$ について, 
$\{\|\cdot\|^{(\ell)} \mid \ell \ge 0\}$
は separating family of seminorms on
$\mathcal{D}^k_A(M)$ であり, 同じ位相を定める.
\end{cor}
\end{tcolorbox}
$\mathcal{K}^k_A(M)$についても, $\{\|\cdot\|^{(0)} \mid \ell \ge 0\}$を考えれば同様の主張が得られる. 


\begin{tcolorbox}[mybox]
\begin{prop}
\label{prop-E-11}
$\{V_\lambda \subset U_\lambda\}$を固定する. 
\begin{enumerate}
\item $\{\varphi_i\}$が$\mathcal{D}^k(M)$で Cauchy 列であることは, 
あるcompact 部分集合 $A \subset M$ で
$\{\varphi_i\} \subset \mathcal{D}^k_A(M)$かつ
$$\lim_{i,j \to \infty} \|\varphi_i - \varphi_j\|^{\ell} = 0
\quad \forall \ell \ge 0 $$
が成り立つことと同値. 
\item $\varphi_i \to \varphi$ in $\mathcal{D}^k(M)$は, 
あるcompact 部分集合  $A \subset M$ で
$\{\varphi_i\} \cup \{\varphi\} \subset \mathcal{D}^k_A(M)$
かつ
$$\lim_{i \to \infty} \|\varphi_i - \varphi\|^{\ell} = 0
\quad \forall \ell \ge 0.$$
が成り立つことと同値. 
\item $\mathcal{D}^k(M)$ は complete.
\end{enumerate}
\end{prop}
\end{tcolorbox}
$\mathcal{K}^k_A(M)$についても, $\{\|\cdot\|^{(0)} \}$を考えれば同様の主張が得られる. 

\begin{proof}

[(1)] $\{\varphi_i\}$ Cauchy 列は boundedである. 
よって\ref{prop-E-7}より,
あるcompact $A \subset M$であって, 
$\{\varphi_i\} \subset \mathcal{D}^k_A(M)$となる. 
これより
\begin{align*}
&\text{$\{\varphi_i\}$ Cauchy 列 in $\mathcal{D}^k(M)$} \\
&\iff 
\text{$\exists A \subset M$ compact s.t. $\{\varphi_i\} \subset \mathcal{D}^k_A(M)$ かつ Cauchy 列 in $\mathcal{D}^k_A(M)$} \\
&\underset{\ref{cor-E-10}}{\iff} 
\text{$\exists A \subset M$ compact s.t.  $\{\varphi_i\} \subset \mathcal{D}^k_A(M)$ かつ
\(
\lim_{i,j \to \infty} \|\varphi_i - \varphi_j\|^\ell = 0
\quad \forall \ell \ge 0.
\) }
\end{align*}

% [page 14]
[(2)] $\{\varphi_i\} \cup \{\varphi\}$ は bounded より,
(1) と同様. 

[(3)] (1) と $\mathcal{D}^k_A(M)$ は complete より,
任意のCauchy 列は収束する. 
\end{proof}


\begin{tcolorbox}[mybox]
\begin{defn}
\label{defn-E-12}
連続線型写像 
$T : \mathcal{D}^k(M) \to \mathbb{C}$
のことを $k$-dimensional current on $M$ という. 
特にdistribution は $0$-dimensional current のことをさす. 
$$
\mathcal{D}'_k(M) := \{\, k\text{-current on } M \,\}
= \mathrm{Hom}_{\text{top } \mathbb{C}\text{-vect sp}}
\bigl(\mathcal{D}^k(M), \mathbb{C}\bigr)
$$
$$\mathcal{K}'_k(M)
:= \mathrm{Hom}_{\text{top } \mathbb{C}\text{-vect sp}}
\bigl(\mathcal{K}^k(M), \mathbb{C}\bigr).
$$
\end{defn}
\end{tcolorbox}



\begin{tcolorbox}[mybox]
\begin{prop}[cf. \ref{prop-M-8.9}, \ref{prop-M-8.11}]
\label{prop-E-13}
$Y$をlocally convex 位相 $\C$-ベクトル空間, 
$T : \mathcal{D}^k(M) \to Y$を
$\mathbb{C}$線型写像とする. 
次は同値である. 
\begin{enumerate}[label=$(\alph*)$]
\item  $T$ は 連続
\item $T$ は bounded, つまり bounded set を bounded set にうつす.
 \item $\varphi_i \to 0$ in $\mathcal{D}^k(M)$
ならば, 
$T(\varphi_i) \to 0$ in $Y$.
\item 任意のcompact $A \subset M$について, 
$T|_{\mathcal{D}^k_A(M)} : \mathcal{D}^k_A(M) \to Y$は連続. 
\item ($Y = \mathbb{C}$ の場合のみ) 任意のcompact $A \subset M$について, 
ある$ \ell \ge 0, C > 0$があって, 
\[
|T(\varphi)| \le C \cdot \|\varphi\|^{\ell}
\quad \text{for } \forall \varphi \in \mathcal{D}^k_A(M).
\]
\end{enumerate}
\end{prop}
\end{tcolorbox}
$\mathcal{K}^k_A(M)$についても, $\ell=0$のみを考えれば同様の主張が得られる. 

\begin{proof}

[$(a) \Leftrightarrow (b) \Leftrightarrow (c) \Leftrightarrow (d)$]
$(a),(b),(c)$ の主張において, 
$$\mathcal{D}^k(M) \mapsto \mathcal{D}^k_A(M)
\quad 
T \mapsto T|_{\mathcal{D}^k_A(M)}
$$
に取り替えた主張を
$(a)_A ,(b)_A (c)_A$
とする. 
\ref{prop-E-6}より$\mathcal{D}^k_A(M)$ は metrizable より
\ref{prop-M-4.2}より, 
$$
(a)_A \iff (b)_A \iff (c)_A
$$
となる. 
また\ref{prop-E-7} (e)から
$$
(d) \text{が成立} \iff 
\forall A \subset M \text{ compact, }(a)_A \text{が成立}
$$
となる. よって$(a)$と$(d)$が同値になる. 
同様にしてこれらは$(b)$や$(c)$と同値となる. 
%同様にして, 
%(b) $\iff \forall A$ cpt. (b)$_A$,
%(c) $\iff \forall A$ cpt. (c)$_A$.
%よって,(a)~(d) は同値。
% [page 19]

[$(e) \Rightarrow (d)$] $Y=\mathbb{C}$ とする. 
compact $ A \subset M$ をとる. 
仮定より $\ell \ge 0, C > 0$があって, 
\[
|T(\varphi)| \le C \cdot \|\varphi\|^{\ell}
\quad \text{for } \forall \varphi \in \mathcal{D}^k_A(M).
\]
$\forall \varepsilon > 0$ に対し,
\[
V := \left\{
\varphi \in \mathcal{D}^k_A(M)
\ \middle|\
\|\varphi\|^{\ell} < \frac{\varepsilon}{C}
\right\}
\]
とおくと, これは $0$ を含む open in $\mathcal{D}^k_A(M)$であり, 
\[
 \varphi \in V \Rightarrow
\bigl|T|_{\mathcal{D}^k_A(M)}(\varphi)\bigr| < \varepsilon
\]
である. よって $T|_{\mathcal{D}^k_A(M)}$ は連続である. 

[$(d) \Rightarrow (e)$]
$A \subset M$ compactとすると, 
仮定より $T|_{\mathcal{D}^k_A(M)}$ は連続. 
よって\ref{cor-E-10}から, $\ell \ge 0, \varepsilon > 0$があって, 
\[
T\bigl(\{\varphi \in \mathcal{D}^k_A(M)\mid \|\varphi\|^{\ell}<\varepsilon\}\bigr)
\subset \{z\in\mathbb{C}\mid |z|<1\}.
\]
よって任意の$\varphi \in \mathcal{D}^k_A(M)$,$\varphi\neq 0$ に対し,
\[
\left|T\left(\frac{\varepsilon}{2\|\varphi\|^{\ell}}\varphi\right)\right|<1.
\]
であるので整理して, 
\(
|T(\varphi)|<\frac{2}{\varepsilon}\,\|\varphi\|^{\ell}
\)となる. よって$C=\frac{2}{\varepsilon}$とおけば良い. 
\end{proof}

\begin{rem}
$k$ dimensional current $\mathcal{D}'_k(M)$のことを$\dim M -k$ degree currentともいう.(かなりややこしい)

$Z \in H_{k}(M, \Z)$の元は積分することで,  $\mathcal{D}'_k(M)$の元となる. (これが"dimensional"の由来だと思われる)また$\dim M - k$次微分形式は外積をとって積分することで$\mathcal{D}'_k(M)$の元となる.

\end{rem}


\section{カレントのorderとorder 0 カレントの特徴づけ}

引き続き$M$を$m$次元 2nd contable 実多様体とする. 

示すことは
\[
\mathcal{L}_{\mathrm{loc}}^{m-k}(M)
\hookrightarrow \mathcal{K}'_{k}(M)
\cong 
D^{'}_k(M)_{\text{ord}=0}
\hookrightarrow  D^{'}_{k}(M)_{\text{ord}\le l}
\hookrightarrow D^{'}_k(M).
\]
そしてこれが$M$の位相$\C$ベクトル空間のsheafでの完全列であることを示す. 

\begin{tcolorbox}[mybox]
\begin{defn}
\label{defn-E-14}
$D_k'(M)$に位相を
$$
\{ ev_{\varphi} :  D^{'}_k(M) \to \C  \mid \varphi \in D^k(M) \}
$$
に関するweak topologyを入れる. 
つまり
$$
\{ ev^{-1}_{\varphi_1}(B_1) \cap \cdots \cap ev^{-1}_{\varphi_l}(B_l)  \mid \varphi_1, \ldots, \varphi_l \in D^k(M), 0 \in B_1, \ldots, B_l \subset \C \text{ open ball } \}
$$
をlocal baseとする位相を入れる.
\end{defn}
\end{tcolorbox}
\begin{rem}
これは各点収束な位相である. 次の同値変形からわかる. 
\begin{align*}
&T_i \to T\quad (i\to\infty) \\
&\iff
\forall U=\mathrm{ev}_{\varphi_1}^{-1}(B_1)\cap \cdots \cap \mathrm{ev}_{\varphi_\ell}^{-1}(B_\ell), \exists N,\ \forall n\ge N,\ T_n-T\in U \\
&\iff
\forall U=\mathrm{ev}_{\varphi_1}^{-1}(B_1)\cap \cdots \cap \mathrm{ev}_{\varphi_\ell}^{-1}(B_\ell), \exists N,\ \forall n\ge N, T_n(\varphi_j)-T(\varphi_j)\in B_j  (\forall j=1,\ldots,\ell) \\
&\iff
\forall \varphi\in D^k(M),\ \forall \varepsilon>0, 
\exists N,\ \forall n\ge N,\ |T_n(\varphi)-T(\varphi)|<\varepsilon \\
&\iff\forall \varphi\in D^k(M),
T_i(\varphi) \to T(\varphi)\quad (i\to\infty) 
\end{align*}

\end{rem}
さて 開集合$U\subset M$ に対し, 
restriction map $D_k'(M)\to D_k'(U)$ を
\[
T\mapsto \Bigl(D^k(U)\hookrightarrow D^k(M)\xrightarrow{T}\C\Bigr)
=:T|_U
\]
これは連続な$\C$-linear mapとなる. 

\begin{tcolorbox}[mybox]
\begin{lem}
\label{lem-E-15}
 $U\subset M\ \mapsto\ D_k'(U)$によって, $M$上の
位相$\C$ベクトル空間のsheafを定める. 
\end{lem}
\end{tcolorbox}
\begin{proof}
% [page 1]
$\{U_\lambda\}_{\lambda\in\Lambda}$ locally finite open cover of $M$, 
$\{\psi_\lambda\}_{\lambda\in\Lambda}$1の分割とする. 

[1. ] $T\in D_k'(M)$, $T|_{U_\lambda}=0\ \forall\lambda$ ならば,
$T=0$を示す. 
これは
$\varphi\in D^k(M)$ について, 
\begin{align*}
T(\varphi)
&=T\Bigl(\sum_\lambda \psi_\lambda\cdot \varphi\Bigr) =\sum_\lambda T(\underbrace{\psi_\lambda \varphi}_{\in D^k (U_\lambda)}) 
=\sum_\lambda \underbrace{T|_{U_\lambda}}_{=0}(\psi_\lambda\varphi) 
 =0
\end{align*}

[2.]
$ T_\lambda\in D_k'(U_\lambda)$かつ
\[
T_\lambda|_{U_\lambda\cap U_\mu}=T_\mu|_{U_\lambda\cap U_\mu}
\]
ならば, ある$T \in D_k'(M)$があって$T|_{U_{\lambda}}=T_{\lambda}$となること. 

$T\in D_k'(M)$ を, $\varphi\in D^k(M)$ に対し
\[
T(\varphi):=\sum_\lambda \underbrace{T_\lambda(\psi_\lambda\cdot \varphi)}_{\text{有限こを除いて0}}
\]
と定義する. 
すると, $T$ は $\C$-linear である. 

また$\varphi_i\to 0\ (i\to\infty)$ のとき, \ref{prop-E-11}から
ある$A \subset M$ コンパクトがあって, $\supp(\varphi_i) \subset A$かつ, 
\[
T(\varphi_i)=\sum_\lambda T_\lambda(\psi_\lambda\varphi_i)\to 0
\qquad (i\to\infty)
\]
となる.
(この$i \to\infty$は$\supp(\varphi_i) \subset A$なので, $\lambda$によらずに取れる. )
よって$T$は連続.

また $\varphi\in D^k(U_\lambda)$ に対し,
\[
T(\varphi)
=\sum_\mu T_\mu(\underbrace{\psi_\mu\varphi}_{\in D^k (U_{\lambda} \cap U_{\mu})})
\underbrace{=}_{T_\lambda|_{U_\lambda\cap U_\mu}=T_\mu|_{U_\lambda\cap U_\mu}}
\sum_\mu T_\lambda(\psi_\mu\varphi)
=T_\lambda\Bigl(\sum_\mu \psi_\mu\varphi\Bigr)
=T_\lambda(\varphi)
\]

よって$T|_{U_\lambda}=T_\lambda$
であり, $D_k'$は$M$上のsheafである. 
\end{proof}


\begin{tcolorbox}[mybox]
\begin{defn}
\label{defn-E-16}
$T\in D_k'(M)$ がorder $\le \ell$であるとは, 
任意のコンパクト$A\subset M$について, ある$\exists C>0$があって次を満たすこと. 
\[
|T(\varphi)|\le C\cdot \|\varphi\|^\ell
\qquad
\forall \varphi\in D_A^k(M).
\]

また$T$がorder $\ell$であることを, $T$がorder $\le \ell$ かつ order $\le \ell-1$ではないとして定める. 
\end{defn}
\end{tcolorbox}

\begin{rem}

これは開集合の取り方によらない. 
これは二つの開被覆
$\{ V_\lambda\subset U_\lambda\}$, $\{ V_\lambda\subset U_\lambda\}$をとり, それに対応するnormを
$\|\varphi\|^\ell,\ \|\varphi\|^{\prime\ell}$とすると, 
任意のコンパクト$A\subset M$について, ある$\exists D>0$があって
$$
\|\varphi\|^\ell \le D\cdot \|\varphi\|^{\prime\ell}
\qquad
\forall \varphi\in D_A^k(M).
$$
となるので. 
\end{rem}

\begin{tcolorbox}[mybox]
\begin{lem}
\label{lem-E-17}
\begin{enumerate}
\item $\{U_\lambda\}_{\lambda\in\Lambda}$ locally finite  open cover of $M$とするとき, 
\begin{center}
$T\in D_k'(M)$ は order $\le \ell$
$\iff$
$T|_{U_\lambda}\ \text{は order }\le \ell\quad \forall\lambda.$
\end{center}
\item $T_i\in D_k'(M)$ order $\le \ell$, $T_i\to T\ (i\to\infty)$
のとき
$T\ \text{も order }\le \ell.$
\end{enumerate}

\end{lem}
\end{tcolorbox}
\begin{proof}
(1)
$\Rightarrow$の証明は明らか.
$\Leftarrow$を示す. 
$\{\psi_\lambda\}$を1の分割とする.
仮定より, $\forall A$ : cpt. $\exists C_\lambda>0$
\begin{equation}
\label{eq-partition}
|T(\psi_\lambda\cdot \varphi)|\le C_\lambda\cdot \|\psi_\lambda\varphi\|^\ell
\qquad
\forall \varphi\in D_A^k(M).
\end{equation}
そこで, 
$A\cap U_\lambda\neq\emptyset$ なる $\lambda$ を
$\lambda_1,\ldots,\lambda_N$ とおくと,
$\lambda_1,\ldots,\lambda_N$ 以外では $\psi_\lambda\varphi=0$である. 
よって
\[
C:=\sum_{i=1}^N C_{\lambda_i}
\]
とおけば, 
\[
|T(\varphi)|
\underset{\text{$A$ compact}}{=}
\left|\sum_{i=1}^N T(\psi_{\lambda_i}\varphi)\right|
\le \sum_{i=1}^N |T(\psi_{\lambda_i}\varphi)|
\underset{\ref{eq-partition}}{\le} \sum_{i=1}^N C_{\lambda_i}\cdot \|\psi_{\lambda_i}\varphi\|^\ell
\le C\cdot \|\varphi\|^\ell.
\]


% [page 1]

(2) 次のBanach--Steinhaus theorem(一様有界性定理)を使う.

\begin{tcolorbox}[mybox]
\begin{thm}\cite[Theorem 2.6]{Rud}
$X$を$F$-space, つまり位相ベクトル空間で位相とcompatibleなcomplete invariant metric
を持つものとし, 
$Y$を位相ベクトル空間とする. 

$\Gamma\subset \Hom_{\text{top vect sp}}(X,Y)$ とする. 
$ x\in X$ に対して,
\[
\Gamma(x):=\{\Lambda(x)\mid \Lambda\in\Gamma\} \subset Y
\]
が$Y$上でboundedならば, 
$\Gamma$ はequi-conti, つまり
任意の0の近傍$0 \in W\subset Y$について, ある近傍$0 \in V\subset X$があって, 
$$
\forall \Lambda\in\Gamma \Rightarrow \Lambda(V)\subset W
$$
\end{thm}
\end{tcolorbox}

% [page 1]

これを $X=(D_A^k(M),\|\cdot\|^\ell)$, $Y=\C$, $\Gamma=\{T_i\}_i$ に用いる. 
ここで, $\varphi\in D_A^k(M)$ならば, 仮定より $T_i(\varphi)\to T(\varphi)\ (i\to\infty)$なので, 
特に
$$
\{T_i(\varphi)\mid i \in \N\} \subset \C \text{ bounded} 
$$
である. 
Banach--Steinhaus theorem(一様有界性定理)より, 
$\Gamma$ は equi-contiである. 

\ref{prop-E-13} (d) $\Rightarrow$ (e)の証明と同様に
\[
W=\{z\in\C\mid |z|<1\},
\qquad
V=\{\varphi\in D_A^k(M)\mid \|\varphi\|^\ell<\varepsilon\}, 
\qquad
 C:=\frac{2}{\varepsilon}
 \]とすると
\[
|T_i(\varphi)|\le C\cdot \|\varphi\|^\ell
\qquad
\text{for }\forall i,\ \forall \varphi\in D_A^k(M).
\]
% [page 1]

よって, $\forall \varphi\in D_A^k(M)$ に対し,
\[
|T(\varphi)-T_i(\varphi)|\le \|\varphi\|^\ell
\quad \text{for } i\gg 0.
\]
ととると, 
\[
|T(\varphi)|
\le |T(\varphi)-T_i(\varphi)|+|T_i(\varphi)|
\le \|\varphi\|^\ell + C\cdot \|\varphi\|^\ell
=(1+C)\|\varphi\|^\ell.
\]
よって $T$ も order $\le \ell$である. 
\end{proof}

\ref{lem-E-17}から$D_{k,\mathrm{order}\le \ell}'\subset D_k'$ は
closed 位相$\C$ベクトル空間からなる subsheafである. 

さて$K'_{k}(M)\to D'_k(M)$ を
\[
T\mapsto \Bigl(D^k(M)\underset{\text{cont.}}{\hookrightarrow} K^k(M)\xrightarrow{T}\C\Bigr)
\]
として定める. これは連続$\C$線型で
またorder $=0$ である.(\ref{cor-E-10}の周り参照)
restriction map とも可換なので, 
sheafとしての写像$K'_k\to D'_{k,\mathrm{order}=0}$
が定める. 


% [page 1]
\begin{tcolorbox}[mybox]
\begin{lem}
\label{lem-E-18}
$K'_k \to D'_{k,\mathrm{order}=0}$
位相$\C$ベクトル空間のsheafの同型を与える.
\end{lem}
\end{tcolorbox}
\xr{同相になる??}
\begin{proof}
\(
M=U\subset \R^m.
\)と仮定して良い(ここに$D'_k(M) \subset \prod D'_k(U_{\lambda})$によって$D'_k(M)$の位相は$\prod D'_k(U_{\lambda}$の部分位相)となることを使う. 

$T\in D_{k,\mathrm{order}=0}'(U)$ をとる.
示すことは次のとおり. 
\[
\xymatrix@C=50pt@R=20pt{
D^k(U) \ar[r]^{T} & \C \\
K^k(U) \ar@{-->}[ru]_{\exists !} \ar@{^{(}->}[u] & \\
}
\]

$\varphi\in K^k(U)$をとり, 
$\varphi=\sum_J \varphi_J\,dx^J$ と書く.
すると各 $J$ に対し, \ref{lem-H-2.12}(\cite[Definition 6.31]{Rud}あたり)からtest function のCauchy 列 $\{\varphi_{i,J}\}\subset D^0(U)$であって, 
\[
\lim_{i\to\infty}\varphi_{i,J}=\varphi_J \ \text{in } K^0(U)
\]
となるものが取れる. 
そこで, 
\[
\varphi_i:=\sum_J \varphi_{i,J}\,dx^J\in D^k(U)
\]
とおくと,これは $D^k(U)$ の Cauchy 列であって
\(\{T(\varphi_i)\}_i \) 
も また$\C$上のCauchy 列となる.

% [page 1]
そこで$\varphi\in K^k(U)$について,
\[
T(\varphi):=\lim_{i\to\infty}T(\varphi_i)
\]
と定義する. 
これは $\{\varphi_i\}$ のとり方によらない.
なぜならば, 
二つのCauchy 列$\{\varphi_i\},\{\varphi_i'\}$について, 
\[
\varphi_{i,J}-\varphi_{i,J}'\to 0\ (i\to\infty).
\underset{}{\Rightarrow} \varphi_i-\varphi_i'\to 0\ (i\to\infty).
\underset{\text{$T$ 連続}}{\Rightarrow} T(\varphi_i)-T(\varphi_i')\to 0\ (i\to\infty).
\]
となるためである. 
よって$\C$-linear map$T:K^k(U)\to \C$が定義される.

連続性を示す.
コンパクト集合$A \subset U$について, $C_A>1$ を
\begin{equation}
\label{eq-K0}
|T(\psi)|\le C_A\|\psi\|^0
\qquad
\forall \psi\in D_A^k(U)
\end{equation}
となるものとする. (これは$T\in D_{k,\mathrm{order}=0}'(U)$なので取れる)
すると$\varphi\in K_A^k(U)$ に対し, ある $A \subset A'$なるコンパクト集合と,  
$\{\varphi_i\}\subset D_{A'}^k(U)$で,  $\varphi_i\to \varphi\ (i\to\infty)$
となるCauchy 列が存在する.

$T$の定義とCauchy 列から
$$
T(\varphi)-T(\varphi_i) \to 0 \quad
\|\varphi_i - \varphi\|^0 \to 0 
\quad (i \to \infty)
$$
なので$i\gg 0$とすれば
\[
|T(\varphi)|
\underset{i \gg 0}{\le} |T(\varphi)-T(\varphi_i)|+|T(\varphi_i)|
\underset{\text{above}}{\le} \|\varphi\|^0+|T(\varphi_i)|
\underset{\ref{eq-K0}}{\le}  \|\varphi\|^0 + C_{A'}\|\varphi_i\|^0
\underset{\|\varphi_i\|^0 \le 2 \|\varphi\|^0}{\le} (2C_{A'}+1)\cdot \|\varphi\|^0.
\]
これより一意性もわかる. $\C$線型同型がいえた. 

\end{proof}




