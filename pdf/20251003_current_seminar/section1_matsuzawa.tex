\chapter{超関数の定義}
\section*{Introduction}

 \(\quad \Omega \subset \mathbb{R}^n\) を開集合として次を定義する. 
\begin{itemize}
\item  \(C^{\infty}(\Omega) := \{ \varphi : \Omega \to \C  \mid  \text{$\varphi$は$C^{\infty}$級} \}\)
\item  \(\mathcal{D}(\Omega) = \{ \varphi \in C^\infty(\Omega) \mid \mathrm{Supp} \ \varphi \ \text{がcompact} \}\)
\end{itemize}
Distributionとは \(\Lambda : \mathcal{D}(\Omega) \to \mathbb{C} \) で\( \mathbb{C}\)-線型  
かつ連続となるものである. 
ここで, 連続とは"$\mathcal{D}(\Omega) $上で\(\varphi_i \to \varphi \)ならば
\(\Lambda(\varphi_\nu) \to \Lambda(\varphi)\) "となることを意味する. 
%\[\int_\Omega \varphi_\nu \, d\mu \quad \dots \dots \quad \int_\Omega f \varphi \, d\mu\]

そのためには, $\mathcal{D}(\Omega) $に次を満たす位相を入れる必要がある:
\begin{tcolorbox}[mybox]
\(\mathcal{D}(\Omega)\) で\(\varphi_i \to \varphi\)であることは, 
あるコンパクト集合\(K \subset \Omega\) があって, \(\mathrm{Supp} \ \varphi_i  \subset K\)であり, 
任意の$\alpha =  (\alpha_1, \dots, \alpha_n)$について,  一様に
\(D^\alpha(\varphi_i- \varphi) \to 0 \)となる.   
\end{tcolorbox}
(野口-落合の本\cite{NO}では, これが既知として書かれていた. ただ収束からは位相がただ一つに定まらない.)
この章では, \(\mathcal{D}(\Omega)\) に収束が上を意味するような位相をいれる.  
%\(\mathcal{D}(\Omega)\) は \(\mathbb{C}\)-vec sp

\begin{thebibliography}{n}
\bibitem[Rud]{Rud}
W. Rudin. \textit{Functional analysis.} 2nd edn. International Series in Pure and Applied Mathematics. McGraw-Hill, Inc., New York. (1991.)
\bibitem[NO]{NO}
J. Noguchi, T.Ochiai \textit{Geometric Function Theory in Several Complex Variables} Translations of Mathematical Monographs
Volume: 80; 1990; 282 pp
\end{thebibliography}
 
 \cite[Chapter 1, 6]{Rud}, \cite[Chapter 3]{NO}を主に参考にしている.
 
\section{Topological vector spaces}
%\(\S 1\) Topological vector spaces  

\(\mathbb{K} = \mathbb{R} \text{ or } \mathbb{C}\), $|\cdot|$を絶対値とする.

\begin{tcolorbox}[mybox]
\begin{defn}
\label{defn-M-1.1}
\(X\)を\(\mathbb{K}\)上のベクトル空間,  \(\tau \)を$X$の位相とする.
  
\((X,\tau)\)が\underline{topological vector space(位相ベクトル空間)}とは次を満たすこと.   
\begin{enumerate}
\item ($T_1$条件) 任意の$x \in X$について, $\{x\} \subset X$がclosed.
\item 加法$X \times X \to X, \ (x,y) \mapsto x+y$, スカラー倍$\mathbb{K} \times X \to X, \ (\alpha,x) \mapsto \alpha x$が(積位相に関して)連続. 
\end{enumerate}
\end{defn}
\end{tcolorbox}

\begin{rem}
位相ベクトル空間はHausdorff.
\begin{proof}
\(f : X \times X \to X, \ f(x,y) =  x-y\)とおくと(2)より連続.  
(1)より$\{ 0\}$は閉集合. よって$\Delta=f^{-1}(\{ 0\}) \subset X \times X$も閉. よってHausdorff.
\end{proof}
\end{rem}

\begin{tcolorbox}[mybox]
\begin{defn}
\label{defn-M-1.2}
\(X\): \(\mathbb{K}\)上のベクトル空間. 以下の用語を定義する.

\begin{enumerate}
\item 部分集合\(E \subset X\)が\underline{convex}とは, 任意の$t \in (0,1)$について, 
$tE + (1-t)E \subset E$となること. (もっと具体的に書くと, 
任意の\(x,y \in E, t \in (0,1)\)について, $t x + (1-t) y \in E$となること.)
\item  部分集合\(E \subset X\)が\underline{balanced}とは, 任意の
\(\alpha \in \mathbb{K}, \ |\alpha| \le 1\)について, $\alpha E \subset E$となること. このとき$0 \in 0\cdot E \subset E$である. 
\item  \(X\)を\(\mathbb{K}\)上の位相ベクトル空間とする. 
部分集合\(E \subset X\) が\underline{bounded}とは, 任意の0を含む開集合$ V \subset X$について, ある$0 < t_0 \in \mathbb{R}$があって, 任意の$t \ge t_0$について, $E \subset tV$となること. 
\item \( d : X \times X \to \mathbb{R}_{\ge 0}\)を\(X\)  の距離とする. 
\(d\)が\underline{invariant metric}であるとは, 任意の$x,y,z \in X$について, $d(x+z, y+z) = d(x,y)$を満たすこと.
\end{enumerate}
\end{defn}
\end{tcolorbox}
定義がややこしいが, convexやbalancedな集合は$\R^n$のopen ballの代わりの役割を果たしていく. 

\begin{rem}
Convex, boundedはtranslation invariantである. つまり, $E$がconvexやboundedならば, 任意の$a \in X$について$a+E$もそうなる. 
\begin{proof}
以下$E \subset X$, $a \in X$とする. 

[Convexの場合] 
\(E\) convexとする. \ref{defn-M-1.2}から任意の$t \in (0,1)$について, $tE + (1-t)E \subset E$である. よって任意の$t \in (0,1)$について
\[
t(a+E) + (1-t)(a+E) = a + tE + (1-t)E \subset a+E
\]
となりいえた. 

[Boundedの場合] 
\(E\) boundedとする. \ref{defn-M-1.2}から \(0 \in V \subset X\) openがあって, 任意の$t \gg 0$について\footnote{"任意の$t\gg0$について"とは, "ある$0 < t_0 \in \R$があって, 任意の$t \ge t_0$について”を意味する.}, $E \subset tV$である.
%$0 < t_0 \in \R$があって, 任意の$t \ge t_0$について$E \subset tV$である. よって, 

よって示すことは任意の$t\gg0$について$a + E \subset tV$である. 
これは任意の$t\gg0$について\( \frac{1}{t} a + \frac{1}{t} E \subset V \)を示せば良い.
そこで次の合成写像を考える:
\[
\xymatrix@C=20pt@R=5pt{
F:K\times X\times (K\times X)\ar[r]^-{f} 
&X\times  X \ar[r]^-{T} &X
\\
(a,x,b,y)\ar@{|->}[r]
&(ax, by)\ar@{|->}[r] &(ax + by)
}
\]
すると次がわかる.
\begin{enumerate}
\item $0 \in V$は開集合なので, $(0,0) \in U \times U \subset X$となる開集合$U \subset X$があって,
$U \times U \subset T^{-1}(V)$となる.(積位相の定義と$T$の連続性)
\item $E$ boundedより, $t \gg 0$について$\frac{1}{t}E \subset U$. よってある$\varepsilon>0$があって, $\varepsilon E \subset U$となる. 
\end{enumerate}
以上を組み合わせると($\varepsilon$は必要に応じて小さくして), 
$ (0, \varepsilon) \times \{a \} \times (0, \varepsilon) \times E \subset  F^{-1}(V)$となる.
これは$t \gg 0$について
$$
\frac{1}{t} a + \frac{1}{t} E \subset V 
$$
を意味する. 
\end{proof}
\end{rem}

\begin{tcolorbox}[mybox]
\begin{defn}
\label{defn-M-1.3}
\(X\)を\(\mathbb{K}\)上の位相ベクトル空間とする. 
$X$の\underline{local basis}とは, 原点\(0 \in X\)でのlocal basis(開基)のこと, つまり
$0$の開近傍からなる集合系$\mathcal{B}$で, 「任意の\(0 \in U \subset X\) openについて, ある$V \in \mathcal{B}$があって, $0 \in V \subset U$」となる集合系のこと. 
\end{defn}
\end{tcolorbox}

\begin{tcolorbox}[mybox]
\begin{defn}
\label{defn-M-1.4}
\(X\)を\(\mathbb{K}\)上の位相ベクトル空間とする. 
\begin{enumerate}
\item \(X\)が\underline{locally convex}とは,  \(X\)が$0$のconvexな開近傍からなるlocal basisを持つこと. 
\item \(X\)が\underline{locally bounded}とは, \(0\) が boundedな開近傍を持つこと. 
\item \(X\)が\underline{locally compact}とは, ある\(0\)の開近傍 \(0 \in V \subset X\)で, \(\overline{V}\)がcompactなものがあること. (通常のlocally compactと同じ)
\item \(X\)が\underline{metrizable}とは, ある距離\(d : X \times X \to \mathbb{R}_{\ge 0}\) があって, $d$の位相が \(X\) の位相と同じであること. 
\item \(X\) が \underline{\(F\)-space}とは, ある完備なinvariant距離\(d : X \times X \to \mathbb{R}_{\ge 0}\)があって, $d$の位相が \(X\) の位相と同じであること. 
\item \(X\)が \underline{Fréchet space}とは,  \(X\) が locally convex  かつ\(F\)-spaceなること. 
\item \(X\)が \underline{Heine-Borel Propertyを持つ}とは,  任意のclosed bounded が compactなること. ("$\R^n$の有界閉集合はコンパクト"というHeine-Borelの定理から来ている. )
\end{enumerate}
\end{defn}
\end{tcolorbox}
これは後々使っていく. \ref{prop-M-3.1}で「locally compact \(\Rightarrow\) 有限次元」や「locally bounded + Heine-Borel Property \(\Rightarrow\) 有限次元」を示す. 
なので, locally compactなどは滅多に起こらないということである. 

%2025/08/28 次はここっから
\section{Separation properties }
%\S 2 Separation properties  

\begin{tcolorbox}[mybox]
\begin{prop}
\label{prop-M-2.1}
\(X\)を\(\mathbb{K}\)上の位相ベクトル空間とする. 
\(K \subset X\)をcompact, \(C \subset X\)をclosedとする.  
\(K \cap C = \varnothing\) ならば, あるopen set \( V \subset X\)で \((K+V) \cap (C+V) = \varnothing\)となるものが存在する. 
\end{prop}
\end{tcolorbox}

\begin{proof}
\(\alpha : X \times X \times X \to X\)を \(\alpha(x, y,z):=x+y-z\)とおく. 
\(K \cap C = \varnothing\)より, $K \times \{ 0\} \times \{ 0\} \subset \alpha^{-1}(X \setminus C)$である. 
$\alpha$連続で, $X \setminus C$openなので, $\alpha^{-1}(X \setminus C)$もopen. 
よって, $K$がcompactなので, あるopen$V \subset X$で
$$
K \times V \times V \subset \alpha^{-1}(X \setminus C)
$$
となるものが存在する. よって\( K+V - V \subset X \setminus C\)であり,  \((K+V) \cap (C+V) = \varnothing\)となる. 
\end{proof}

\begin{tcolorbox}[mybox]
\begin{prop}
\label{prop-M-2.2}
\(X\)を\(\mathbb{K}\)上の位相ベクトル空間とする. 
\begin{enumerate}
\item  \(C \subset X\) convexならば, \( \overline{C}, C^\circ\) もconvex. 
\item \(B \subset X\) balancedならば, \(\overline{B}\)もbalanced.
さらに,  \(0 \in B^\circ\)であるならば, \(B^\circ\) もbalanced
\item \(E \subset X\) boundedならば,  \(\overline{E}, E^\circ\) もbounded. 
\end{enumerate}
\end{prop}
\end{tcolorbox}
\begin{proof}
(0). 証明において使う事柄をまとめておく
\begin{enumerate}
\item $a \in X$について$f_a : X \to X$, $f_a(x):=a+x$は同相写像. 連続は明らかで$f_{-a}$が逆写像になるから. 同様に$s \in \mathbb{K} \setminus \{0 \}$について, $f_s: X \to X$, $f_s(x):=s \cdot x$も同相写像. 
\item $T : X \times X \to X$, $t(x,y):=x+y$とおくとき, $T$は開写像. なぜなら$U, V \subset X$openについて$T(U \times V)=\cup_{x \in U}(x+ V)$であり, $x+V$は上より開集合であるので. 
\end{enumerate}


(1). 示すことは, 任意の$t \in (0,1)$について, $tC^\circ + (1-t)C^\circ \subset C^\circ$である($\overline{C}$も同じ). 
$t \in (0,1)$を固定する. 

[$C^{\circ}$について]. 
%次の写像を考える. 
%\[\xymatrix@C=20pt@R=5pt{F: X\times X\ar[r]^-{f}&X\times  X \ar[r]^-{t} &X\\(x,y)\ar@{|->}[r]&(tx, (1-t)y)\ar@{|->}[r] &tx + (1-t)y}\]この写像により, $C$はconvexなので, $$F(C^{\circ} \times C^{\circ}) = tC^{\circ}  + (1 - t)C^{\circ} \subset C$$
$C$はconvexなので, 
$$T(tC^{\circ} \times  (1 - t)C^{\circ}) = tC^{\circ}  + (1 - t)C^{\circ}\subset C$$
である. $T$は開写像より, $tC^{\circ}  + (1 - t)C^{\circ} $はopenなので, $C^{\circ}$に含まれる. 

[$\overline{C}$について]. 次の写像を考える. 
\[\xymatrix@C=20pt@R=5pt{
F: X\times X\ar[r]^-{f}&X\times  X \ar[r]^-{T} &X\\
(x,y)\ar@{|->}[r]&(tx, (1-t)y)\ar@{|->}[r] &tx + (1-t)y}
\]
この$F$は連続である. 
よって連続の閉包を用いた同値性\footnote{位相空間の間の写像$f : X \to Y$が連続であることと, 任意の$A \subset X$について$f(\overline{A}) \subset \overline{f(A)}$であることは同値である. }より
$F(\overline{C\times C}) \subset \overline{F(C \times C)}$となる. 
$\overline{C\times C} = \overline{C} \times \overline{C}$なので, 展開すると
$$
  t\overline{C} +  (1-t)\overline{C}
  = F(\overline{C\times C}) 
\subset 
\overline{F(C \times C)}
= \overline{tC + (1-t)C} \subset \overline{C}.
$$

(2). \(\alpha \in \mathbb{K}, \ |\alpha| \le 1\)とする. 
$f_{\alpha}(x):=\alpha x$とおくと, 連続の閉包を用いた同値性より
$$
\alpha \overline{B} = f_{\alpha}( \overline{B}) \subset 
 \overline{ f_{\alpha}(B)}  = \overline{\alpha B} \subset \overline{B}
 $$
 よってbalancedである. ( 最後の$\overline{\alpha B} \subset \overline{B}$に$B$がbalancedを用いた)
 
\(0 \in B^\circ\) をさらに仮定すると, 
$0 \cdot B^\circ = 0 \in B^\circ$であり, 任意の\(\alpha \in \mathbb{K}, \ |\alpha| \le 1\)についても, $\alpha$倍が同相写像であるので, $ \alpha B^\circ = (\alpha B)^\circ \subset B^\circ\)となる. よってbalanced.

(3). \(E^\circ\)がboundedは$E^\circ \subset E$より明らか. 
$\overline{E}$がboundedを示す. 
\(0 \in V \subset X\) openを固定する. 
示すことは任意の$t \gg0$について$\overline{E} \subset tV$である.

当たり前なこととして, \(\{\overline{0}\} \cap (X \setminus V) = \varnothing\)である. 
$\{ 0\}$ compact, $X \setminus V$ closed より, \ref{prop-M-2.1}から, 
あるopen \(0 \in {W} \subset X\) で
\[\{ 0\}+W \cap \left( (X \setminus V) + W\right) = \varnothing\]
となる. 
これは, \( 0 \in W \subset \overline{W} \subset V\) を意味する.
(もし$\overline{W} \cap (X \setminus V) \neq \varnothing$なら, その元の$W$近傍が$W$と交わりをもち, それは上に矛盾する).
\(E\) は boundedなので,   任意の$t \gg0$について$E \subset t W$ 
以上より$t$倍は同相なので, 
\[\overline{E} \subset t \overline{W} = t \overline{W} \subset t V\]
よって$\overline{E}$はbounded.
\end{proof}

\begin{tcolorbox}[mybox]
\begin{prop}
\label{prop-M-2.3}
\(X\)を\(\mathbb{K}\)上の位相ベクトル空間とする. 
\begin{enumerate}
\item  \(0 \in U \subset X\) openならば, あるbalanced open $W$で, \(0 \in W \subset U\) となるものがある.
\item  \(0 \in U \subset X\) convex open ならば, あるconvex balanced open $W$で, \(0 \in W \subset U\) となるものがある.
\end{enumerate}
\end{prop}
\end{tcolorbox}

\begin{proof}
以下$\delta >0$に対して, 
$B_\delta(0):= \{ \alpha \in \mathbb{K} | |\alpha|<\delta \}$とおく. 

(1).  \(f : \mathbb{K}  \times X \to X\)を$f(\alpha, x):=\alpha x$とする. 
これは連続より, $f^{-1}(U)$は$(0,0)$を含む開集合である. 
よって, ある$\delta >0$と$0 \in V \subset X$となるopenがあって, 
\(B_\delta(0) \times V \subset f^{-1}(U)\)  となる. 

$W:=f(B_\delta(0) \times V )$とおく. $W \subset U$は明らか. 
また$ W = \bigcup_{|\alpha| <\delta} \alpha V  $であり, $X$の開集合である. 
任意の\( \beta \in \mathbb{K}, \ |\beta| \le 1\)  について, 
\(\beta W \subset \bigcup_{|\alpha| <\delta} \beta \alpha V \subset W\) となる. ($|\alpha \beta|<\delta$のなので). よって$W$はbalancedである. 

(2).  \(A := \bigcap_{|\alpha| = 1, \ \alpha \in \mathbb{K}} \alpha U\) とする. ($\mathbb{K}=\C$なら$S^1$のように回転させて合併をとる)

\begin{tcolorbox}[mybox]
\begin{claim}
\(A\)はconvex balancedである. 
\end{claim}
\end{tcolorbox}
\begin{proof}[Claimの証明]
Convexに関しては(元をとって考えれば)明らか.
balancedを示す. \( \beta \in \mathbb{K}, \ |\beta| \le 1\)  をとる. $\beta A \subset A$を示せば良い. 
$0 \in U$より, $0 \in A$である. よって$0 \cdot  A \subset  A$である. これより$\beta \neq 0$として良い. 
すると$|\alpha|=1$ならば$\frac{\beta \alpha}{|\beta|}=1$であるので, 
\[
\beta A 
= \bigcap_{|\alpha|=1} \beta \alpha U 
= \bigcap_{|\alpha|=1} \frac{\beta \alpha}{|\beta|} |\beta|U 
\subset  \bigcap_{|\alpha|=1} \frac{\beta \alpha}{|\beta|}U 
\subset A
\]  
となる. (\( |\beta| U \subset |\beta| U + (1-|\beta|) U \subset U\)に注意. $0 \in U$はここにも使う) よってbalancedである.  
\end{proof}
% [page 15]
この$A^{\circ}$が欲しいconvex balanced open であることを示す (命題の主張の$W$).
そのためには, \ref{prop-M-2.2}より, $0 \in A^{\circ}$であることを示せば良い.

$0 \in U$より, (1)からあるbalanced open $0 \in V \subset U$がある. 
$V$はbalancedなので, $ \alpha \in \mathbb{K}, |\alpha| = 1$について, $\alpha^{-1} V\subset V$である(\(|\alpha^{-1}|=1\)なので). 
よって, \( V \subset \alpha V \subset \alpha U\) であるので, 
共通部分をとって, 
\[ V \subset \bigcap_{|\alpha|=1} \alpha U = A\]  
を得る. $V$ openより, $0 \in V \subset A^{\circ}$となりいえた.
\end{proof}

\begin{tcolorbox}[mybox]
\begin{cor}
\label{cor-M-2.4}
\(X\)を\(\mathbb{K}\)上の位相ベクトル空間とする. 
このとき$X$はbalancedな$0$の開近傍からなるlocal basisを持つ.   

さらに$X$が locally convex $($\ref{defn-M-1.4}参照$)$ならば, convex balancedな$0$の開近傍からなるlocal basisを持つ. 
\end{cor}
\end{tcolorbox}
\begin{tcolorbox}[mybox]
\begin{cor}
\label{cor-M-2.5}
\(X\)を\(\mathbb{K}\)上の位相ベクトル空間とする. 
任意のコンパクト集合$K \subset X$はboundedである. 
\end{cor}
\end{tcolorbox}

\begin{proof}
\(0 \in V \subset X\) openをとる. 任意の$t \gg 0 $ について$E \subset tV$を示す. 
\ref{prop-M-2.3}から, あるbalanced openで
\(0 \in W \subset V\)となるものがある. 

まず\(X = \bigcup_{n \in \N}nW\) であることを示す. $X \subset \bigcup_{n \in \N} n W$のみを示せば良い. 
\(f : \mathbb{K}  \times X \to X\)を$f(\alpha, x):=\alpha x$とおく. 
任意の$y \in X$について $f(0, y)=0 \in W$である. よって$f$は連続なので, $(0,y) \in f^{-1}(W)$となる.
これより, ある$\delta>0$と0の開近傍$U \subset X$があって
$(0,y) \in B_{\delta}(0) \times (y+U)  \subset f^{-1}(W)$となる. 
特に$\frac{1}{n}<\delta$なる$n$をとれば$\frac{1}{n} y \in W$となる. よっていえた


今$K \subset X = \bigcup_{n \in \N} n W $で$K$コンパクトなので, ある$n$があって
$K \subset nW$となる. $W$はbalancedなので任意の$t \ge n$について\(nW \subset tW\)である. 
よって任意の$t \ge n$について
\[
 \ K \subset nW \subset tW \subset tV
 \] となり$K$はboundedである. 
 \end{proof}

\section{Types of topological vector space}

\begin{tcolorbox}[mybox]
\begin{prop}
\label{prop-M-3.1}
\(X\)を\(\mathbb{K}\)上の位相ベクトル空間とする. 
$X$がlocally compact $($\ref{defn-M-1.4}参照$)$ならば, 有限次元.

特に$X$がlocally boundedかつHeine-Borel propertyを持つ$($\ref{defn-M-1.4}参照$)$
ならば, 有限次元.
\end{prop}
\end{tcolorbox}
\begin{proof}

(1). $X$をlocally compactとする.
定義からある開集合\( 0 \in V \subset X\)で \(\overline{V}\)がcompactなものが存在する. 
\ref{cor-M-2.5}から, $\overline{V}$はboundedである. よって$V$もboundedである.
よって, $\{ 2^{-n}V\}_{n \ge 1}$が0のlocal basisになる. (任意のopen $0 \in W$について, $V$ boundedなので$V \subset 2^{n_0}W$となる$n_0$が取れるから)

さて$0 \in V$より, $\overline{V} \subset \bigcup_{x \in \overline{V}}(x + \frac{1}{2}V)$である. 
\(\overline{V}\)はcompactより, ある \(x_1, \dots, x_m \in X\)があって, 
\begin{equation}
\label{eq-prop-M-3.1}
V \subset (x_1 + \tfrac{1}{2} V) \cup \dots \cup (x_m + \tfrac{1}{2} V)
\end{equation}
となる. そこで次のようにおく.
\begin{itemize}
\item \(Y := \sum_{i=1}^m \mathbb{K} x_i \subset X\)  
\item  \(d := \dim_{\mathbb{K}} Y\)  
\item\(v_1, \dots, v_d \in Y\) \(\mathbb{K}\)上の基底.
\end{itemize}

\begin{tcolorbox}[mybox]
\begin{claim}
\[
\xymatrix@C=20pt@R=5pt{
f : \mathbb{K}^d \ar[r]&(\mathbb{K} \times X)^d \ar[r]& X^d \ar[r]&X \\  
(a_i)_{i=1}^d \ar@{|->}[r] & (a_i, v_i)_{i=1}^d \ar@{|->}[r] & (a_i \cdot v_i)_{i=1}^d \ar@{|->}[r] &\sum_{i=1}^d a_i \cdot v_i \\
}
\]
とおくと$f : \mathbb{K}^d \to X$は連続写像.
そして, $f : \mathbb{K}^d \to Y$は同相写像で, $Y \subset X$は$X$の閉集合
\end{claim}
\end{tcolorbox}
\begin{proof}[Claimの証明]
スカラー倍や足し算が連続なので$f$は連続. また$f : \mathbb{K}^d \to Y$は全単射である. 
これが同相になるのを見るために, $f^{-1}$を次のように構成する 
 
 \(S := \{ z \in \mathbb{K}^d \mid ||z|| = 1 \}, B := \{ z \in \mathbb{K}^d \mid ||z|| \le 1 \}\) とおく($\R^d$の球面と閉球である)
\(0 \in S \)で\(f\) 連続単射より. $ 0 \not \in f(S) \subset X$かつ$f(S)$compactである. ($X$はhausdorffより閉集合でもある). 
よって\ref{prop-M-2.3}から, あるbalanced open \( W \subset X \)で$0 \in W$かつ$W \subset X \setminus f(S)$なものが存在する. 

$ f^{-1}W \subset B$であることを示す. 
もし$z \in f^{-1}W \setminus B$が存在したとする. 
定義から$||z||>1$である. 
$W$balancedなので,  $f^{-1}W$もbalanced, 
よって, 
$$
\frac{1}{|| z||} f^{-1}(W) \subset f^{-1}(W)
$$
である. これは$\frac{z}{|| z||} \in f^{-1}(W)$となるが, ノルムが1なので, $W \subset X \setminus f(S)$に矛盾する. 

特に任意の$r > 0$について, $f^{-1}(rW) \subset rB$である. 
よって任意の$r>0$について, $ f^{-1}(rW \cap Y) \subset rB$である. 
これは$f^{-1} : Y \to \mathbb{K}^d$が$0 \in Y$で連続であることを意味する.\footnote{位相空間の写像$f : X \to Y$が点$x \in X$で連続とは, $f(x)$の任意の近傍$V$に対して, ある$x$の近傍$U$が存在して, $f(U) \subset V$となること. }
任意の点$y \in Y$については, 以下の図を考える. 
\[
\xymatrix@C=36pt@R=18pt{
  Y \ar[r]^{f^{-1}} \ar[d]_{+y} &
  \mathbb{K}^d \ar[d]^{+f^{-1}(y)} \\
  Y  \ar[r]^{f^{-1}} &
  \mathbb{K}^d
}
\]
この縦の矢印は同相である. よって, $f^{-1} : Y \to \mathbb{K}^d$は点$y$でも連続である. 
これより$f : \mathbb{K}^d \to Y$は同相.

$Y \subset X$が閉集合を示す. $y \in \overline{Y}$をとる. 
$X = \bigcup_{t \in \N}W$ (\ref{cor-M-2.5}の証明参照)であるので, 
$t >0$があって$y \in tW$となる.
$tW$openなので, $y \in \overline{Y \cap tW}$である. 
$f^{-1}(tW) \subset tB$で$f^{-1} : Y \to\mathbb{K}^{d}$が同相なので, 
$\overline{Y \cap tW} \subset \overline{f(tB)}$である. 
最後に$tB \subset \mathbb{K}^{d}$コンパクトより, $f(tB)$もそう, よって閉集合なので
$\overline{f(tB)} = f(tB)$である. 
以上をつなぎ合わせると
\[
y \in \overline{Y \cap tW} \subset  \overline{f(tB)} = f(tB) \subset Y
\]
である. よって$y \in Y$であり, $Y=\overline{Y}$で閉集合である. 
\end{proof}


証明に戻る. 
(\ref{eq-prop-M-3.1})より$Y$の定義から$ V \subset Y + \frac{1}{2} V$である. 
これより
$$
V \subset Y + \frac{1}{2} V 
\subset Y + \left(\frac{1}{2}Y + \frac{1}{4}V\right)
= Y + \frac{1}{4}V
$$
となる. これを繰り返して, $V \subset \bigcup_{n \ge 1} (Y + \frac{1}{2^n}V)$を得る. 
今 $\{ 2^{-n}V\}_{n \ge 1}$が0のlocal basisになることと, $Y$が閉集合なので, 
$$
 V \subset \bigcap_{n \geq 1} \left(Y + \frac{1}{2^n} V\right) \subset \overline{Y} = Y
$$
となる. $X = \bigcup_{k \in N} kV$なので, $X\subset Y$となる. 
よって$Y \cong \mathbb{K}^d$なので, $X$は有限次元


(2). \(X\) locally boundedかつHeine-Borel Propertyを満たすとする.  
locally boundedなので, ある$0 \in V \subset X$でbounded openが存在する.
\ref{prop-M-2.2}より$\overline{V}$もbounded. 
Heine-Borel Propertyより, $\overline{V}$はコンパクト. 
よって, $X$は locally compactなので有限次元.
\end{proof}

\begin{rem}
上の証明の議論から「\(\mathbb{K}\)上の位相ベクトル空間$Y$が有限次元ならば, $Y$は$\mathbb{K}^{d}$と同相である」ことがこの議論からわかる. 

もっと強く「\(X\)を\(\mathbb{K}\)上の位相ベクトル空間, $Y \subset X$を$d$次元\(\mathbb{K}\)部分空間とするとき, ある\(f : \mathbb{K}^d \to Y \)で同相かつ$\mathbb{K}$線形なものが存在する」ということもわかる. 
\end{rem}

\begin{tcolorbox}[mybox]
\begin{prop}
\label{prop-M-3.2}
\(X\)を\(\mathbb{K}\)上の位相ベクトル空間とする. 
$X$のlocal baseが高々可算とする$($特に第一可算である$)$.
このとき次の三つを満たす距離$d : X \times X \to \mathbb{R}$が存在する.
\begin{enumerate}
\item $d$は$X$の位相を誘導する.
\item $d$はtranslation invariant, つまり$d(x+z, y+z)=d(x,y)$.
\item 任意の$r>0$について, $\{ x \in X \mid d(x,0) < r \}$はbalanced.
\end{enumerate}
さらに\(X\)がlocally convexであると仮定する. 
このとき距離\(d\)は, "任意の$ y \in X,$, $r>0$について
\(
\{ x \in X \mid d(x,y) < r \}
\)
がconvex" となるようにとることができる. 
\end{prop}
\end{tcolorbox}

\begin{proof}
以下$X$のlocal baseが高々可算とする.段階を追って示していく. 

(1). translation invariantな距離$d : X \times X \to \mathbb{R}$が存在すること. 
\ref{prop-M-2.3}より, balanced openからなるlocal base $\{V_n\}_{n=1}^\infty $で任意の
$n \in \N$について
\[
V_{n+1} + V_{n+1} + V_{n+1} + V_{n+1} \subset V_n 
\]
%V_n \ \text{: balanced} \quad (\text{\textcolor{red}{bal convex if $X$ is loc convex}})
となるように取れる. 
そこで
\[
D := \left\{ \sum_{n=1}^\infty c_n 2^{-n} \mid \text{$c_n =0, 1$で有限個を除いて$0$} \right\}
\]
とおく. $D \subset [0, 1)$かつ任意の$r \in D$について
\( r = \sum_{n=1}^\infty c_n(r) 2^{-n}$となる表示は一意的である. 
そこで, $r \in D \cup [1,\infty)$に関して, 
\[
A(r) := 
\begin{cases}
X & (r \ge 1),\\
c_1(r) V_1 + c_2(r) V_2 + \cdots & (r \in D)
\end{cases}
\]
と定義する. (下の表示は無限和に見えるが, $D$の定義から有限和である).
このとき$0 \in A(r)$かつ任意の$r \ge 0$について$ A(r)$はbalanced openである. 
%\](\textcolor{red}{balanced convex if $X$ is convex})\[

そこで次の関数を定義する.
\begin{itemize}
\item $f: X \to \mathbb{R}$, $x \mapsto \inf\{ r \in D \cup [1,\infty) \mid x \in A(r) \}$
\item $d: X \times X \to \mathbb{R}$, $(x,y) \mapsto f(x-y)$
\end{itemize}
$d$は translation invariant かつsymmetricである. 
(symmetricなのは, $A(r)$がbalancedなので, $x-y \in A(r)$は$y-x \in A(r)$を意味するから)

\begin{tcolorbox}[mybox]
\begin{claim}
任意の$ r,s \in D \cup [1,\infty)$について, $A(r) + A(s) \subset A(r+s)$が成り立つ. 
\end{claim}
\end{tcolorbox}
\begin{proof}[Claimの証明]
$r+s \ge 1$の場合は自明. よって, $r+s \in D$として良い. 
$r,s,r+s$を次のように表示する. 
\begin{align*}
\begin{split}
r &= \alpha_1 2^{-1} + \cdots + \alpha_{N-1} 2^{-(N-1)} + \alpha_N 2^{-N} + \cdots \\
s &= \beta_1 2^{-1} + \cdots + \beta_{N-1} 2^{-(N-1)}+ \beta_N 2^{-N} + \cdots \\
r+s &= \gamma_1 2^{-1} + \cdots + \gamma_{N-1} 2^{-(N-1)} + \gamma_N 2^{-N} + \cdots \\
\end{split}
\end{align*}

Case 1: ある$N$があって, $\alpha_i + \beta_i = \gamma_i (i = 1,\ldots,N-1) $かつ, $\alpha_N + \beta_N \neq \gamma_N$となる場合. 
このとき, 
$\alpha_N = \beta_N = 0$かつ, $\gamma_N = 1$にならざるを得ない. 
(これは要するに以下のように
\[
\begin{matrix}
\alpha_N : &1 & 0 & \cdots & 0 & 0 & 1\\
\beta_N: & 0 & 1 & \cdots & 1 & 0 & 1\\
\gamma_N:&\underbrace{1}_{1} & \underbrace{1}_{2}& \cdots & \underbrace{1}_{N-1} &  \underbrace{1}_{N}&  \underbrace{0}_{N+1} \\
\end{matrix}
\]
と, $N-1$まで繰り上がりが起こっておらず, $N$の時に繰り上がりが起こるパターンである. )
今$M \ge 2$について
$$
\alpha_{M-1} V_{M-1}+ \alpha_{M}V_{M}
\subset V_{M-1} + V_{M-1} \subset V_{M-2}
$$
である. 
以下$M$を十分に大きい整数とすると, $\alpha_{N+2} V_{N+2} + \cdots   + \alpha_M V_M \subset V_{N+1}$である. 
よって$\alpha_N =0$に注意すると
\begin{align*}
\begin{split}
A(r) &= \alpha_1 V_1 + \cdots + \alpha_{N-1} V_{N-1} + \underbrace{\alpha_N V_N}_{0} + \alpha_{N+1} V_{N+1}  + \underbrace{\alpha_{N+2} V_{N+2} + \cdots + \alpha_M V_M}_{\subset V_{N+1}} \\
&\subset \alpha_1 V_1 + \cdots + \alpha_{N-1} V_{N-1} + V_{N+1} + V_{N+1}
\end{split}
\end{align*}
同様に, $A(s) \subset \beta_1 V_1 + \cdots + \beta_{N-1} V_{N-1} + V_{N+1} + V_{N+1}$
であるので, 
\begin{align*}
\begin{split}
 A(r) + A(s) 
 &\subset (\alpha_1 + \beta_1) V_1 + \cdots + (\alpha_{N-1} + \beta_{N-1}) V_{N-1}
+ V_{N+1} + V_{N+1} + V_{N+1} + V_{N+1} \\
& \subset \gamma_1 V_1 + \cdots + \gamma_{N-1} V_{N-1} + \underbrace{\gamma_N}_{1} V_N \\
&\subset A(r+s)
\end{split}
\end{align*}

% [page 30]
Case 2: そのような $N$ がないとき, つまり任意の$i$で \( \alpha_i + \beta_i = \gamma_i \)が成り立つ時は, 自明に$A(r) + A(s) = A(r+s)$である.
\end{proof}

このClaimより次の三つがわかる.
\begin{tcolorbox}[mybox]
\begin{enumerate}[label=(\alph*)]
\item $r,t \in D \cup [1,\infty)$について, $r \le t$ならば$A(r) \subset A(t)$.
\item $f(x) = 0 \Leftrightarrow x = 0$.
\item $ f(x+y) \leq f(x) + f(y)$.
\end{enumerate}
特に$d$は$X$上の transrate invariant metricとなる. 
\end{tcolorbox}
\begin{proof}[上の3つのことの証明]
(a).$ t \ge 1$の場合は明らか. $t < 1$ の場合は, $t - r \in D$であるので(筆算を考える), 上のClaimから
\[
A(r) \subset A(r) + A(t-r) \subset A(t) 
\]

(b).
$x = 0$ならば
$$
f(0) = \inf \{ r \in \mathbb{R}_{>0} \mid 0 \in A(r) \} = 0
$$である.
逆に$f(x) = 0$ならば, 任意の$r \in D$について, $x \in A(r)$
である, 特に任意の$n \in \N$について, $x \in A\left( \frac{1}{2^n} \right) = V_n$である. 
$X$はHausdorffで$\{V_n\}_{n \ge 1}$は$0$のlocal baseなので, $x = 0$である. 

(c). $x \in A(r),\ y \in A(s)$について, 
$x+y \in A(r) + A(s) \subset A(r+s)$であるので, 
$f(x+y) \leq r+s$. よって$r,s$に関してinfをとれば言える.

また(b)は$d$の正定値性, (c)は$d$の三角不等式を表していて, 
$d$はsymmetric transrate invariantであることはわかっているので, $d$はほしい距離となる. 
\end{proof}

(2). \(d\) が \(X\) の topology を誘導することを示す.
\(\delta > 0\)について, \(B_\delta(0) :=\{ x \in X \mid d(x,0) < \delta \}\)
と定義する. $d(x,0)=f(x)$であるので, $f$の定義から
\[
B_\delta(0) = \bigcup_{r < \delta, r \in D \cup [1, \infty)} A(r) 
\]
である. この表示から$B_\delta(0)$は$X$のbalanced open setである. 
また$B_{2^{-n}}(0) \subset A(2^{-n}) = V_n$である. 
これより$\{B_\delta(0)\}_{\delta > 0}$は$X$のlocal baseになり, $d$は$X$の位相を誘導する. 

また”任意の$r>0$について, $\{ x \in X \mid d(x,0) < r \}$はbalanced"はすでに示した. 
よって$d$が欲しい距離となる. 

\(X\) が locally convex ならば, balanced convexとなる$V_{n}$をとることができる. 
よって$A(r)$もbalanced convexになり, $B_\delta(0)$もそうなる. 
convexityはtranslation invariantなので, 任意の$y \in Y$について$B_\delta(y)$もconvexとなる.

\end{proof}


% [page 34]
\section{Bounded linear maps}

\begin{tcolorbox}[mybox]
\begin{defn}
\label{defn-M-4.1}
\(X, Y\)を\(\mathbb{K}\)上の位相ベクトル空間とする. 
\(\mathbb{K}\)-linear map \(\Lambda: X \to Y\)  が\underline{bounded}であるとは, 
任意のbounded set $ E \subset X$について, $\Lambda(E)$もboundedであること. 
(つまり任意のopen $0 \in V  \subset Y$について, ある$t_0 > 0$があって, 任意の$t \geq t_0$について, $\Lambda(E) \subset tV$となること.)
\end{defn}
\end{tcolorbox}

\begin{tcolorbox}[mybox]
\begin{prop}
\label{prop-M-4.2}
\(\mathbb{K}\)上の位相ベクトル空間の線型写像 \(\Lambda: X \to Y\) について, 次の条件を考える. 
\begin{enumerate}[label=$(\alph*)$]
\item \(\Lambda\)は連続
\item  \(\Lambda\)はbounded
\item  任意の$X$の点列$\{ x_n\}_{n \ge 1}$について, 
\(x_n \to 0\) $(n \to \infty)$ならば, 
$\{\Lambda x_n \mid n=1,2,\dots\} \subset Y $はbounded
\item  任意の$X$の点列$\{ x_n\}_{n \ge 1}$について, 
\(x_n \to 0\) $(n \to \infty)$ならば, $Y$上で\(\Lambda(x_n) \to 0\) ($n \to \infty$).
\end{enumerate}
この時$(a) \Rightarrow (b) \Rightarrow (c)$は常に成り立つ. 

さらに,  \(X\) がmetrizableならば, \((c) \Rightarrow (d) \Rightarrow (a)\) が成り立つ. 
つまり上の条件は同値である. 特に\ref{prop-M-3.2}から, $X$のlocal baseが高々可算ならば, 上の条件は同値である. 
\end{prop}
\end{tcolorbox}

% [page 35]

\begin{proof} 

[(a) \(\Rightarrow\) (b)]
\(E \subset X\) boundedとする. 
\(0 \in V \subset Y\) openをとる. 
\(\Lambda\)は連続なので, \(0 \in \Lambda^{-1}(V) \subset X\) openである.
よって$E$ boundedなので, $t \gg0$について
\( E \subset t\Lambda^{-1}(V)\)
である. よって$t \gg0$について$\Lambda(E) \subset tV $より, $\Lambda(E)$はboundedとなる. 

[(b) \(\Rightarrow\) (c)]
$X$の点列$\{ x_n\}_{n \ge 1}$で\(x_n \to 0\) ($n \to \infty$)となるものを取る. 
$\{ x_n \mid n=1,2,\dots\} \subset X$がboundedであることを示せば良い. 
$0 \in  V \subset X $openをとる. 
\ref{prop-M-2.3}から, ある$0 \in  U \subset V$なるbalanced openがある. 
$x_n \to 0$よりある$n_0$があって次が成り立つようにできる.
\begin{itemize}
\item $n > n_0$ならば $x_n \in U$. これは$x_n \to 0$の定義そのもの.
\item ある$t_0 > 0$があって, 任意の$t \geq t_0$について, $x_1, \ldots, x_{n_0} \in tU$. これは\ref{cor-M-2.5}の証明から. ($U$がbalancedはここに使う.)
\end{itemize}
$U$はbalancedなので, (必要ならば$t_0>1$となるように$t_0$を取り替えて), $t \ge t_0$ならば
$U \subset tU$となる. よって$t \ge t_0$ならば$\{ x_n \mid n=1,2,\dots\}  \subset tU$より, boundedである. 

[(c) $\Rightarrow$ (d)] 
$X$はmetrizableとする. 
すると$X$はcountable local baseを持つので, \ref{prop-M-3.2}より, 
translate invariant metric $d : X \times X \to \mathbb{R}$で
$X$の位相を生成するものが存在する.

$X$の点列$\{ x_n\}_{n \ge 1}$で\(x_n \to 0\) ($n \to \infty$)となるものを取る. 
ここで$k_n$を
$$
 d(x_n,0) \leq \frac{1}{k_{n}^{2}}
$$
となる最大の自然数とおく. (ただし$x_n =0$なら$k_n:=n$)
$k_n \to 0 $である. 
$d$がtranslate invariantなので, 
\begin{align*}
\begin{split}
d(k_n x_n, 0)
&\le d(k_n x_n, (k_{n} - 1)x_n) + d((k_{n} - 1)x_n, 0) \\
&= d(x_n, 0) + d((k_{n}-1) x_n,0) \\
&\le k_n d(x_n,0) 
\le \frac{1}{k_n} \to 0
\end{split}
\end{align*}
よって$X$上で$k_n x_n \to 0$である. 
(c)の仮定を使って, 
$\{\Lambda(k_n x_n) \mid n = 1,2,\dots\} \subset Y $はboundedである. 

さて今から$\Lambda(x_n) \to 0$を示す. 
任意の$ 0 \in V \subset Y $openをとる. \ref{prop-M-2.3}よりbalancedを仮定して良い .
$\{\Lambda(k_n x_n) \mid n = 1,2,\dots\} \subset Y $はboundedなので, 
ある$t>0$があって, 
$$
\{\Lambda(k_n x_n) \mid n = 1,2,\dots\}
\subset tV
$$
となる. $k_n \to \infty$より, ある$n_0$があって, 任意の$n \ge n_0$について
$\frac{t}{k_n} \ge 1$となる. 
今$V$はbalancedなので, $\frac{t}{k_n}V \subset V$となる. 
まとめると$n \ge n_0$ならば,
$\Lambda x_n \in V$である. 
よって収束の定義から$\Lambda(x_n) \to 0$である.

[(d) $\Rightarrow$ (a)] 
\(X\) metrizableとする. 
 \(\Lambda\) が0で連続を示せば良い. 背理法.
  \(\Lambda\) が0で連続でないとすると, 
ある開近傍$0 \in V \subset Y$があって, 任意の近傍$0 \in U \subset X$について$\Lambda(U) \not\subset V$である. 
よって$X$は距離空間なので,  任意の$n \geq 1$について, ある$x_n \in X$があって, 
$d(x_n,0) < \frac{1}{n}$かつ$ \Lambda x_n \not \in V$となるものがある. 
これは$x_n \to 0$だが,  $\Lambda x_n$は0に収束しないので, (d)に矛盾する. 
\end{proof}


% [page 39]
\section{Seminorms and local convexity}

\begin{tcolorbox}[mybox]
\begin{defn}
\label{defn-M-5.1}
\(X\)を\(\mathbb{K}\)上の位相ベクトル空間とする. 
写像\(p : X \to \mathbb{R}\)が\underline{seminorm}とは以下の2条件を満たすこと. 
\begin{itemize}
\item 任意の$x,y \in X$について, \(p(x+y) \leq p(x) + p(y)\). 
\item 任意の$\alpha \in \mathbb{K}$, $x \in X$について, \(p(\alpha x) = |\alpha| p(x) \).
\end{itemize}
\end{defn}
\end{tcolorbox}
Berkovichの文脈では, このseminormは"faithful seminorm"と呼ばれるものらしい. 

\begin{tcolorbox}[mybox]
\begin{prop}
\label{prop-M-5.2}
\ref{defn-M-5.1}の記法において, 次が成り立つ. ただし$p$はseminormとする.
\begin{enumerate}
\item \(p(0) = 0\).
\item \(|p(x) - p(y)| \leq p(x-y)\).
\item \(p(x) \geq 0\).
\item \(\{x \in X \mid p(x) = 0\} \subset X\) は\(\mathbb{K}\)-線形部分空間
\item \(A = \{x \in X \mid p(x) < 1\} \subset X\) はconvex balanced.
\item $p(x) = \inf \{ t > 0 \mid t^{-1}x \in A \}$である. 以下右のinfを$p_{A}(x)$と表す. 
\end{enumerate}
\end{prop}
\end{tcolorbox}
% [page 40]
\begin{proof}
(1). \(p(0) = p(0 \cdot 0) = |0|\, p(0) = 0\)より.

(2). \(p(x) = p(x - y + y) \leq p(x-y) + p(y)\), 
\(p(y) = p(y - x + x) \leq p(y-x) + p(x)\), 
\(p(y-x) = |-1| p(x-y)\)である. これを組み合わせると言える. 
%\(|p(x) - p(y)| \leq p(x-y)\)である. 

(3). (2)より \(p(x) = p(x - 0) \geq |p(x) - p(0)| \geq 0\) なので. 

(4). (3)と\ref{defn-M-5.1}より. 

(5).  convexについて. 任意の\(x,y \in A,\ t \in (0,1)\)について, 定義から
\[
p(tx + (1-t)y) \leq t p(x) + (1-t)p(y) < 1 
\]
であるので. balancedについては, 
 任意の\(x \in A, \ \alpha \in \mathbb{K},\ |\alpha| \leq 1\) について
\[ p(\alpha x) = |\alpha|\, p(x) < 1\]
なので$\alpha x \in A$となる.

(6).  \(x \in X, \ t>0\) に対し
\[
t^{-1} x \in A \ \Leftrightarrow\ p(t^{-1}x) < 1 \ \Leftrightarrow\ t^{-1} p(x) < 1 \ \Leftrightarrow\ p(x) < t
\]
であるので, $\inf\{ t>0 \mid t^{-1} x \in A \} = p(x) $が言える. 
\end{proof}

\begin{tcolorbox}[mybox]
\begin{defn}
\label{defn-M-5.3}
\(X\)を\(\mathbb{K}\)上の位相ベクトル空間とする. 
\(\mathcal{P}\) を$X$のseminormsの族とする. 
\(\mathcal{P}\) が\underline{separating}とは, 任意の$x \in X\setminus\{0\}$について, ある$p \in \mathcal{P}$があって, $p(x) > 0$となること.
\end{defn}
\end{tcolorbox}

\begin{tcolorbox}[mybox]
\begin{thm}
\label{thm-M-5.4}
\(X\)を\(\mathbb{K}\)上のベクトル空間とする. 
\(\mathcal{P}\) を$X$のseparatingなseminormsの族とする. 
\(p \in \mathcal{P},\ n \in \N\)について,   
 \(
 V(p,n) := \{ x \in X \mid p(x) < \frac{1}{n} \}\)とおき, 
\[
\mathcal{B} := \{ V(p_1,n_1) \cap \dots \cap V(p_r, n_r) \mid r \ge 0, p_i \in \mathcal{P},\ n_i \in \N \}
 \]  
 とする. このとき$X$の位相  \(\tau\) で次を満たすものがただ一つ存在する. 
\begin{itemize}
\item \((X,\tau)\)は\(\mathbb{K}\)上のlocally convex 位相ベクトル空間.
\item \(\mathcal{B}\)は\((X,\tau)\)の local base.
\end{itemize}
さらにその位相ベクトル空間$(X, \tau)$は次を満たす. 
\begin{enumerate}[label=(\alph*)]
\item 任意の\(p \in \mathcal{P},\ p: X \to \mathbb{R}\)は連続.
\item  任意の部分集合\(E \subset X\)について,  "\(E\) がboundedである"ことは, "任意の$\in \mathcal{P}$について, \( p(E) \subset \mathbb{R}\)がboundedである"ことと同値. 
\end{enumerate}
\end{thm}
\end{tcolorbox}
よってseparatingなseminormから, ただ一つの locally convex 位相ベクトル空間の構造が定まり, それはseminormが作るopen ballがlocal baseとなる. boundednessは$p$からわかる. 

またもし\(\mathcal{P}\)が可算ならば, \( \mathcal{B}\) は可算. よって\ref{prop-M-3.2}より
\((X,\tau)\)はmetrizableである. 
特にFréchet space, つまりlocally convexかつcomplete invariant metricを持つ(F-space)空間となる
(\ref{defn-M-1.4}参照.)

% [page 43]
\begin{proof}

\(\tau\)を"\(\mathcal{B}\)の元を並行移動したものの合併集合全体"とする. 
つまり
$$
\tau:=\left\{ \bigcup_{i \in \Lambda \in A} (B_i + a_i) \mid B_i\in \mathcal{B}, a_i \in X\right\}
$$
とする. ただし \(\Lambda = \emptyset\)の場合は\(\bigcup_{i \in \Lambda \in A} (B_i + a_i) =\emptyset\) と定める. 
この$\tau$が位相になることを示す.(これが示されれば唯一性も言える.)

(1).  \(\varnothing \in \tau\) は自明. $X \in \tau$も
\(X = \bigcup_{x \in X} (V(p,1) + x)\)より. 

(2).  \(U_\lambda \in \tau \ \Rightarrow \ \bigcup_{\lambda } U_\lambda \in \tau\) は$\tau$の定義から.

(3). \(U_1, U_2 \in \tau \ \Rightarrow\ U_1 \cap U_2 \in \tau\). 
これはかなりややこしいがので丁寧にやる. 
$$
U_1 = \bigcup_{\alpha \in \Lambda} (x_\alpha + B_\alpha)
\quad
U_2 = \bigcup_{\beta \in \Lambda'} (y_\beta + B'_\beta)
$$
とする.$x_\alpha, y_\beta \in X$かつ$B_\alpha, B'_\beta \in \mathcal{B}$である. 
すると
\[
U_1 \cap U_2 = \bigcup_{\alpha \in \Lambda,  \beta \in \Lambda'} 
\left[ 
(x_\alpha + B_\alpha) \cap (y_\beta + B'_\beta)
\right]
\]
よって示すべきことは,$x, y\in X$かつ$B, B'\in \mathcal{B}$について
\((x+B) \cap (y+B') \in \tau \)である. 
\(\tau\) は平行不変より,  \(y=0\) としてよい.
さらに以下のように$B, B'$を定める
$$
B := V(p_1,n_1) \cap \dots \cap V(p_r, n_r)
\quad
B' := V(p'_1, n'_1) \cap \dots \cap V(p'_{r'}, n'_{r'})
$$

$w \in (x+B) \cap B'$とする. 
すると$m_i, m'_{i'} \in \N$を
$$
\frac{1}{m_i} < \frac{1}{n_i} - p_{i}(w -x)
\quad
\frac{1}{m^{'}_{i'}} < \frac{1}{n^{'}_{i'}} - p_{i'}(w)
$$
と定めると, $w +  \bigcap_{i=1}^{r} V(p_i, m_i) \cap \bigcup_{i=1}^{r} V(p'_i, m'_i) \subset (x + B) \cap B'$となることを示す. (なおこれが示されれば$(x+B) \cap (y+B') \in \tau$は$w$に関して合併集合をとれば言える. )

$\xi \in w +  \bigcap_{i=1}^{r} V(p_i, m_i) \cap \bigcup_{i=1}^{r} V(p'_i, m'_i) $とする. 示すことは, 
$$
p_i (\xi - x) < \frac{1}{n_i}
\quad
p_i (\xi) < \frac{1}{n^{'}_{i'}}
$$
である. 1つ目については, $w \in (x+B)$と$\xi \in w +  \bigcap_{i=1}^{r} V(p_i, m_i) $より, 
seminormの劣加法性を使って
$$
p_i (\xi - x)
=
p_i (\xi -w + w - x)
\le 
p_i (\xi -w) + p_i(w - x)
<\frac{1}{n_i}
$$
である. 二つ目は$w \in B'$と$\xi \in \bigcup_{i=1}^{r} V(p'_i, m'_i) $を使って上と同様にしめせる.
よっていえた. 

これより次がわかる. 
\begin{tcolorbox}[mybox]
\begin{itemize}
\item 平行移動は \((X,\tau)\)で同相写像. これは\(U \in \tau \Rightarrow U = \bigcup_i B_i + a_i\) の形に書けるので.
\item 任意の\(p \in \mathcal{P}\)について, $p: X \to \mathbb{R}$は連続. 
これは\ref{prop-M-5.2}より\(|p(x+y) - p(x)| \leq p(y)\)から, 
\[p(x + V(p,n)) \subset (p(x) - \tfrac1n, \ p(x) + \tfrac1n) \subset \R\]
であるので. 
\item  任意の\(V \in \mathcal{B}\)について\( V\)はbalanced convexな\(0\)の開近傍. これはseminormの定義からわかる. 
\end{itemize}
\end{tcolorbox}
以下残りの事柄も示していく. 

[\(\mathcal{B}\) は\((X,\tau)\)でのlocal baseなること]
\(0 \in U \subset X\) open, つまり\(U \in \tau\)とする. 定義からある$x \in X$と$p_i \in \mathcal{P}$, $n_i \in \N$があって
\[
0 \in x + V(p_1, n_1) \cap \dots \cap V(p_r, n_r) \subset U
\]  
である. 特に$p_{i}(x) < \frac{1}{n_{i}}$なので, 
$\frac{1}{m_i} < \frac{1}{n_i} - p_{i}(-x)$ととると, 位相の時の議論と同じくして
$$
0 \in V(p_1, m_1) \cap \dots \cap V(p_r, m_r) 
\subset x + V(p_1, n_1) \cap \dots \cap V(p_r, n_r) \subset U
$$
となる. $V(p_1, m_1) \cap \dots \cap V(p_r, m_r) \in \mathcal{B}$なので, $\mathcal{B}$は0の基本近傍系となり, つまりlocal baseである. 

[$(X,\tau)$は$T_1$空間なること]
 平行移動は同相写像なので\( \{0\} \) が closedを示せば良い.
 $\mathcal{P}$はseparatingより, 任意の$x \in X \setminus \{ 0 \}$についてある$p_x \in \mathcal{P}$があって$ p_{x}(x) \neq 0$である. よって, $\frac{1}{n_x}<p(x)$となる自然数をとれば, 
 \ref{prop-M-5.2}より
 $$
  X \setminus \{0\} = \bigcup_{x \in   X \setminus \{0\} } (x + V(p_x, n_x))
$$
であることがわかる. よって$  X \setminus \{0\}$はopenで, \( \{0\} \)はclosed.

[足し算は連続なること]
\(X \times X \to X,\ (x,y) \mapsto x+y\)とする. 
$(x,y) \in X \times X$で連続であることを示せば良い. $\mathcal{B}$はlocal baseなので, 
任意の$U:=V(p_1, n_1) \cap \dots \cap V(p_r, n_r) \in \mathcal{B}$について, ある$W \in \mathcal{B}$があって, 
$$
(x+W) + (y+W) \subset x + y + U
$$
となることを示せば良い. 
そうなる$W$として\(V(p_1, 2n_1) \cap \dots \cap V(p_r, 2n_r)\) を取れば良い.

[スカラー倍は連続なること]
\(\mathbb{K} \times X \to X,\ (a, x) \mapsto \alpha x\)  
$(a, x) \in \mathbb{K} \times X$で連続であることを示せば良い. $\mathcal{B}$はlocal baseなので, 
任意の$U:=V(p_1, n_1) \cap \dots \cap V(p_r, n_r) \in \mathcal{B}$について, ある$W \in \mathcal{B}$と, $a \in \mathbb{K} $の半径$\delta$の開球$D$があって
$$
D \cdot (x+W) \subset ax + U
$$
となることを示せば良い. 
これは$W=V(p_1, m_1) \cap \dots \cap V(p_r, m_r)$とおき, 
$a + \alpha \in D, \xi \in W$について. 
$$
a\xi + \alpha (x + \xi)
= (a +\alpha) (x + \xi) - ax \in V(p_1, n_1) \cap \dots \cap V(p_r, n_r) 
$$
となるように$\delta, m_i$が取れれば良い. 
これは$|a| \frac{1}{m_i} < \frac{1}{2n_i}$となるように$m_i$を十分大きくとった後に
$\delta(|x| + \frac{1}{m_i}) < \frac{1}{2n_i}$となるように$\delta$を十分小さくとれば良い.


以上の結論として次が言える. 
\begin{tcolorbox}[mybox]
\begin{enumerate}
\item \((X,\tau)\)は\(\mathbb{K}\)上の位相ベクトル空間. 
\item \(\mathcal{B}\)は\((X,\tau)\)の balanced convex setからなるlocal base. 
\item 上の1,2を満たす位相$\tau$は唯一.
\item 任意の\(p \in \mathcal{P}\)について, $p: X \to \mathbb{R}$は連続. 
\end{enumerate}
\end{tcolorbox}

最後にboundednessの特徴付けに関して. 
「\(E \subset X\)がboundedである」ことは, 「任意の\( V \in \mathcal{B}\)について, $t \gg 1$ならば\(E \subset tV\)である」ことと同値.
それは「任意の\(p \in \mathcal{P}\)について, $n\gg 1$ならば\(E \subset tV(p,n)\)である」ことと同値. 
これは「任意の\(p \in \mathcal{P}\)について, \(p(E) \subset \mathbb{R}\)がbounded」と同値である. 
(\(E \subset tV(p,n)\)は$p(E)<\frac{t}{n}$と同じ意味であることに注意)
\end{proof}

% [page 31]
\section{Cauchy Sequence and Completeness}

\begin{tcolorbox}[mybox]
\begin{defn}
\label{defn-M-6.1}
\(X\)を\(\mathbb{K}\)上の位相ベクトル空間とする. 
\(\{ x_n \}_{n=1}^\infty \subset X\)が\underline{Cauchy列}とは,   
任意のopen \( 0 \in V \subset X \)について, ある$n_0 \in \N$があって, 
任意の\(n \geq m \geq n_0\)について,  \( x_n - x_m \in V \)となること. 

\( X \)が\underline{complete}とは, 任意のCauchy列が収束すること. 
\end{defn}
\end{tcolorbox}
\ref{prop-M-2.3}よりCauchy列の定義における$V$はbalancedを仮定して良い. 

\begin{rem}
\(X\)がmetrizableで invariant metric
\( d : X \times X \to \mathbb{R} \)を持つとする. 

この時\(\{x_n\}_{n=1}^\infty\)がCauchy列であることは,  通常のコーシー列の定義
「任意の$ \varepsilon > 0$について, ある$n_0 \in \N$があって, 任意の\(n \geq m \geq n_0\)について,  \( d(x_n, x_m) < \varepsilon\)」と同値である.  
理由はinvariantなら$d(x_n - x_m, 0)=d(x_n, x_m) $であるので. 
\end{rem}
\begin{rem}
Cauchy列(のなす集合)はbounded.
\begin{proof}
任意のopen \( 0 \in V \subset X \)をとる. 
\ref{prop-M-2.3}より, $0 \in W \subset V$となるbalanced open$W$をとる. 
ある$n_0 \in \N$があって, 
任意の\(n \geq m \geq n_0\)について,  \( x_n - x_m \in W \)となる.
また$x_1, \ldots, x_{n_0} \in t_0W$となるような$t_0 \ge 1$を取れる. 
$t\ge 2t_0$とすると, $n > n_0$ならば, $W$はbalancedより
$$
 x_n = x_{n_0} + (x_n - x_{n_0}) \in t_0W + W \subset (2t_0)W \subset tV.
$$
また$n \le n_0$ならば$x_{n} \in t_0W \subset tV$. よって. 
よって$t\ge 2t_0$ならば$x_i \in tV$である. 
\end{proof}
\end{rem}

\section{The space \(C^\infty(\Omega)\) and \(\mathcal{D}_K\)}

\begin{tcolorbox}[mybox]
\begin{defn}
\label{defn-M-7.1}
$\Omega$を空でない$\mathbb{R}^n$の開集合, $K \subset \mathbb{R}^n$をコンパクト集合とする. 
\begin{itemize}
\item  \(C^{\infty}(\Omega) := \{ f : \Omega \to \C  \mid  \text{$f$は$C^{\infty}$級} \}\)
\item  \(\mathcal{D}_{K} = \{f \in C^\infty(\Omega) \mid \mathrm{Supp} f \subset K\}\)
\end{itemize}
とおく. これらは\(\mathbb{C}\)-vector spaceである.
\end{defn}
\end{tcolorbox}

Distributions の定義に\(\mathcal{D}_K\) が必要である. この節の目標は次である:
(用語に関しては\ref{defn-M-1.4}参照.)
\begin{tcolorbox}[mybox]
\textbf{Goal.} 
ある \( C^\infty(\Omega) \)  の位相で, 次を満たす位相を入れる.
\begin{itemize}
\item  \( C^\infty(\Omega) \) は位相ベクトル空間になる. 
\item Fréchet space. つまりlocally convexかつcomplete invariant metricを持つ(F-space)
\item Heine-Borel propertyを持つ. 
\item 任意のコンパクト集合$K \subset \Omega$について, \(\mathcal{D}_K \subset C^\infty(\Omega)\)はclosed. 
\end{itemize}
\end{tcolorbox}

以下$\Omega$を空でない$\mathbb{R}^n$の開集合とする. 

\begin{tcolorbox}[mybox]
\begin{lem}
\label{lem-M-7.2}
あるコンパクト集合の列
\(K_1 \subset K_2 \subset \cdots \subset \Omega\) 
があって次を満たすものが存在する.
\begin{enumerate}
%\item \(K_i\) is compact.  
\item \(K_i \subset K_{i+1}^{\circ}\)  
\item  \(\Omega = \bigcup_{i=1}^\infty K_{i}^{\circ}\)
\end{enumerate}
\end{lem}
\end{tcolorbox}

\begin{proof}
$a \in \R^n$, $r >0$について, 
\( B(a, r) := \{ x \in \mathbb{R}^n \mid ||x-a|| < r \} \)  
と定める. 
\( \overline{B(a, r)} := \{ x \in \mathbb{R}^n \mid ||x-a|| \le r \} \) である. 
\[
\mathcal{B}:=
\{  \overline{B(a, r)}\mid a \in \mathbb{Q}^n \cap \Omega, \ r \in \mathbb{Q}_{>0}, \  \overline{B(a, r)} \subset \Omega \}
\]  
とする. これは可算なので,  \(\mathcal{B}=\{B_1, B_2, \dots \}\) と添字をふる. 

\(K_1 := B_1\)とする. 以下\(K_1, \ldots, K_m\) が作れた時に, 
コンパクト集合$K_{m+1} \subset \Omega$で\(K_m \subset K_{n+1}^{\circ}\) をみたすものを帰納的に構成する. 
$x \in K_{m}$について$B(x, r_x) \subset \Omega$となる$r_x >0$をとる.
 $K_m$コンパクトなので, $K_m \subset \bigcup_{j=1}^{l}B(x_j, r_{x_j})$とできる. 
 今$C:=\bigcup_{j=1}^{l}\overline{B(x_j, r_{x_j})}$とおくと, 
 $C$コンパクトで
$$
K_m \subset C^{\circ} \subset C \subset \Omega
$$
となる.
よって\(K_{m+1} := C \cup B_{m+1}\)とおけば良い.

また上の\(K_1, \ldots, K_m, \ldots\)の構成法から,  \(\Omega = \bigcup_{i=1}^\infty K_{i}^{\circ}\)となる. 
\end{proof}

以下\(\alpha = (\alpha_1, \dots, \alpha_n) \in \N^n\)について
\[
D^\alpha := \partial_{x_1}^{\alpha_1} \cdots \partial_{x_n}^{\alpha_n}
\]
と定める.


\begin{tcolorbox}[mybox]
\begin{prop}
\label{prop-M-7.3}
\ref{lem-M-7.2}のように\(\{ K_i \}_{i=1}^\infty\)をとる. 
任意の\(N \in \N\)について
\[
p_N :  C^\infty(\Omega) \to \R \quad
f \mapsto p_N(f)=\max \left\{|D^\alpha f(x)| \mid x \in K_N, |\alpha| \le N \right\} 
\]
%\[\xymatrix@C=pt@R=5pt{p_N :  C^\infty(\Omega) \ar[r]& \R \\f \ar@{|->}[r]&\max \left\{|\partial^\alpha f(x)| \mid x \in K_N, |\alpha| \le N \right\} \\}\]
とおく. この時次が成り立つ.
\begin{enumerate}
\item  \(\mathcal{P}:=\{ p_N \mid N \ge 1 \}\)は\(C^\infty(\Omega)\)のseminormからなるseparating family. 
特に\ref{thm-M-5.4}から \(C^\infty(\Omega)\)はlocally convexかつinvariant metricを持つ\(\mathbb{C}\)上の位相ベクトル空間となる. 
\item  任意の\(x \in \Omega\)について, 
\[
 ev_x : C^\infty(\Omega) \to \mathbb{C}
\quad
f \mapsto  ev_{x}(f)=f(x).
\]
は連続である. 特に任意のコンパクト\(K \subset \Omega\) について, 
\(\mathcal{D}_{K} = \bigcap_{x \in \Omega \setminus K} \mathrm{Ker}ev_x\subset C^\infty(\Omega)
\)であるので, $\mathcal{D}_{K} \subset C^\infty(\Omega)$は$(1)$の位相でclosedである. 
\item  \(V_N := \{ f \in C^\infty(\Omega) \mid p_N(f) < \frac{1}{N} \}\)とおくと, 
\(\{ V_N \mid N \ge 1 \}\)は\(C^\infty(\Omega)\)のlocal baseとなる.
\item (1)の\(C^\infty(\Omega)\)の位相は, \(\{ K_i \}_{i=1}^\infty\)の取り方によらない. 
\item \(\{ f_i \}_{i=1}^\infty \)を$C^\infty(\Omega)$の点列とし, $f \in C^\infty(\Omega)$とする. 
"$(1)$の位相で\(\{ f_i \}_{i=1}^\infty \) が$f$に収束する"ことは, "任意の$\alpha \in \N^n$について
$D^\alpha f_i$が$D^\alpha f$に局所一様収束する"ことと同値である. 
\end{enumerate}
\end{prop}
\end{tcolorbox}
$D^\alpha f_i$が$D^\alpha f$に局所一様収束するとは, 任意のコンパクト集合$K \subset \Omega$について$\sup_{x \in K}\{ |D^\alpha f_i - D^\alpha f|\} \to 0$となること. 

\begin{proof}
(1). $p_N$がseminormとなることは簡単にわかる. 
\(\mathcal{P}:=\{ p_N \mid N \ge 1 \}\)separatingになることを示す. 
$f \in C^\infty(\Omega) \setminus \{0\}$とする. 
定義から\(x \in \Omega\)で\(f(x) \neq 0\)となるものがある. 
よって\ref{lem-M-7.2}より, ある$N \ge 1$で$x \in K_N$となるものが存在する. 
$p_N$の定義から, $0 < |f(x)| \le  p_N(f)$となる. よっていえた. 

(2). \(x \in \Omega, \ f \in C^\infty(\Omega)\)とする. 
示すことは「任意の$\varepsilon >0$について, ある(1)での位相における開集合$0 \in V \subset C^\infty(\Omega)$が存在して, 任意の$g \in V$について
$|ev_{x}(f + g) - ev_{x}(f)| < \varepsilon$となる」である. ($f + V$は$f \in C^\infty(\Omega)$の開近傍になる.)

$\varepsilon >0$とする. \(V := \{ g \in C^\infty(\Omega) \mid p_N(g) < \varepsilon \}\)とおく. 
\ref{thm-M-5.4}より$p_N :  C^\infty(\Omega) \to \R$は(1)の位相で連続になるので, $V$は(1)での位相における開集合となる. 
任意の$g \in V$について, 
\[
|ev_{x}(f + g) - ev_{x}(f)| 
= |g(x)| \le p_N (g) < \varepsilon
\]
となりいえた. 

(3).  %\(p_1 \le p_2 \le \cdots\)であることに注意する. 
\ref{thm-M-5.4}から, 
$i \in \N, n_i \in \N$について, 
\(V(p_{i}, n_{i}) := \{ f \in C^\infty(\Omega) \mid p_{i}(f) < \frac{1}{n_{i}} \}\)と定め, 
\[
\mathcal{B}:= \{ V(p_{i_1}, n_{i_1}) \cap \cdots \cap V(p_{i_r}, n_{i_r}) \}
\]
とする. \ref{thm-M-5.4}から$\mathcal{B}$はlocal baseである. 
よって示すことは, 「任意の$V:=V(p_{i_1}, n_{i_1}) \cap \cdots \cap V(p_{i_r}, n_{i_r}) $について, ある$N \in \N$があって, $0 \in V_N \subset V $となる」ことである. 

$V:=V(p_{i_1}, n_{i_1}) \cap \cdots \cap V(p_{i_r}, n_{i_r})$とする. 
$i_1 \le \cdots \le i_r$として良い. 
$m:=\max\{ n_{i_1}, \ldots, n_{i_r}\}$とし$n:= \max\{ i_r, m\}$とおく. 
\(p_1 \le p_2 \le \cdots\)であることに注意すると
\[
V_{N}=V(p_N, N)
\subset V(p_{i_r}, m)
\subset V
\]
となる. よって$\{ V_N \}_{N \ge 1}$はlocal baseとなる. 

(4). $\{K'_{i}\}$を\ref{lem-M-7.2}を満たす別のコンパクト集合族とし, $p'_{i}$を$K'_{i}$に対応するものとする. 
「任意の$N \ge 1$について, ある$N' \ge N$が存在して, 
$V'_{N'}=V(p'_{N'}, N') \subset V(p_{N}, N)=V_N$である」ことを示す. 
これを示せば対称性より逆も成り立ち, 二つの位相が同じことが言える((3)より$\{ V_N \}_{N \ge 1}$はlocal baseとなるので).

$N \ge 1$とする. 
$K_N \subset \Omega = \bigcup_{i' \ge 1} K'^{\circ}_{i'}$
であるので, $K_N$のコンパクト性より, ある$N' \ge N$で$ K_N \subset K'_{N'}$となるものがある. 
よって$p_N \le p'_{N'}$であるので, $V(p'_{N'}, N') \subset V(p_{N}, N)$. よっていえた.

(5). 
「$C^{\infty}(\Omega)$の位相で$f_i \to f$」は「任意の
\(N \ge 1\)について, ある$ i_0 \ge 1$があって, 任意の$i \ge i_0$について, $f_i \in f + V_N$
である」ことと同値である.
($\{ V_N \}_{N \ge 1}$はlocal baseとなるので)
ここで
\[
f_i \in f + V_N
\Leftrightarrow\
p_N(f_i - f) < \frac{1}{N}
\Leftrightarrow
\text{任意の$|\alpha|\le N$について$K_N$上で}|D^{\alpha} f_i - D^{\alpha} f| < \tfrac{1}{N}
\]
であることに注意する. 

$C^{\infty}(\Omega)$の位相で$f_i \to f$とする. 
任意の$K \subset \Omega$コンパクトについて, ある$l \ge 1$があって, $K \subset K_{l}$となる
任意の$\alpha \in \N^n$について$N \ge \max(|\alpha|, l)$なるように$N$をとる. 
するとある$ i_0 \ge 1$があって, 任意の$i \ge i_0$について, $f_i \in f + V_N$となる. 
よって$i \ge i_0$ならば
\[
\sup_{x \in K}\{ |D^\alpha f_i - D^\alpha f|\} 
\le 
\sup_{x \in K_{N}}\{ |D^\alpha f_i - D^\alpha f|\} 
\le 
p_N(f_i - f) < \frac{1}{N}
\]
である. よって$K$上で任意の$\alpha \in \N^n$について$D^\alpha f_i$が$D^\alpha f$に一様収束する.

逆に「任意の$\alpha \in \N^n$について$D^\alpha f_i$が$D^\alpha f$に局所一様収束」すると仮定する .
特に$K_N$上で一様収束する. 
$|\alpha|\le N$となる$\alpha$は有限なので, $C^{\infty}(\Omega)$の位相で$f_i \to f$が言える. 
((5).の証明の初めに言った同値性に注意する.)
\end{proof}
% [page 58]
\begin{tcolorbox}[mybox]
\begin{prop}
\label{prop-M-7.4}
\ref{prop-M-7.3}での$C^{\infty}(\Omega)$の位相によって, 
$C^{\infty}(\Omega)$は位相ベクトル空間になり, Fréchet$($locally convexかつcomplete invariant metricを持つ$)$かつHeine-Borel propertyを持つ. 
\end{prop}
\end{tcolorbox}
\begin{proof}
\ref{lem-M-7.2}のように\(\{ K_i \}_{i=1}^\infty\)をとり. 
\ref{prop-M-7.3}のように
$$
p_N :  C^\infty(\Omega) \to \R \quad
f \mapsto p_N(f):=\max \left\{|D^\alpha f(x)| \mid x \in K_N, |\alpha| \le N \right\}
$$
をとる. \(V_N := \{ f \in C^\infty(\Omega) \mid p_N(f) < \frac{1}{N} \}\)とおくと, 
\ref{prop-M-7.3}から$\{ V_N \}_{N \ge 1}$はlocal baseとなる. 
\ref{thm-M-5.4}から \(C^\infty(\Omega)\)はlocally convexかつinvariant metricを持つ\(\mathbb{C}\)上の位相ベクトル空間となる. 
よって残りは完備性とHeine-Borel propertyである. 

[完備性]
$\{f_i\}^{\alpha} \subset C^{\infty}(\Omega)$をCauchy列とする. 
定義から任意の$N \ge 1$について, ある$i_0 \ge 1$があって, 任意の$i,j \ge i_0$について
$f_i - f_j \in V_N$となる. 
ここで$f_i - f_j \in V_N$とは, 「任意の$|\alpha| \le N$となる$\alpha$について, $\sup_{K_N} |D^{\alpha} f_i - D^{\alpha} f_j| < \frac{1}{N}$となる」ことと同値である. 

よって任意の$\alpha \in \N^n$について, $|\alpha| \le N$ならば, 
$\{ D^{\alpha} f_i \}$は$K_N$上で一様Cauchy列となる. 
(つまり任意の$\varepsilon>0$について, ある$i_0 \ge 1$があって, 任意の$i,j \ge i_0$について, $\sup_{K_N} |D^{\alpha} f_i - D^{\alpha} f_j| < \varepsilon$である.)
よって任意の$\alpha \in \N^n$について, $\{ D^{\alpha} f_i \}_{i=1}^{\infty}$は任意のコンパクト集合$K \subset \Omega$上で一様Cauchy列となる. 

$\alpha=0$とすれば, $\{  f_i \}_{i=1}^{\infty}$は任意のコンパクト集合上で一様Cauchy列より, ある
$\Omega$上の連続関数$f$があって, 任意のコンパクト集合上で$\{  f_i \}_{i=1}^{\infty}$は$f$に一様収束する. 次のClaimより, $f\in C^{\infty}(\Omega)$となり, 任意の$\alpha \in \N^n$について$D^\alpha f_i$が$D^\alpha f$に局所一様収束する. よって\ref{prop-M-7.3}から\(\{ f_i \}_{i=1}^\infty \) は$f$に収束する.

\begin{tcolorbox}[mybox]
\begin{claim}
\label{claim-prop-M-7.4}
$(a,b) \subset \mathbb{R}$を開区間とし, $f_i : (a,h) \to \mathbb{R}$を$C^1$級関数とする. 
$f, g$を連続関数として, $(a,b)$上で一様収束$f_i \to f$, $f'_i \to g$すると仮定する. 
この時$f$は微分可能かつ$g=f'$となる. 
\end{claim}
\end{tcolorbox}
\begin{proof}
$h >0$とすると
\begin{align}
\begin{split}
\label{eq-prop-M-7.4}
& \left| \frac{f(x+h) - f(x)}{h} - g(x) \right| \\
 & \le  \left| \frac{f(x+h) - f(x)}{h} - \frac{f_n(x+h) - f_n(x)}{h} +  \frac{f_n(x+h) - f_n(x)}{h} - g(x) \right| \\
&\le \left| \frac{f(x+h) - f(x)}{h} - \frac{f_n(x+h) - f_n(x)}{h} \right|
+ \left| f_n'(x+\theta h) - g(x) \right| \\
&\le
 \frac{1}{h} \left| f(x+h) - f_n(x+h) \right|
 + \frac{1}{h} \left| f(x) - f_n(x) \right|
 + \left| f_n'(x+\theta h) - g(x+\theta h) \right|
 + \left| g(x+\theta h) - g(x) \right| \\
 &\le
\frac{2}{h} \sup_{x \in(a,b)}|f(x)- f_n(x)| + 
 \sup_{x \in(a,b)}|g(x)- f_n'(x)| + \left| g(x+\theta h) - g(x) \right|
\end{split}
\end{align}
となる. ここで$\theta \in [0,1]$は$\frac{f_n(x+h) - f_n(x)}{h}=f_n'(x+\theta h)$となるようにとる.(平均値の定理より存在する. $\theta$は$n, h$に依存する)

よって任意の$\varepsilon >0$について, $h>0$を$ \sup_{\theta \in (-1,1)}\left| g(x+\theta h) - g(x) \right| < \frac{\varepsilon}{2}$となるように取り, そして$n$を
$$
\frac{2}{h} \sup_{x \in(a,b)}|f(x)- f_n(x)| + 
 \sup_{x \in(a,b)}|g(x)- f_n'(x)|
 < \frac{\varepsilon}{2}
$$
となるように大きくとれば, (\ref{eq-prop-M-7.4})より, $\varepsilon >0$について, ある$h>0$があって, 
$\left| \frac{f(x+h) - f(x)}{h} - g(x) \right| <\varepsilon$となる. よっていえた. 
\end{proof}

%次はここから. 
[Heine-Borel Propertyについて]
示すことは任意のclosed bounded setがコンパクトとなること. 
\(E \subset C^\infty(\Omega)\) closed boundedとする. 
$C^\infty(\Omega)$はmetrizableより, $E$もそうである.
よってコンパクトであることは点列コンパクトであることと同値である. ($E$の点列コンパクト性を示していく)

\ref{thm-M-5.4}から任意の$N \ge 1$について, 
$p_N (E) \subset \R$はboundedである. 
よってある$M_N > 0$があって, 任意の$|\alpha|<N, f \in E$について, 
$K_N$上で$\sup_{K_N}|D^\alpha f| \le M_N$となる.

\begin{tcolorbox}[mybox]
\begin{claim}
$\alpha \in \N^n$とする.
任意の$|\alpha|<N$となる自然数$N$について
$$
\mathcal{F}:=\{D^\alpha f \mid f \in E\}
$$
は$K_N$上で次を満たす.
\begin{itemize}
\item 一様有界. つまりある$M>0$があって, 任意の$D^\alpha f  \in \mathcal{F}$について, $|D^\alpha f| \le M$.
\item 一様同程度連続. つまり任意の$\varepsilon >0$についてある$\delta >0$があって, 任意の$D^\alpha f  \in \mathcal{F}$について, $|x - x'| < \delta$ならば$|f(x) - f(x')| < \varepsilon$.
\end{itemize}
\end{claim}
\end{tcolorbox}

\begin{proof}
一様有界はもう示した. 一様同程度連続を示す. 
$f \in E, F:=D^\alpha f $とする.
$x = (x_1, \ldots, x_n), y=(y_1, \ldots, y_n) \in K_N$とする.
平均値の定理より. ある$M_{n+1}$があって, $K_{N+1}$上で$|\frac{\partial F}{\partial x_1}F(x)| \le M_{N+1}$となる. よって
\begin{align}
\begin{split}
|F(x) - F(y)|
&\le |F(x_1,\dots,x_n) - F(y_1,x_2,\dots,x_n)| + \cdots
+ |F(y_1,\dots,y_{n-1},x_n) - F(y_1,\dots,y_n)| \\
& \le  M_{N+1} \left[ |x_1-y_1| + \cdots + |x_n-y_n| \right].
\end{split}
\end{align}
がいえる. これは一様同程度連続を導く.
\end{proof}

%$\alpha \in \N^n$とし$|\alpha|<N$となる自然数$N$とする. $\{g_n\}_{n\ge 1} \subset\{D^\alpha f|_{K_N} \mid f \in E \}$とする. このとき$\{g_n\}_{n\ge 1}$は一様収束する部分列を持つことを示す. (これはAscoliの定理の議論をまねる)

$N \in \N$とする. $\{g_n\}_{n\ge 1} \subset\{f|_{K_N} \mid f \in E \}$とする. 
このとき$\{g_n\}_{n\ge 1}$の部分列$\{ g_{n_k}\}_{k \ge 1} $で, 任意の$|\alpha|<N$となる$\alpha \in \N^n$について, $D^{\alpha}g_{n_k}$が一様収束する部分列を持つことを示す. (これはAscoliの定理の議論をまねる)

%\{g_n\}_{n\ge 0} \subset \{\partial^\alpha f|_{K_N} \mid f \in E\}
%uniformly bdd, uniformly equicontinuous.  
$\{x_1, x_2, \ldots, \}\subset K_N$を$K_N$の可算で稠密な部分集合とする. 
$\{g_n\}_{n\ge 1}$は 一様有界より, ある$M>0$があって
$|g_i(x_j)|<M$となる. 

以下「ある部分列$\{ g_{n_k}\}_{k \ge 1}$があって, 任意の$ i \ge 1$について
$\{ g_{n_k}(x_i)\}_{k \ge 1}$は収束する」ことを示す. これは対角線論法. 
$i=1$の時は, ボルツァーノ・ワイエルシュトラスの定理から$\{ g_{11}(x_1), g_{12}(x_1),  g_{13}(x_1), \ldots\} \subset \C$が収束するように$g_{1k}$が取れる.
次に$\{g_{1k}\}$の部分列をうまく取ることで, $\{ g_{21}(x_2), g_{22}(x_2),  g_{23}(x_2), \ldots\} \subset \C$が収束するように$g_{2k}$が取れる.これを繰り返すと次のような点列が取れる. 
\[
\begin{matrix}
g_{11}(x_1) & g_{12}(x_1) & g_{13}(x_1)& \cdots\\
g_{21}(x_2) & g_{22}(x_2)  & g_{23}(x_2)& \cdots\\
\vdots & \vdots &  & \ddots \\
\end{matrix}
\]
そこで$g_m:=g_{mm}$とおけば, これが欲しい点列になる. 

さて上の$\{ g_m\}_{m \ge 1}$は$K_N$上である$g$に一様収束することを示す.
$g : K_{N} \to \C$を以下のように構成する. $x \in \Omega \setminus K_N$ならば$g(x)=0$とする. 
$x \in K_N$とする. $\{g_m(x) \} \subset \C$はCauchy列であることを示す. 
$\varepsilon >0$とする.
すると次のようにできる.
\begin{itemize}
\item $\{g_m \}_{ \ge 1}$は一様同程度連続なので, ある有限この開集合$U_j \subset \Omega$があって$K_N \subset \bigcup_{j=1}^{l}U_j$かつ, 任意の$m \ge 1$かつ$x , y\in U_j$について$|g_m(x) - g_m (y) |<\varepsilon$となる. 
\item  $x \in U_1$として良い. すると上の$x_j$で$x_j \in U_1$となるものが取れる. 
するとある$M_0$があって, 任意の$m, m\ge M_0$について$|g_m(x_j) - g_{m'} (x_j) |<\varepsilon$となる.
\end{itemize}
以上より任意の$m, m\ge M_0$について
\[
|g_m(x) - g_{m'}(x)|
\le |g_m(x) - g_m(x_j)|
+ |g_m(x_j) - g_{m'}(x_j)|
+ |g_{m'}(x_j) - g_{m'}(x)|
\le 3\varepsilon.
\]
となる. よって$\{g_m(x) \} \subset \C$はCauchy列である.
これより$g(x) :=\lim_{m \to \infty}g_m(x)$として定義できる. 

あとはこの議論を繰り返す. ($\{D^1g_m\}_{m \to \infty}$の部分列をとる.) \ref{claim-prop-M-7.4}によって, 部分列$\{g_m(x) \} $と$\Omega$の$C^N$級関数$g$があって, $D^{\alpha}g_m \to D^{\alpha}g$は一様収束する. \footnote{$D^{\alpha}g_m \to D^{\alpha}g$の収束は, おそらく$K_{n}^{\circ}$上にした方が良い. 微分をしているので境界を考えるのは面倒である. }

以下$E$が点列コンパクトであることを示す.
$\{ f_{m} \}_{m \ge 1} \subset E$とする. 
任意の$N \in \N$について, 
ある部分列$\{ f_{m_k}\}$が$K_N$上で$C^N$級関数に一様収束するものがある
よって対角線論法を用いることで, 
ある部分列$\{ f_{m_k}\}$があって, 任意の$\alpha \in \N$について$\{ D^{\alpha}f_{m_k}\}$局所一様収束するものが取れる.
この$f_{m_k} $は$\Omega$上の$C^{\infty}$級関数$f$に局所一様収束する. よって点列コンパクトである. 
\end{proof}

\section{Space of test functions and distributions}

引き続き$\Omega$を空でない$\mathbb{R}^n$の開集合とする. 

\begin{tcolorbox}[mybox]
\begin{defn}
\label{defn-M-8.1}
\[
\mathcal{D}(\Omega) = \{\, \varphi : \Omega \to \mathbb{C} \mid \varphi C^\infty,\ \mathrm{supp}(\varphi)\ \text{コンパクト} \,\}
\]
とする. $\mathcal{D}(\Omega)$の元を\underline{test function}という. 
\end{defn}
\end{tcolorbox}

この節では以下を示す.
\begin{tcolorbox}[mybox]
\textbf{Goal.} 
\begin{itemize}
\item $\mathcal{D}(\Omega)$がlocally convex complete with Heine-Borel propertyを満たす位相ベクトル空間になるような位相$\tau$が存在すること. (この位相は距離化可能とは限らない. よってFrechetとは言えない)
\item $C^{\infty}(\Omega)$に\ref{prop-M-7.3}の位相を入れる. この時$\mathcal{D}(\Omega) \hookrightarrow C^{\infty}(\Omega)$は連続である.
\item $K \subset \Omega$コンパクトとし. \(\mathcal{D}_{K} = \{f \in C^\infty(\Omega) \mid \mathrm{Supp} f \subset K\}\)とする. するとこれは$C^{\infty}(\Omega), \mathcal{D}(\Omega)$の部分集合である. この時$\mathcal{D}_{K}$に誘導される2つの部分位相は同じである.
よって以下の図式が成り立つ. 
\[
\xymatrix@C=70pt@R=25pt{
\mathcal{D}_{K} \ar@{^{(}->}[r]^{\text{closed subspace}}\ar@{^{(}->}[rd]_{\text{closed subspace}}
&\mathcal{D}(\Omega) \ar@{^{(}->}[d]^{\text{conti}}\\
&C^{\infty}(\Omega)
}
\]
\end{itemize}
\end{tcolorbox}
そこで任意の$N \ge 0$について
\[
\|\cdot\|_N : \mathcal{D}(\Omega) \to \mathbb{R}_{\ge 0},\quad
\varphi \mapsto \|\varphi\|_N = \max\left\{\, \left| D^\alpha \varphi(x) \right| \ \middle| \ \alpha \in \mathbb{Z}^n_{\ge 0},\ |\alpha| \le N,\ x \in \Omega \,\right\}
\]
とする. $\|\cdot\|_N$は$\mathbb{C}$ベクトル空間$\mathcal{D}(\Omega)$のノルムであり, 
$\|\cdot\|_N \le \|\cdot\|_{N+1}$となる. 

% [page 75]
\begin{tcolorbox}[mybox]
\begin{lem}
\label{lem-M-8.2}
$K \subset \Omega$をコンパクトとする. 
$\{\|\cdot\|_N\}_{N \ge 0}$で誘導される$\mathcal{D}_K$の位相は, \ref{prop-M-7.3}での$\{p_N\}_{N \ge 1}$による位相と同じである.
\end{lem}
\end{tcolorbox}
ここで$\{\|\cdot\|_N\}_{N \ge 0}$で誘導される$\mathcal{D}_K$の位相とは, 
$f \in \mathcal{D}_K$として, $N \in \N, \varepsilon >0$について, $\{\varphi \in \mathcal{D}_K \mid \|\varphi - f\|_N < \varepsilon\}$が生成する位相である. 

\begin{proof}
\ref{prop-M-7.3}のようにコンパクト集合の列$K_1 \subset K_2 \subset \cdots \subset \Omega$で$\Omega = \bigcup_{i \ge 1} K_i$, $K_i \subset K_{i+1}^{\circ}$で,
$$
p_N(\varphi) = \max\left\{\, |D^\alpha \varphi(x)| \ \middle|\
 \alpha \in \N^n,\ |\alpha| \le N,\ x \in K_N \,
\right\}.
$$
となるようにとる. $K \subset \Omega$はコンパクトなので, ある$N_0$があって
$N \ge N_0$ならば$K \subset K_N$となる. 

任意の$m \in \N$について, ある$N$があって, 
$$
\{\varphi \in \mathcal{D}_K \mid \|\varphi \|_N < \varepsilon\}
\subset 
\{\varphi \in \mathcal{D}_K \mid p_m( \varphi ) < \varepsilon\}
$$
であることを示す.
$m \in \N$を固定する. すると$N \ge \max\{ m, N_0\}$なる$N$について, 
$\varphi \in \mathcal{D}_K$ならば, $\mathrm{Supp}(\varphi) \subset K_N$である. 
よって定義から
\(\|\varphi\|_N = p_N(\varphi)\)となる. 
以上より$N \ge \max\{ m, N_0\}$ならば, 
\[
\{\varphi \in \mathcal{D}_K \mid \|\varphi\|_N < \varepsilon\}
=\{\varphi \in \mathcal{D}_K \mid p_N(\varphi)< \varepsilon\}
\subset \{\varphi \in \mathcal{D}_K \mid p_m(\varphi) < \varepsilon\}.
\]

上の証明で$p_N$と$\|\cdot\|_N$ の役割を入れ替えて議論することができる. 
よって2つの位相は同じである. 
\end{proof}

% [page 77]
\begin{rem}
\label{rem-M-8.3}
$\{\|\cdot\|_N\}_{N \ge 0}$によって, $\mathcal{D}(\Omega)$
locally convex metrizableな$\C$上の位相ベクトル空間
の構造を持つ. がそれは完備ではない (そしてこれはほしい位相ではない).

$\Omega = \mathbb{R}$ 
とする. 自然数$i \in \N$について, $f$を
$\mathrm{Supp}(f) \subset (\frac{1}{3}, \frac{2}{3})$, $f(\frac{1}{2}) =1$, $0 \le f\le 1$となるような滑らかな関数をとる. (1の分割に出てくるBump関数みたいなもの)
そして, 
$$
\varphi_n := \sum_{i=1}^{n} \frac{1}{2^{i -1}} f(x - i)
$$
とする. 
$\varphi_n$は関数として次を満たす.
\begin{itemize}
\item $\varphi_n$は$C^\infty$で$\mathrm{Supp}(\varphi_n) \subset (0,n)$. 特に$\mathcal{D}(\Omega)$の元である.
\item 各$1 \le i \le n$について, $(i -1, i)$上では$0 \le \varphi_n \le \frac{1}{2^{i-1}}$.
\item ある$M>0$があって, 任意の$n, \alpha \in \N$について$|D^\alpha \varphi_n| < M$.($\varphi_n$の微分は$f$にしかよらない定数で抑えられる.)
\end{itemize}

$\{ \varphi_n\}$は$\{\|\cdot\|_N\}_{N \ge 0}$の位相に関してCauchy列である. これを示す. 
まず, $\{\|\cdot\|_N\}_{N \ge 0}$の0でのlocal baseは
$$
V_{N,r} = \{  f \in \mathcal{D}(\mathbb{R}) \mid \|f\|_N < r \,\}
$$
という形をしている. 
そこで, 
すると$m' \ge m \ge 1$ならば, 
\[
\|\varphi_m' - \varphi_{m}\|_N =
\left\| \sum_{i=m+1}^{m'} \frac{1}{2^{i -1}} f(x - i)\right\|_N
\le \frac{1}{2^{m}}M
\]
となる. よってCauchy列である. 

しかし極限は存在しない. 極限$\varphi$が存在したら, $\varphi(i+\frac{1}{2})=\frac{1}{2^{i -1}}$にならないといけず, Supportはコンパクトにならない. 
\end{rem}

% [page 81]
\begin{tcolorbox}[mybox]
\begin{defn}
\label{defn-M-8.4}
\begin{enumerate}
\item $K \subset \Omega$コンパクトとする. $\mathcal{D}_K$の位相$\tau_K$を$\{\|\cdot\|_N\}_{N \ge 0}$で
定められる位相とする. 
これは\ref{lem-M-8.2}により, \ref{prop-M-7.3}での位相と同じであり, 特に$\mathcal{D}_K$はlocally convex, complete, metrizable, with the Heine-Borel propertyである.
\item $\mathcal{D}(\Omega)$の集合族$\beta$を, "空でないconvex, balanced set
 $W \subset \mathcal{D}(\Omega)$で, 任意のコンパクト集合$K \subset \Omega$について
$W \cap \mathcal{D}_K \in \tau_K$となるもの"の集まりとする. 
\item $\mathcal{D}(\Omega)$の集合族$\tau$を, "$\bigcup_{i \in I} (\varphi_i + W_i )$とかけるもの"の集まりとする. 
ただし$i \in I$について, $\varphi_i \in \mathcal{D}(\Omega), W_i \in \beta$とする. 
\end{enumerate}
\end{defn}
\end{tcolorbox}

\begin{rem}
$(\mathcal{D}(\Omega), \tau)$は位相ベクトル空間となる. (後で示す).
$(\mathcal{D}(\R), \tau)$においては, \ref{rem-M-8.3}での点列$\{\varphi_n\}$
はCauchy列にはならない. 

\begin{proof}
$x_m := m+\frac{1}{2}$とする, $c_m >0$について
$$
V:= \{ \varphi \in \mathcal{D}(\R) \mid 
|\varphi (x_m)| < c_m \forall m \ge 1
\}
$$
とする. まず$V \in \beta$であること示す. 
convex, balancedなのは明らか. 任意の$K \subset \Omega$コンパクトについて$K \cap \{ x_m\}$は有限集合より, 
 $V \cap \mathcal{D}_K \in \tau_K$となる. 
 特に$V \in \tau$で$(\mathcal{D}(\R), \tau)$における0の開近傍である. 

$c_m :=\frac{1}{2^m}$とする. すると
任意の$m' > m \ge 1$について,
$$
|\varphi_{m'}(x_{m'}) - \varphi_{m} (x_{m'})|
= \frac{1}{2^{m'-1}}
> \frac{1}{2^{m'}}
= c_{m'}
$$
となる. よっ$\varphi_{m'} - \varphi_{m} \not \in V$となる. これより$(\mathcal{D}(\R), \tau)$においては, \ref{rem-M-8.3}での点列$\{\varphi_n\}$はCauchy列にはならない. 
\end{proof}
\end{rem}

\begin{tcolorbox}[mybox]
\begin{thm}
\label{thm-M-8.6}
$\beta, \tau$を\ref{defn-M-8.4}における$\mathcal{D}(\Omega)$の集合族とする. 
\begin{enumerate}
\item $\tau$ は$\mathcal{D}(\Omega)$の位相であり, $\beta$は$\tau$の$0$でのlocal baseである.
\item $(\mathcal{D}(\Omega), \tau)$は$\C$上のlocally convexな位相ベクトル空間. 
\end{enumerate}
\end{thm}
\end{tcolorbox}

% [page 84]
\begin{proof}
(1). まず$\tau$ は$\mathcal{D}(\Omega)$の位相であることを示す. 
$\varnothing \in \tau$は\ref{defn-M-8.4}において, $I = \varnothing$とおけば良い. 
 $\mathcal{D}(\Omega) \in \beta$より, $\mathcal{D}(\Omega) \in \tau$も明らか. 
 また, $\tau$はunion "$\bigcup$" という操作で閉じている.
 よって示すことは, 「$V_1, V_2 \in \tau$ならば $V_1 \cap V_2 \in \tau$」のみとなる. 

$V_1, V_2 \in \tau$とする. 
$\varphi \in V_1 \cap V_2$をとる. 
すると, $i=1,2$について, ある$\varphi_i \in \mathcal{D}(\Omega)$と$W_i \in \beta$があって, 
\[
\varphi \in \varphi_i + W_i \subset V_i
\]
となる. 
ある$\delta_i \in (0,1)$で
$$
\delta_1 W_1 \cap \delta_2 W_2 \in \beta 
\quad \text{かつ} \quad
\varphi + \delta_1 W_1 \cap \delta_2 W_2  \subset V_1 \cap V_2
$$
となるものが存在することを示せば良い. 
コンパクト集合$K \subset \Omega$で$\varphi, \varphi_1, \varphi_2 \in \mathcal{D}_K$となるものを取る. 
すると\(\varphi - \varphi_i \in W_i \cap \mathcal{D}_K \)である.
$W_i \in \beta$であるので, $W_i \cap \mathcal{D}_K $は$\mathcal{D}_K$
で開集合である. 
$\mathcal{D}_K$は$\mathbb{C}$上の位相ベクトル空間であるので, ある$\delta_i \in (0,1)$があって, 
\( \varphi - \varphi_i \in (1 - \delta_i) W_i \cap \mathcal{D}_K\)とできる. 
以上より, $W_i$はconvexであるので, 
\[ 
\varphi + \delta_i W_i \subset  \varphi_i + \delta_i W_i  + \delta_i W_i 
\subset \varphi_i + W_i
\subset V_i
\]
となる, よって$\varphi + (\delta_1 W_1 \cap \delta_2 W_2) \subset V_1 \cap V_2$となりいえた. 
($\delta_1 W_1 \cap \delta_2 W_2 \in \beta$は簡単にわかる.)
また$\beta$が$0$のlocal baseであることは, 上の議論において$\varphi=0$, $V_1=V_2$として議論すればわかる. 

% [page 85]
(2). \ref{defn-M-8.4}から, $(\mathcal{D}(\Omega), \tau)$において平行移動は同相写像になる.

[$(\mathcal{D}(\Omega), \tau)$は$T_1$であること]
$\varphi_{1}, \varphi_{2} \in \mathcal{D}(\Omega)$で$\varphi_{1} \neq \varphi_{2}$となるものを取る. 
\[
W :=
\left\{
 \varphi \in \mathcal{D}(\Omega) \mid \|\varphi\|_{0} <  \|\varphi_{1} - \varphi_{2}\|_{0}
 \right\}.
\]
$W$はconvex, balanced, $0 \in W$で 任意のコンパクト$K \subset \Omega$について, 
\(W \cap \mathcal{D}_{K}\)は\(\mathcal{D}_{K}\)で開集合になる. 
つまり$W \in \beta$である.
そして, $\varphi_{1} \notin \varphi_{2} + W$かつ$\varphi_{2} \in \varphi_{2} + W$であり, $\varphi_{2} + W \subset \mathcal{D}(\Omega)$でopenなので, 
$$
\mathcal{D}(\Omega) \setminus \{ \varphi_1\} = \bigcup_{\varphi_2 \in \mathcal{D}(\Omega) \setminus \{ \varphi_1\}} \varphi_2 + W
$$
となり, $\{\varphi_{1}\} \subset \mathcal{D}(\Omega)$はclosedである. 

[加法が連続なること.]
$T : \mathcal{D}(\Omega) \times \mathcal{D}(\Omega) \to \mathcal{D}(\Omega)$を$(\varphi_1, \varphi_2)\mapsto \varphi_1 + \varphi_2$とする. 
$\varphi_1 + \varphi_2$の開近傍は$\beta \in W$を使って, $\varphi_1 + \varphi_2 + W$とかける. 
よって, 
$$
T\left(\varphi_1 + \frac{1}{2} W ,  \varphi_2 + \frac{1}{2} W \right)
\subset \varphi_1 + \varphi_2 + W
$$
であるので, $T$は$(\varphi_1, \varphi_2)$で連続となる. 
% [page 87]

[スカラー倍が連続なること.]
$S : \C \times \mathcal{D}(\Omega) \to \mathcal{D}(\Omega)$を$(\alpha_0, \varphi_0)\mapsto \alpha \varphi$とする. 
これが$(\alpha_0, \varphi_0)$で連続であることを示す.

$K \subset \Omega$を$\varphi_0  \in \mathcal{D}_{K}$となるコンパクト集合とする. 
$W \in \beta$とする. すると次が成り立つ. 
\begin{itemize}
\item $W \cap \mathcal{D}_{K} \subset \mathcal{D}_{K}$は開集合なので, ある$\varepsilon>0$があって, 
任意の$ |\alpha| < \varepsilon$について, $\alpha \varphi_{0} \in \frac{1}{3} W$である. 
\item $\alpha_{0} = 0 $または$\varphi \in \frac{1}{3|\alpha_{0}|}\} W$の時は, $W$はbalancedなので, $\alpha_0 \varphi \in \frac{1}{3} W$となる. 
\item $\varphi \in \frac{1}{3}W$かつ$|\alpha| \le 1$ならば, $\alpha \varphi \in \frac{1}{3}W$である. 
\end{itemize}

よって, $|\alpha| < \min \{ \varepsilon, 1\}$かつ$\varphi \in \min \{\frac{1}{3},  \frac{1}{3|\alpha_{0}|}\} W$ならば
$$
(\alpha_0 + \alpha)(\varphi_0 +\varphi)
\in  \alpha_{0}\varphi_{0} + \frac{1}{3} W+ \frac{1}{3} W+\frac{1}{3} W
\subset  \alpha_{0}\varphi_{0} + W
$$
となる. ($W$はconvex balancedを使う)
よってスカラー倍も連続.

以上より$(\mathcal{D}(\Omega), \tau)$は$\C$上の位相ベクトル空間. 
そして, $\beta$の元はconvexであるため, $(\mathcal{D}(\Omega), \tau)$はlocally convex.
\end{proof}


\begin{tcolorbox}[mybox]
\begin{thm}
\label{thm-M-8.7}
ひき続き, $\beta, \tau$を\ref{defn-M-8.4}における$\mathcal{D}(\Omega)$の集合族とする. 
 $(\mathcal{D}(\Omega)$には位相$\tau$を入れる. 
\ref{thm-M-8.6}より, $(\mathcal{D}(\Omega), \tau)$は$\C$上のlocally convexな位相ベクトル空間で, $\beta$は$0$のlocal baseである. 
 
 この時次が成り立つ. 
 \begin{enumerate}
\item $V \subset \mathcal{D}(\Omega)$をconvex balanced setとする. 
$V$が開集合であることは, 任意の$K \subset \Omega$について, $V \cap \mathcal{D}_{K}$は$\mathcal{D}_{K}$で開集合になること$($つまり$V \cap \mathcal{D}_{K} \in \tau_K$$)$と同値.
\item $K \subset \Omega$をコンパクト集合とする. 
この時, $\mathcal{D}_{K} \hookrightarrow \mathcal{D}(\Omega)$によって誘導される部分位相は,  \ref{defn-M-8.4}での$\tau_{K}$と同じである. 
\item  $E \subset \mathcal{D}(\Omega)$がboundedならば, あるコンパクト集合$K \subset \Omega$で$E \subset \mathcal{D}_{K}$となるものが存在し, 任意の$N \in \N$について, 
$\|\cdot\|_{N}:E \to \mathbb{R}$はbounded.
\item $\mathcal{D}(\Omega)$はHeine-Borel propertyを持つ.
\item $\{\varphi_{n}\}_{n=0}^{\infty} \subset \mathcal{D}(\Omega)$がCauchy列ならば, 
あるコンパクト集合$K \subset \Omega$があって, $\{\varphi_{n}\}_{n=0}^{\infty} \subset \mathcal{D}_{K}$かつ, 任意の$N \in \N$について, 
\[
\lim_{n,m \to \infty} \|\varphi_{m} - \varphi_{n}\|_{N} = 0.
\]
となる. $($最後の意味は$\| \cdot \|_{N}$に関してCauchy列になるということ.$)$
\item $\varphi_{n} \to \varphi_{0}$ in $\mathcal{D}(\Omega)$ならば, 
あるコンパクト集合$K \subset \Omega$があって, $\{\varphi_{n}\}_{n=0}^{\infty} \subset \mathcal{D}_{K}$かつ
任意の$\alpha \in \N^n$について, 一様に$D^{\alpha}\varphi_{n} \to \varphi_{0}$と収束する. 
\item $\mathcal{D}(\Omega)$はこの位相において完備である. 
\end{enumerate}
\end{thm}
\end{tcolorbox}

\begin{proof}
次のclaimを示す.

\begin{tcolorbox}[mybox]
\begin{claim}
$V \subset \mathcal{D}(\Omega)$を開集合, $K \subset \Omega$をコンパクト集合とする. 
この時$V \cap \mathcal{D}_K$は$\mathcal{D}_K$上で開集合である. (つまり$V \cap \mathcal{D}_K \in \tau_K$ということ. )
特に \(\mathcal{D}_K \hookrightarrow \mathcal{D}(\Omega)\)は連続
\end{claim}
\end{tcolorbox}

\begin{proof}
任意の$\varphi \in V \cap \mathcal{D}_K$について, ある0のlocal base $W \in \beta$で$\varphi  + W \subset V$となるものが存在する. 
すると$\varphi +  W \cap \mathcal{D}_K \subset V \cap \mathcal{D}_K$であり, $W \cap \mathcal{D}_K$は$\mathcal{D}_K$で開集合である. 
よって, $V \cap \mathcal{D}_K \subset \mathcal{D}_K$は開集合である.
\end{proof}

(1). $\Rightarrow$はclaimから. 
\(\Leftarrow\)について, \(V = \varnothing\)の時は明らか. 
そうでない時は, \(V \in \beta\)よりOK.

% [page 91]
(2). Claimより,  \(\mathcal{D}_K \hookrightarrow \mathcal{D}(\Omega)\)は連続である. 
よってあとは$E \in \tau_K$について, ある$V \subset \mathcal{D}(\Omega)$openであって, $E = V \cap \mathcal{D}_K$となるものが存在することを示せば良い.

$E \in \tau_K$, $\varphi \in E$とする. 
$\tau_K$の位相は\ref{lem-M-8.2}によって,  \(\{\|\cdot\|_N\}_{N \ge 0}\)が誘導する位相と同じであるので, 
ある$N \in\N$, $\delta > 0$があって, 
$$
W_{\phi}: = \{\psi \in \mathcal{D}(\Omega) \mid \|\psi \|_N < \delta\}
$$
とおくと, \( \phi + W_{\phi} \cap \mathcal{D}_K \subset E\)となる.
そして, \(W_{\phi} \in \beta\)であるので, $\phi + W_{\phi} \subset\mathcal{D}(\Omega)$でopenである. 
また
$$
(\phi + W_{\phi}) \cap \mathcal{D}_K 
= \phi + (W_{\phi} \cap \mathcal{D}_K) \subset E
$$
である. 
今$V : = \bigcup_{\phi \in E} (\phi + W_{\phi})$とおくと$V \subset \mathcal{D}(\Omega)$openであり, 
\(V \cap \mathcal{D}_K = E\)となる. よっていえた. 

% [page 92]
(3). \(E \subset \mathcal{D}(\Omega)\) boundedとする. 
まず$E \subset \mathcal{D}_K$となるコンパクト集合の存在を示す.
背理法. 「任意の\(K \subset \Omega\)コンパクト集合について, \( E \not\subset \mathcal{D}_K\)とする. 
すると任意の$m \in \N$について, ある$\varphi_m \in E$と$x_m \in \Omega$があって, 
\begin{itemize}
\item \(\varphi_m(x_m) \neq 0\) かつ
\item  \(\{x_m\}\)は$\Omega$上で集積点を持たない
\end{itemize} 
ものを構成できる. 
これは次のように帰納的に構成する: 
\ref{lem-M-7.2}のようなコンパクト集合の列
\(\Omega = \bigcup_{i \ge 1} K_i\)をとる. 
\(\varphi_1, \ldots, \varphi_m, x_1, \ldots, x_m\)が構成されたとする. 
$x_1, \ldots, x_m \in K_{i}$なる$i$をとる. 
\(E \not\subset \mathcal{D}_{K_{i}}\)より, ある \(\varphi_{m+1} \in E\)で
\(\mathrm{supp}(\varphi_{m})  \not \subset K_{i}\)となるものがある. 
そこである\(x_{m+1} \in \Omega \setminus K_{i}\)で
\(\varphi_{m+1}(x_{m+1}) \neq 0\)となるものが取れる. 
これを繰り返せば構成できる. 

さて$W \subset \mathcal{D}(\Omega)$を
\[
W := 
\left\{ \psi \in \mathcal{D}(\Omega) \mid
 |\psi(x_m)| < \frac{|\varphi_m(x_m)|}{m} (\forall m \ge 1) \right\}
\]
とおく. 
$W$はconvex balancedかつ$ 0 \in W$である.
そして, 任意のコンパクト集合$K \subset \Omega$に関して, 
$K \cap \{ x_m\}_{m \ge 1}$は有限集合となるので, $W \cap \mathcal{D}_K \in \tau_K$となる. 
以上より, \(W \in \beta\)となる. 
一方で$\varphi_m$の取り方から$\varphi_m \not \in W$である.
つまり任意の$m \in \N$について, $E \not \subset m W$である. 
これは$E$がboundedに矛盾する. 

後半の主張に関しては, コンパクト集合$K \subset \Omega$で
\(E \subset \mathcal{D}_K \subset \mathcal{D}(\Omega)\)となるものを取る. 
\(E\)は\(\mathcal{D}_K\)でもboundedである. よって,\ref{thm-M-5.4}より$\|\cdot\|_{N}:E \to \mathbb{R}$はboundedとなる. 

(4). \(E \subset \mathcal{D}(\Omega)\)をbounded closed とする. 
(3)より, あるコンパクト集合\(K \subset \Omega\)があって, \(E \subset \mathcal{D}_K\)となる. 
(2)より, \(E\)は\(\mathcal{D}_K\)の上でbounded closedである. 
\ref{prop-M-7.4}から$E$は$\mathcal{D}_K$上でコンパクトである. よって$\mathcal{D}(\Omega)$でもコンパクトである. 

(5). \(\{\varphi_i\}_{i=1}^\infty \subset \mathcal{D}(\Omega)\) Cauchy列とすると, 
\(\{\varphi_i\}_{i=1}^\infty\)はbounded setである. 
よって(3)からあるコンパクト集合$K \subset \Omega$があって,  \(\{\varphi_i\} \subset \mathcal{D}_K\)となる.
(2)から\(\mathcal{D}_K \subset \mathcal{D}(\Omega)\)は部分位相が入っているので, 
\(\{\varphi_i\}\)は\(\mathcal{D}_K\)上でもCauchy列である. 
つまり任意の$N \in \N$について$\lim_{n,m \to \infty} \|\varphi_{m} - \varphi_{n}\|_{N} = 0$となる. 


(6). $\mathcal{D}(\Omega)$上で$\varphi_i \to 0$とする. (この場合に示せば良い.)
$\{\varphi_i\} \subset \mathcal{D}(\Omega)$はCauchy列になるので, (5)から
あるコンパクト集合$K \subset \Omega$があって, $ \{\varphi_i\} \subset \mathcal{D}_K$となる .
よって$\mathcal{D}_K $上でも$\varphi_i \to 0$となるので, 
$N \in \N$について$ \|\varphi_i\|_N \to 0$である. 
これは任意の$\alpha \in \N^n$について, 一様に$D^\alpha \varphi_i \to 0$と収束する. 

(7). $\{\varphi_i\}  \subset \mathcal{D}(\Omega)$をCauchy列とする. 
(5)からあるコンパクト集合$K \subset \Omega$があって, $ \{\varphi_i\} \subset \mathcal{D}_K$となる .
\ref{prop-M-7.4}によって, $ \mathcal{D}_K$は完備である. (完備な距離空間の閉集合なので.)
よってある$\varphi \in \mathcal{D}_K$があって, 
$\mathcal{D}_K$上で$\varphi_i \to \varphi$
となる. これは$\mathcal{D}(\Omega)$上でも$\varphi_i \to \varphi$となる. よって$\varphi$こそがほしい収束先である. 
\end{proof}

以下\(\mathcal{D}(\Omega)\) には常に\ref{defn-M-8.4}における位相$\tau$を入れる. 
\ref{thm-M-8.7}より\(\mathcal{D}(\Omega)\)はlocally convex completeでHeine-Borel Propertyを持つ位相ベクトル空間である. 

\begin{rem}
$x \in \Omega$について$\text{ev}_x : \mathcal{D}(\Omega) \to \mathbb{C}$を
$\text{ev}_x(\varphi) :=\varphi(x)$として定めると, 
これは連続である. 
なぜならば$\varphi \in \mathcal{D}(\Omega)$と$\varepsilon>0$について
$$
W:=\{ f \in \mathcal{D}(\Omega) \mid \|  f\|_0 < \varepsilon \}
$$
とおくと, これは開集合であり, $\text{ev}_x(\varphi + W) \subset B(\varphi(x), \varepsilon)$となる.
つまり$\text{ev}_x$は$\varphi \in \mathcal{D}(\Omega)$で連続であり, $\varphi$は任意なので, $\text{ev}_x$は連続である. 

また$\mathcal{D}_K = \bigcap_{x \in \Omega \setminus K} \mathrm{ev}_x^{-1}(0)$
 とかけるので, 特に$\mathcal{D}_K \subset \mathcal{D}(\Omega)$はclosedである. 
 \end{rem}

\begin{tcolorbox}[mybox]
\begin{defn}
\label{defn-M-8.8}
\(\Omega \subset \mathbb{R}^n\)を空でない開集合とする. 
\(\Omega\)上の\underline{超関数(distribution)}とは, 連続な\(\mathbb{C}\)線型写像\(\Lambda : \mathcal{D}(\Omega) \to \mathbb{C}\)のことである. 
その集まりを$\mathcal{D}'(\Omega)$と表す. 
\end{defn}
\end{tcolorbox}

\begin{tcolorbox}[mybox]
\begin{prop}
\label{prop-M-8.9}
$Y$を$\C$上のlocally convexな位相ベクトル空間とし, 
\(\Lambda : \mathcal{D}(\Omega) \to Y\)を \(\mathbb{C}\)-線型写像とする. 
以下は同値である. 
\begin{enumerate}[label=$(\alph*)$]
\item \(\Lambda\)は連続.  
\item \(\Lambda\)は有界
\item $\mathcal{D}(\Omega)$上で\(\varphi_i \to 0\)となる点列に対し, $Y$上で$\Lambda \varphi_i \to 0$となる. 
\item 任意のコンパクト集合\(K \subset \Omega\)について, \(\Lambda|_{\mathcal{D}_K} : \mathcal{D}_K \to Y\)は連続. 
\end{enumerate}
\end{prop}
\end{tcolorbox}
よって特に$\C$線型写像\(\Lambda : \mathcal{D}(\Omega) \to \mathbb{C}\)に対して, 
\(\Lambda\)が超関数であることは, \(\Lambda\)が有界であることと同値であり, そして
任意のコンパクト集合\(K \subset \Omega\)について, \(\Lambda|_{\mathcal{D}_K} : \mathcal{D}_K \to Y\)は連続であることと同値である. 
% [page 102]
\begin{proof}

[(a) \(\Rightarrow\) (b)]
\ref{prop-M-4.2}より. 

[(b) \(\Rightarrow\) (c)]
 \(\{\varphi_i\}_{i\ge1} \subset \mathcal{D}(\Omega)\)を$\mathcal{D}(\Omega)$上で
 \(\varphi_i \to 0\)となる点列とする.
  \ref{thm-M-8.7}よりあるコンパクト集合\(K \subset \Omega\)があって, 
  \(\{\varphi_i\} \subset \mathcal{D}_K\)となる. 
今
\[
\Lambda|_{\mathcal{D}_K} : 
\mathcal{D}_K \overset{\text{bounded}} {\hookrightarrow}\mathcal{D}(\Omega) \xrightarrow{\Lambda} Y
\]
という写像もまたboundedになる. 
$\mathcal{D}_K $は\ref{prop-M-7.3}からmetrizableであるので,
\ref{prop-M-4.2}から\(\Lambda|_{\mathcal{D}_K}\)は連続である. 
よって, $\Lambda(\varphi_i)=\Lambda|_{\mathcal{D}_K} (\varphi_i) \to 0$となる. 

% [page 103]
[(c) \(\Rightarrow\) (d)]
\(\{\varphi_i\} \subset \mathcal{D}_K\)を\(\varphi_i \to 0\)となる点列とする. 
(c)の仮定から, \(\Lambda|_{\mathcal{D}_K} (\varphi_i) = \Lambda \varphi_i \to 0\)である. 
これは\ref{prop-M-4.2}から\(\Lambda|_{\mathcal{D}_K}\)は連続であることを意味する. 

[(d) \(\Rightarrow\) (a)]
示すべきことは, 任意の convex balanced open set\(0 \in U \subset Y\)について, 
$\Lambda^{-1}(U) \subset \mathcal{D}(\Omega)$がopenとなることである. 
まず\(\Lambda^{-1}(U)\)はconvex balancedで$0 \in \Lambda^{-1}(U)$となる. 
そして任意のコンパクト集合$K \subset \Omega$について, 
(d)の仮定から\(\Lambda^{-1}(U) \cap \mathcal{D}_K \in \tau_K\)となる. 
よって\ref{thm-M-8.7}より
$\Lambda^{-1}(U)$は$\mathcal{D}(\Omega)$上でopenとなる.
\end{proof}
 

\begin{tcolorbox}[mybox]
\begin{cor}
\label{cor-M-8.10}
$\alpha \in \N^n$について, 
\(D^\alpha : \mathcal{D}(\Omega) \to \mathcal{D}(\Omega)\)を\(D^\alpha(\varphi) :=D^\alpha \varphi\)とすると, これは連続である.
\end{cor}
\end{tcolorbox}

\begin{proof}
$K \subset \Omega$をコンパクト集合とすると, 以下の可換図式が成り立つ. 
\[
\xymatrix{
\mathcal{D}(\Omega) \ar[r]^{D^\alpha} & \mathcal{D}(\Omega) \\
\mathcal{D}_K \ar[r]^{D^\alpha}  \ar@{^{(}->}[u] & 
\mathcal{D}_K \ar@{^{(}->}[u]
}
\]
よって\(D^\alpha : \mathcal{D}_K \to \mathcal{D}_K\)が連続であることを示せば良い .
任意の\(\varphi \in \mathcal{D}_K, N \in \N$について, 
\[
 \|\varphi\|_{N} \leq |\varphi|_{N + |\alpha|} 
\]
である. 
\( \mathcal{D}_K \)の位相は\ref{lem-M-8.2}より\(\{ \| \cdot \|_{N} \}_{N \geq 0} \) で定まっていたので
いえた. 
\end{proof}

\begin{tcolorbox}[mybox]
\begin{prop}
\label{prop-M-8.11}
$\Lambda : \mathcal{D}(\Omega) \to \mathbb{C}$を$\mathbb{C}$線型写像とする. 
以下は同値である. 
\begin{enumerate}[label=$(\alph*)$]
\item \(\Lambda \in D'(\Omega)\), つまり$\Lambda$は超関数である. 
\item 任意のコンパクト集合\(K \subset \Omega$について, ある$N \geq 0$とある$C>0$があって, 任意の
$\varphi \in \mathcal{D}_{K}$について, 
\[
|\Lambda \varphi| \leq C \|\varphi\|_{N}
\]
が成り立つ. 
\end{enumerate}
\end{prop}
\end{tcolorbox}


\begin{proof}

[(b) \(\Rightarrow\) (a)]
\(K \Subset \Omega\)をコンパクト集合とする. 
\ref{prop-M-8.9}から示すことは, 
\(\Lambda|_{\mathcal{D}_{K}} : \mathcal{D}_{K} \hookrightarrow D(\Omega) \xrightarrow{\Lambda} \mathbb{C}\)
が連続となることである. 
平行移動して\(\Lambda|_{\mathcal{D}_{K}}\) が0で連続であることを示せば良い. 
仮定(b)にあるような $N \geq 0, C>0$を固定する. 
任意の$\varepsilon >0$について, 
$$
V = \{\varphi \in \mathcal{D}_{K} \mid \|\varphi\|_{N} < \frac{\varepsilon}{C} \} 
$$
とする. 
$0 \in V \subset \mathcal{D}_{K}$openであり, 任意の$\varphi \in V$について
\(|\Lambda|_{D_{K}}(\varphi)| < \varepsilon\)である. 
よって$\Lambda|_{D_{K}}$は連続である. 

[(a) \(\Rightarrow\) (b)]  
\(K \subset \Omega\)をコンパクト集合とする. 
すると$\Lambda|_{\mathcal{D}_{K}} : \mathcal{D}_{K} \to \mathbb{C}$もまた連続である. 
よってある$N \geq 0$とある$\varepsilon>0$があって,
\[
\Lambda \left( \{ \varphi \in \mathcal{D}_{K} \mid \|\varphi\|_{N} < \varepsilon\} \right)
 \subset 
\{ z \in \mathbb{C} \mid |z| < 1 \} 
\]  
となる. 
これは任意の0でない$\varphi \in \mathcal{D}_{K}$について, 
$|\Lambda( \frac{\varepsilon}{2\|\varphi\|_{N}} \varphi)| < 1$となる. 
よって任意の$\varphi \in \mathcal{D}_{K}$について, 
$$
 |\Lambda(\varphi)| < \frac{2}{\varepsilon} \|\varphi\|_{N} 
 $$
 となる. $C=\frac{2}{\varepsilon}$とおけばいえた. 
 \end{proof}

\begin{tcolorbox}[mybox]
\begin{defn}
\label{defn-M-8.12}
%\underline{Def 8.12}  
\(\Lambda \in \mathcal{D}'(\Omega)\)とする. 
\(\Lambda\)が\underline{finite order}を持つとは, 
"ある$N \geq 0$があって, 任意のコンパクト集合\(K \subset \Omega$について, ある$C>0$があって, 任意の
$\varphi \in \mathcal{D}_{K}$について, 
\[
|\Lambda \varphi| \leq C \|\varphi\|_{N}
\]
が成り立つ"こと. 
これが成り立つ最小の自然数\(N\)を\(\Lambda\)のorderという. 
\end{defn}
\end{tcolorbox}
% [page 109]

\begin{ex}[Diracの超関数]
\(x\in \Omega\)について, 
\(\delta_{x} = \mathcal{D}(\Omega) \to \mathbb{C}\)を$\delta_{x}(\varphi):=\varphi(x)$とする. 
\(\delta\)はdistributionである. 
なぜならば
任意の$\varphi \in \mathcal{D}(\Omega)$について, 
\[
|\delta_{x}(\varphi)| = |\varphi(x)| \leq \|\varphi\|_{0}
\]
が成り立つので, 
\ref{prop-M-8.11}からわかる. 
さらに \(\delta_{x}\)はfinite orderを持つ, orderは0である. 
\end{ex}

