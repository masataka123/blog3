%\documentclass[dvipdfmx,a4paper,12pt]{amsart}
\documentclass[dvipdfmx,a4paper,11pt]{report} %titleとe-mailをコメントアウトする.


%%% Packages %%%
\setlength{\lineskip}{0pt}

% --- 和文フォント設定 (ゴシックを使うなら) ---amsartを使う時はコメントアウト
\renewcommand{\kanjifamilydefault}{\gtdefault}
\usepackage{otf}          % min10を避けるため
\usepackage{pxrubrica}    % 和文ルビ

% --- 基本パッケージ ---
\usepackage{graphicx}
\usepackage[all]{xy}
\usepackage{wrapfig}
\usepackage{pgfplots}
\usepackage{color}
\usepackage[dvipsnames]{xcolor}

% --- 数学関連 ---
\usepackage{amsmath,amssymb,amsthm,amsfonts,mathtools}
\usepackage{amscd,dsfont,bigdelim,braket,physics,mathrsfs,bm}

% --- 書式・リスト関連 ---
\usepackage{latexsym}
\usepackage{setspace}
\usepackage{multirow}
\usepackage{enumerate}
\usepackage{enumitem}

% --- コメント・取消線など ---
\usepackage{comment}
\usepackage[normalem]{ulem} % \emph の下線化を抑止(cancelと共存)

% --- URL・文字コード ---
\usepackage{url}
% \usepackage[utf8]{inputenc}  % ← 使用しているエンジンがuplatexなら不要、pdflatexなら有効に

% --- showkeys(常に表示) ---
\usepackage{showkeys}
\renewcommand*{\showkeyslabelformat}[1]{%
 \fbox{\parbox{1.6cm}{\normalfont\tiny\sffamily#1\vspace{6mm}}}%
}
% --- hyperrefは最後に読み込む ---
\usepackage[dvipdfmx,colorlinks,linkcolor=blue,anchorcolor=blue,citecolor=blue]{hyperref}

%%% レイアウト調整 %%%
%%% レイアウト調整(geometryに統一) %%%
\usepackage[
  top=30mm,        % 上余白
  bottom=30mm,     % 下余白
  left=25mm,       % 左余白
  right=25mm,      % 右余白
  headheight=12pt, % ヘッダー高さ(必要なら)
  headsep=10mm,    % ヘッダーと本文の間
  footskip=32pt,   % 本文とフッターの距離
  includehead,     % ヘッダー分も高さに含める
  includefoot      % フッター分も高さに含める
]{geometry}

%%% 行間調整(適宜 1.2 などに変更) %%%
\usepackage{setspace}
\setstretch{1.1}

% --- 段落設定 --- 文字の行間ならここを変更する
\setlength{\parskip}{5pt}   % 段落間のスペース
\setlength{\parindent}{0pt}   % 段落先頭の字下げをなくす


%%% 追加(重複なし)パッケージ・設定 %%%

% --- 目次の体裁調整 ---
%\usepackage{tocloft}
%\renewcommand{\contentsname}{目次} % 日本語化
%\setlength{\cftbeforesecskip}{0pt}
%\setlength{\cftbeforesubsecskip}{0pt}
%\setlength{\cftbeforesubsubsecskip}{0pt}

% --- セクション見出しの体裁調整 ---
%\usepackage{titlesec}
%\titleformat*{\section}{\Large\bfseries}
%\titleformat*{\subsection}{\large\bfseries}
%\titlespacing*{\section}{0pt}{1.5ex plus .2ex minus .2ex}{0.8ex plus .1ex}
%\titlespacing*{\subsection}{0pt}{1.0ex plus .2ex minus .2ex}{0.5ex plus .1ex}

% --- ヘッダー/フッター設定 ---
%\usepackage{fancyhdr}
%\pagestyle{fancy}
%\fancyhf{}
%\rhead{岩井 雅崇}
%\lhead{大阪大学 数学専攻}
%\cfoot{\thepage}


% -- enumerate, itemize行間設定
\usepackage{enumitem} % デフォルト設定
\setlist[itemize]{itemsep=3pt, parsep=0pt}
\setlist[enumerate]{itemsep=3pt, parsep=0pt}
% "変更する際は右を使う" \setlength{\parskip}{0cm} % 段落間\setlength{\itemsep}{5pt} % 項目間

% --- tcolorbox 設定 ---%\begin{tcolorbox}[mybox]と使う
\usepackage[most]{tcolorbox}
\tcbuselibrary{breakable, skins, theorems}
\tcbset{
  mybox/.style={
    colback = white,
    colframe = green!35!black,
    fonttitle = \bfseries,
    breakable = true
  }
}

% --- TikZ 設定 ---
\usepackage{tikz}
\usetikzlibrary{positioning, arrows.meta, fit, calc, backgrounds}
\pgfdeclarelayer{background}
\pgfdeclarelayer{foreground}
\pgfsetlayers{background,main,foreground}

% --- footnote がページをまたがない設定 ---
\interfootnotelinepenalty=10000

% --- 目次に表示する階層の深さ ---
\setcounter{tocdepth}{2}

% --- 日本語目次---
\usepackage{pxjahyper}

%--newtheorem%--newcommand----

\newtheorem{thm}{Theorem}[section] 
\newtheorem{theo}[thm]{Theorem}
\newtheorem{cor}[thm]{Corollary}
\newtheorem{prop}[thm]{Proposition}
\newtheorem{conj}[thm]{Conjecture}
\newtheorem*{mainthm}{Theorem}
\newtheorem{deflem}[thm]{Definition-Lemma}
\newtheorem{lem}[thm]{Lemma}
\theoremstyle{definition} 
\newtheorem{defn}[thm]{Definition}
\newtheorem{propdefn}[thm]{Proposition-Definition} 
\newtheorem{lemdefn}[thm]{Lemma-Definition} 
\newtheorem{thmdefn}[thm]{Theorem-Definition} 
\newtheorem{eg}[thm]{Example} 
\newtheorem{ex}[thm]{Example} 
\newtheorem{ques}[thm]{Question}
\newtheorem{remin}[thm]{Reminder}
\theoremstyle{remark}
\newtheorem{rem}[thm]{Remark}
\newtheorem{setup}[thm]{Setup}
\newtheorem{obs}[thm]{Observation}
\newtheorem{notation}[thm]{Notation}
\newtheorem{cl}{Claim}
\newtheorem{claim}[thm]{Claim}
\newtheorem{assup}[thm]{Assumption}
\newtheorem{step}{Step}
\newtheorem*{clproof}{Proof of Claim}
\newtheorem{cln}[thm]{Claim}
\newtheorem*{ack}{Acknowledgements} 
\numberwithin{equation}{section}
\newtheorem{case}{Case}

\newcommand{\rk}[0]{\operatorname{rk}}
\newcommand{\supp}[0]{\operatorname{Supp}}
\newcommand{\Rad}[0]{\operatorname{Rad}}
\newcommand{\Sha}[0]{\operatorname{Sha}}
\newcommand{\sha}[0]{\operatorname{sha}}
\newcommand{\eend}[0]{\operatorname{End}}
\newcommand{\codim}[0]{\operatorname{codim}}
\newcommand{\nd}[0]{\operatorname{nd}}
\renewcommand{\rank}[0]{\operatorname{rank}}
\newcommand{\degree}[0]{\operatorname{deg}}
\newcommand{\Exc}[0]{\operatorname{Exc}}
\newcommand{\pr}{{\rm pr}}
\newcommand{\id}{{\rm id}}
\newcommand{\Sym}{{\rm Sym}}
\newcommand{\End}[0]{\operatorname{End}}
\newcommand{\Coker}[0]{\operatorname{Coker}}

\newcommand{\Supp}{{\rm Supp}}
\newcommand{\Hom}[0]{\mathscr{H}\!\textit{om}}
\newcommand{\Ext}[0]{\mathscr{E}\!\textit{xt}}
\newcommand{\GL}[0]{\operatorname{GL}}
\newcommand{\SheafHom[1]}{\mathscr{H}\!\!\!\text{\calligra om}_{\,{#1}}}
\newcommand{\PGL}[0]{\mathbb{P}\GL(r,\C)}

\newcommand{\Alb}{{\rm Alb}}
\newcommand{\verti}{{\rm vert}}
\newcommand{\hor}{{\rm hor}}
\newcommand{\univ}{{\rm univ}}
\newcommand{\Tor}{{\rm tor}}
\newcommand{\shaf}{\mathrm{sha}}
\newcommand{\Shaf}{\mathrm{Sha}}
\newcommand{\reg}{{\rm{reg}}}
\newcommand{\sing}{{\rm{sing}}}
\newcommand{\qt}{{\rm{qt}}}
\newcommand\sO{{\mathcal O}}
\newcommand{\Div}[0]{\operatorname{div}}
\newcommand{\ddbar}{dd^c}
\newcommand{\cV}{\mathcal{V}}
\newcommand{\deldel}{\sqrt{-1}\partial \overline{\partial}}
\newcommand{\dbar}{\overline{\partial}}
\newcommand{\I}[1]{\mathcal{I}(#1)}
\newcommand{\Aut}[1]{\mathrm{Aut}(#1)}
\newcommand{\Ker}[1]{\mathrm{Ker}(#1)}
\newcommand{\Image}[1]{\mathrm{Im}(#1)}
\DeclareMathOperator{\Ric}{Ric}
\DeclareMathOperator{\Vol}{Vol}
 \newcommand{\pdrv}[2]{\frac{\partial #1}{\partial #2}}
 \newcommand{\drv}[2]{\frac{d #1}{d#2}}
  \newcommand{\ppdrv}[3]{\frac{\partial #1}{\partial #2 \partial #3}}
\newcommand{\underalign}[2]{\quad \underset{\mathclap{\strut #1}}{#2}\quad}
\newcommand{\polar}{\beta}
  
\newcommand{\R}{\mathbb{R}}
\newcommand{\Z}{\mathbb{Z}}
\newcommand{\N}{\mathbb{Z}_+}
\newcommand{\C}{\mathbb{C}}
\newcommand{\Q}{\mathbb{Q}}
\newcommand{\D}{\mathbb{D}}
\newcommand{\mP}{\mathbb{P}}
\newcommand{\mO}{\mathcal{O}}
\newcommand{\B}{\mathds{B}}
\newcommand{\tl}{\hspace{-0.8ex}<\hspace{-0.8ex}}
\renewcommand{\tr}{\hspace{-0.8ex}>}

\newcommand{\xb}[1]{\textcolor{blue}{#1}}
\newcommand{\xr}[1]{\textcolor{red}{#1}}
\newcommand{\xm}[1]{\textcolor{magenta}{#1}}

\newcommand{\illegible}[1]{\textcolor{red}{[ILLEGIBLE: #1]}}

\title{カレントセミナーのノート} 
\author{岩井雅崇}
%\address{Department of Mathematics, Graduate School of Science, Osaka City University 3-3-138, Sugimoto, Sumiyoshi-ku Osaka, 558-8585Japan} 
%\email{{\tt masataka.math@gmail.com}}
%\email{{\tt masataka.math@gmail.com, masataka@sci.osaka-cu.ac.jp}}
\date{\today, version 0.01}

\renewcommand{\thefootnote}{\arabic{footnote}}

\baselineskip = 15pt
\footskip = 32pt

\begin{document}
\begin{thebibliography}{n}
\bibitem[Rud]{Rud}
W. Rudin. \textit{Functional analysis.} 2nd edn. International Series in Pure and Applied Mathematics. McGraw-Hill, Inc., New York. (1991.)
\bibitem[NO]{NO}
J. Noguchi, T.Ochiai \textit{Geometric Function Theory in Several Complex Variables} Translations of Mathematical Monographs
Volume: 80; 1990; 282 pp
\end{thebibliography}
 
\cite[Chapter 3]{NO}を参考にしている. 

\section{Current}

$M$ 2nd countable$m$次元$C^\infty$級多様体とする. 

\textbf{Recall}
$M \subset \R^m$の時$C^{\infty}(M)$には次の位相を次で入れていた. 

$K_1 \subset K_2 \subset \cdots \subset M$というコンパクト集合であって
$$
K_i \subset K_{i+1}^{\circ} \quad \text{and} \quad M = \bigcup K_{i}^{\circ} 
$$ 
となるものを一つ固定し, $f \in C^{\infty}(M)$について, $N \in \N$として
$$
P_{N}(f) := \max\{|D^{\alpha} f(x)| \mid |\alpha| \le N \& x \in K_{N} \}
$$
$$
V_{N} := \max\{ f \in C^{\infty}(M) \mid P_N (f) < \frac{1}{N}\}
$$
を$0$のopen baseとするような位相を入れていた.

$K \subset M$コンパクトに対して
$$
\mathcal{D}_{K}(M):=
\{ \phi \in C^{\infty}(M) \mid \Supp(\phi) \subset K\} \subset C^{\infty}(M)
$$
に対して相対位相を入れて
$$
\mathcal{D}(M):= \bigcup_{K \subset M \mid \text{$K$ cpt}} \mathcal{D}_{K}(M)
\cong 
\varinjlim_{K} \mathcal{D}_{K}(M)
$$
このcolimitはlocally convex vector spaceでのcolimitである. 

\begin{tcolorbox}[mybox]
\begin{lem}
\label{lem-E-1}
 \begin{enumerate}
\item  $U, U'\subset \R^m$ を開集合とする.  $U \cong U'$を微分同相とする時, $C^{\infty}(U) \cong C^{\infty}(U')$である. ここでこの同型は位相$\C$ベクトル空間の同型である
\item $M \subset \R^m$ 開集合とし, $M = \bigcup U_i$を可算個の開被覆とする時
$$
C^{\infty}(M) \hookrightarrow \prod C^{\infty}(U_i)
\quad
f \mapsto \{ f|_{U_i}\}  
$$
とすると, $C^{\infty}(M)$の位相はこの直積位相$\prod C^{\infty}(U_i)$によって引き起こされる位相となる. 
特に
$$
C^{\infty}(M)  \cong \mathrm{Eq}(\prod C^{\infty}(U_i) \rightrightarrows \prod C^{\infty}(U_i \cap U_j) )
$$
という位相$\C$ベクトル空間の同型を得る. 
\end{enumerate}
\end{lem}
\end{tcolorbox}
\begin{proof}

(1). 以下$K_1 \subset K_2 \subset \cdots \subset U$というコンパクト集合で$C^{\infty}(U)$の位相を誘導するものを一つ固定する. 
$\Phi : U' \to U, (x_1, \ldots, x_m)\mapsto (y, \ldots, y_m)$を微分同相写像とする. 
すると$K'_{i}:=\Phi(K_i)$によって, $C^{\infty}(U')$の位相を誘導する.
$V_{N}, V'_{N}$を上の通りとする. 

さてその引き戻し
$$
\Phi^{*} : C^{\infty}(U) \rightarrow C^{\infty}(U')
\quad
f(y) \mapsto f \circ \Phi(x)=f (y (x))
$$
が位相$\C$ベクトル空間の同型を誘導することを示す. 

$f \in C^{\infty}(U)$についてchain ruleより, $|\alpha | \le N'$について
\begin{equation}
\label{eq-lem-E-1}
D_{x}^{\alpha}f \circ \Phi(x)
:=
\sum_{|\beta| \le N'} (D_{y}^{\beta} f) (y(x)) \cdot \Phi_{\alpha \beta}(x)
\end{equation}
である. 

$\Phi$が$0$の近傍で連続であることを示せば良い. 
任意の$N'$について, $K_{N'}$コンパクトなので, $K_{N'}$上では$|\Phi_{\alpha \beta}(x)| \le C_{N'}$となる$C_{N'}$が取れる. 
よって$N' \cdot \frac{1}{N} \cdots C_{N'} < \frac{1}{N'}$となる$N$をとれば, 
$f \circ \Phi \in \Phi^{*}(V_N)$について
$$
|D_{x}^{\alpha}f \circ \Phi(x)|
\underset{(\ref{eq-lem-E-1})}{\le}
\sum_{|\beta| \le N} \underbrace{|(D_{y}^{\beta} f) (y(x))|}_{<\frac{1}{N}} \cdot |\Phi_{\alpha \beta}(x)|
\le N' \cdot \frac{1}{N} \cdot C_{N'} < \frac{1}{N}
$$
よって「任意の$N'>0$について, ある$N>0$があって, $\Phi^{*}(V_N)  \subset V'_{N'}$である」ため$\Phi$は$0$の近傍で連続である. 

(2) $M = \bigcup_{i=1}^{\infty} U_i$とする. 
各$U_i$で
$$
K_{i1} \subset K_{i2} \subset \cdots \subset U_i
$$
で$U_i = \bigcup_{j} K_{ij}^{\circ} $となるコンパクト列をとる.
そこで$K_N := \bigcup_{i=1}^{N}K_{i, N}$とすると
$$
K_{1} \subset K_{2} \subset \cdots \subset M
$$
であって, $K_i \subset K_{i+1}^{\circ}$ かつ$M=\bigcup K_{i}^{\circ}$となる. 

さて$\prod C^{\infty}(U_i)$の$0$でのlocal baseは
$$
\{ 
V_{\bm{i}, \bm{N}}:=V_{i_1, N_{i_1}} \times V_{i_2, N_{i_2}} \times \cdots \times V_{i_l, N_{i_l}} \times 
\prod_{i \neq i_k} C^{\infty}(U_i) \mid i_1 < i_2 < \cdots <i_l, N_{i_k} \in \N
\}
$$
となる形のものである. 
ここで
\begin{itemize}
\item $\bm{i}:=(i_1, i_2, \ldots, i_{l}), \bm{N}:=(N_{i_1}, N_{i_2},\ldots, N_{i_l})$と定める. 
\item $V_{i, N}:= \{ f \in C^{\infty}(U_i) \mid P_{N}(f) < \frac{1}{N}\}$である(ここの定義での$ P_{N}(f) $には$K_{i, N}$をつかう.)
\end{itemize}

一方で$V_{N}:= \{ f \in C^{\infty}(M) \mid P_{N}(f) < \frac{1}{N}\}$(ここの定義での$ P_{N}(f) $には$K_{iN}$をつかう.)であり, これは$C^{\infty}(M)$の0でのlocal baseである. 
以上より, 次の二つを示せば良い.
\begin{itemize}
\item 任意の$\bm{i}, \bm{N}$について, ある$N$があって, $V_N \subset V_{\bm{i}, \bm{N}} \cap  C^{\infty}(M)$が成り立つ. 
\item 任意の$N$について, ある$\bm{i}, \bm{N}$があって, $V_N \supset V_{\bm{i}, \bm{N}} \cap  C^{\infty}(M)$が成り立つ. 
\end{itemize}


\begin{align*}
V_N &:= 
\left\{ f \in C^{\infty}(M) \mid x \in K_{N}:=\bigcup_{i=1}^{N}K_{i, N}, |\alpha| \le N, |D_\alpha f(x)|<\frac{1}{N}\right\} \\
&\underset{\text{def}}{=}
\left( V_{1, N} \times V_{2, N} \times \cdots \times V_{N, N} \times 
\prod_{i > N} C^{\infty}(U_i)  \right) \cap C^{\infty}(M) \\
&\underset{\text{def}}{=} V_{(1,2, \ldots, N), (N, \ldots, N)} \cap C^{\infty}(M) \\
\end{align*}
である. これより二つ目の主張が正しいことがいえる.
一つ目の主張は任意の$\bm{i}, \bm{N}$について, 
$$
V_{\max\{\bm{i}, \bm{N}\} }\subset V_{\bm{i}, \bm{N}} \cap C^{\infty}(M)
$$
なので言える. 
\end{proof}


\begin{tcolorbox}[mybox]
\begin{defn}
\label{defn-E-2}
$M$ 2nd countable$m$次元$C^\infty$級多様体とする. 
$C^\infty(M)$に位相を次のように入れる.

まず$M = \bigcup_{l=1}^{\infty} U_i $で$U_i \subset \mathbb{R}^m $となるcountable open coverを一つ固定する. 
そして, 
\[
C^\infty(M) \hookrightarrow \prod_i C^\infty(U_i) \quad 
f \longmapsto (f|_{U_i})
\]
による部分位相を$C^\infty(M)$に入れる. 
つまり,
\[
C^{\infty}(M)  \cong \mathrm{Eq}(\prod C^{\infty}(U_i) \rightrightarrows \prod C^{\infty}(U_i \cap U_j) )
\]
となるように入れる. 
\end{defn}
\end{tcolorbox}

\begin{rem}
\ref{defn-E-2}による位相の定義において, $U_i$の取り方によらない.
\end{rem}

\begin{proof}
別の$U'_j$をとると細分$U_i \cap U'_j$が取れる.
よって次の図式が考えられる. 
  \[
\xymatrix@C=50pt@R=20pt{
C^{\infty}(M) \ar@{^{(}-_>}[r] \ar@{^{(}-_>}[rd]
&\prod C^{\infty}(U_i) \ar@{^{(}-_>}[rd]
&   \\
 &\prod C^{\infty}(U'_j)  \ar@{^{(}-_>}[r] & \prod C^{\infty}(U_i \cap U'_j)  \\
}
\]
そこで"相対位相の直積は相対位相になる"ので,   
\ref{lem-E-1}より言える.
\end{proof}

これにより
$$
C^{\infty} : (U \underset{open}{\subset} M) \mapsto C^{\infty}(U)
$$
は$\C$ベクトル空間のsheafとなる. 
よって$K \subset M$に対して
$$
\mathcal{D}_{K}(M) := \{ \phi \in C^{\infty}(M) \mid \Supp(\phi) \subset K\}
$$
として$C^{\infty}(M)$の部分位相を入れる. 
$$
\mathcal{D}(M):= \bigcup_{K \subset M \mid \text{$K$ cpt}} \mathcal{D}_{K}(M)
\cong 
\varinjlim_{K} \mathcal{D}_{K}(M)
$$
このcolimitはlocally convex vector spaceでのcolimitである. 

[指摘]
このcolimは存在する?
やっぱりSection 1でやったような位相の入れ方にもう一回戻る?
$\varinjlim_{K} \mathcal{D}_{K}(M)$については次回再考. 



\end{document}