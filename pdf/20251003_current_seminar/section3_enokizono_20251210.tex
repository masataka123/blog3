\chapter{Currentの定義}

\begin{thebibliography}{n}
\bibitem[Rud]{Rud}
W. Rudin. \textit{Functional analysis.} 2nd edn. International Series in Pure and Applied Mathematics. McGraw-Hill, Inc., New York. (1991.)
\bibitem[NO]{NO}
J. Noguchi, T.Ochiai \textit{Geometric Function Theory in Several Complex Variables} Translations of Mathematical Monographs
Volume: 80; 1990; 282 pp
\end{thebibliography}
 
\cite[Chapter 3]{NO}を参考にしている. 

\section{Current}

$M$ 2nd countable$m$次元$C^\infty$級多様体とする. 

\textbf{Recall}
$M \subset \R^m$の時$C^{\infty}(M)$には次の位相を次で入れていた. 

$K_1 \subset K_2 \subset \cdots \subset M$というコンパクト集合であって
$$
K_i \subset K_{i+1}^{\circ} \quad \text{and} \quad M = \bigcup K_{i}^{\circ} 
$$ 
となるものを一つ固定し, $f \in C^{\infty}(M)$について, $N \in \N$として
$$
P_{N}(f) := \max\{|D^{\alpha} f(x)| \mid |\alpha| \le N \& x \in K_{N} \}
$$
$$
V_{N} := \max\{ f \in C^{\infty}(M) \mid P_N (f) < \frac{1}{N}\}
$$
を$0$のopen baseとするような位相を入れていた.

$K \subset M$コンパクトに対して
$$
\mathcal{D}_{K}(M):=
\{ \phi \in C^{\infty}(M) \mid \Supp(\phi) \subset K\} \subset C^{\infty}(M)
$$
に対して相対位相を入れて
$$
\mathcal{D}(M):= \bigcup_{K \subset M \mid \text{$K$ cpt}} \mathcal{D}_{K}(M)
\cong 
\varinjlim_{K} \mathcal{D}_{K}(M)
$$
このcolimitはlocally convex vector spaceでのcolimitである. 

\begin{tcolorbox}[mybox]
\begin{lem}
\label{lem-E-1}
 \begin{enumerate}
\item  $U, U'\subset \R^m$ を開集合とする.  $U \cong U'$を微分同相とする時, $C^{\infty}(U) \cong C^{\infty}(U')$である. ここでこの同型は位相$\C$ベクトル空間の同型である
\item $M \subset \R^m$ 開集合とし, $M = \bigcup U_i$を可算個の開被覆とする時
$$
C^{\infty}(M) \hookrightarrow \prod C^{\infty}(U_i)
\quad
f \mapsto \{ f|_{U_i}\}  
$$
とすると, $C^{\infty}(M)$の位相はこの直積位相$\prod C^{\infty}(U_i)$によって引き起こされる位相となる. 
特に
$$
C^{\infty}(M)  \cong \mathrm{Eq}(\prod C^{\infty}(U_i) \rightrightarrows \prod C^{\infty}(U_i \cap U_j) )
$$
という位相$\C$ベクトル空間の同型を得る. 
\end{enumerate}
\end{lem}
\end{tcolorbox}
\begin{proof}

(1). 以下$K_1 \subset K_2 \subset \cdots \subset U$というコンパクト集合で$C^{\infty}(U)$の位相を誘導するものを一つ固定する. 
$\Phi : U' \to U, (x_1, \ldots, x_m)\mapsto (y, \ldots, y_m)$を微分同相写像とする. 
すると$K'_{i}:=\Phi(K_i)$によって, $C^{\infty}(U')$の位相を誘導する.
$V_{N}, V'_{N}$を上の通りとする. 

さてその引き戻し
$$
\Phi^{*} : C^{\infty}(U) \rightarrow C^{\infty}(U')
\quad
f(y) \mapsto f \circ \Phi(x)=f (y (x))
$$
が位相$\C$ベクトル空間の同型を誘導することを示す. 

$f \in C^{\infty}(U)$についてchain ruleより, $|\alpha | \le N'$について
\begin{equation}
\label{eq-lem-E-1}
D_{x}^{\alpha}f \circ \Phi(x)
:=
\sum_{|\beta| \le N'} (D_{y}^{\beta} f) (y(x)) \cdot \Phi_{\alpha \beta}(x)
\end{equation}
である. 

$\Phi$が$0$の近傍で連続であることを示せば良い. 
任意の$N'$について, $K_{N'}$コンパクトなので, $K_{N'}$上では$|\Phi_{\alpha \beta}(x)| \le C_{N'}$となる$C_{N'}$が取れる. 
よって$N' \cdot \frac{1}{N} \cdots C_{N'} < \frac{1}{N'}$となる$N$をとれば, 
$f \circ \Phi \in \Phi^{*}(V_N)$について
$$
|D_{x}^{\alpha}f \circ \Phi(x)|
\underset{(\ref{eq-lem-E-1})}{\le}
\sum_{|\beta| \le N} \underbrace{|(D_{y}^{\beta} f) (y(x))|}_{<\frac{1}{N}} \cdot |\Phi_{\alpha \beta}(x)|
\le N' \cdot \frac{1}{N} \cdot C_{N'} < \frac{1}{N}
$$
よって「任意の$N'>0$について, ある$N>0$があって, $\Phi^{*}(V_N)  \subset V'_{N'}$である」ため$\Phi$は$0$の近傍で連続である. 

(2) $M = \bigcup_{i=1}^{\infty} U_i$とする. 
各$U_i$で
$$
K_{i1} \subset K_{i2} \subset \cdots \subset U_i
$$
で$U_i = \bigcup_{j} K_{ij}^{\circ} $となるコンパクト列をとる.
そこで$K_N := \bigcup_{i=1}^{N}K_{i, N}$とすると
$$
K_{1} \subset K_{2} \subset \cdots \subset M
$$
であって, $K_i \subset K_{i+1}^{\circ}$ かつ$M=\bigcup K_{i}^{\circ}$となる. 

さて$\prod C^{\infty}(U_i)$の$0$でのlocal baseは
$$
\{ 
V_{\bm{i}, \bm{N}}:=V_{i_1, N_{i_1}} \times V_{i_2, N_{i_2}} \times \cdots \times V_{i_l, N_{i_l}} \times 
\prod_{i \neq i_k} C^{\infty}(U_i) \mid i_1 < i_2 < \cdots <i_l, N_{i_k} \in \N
\}
$$
となる形のものである. 
ここで
\begin{itemize}
\item $\bm{i}:=(i_1, i_2, \ldots, i_{l}), \bm{N}:=(N_{i_1}, N_{i_2},\ldots, N_{i_l})$と定める. 
\item $V_{i, N}:= \{ f \in C^{\infty}(U_i) \mid P_{N}(f) < \frac{1}{N}\}$である(ここの定義での$ P_{N}(f) $には$K_{i, N}$をつかう.)
\end{itemize}

一方で$V_{N}:= \{ f \in C^{\infty}(M) \mid P_{N}(f) < \frac{1}{N}\}$(ここの定義での$ P_{N}(f) $には$K_{iN}$をつかう.)であり, これは$C^{\infty}(M)$の0でのlocal baseである. 
以上より, 次の二つを示せば良い.
\begin{itemize}
\item 任意の$\bm{i}, \bm{N}$について, ある$N$があって, $V_N \subset V_{\bm{i}, \bm{N}} \cap  C^{\infty}(M)$が成り立つ. 
\item 任意の$N$について, ある$\bm{i}, \bm{N}$があって, $V_N \supset V_{\bm{i}, \bm{N}} \cap  C^{\infty}(M)$が成り立つ. 
\end{itemize}


\begin{align*}
V_N &:= 
\left\{ f \in C^{\infty}(M) \mid x \in K_{N}:=\bigcup_{i=1}^{N}K_{i, N}, |\alpha| \le N, |D_\alpha f(x)|<\frac{1}{N}\right\} \\
&\underset{\text{def}}{=}
\left( V_{1, N} \times V_{2, N} \times \cdots \times V_{N, N} \times 
\prod_{i > N} C^{\infty}(U_i)  \right) \cap C^{\infty}(M) \\
&\underset{\text{def}}{=} V_{(1,2, \ldots, N), (N, \ldots, N)} \cap C^{\infty}(M) \\
\end{align*}
である. これより二つ目の主張が正しいことがいえる.
一つ目の主張は任意の$\bm{i}, \bm{N}$について, 
$$
V_{\max\{\bm{i}, \bm{N}\} }\subset V_{\bm{i}, \bm{N}} \cap C^{\infty}(M)
$$
なので言える. 
\end{proof}


\begin{tcolorbox}[mybox]
\begin{defn}
\label{defn-E-2}
$M$ 2nd countable$m$次元$C^\infty$級多様体とする. 
$C^\infty(M)$に位相を次のように入れる.

まず$M = \bigcup_{l=1}^{\infty} U_i $で$U_i \subset \mathbb{R}^m $となるcountable open coverを一つ固定する. 
そして, 
\[
C^\infty(M) \hookrightarrow \prod_i C^\infty(U_i) \quad 
f \longmapsto (f|_{U_i})
\]
による部分位相を$C^\infty(M)$に入れる. 
つまり,
\[
C^{\infty}(M)  \cong \mathrm{Eq}(\prod C^{\infty}(U_i) \rightrightarrows \prod C^{\infty}(U_i \cap U_j) )
\]
となるように入れる. 
\end{defn}
\end{tcolorbox}

\begin{rem}
\ref{defn-E-2}による位相の定義において, $U_i$の取り方によらない.
\end{rem}

\begin{proof}
別の$U'_j$をとると細分$U_i \cap U'_j$が取れる.
よって次の図式が考えられる. 
  \[
\xymatrix@C=50pt@R=20pt{
C^{\infty}(M) \ar@{^{(}-_>}[r] \ar@{^{(}-_>}[rd]
&\prod C^{\infty}(U_i) \ar@{^{(}-_>}[rd]
&   \\
 &\prod C^{\infty}(U'_j)  \ar@{^{(}-_>}[r] & \prod C^{\infty}(U_i \cap U'_j)  \\
}
\]
そこで"相対位相の直積は相対位相になる"ので,   
\ref{lem-E-1}より言える.
\end{proof}

これにより
$$
C^{\infty} : (U \underset{open}{\subset} M) \mapsto C^{\infty}(U)
$$
は$\C$ベクトル空間のsheafとなる. 
よって$K \subset M$に対して
$$
\mathcal{D}_{K}(M) := \{ \phi \in C^{\infty}(M) \mid \Supp(\phi) \subset K\}
$$
として$C^{\infty}(M)$の部分位相を入れる. 
$$
\mathcal{D}(M):= \bigcup_{K \subset M \mid \text{$K$ cpt}} \mathcal{D}_{K}(M)
\cong 
\varinjlim_{K} \mathcal{D}_{K}(M)
$$
このcolimitはlocally convex vector spaceでのcolimitである. 

[指摘]
このcolimは存在する?
やっぱりSection 1でやったような位相の入れ方にもう一回戻る?
$\varinjlim_{K} \mathcal{D}_{K}(M)$については次回再考. 


\subsection{current続き}
以下\cite{NO}の記法に合わせる. ($M$ 2nd countable $m$次元$C^\infty$級多様体, $k \in \N$とする.)\footnote{おそらく通常の記法とかなり違う記法であると思われる. }

\begin{itemize}
\item $C(M) := \left\{\, f : M \to \mathbb C \,;\ \text{連続} \,\right\}$
\item $\mathcal E(M) := \left\{\, f : M \to \mathbb C \,;\ C^\infty\text{-関数} \,\right\}$  
普通は"$C^\infty(M)$"である.
\item $C^k(M) := \left\{\, \varphi : k\text{-forms on } M\ \text{with coeff} \in C \,\right\}$
つまりlocallyに
$$\varphi = \displaystyle \sum_J \varphi_J\, d\bar z^J
\quad 
(d\bar z^J = d\bar z^{j_1} \wedge \cdots \wedge d\bar z^{j_k})
$$
とかけるものとする. ここで$(U; z^1, \dots, z^n)$を局所座標とし, $J=(j_1, \ldots, j_k)$, $\varphi_J \in C(U)$とする. 
特に$C^0(M) = C(M)$.
\item $\mathcal{E}^k(M) := \left\{\, \varphi : k\text{-forms on } M\ \text{with coeff} \in \mathcal E \,\right\}$ 普通は"$\mathcal{A}^k(M)$"である.
\item $\mathcal{K}^k(M) := \left\{\, \varphi \in C^k(M)\ \Big|\ \operatorname{supp}\varphi\ \text{cpt} \,\right\}$
ここで$\Supp\varphi :=\overline{ \{ x \in M,\ \varphi(x) \neq 0 \}}$
である. $\varphi(x) \neq 0 $とはある$J$があって, $\varphi_J(x) \neq 0$であることを意味する. 
\item $\mathcal{D}^k(M) := \left\{\, \varphi \in \mathcal E^k(M)\ \Big|\ \operatorname{supp}\varphi\ \text{cpt} \,\right\}$
\item $A \subset M$について以下のようにおく
$$
\mathcal{K}_{A}^k(M) := \left\{\, \varphi \in \mathcal{K}^k(M)\ \Big|\ \operatorname{supp}\varphi \subset A \,\right\}
\quad 
\mathcal{D}_{A}^k(M) := \left\{\, \varphi \in \mathcal{D}^k(M)\ \Big|\ \operatorname{supp}\varphi \subset A \,\right\}
$$
\end{itemize}


\[
\begin{array}{c|c|c}

& \text{連続}
& \text{$C^{\infty}$}
\\ \hline
\text{関数} 
&  C(M)
& \mathcal E(M) 
\\ \hline
\text{form} 
&  C^k(M)
& \mathcal E^k(M) 
\\ \hline
\text{form with  support compact}
&  \mathcal K^k(M)
& \mathcal D^k(M) 
\\ \hline
\text{form with  support $\subset A$} 
&  \mathcal K_{A}^k(M)
& \mathcal D_{A}^k(M) 
\\ \hline
\end{array}
\]


\underline{Recall}
% [page 1]

$U \subset \mathbb R^m$ open のとき、$C(U),\ \mathcal E(U)$ には  
次のような 位相をいれていた.

コンパクト集合の列:$K_1 \subset K_2 \subset \cdots \subset U$  
で$K_j \subset K_{j+1}^\circ$ かつ $U = \bigcup_j K_{j}^{\circ}$となるものを取り, $f \in \mathcal E(U)$ に対し

\[
P_N(f) := \max 
\Bigl\{\, |D^\alpha f(x)|\ \Big|\ x \in K_N,\ |\alpha| \le N,\ \omega \in K_N \Bigr\}
\]

\[
V_N := \Bigl\{\, f \in \mathcal E(U)\ \Big|\ P_N(f) < \tfrac1N \Bigr\}
\]
を open base at $0$(0での開基) とする位相を入れていた. 

\ref{lem-E-1}により, これがコンパクト集合の列や座標近傍$U \subset \mathbb R^m$ によらない  


一般の多様体に関しては$M$ に対しては、$M = \bigcup_j U_j$という座標近傍の被覆を使って
% [page 1]
$$
\mathcal E(M) \subset \prod \mathcal E(U_j)
$$
部分位相を入れた.
\begin{tcolorbox}[mybox]
\begin{defn}
\label{defn-E-3}
 $U \subset \mathbb R^m$ open に対し, $C^k(U),\ \mathcal E^k(U)$ に対して位相を
$$\mathcal E^k(U) \simeq \prod\limits_{J=(j_1<\cdots<j_k)} \mathcal E(U)\, d\bar z^J
$$ による  
直積位相を入れる. 
\end{defn}
\end{tcolorbox}
これは次のノルム
\[
P_N(\varphi) := \max 
\Bigl\{\, |D^\alpha \varphi_{J}(x)|\ \Big|\ x \in K_N,\ |\alpha| \le N,\ \omega \in K_N \Bigr\}
\]
について, $\mathcal E (U)$と同じように位相を入れたものに一致する.

\begin{tcolorbox}[mybox]
\begin{lem}
\label{lem-E-4}
\begin{enumerate}
\item$U \subset \mathbb R^m,\ U' \subset \mathbb R^m$ openについて, 
$\Phi : U' \to U$がdiffeoのとき
\[
\Phi ^{*} : \mathcal E^k(U) \xrightarrow{\ \sim\ } \mathcal E^k(U') 
\]
が$\C$上の位相ベクトル空間の同型を誘導する.
\item $U \subset \mathbb R^m$ open, $U = \bigcup_j U_j$:countable open coverについて, 
\[
\mathcal E^k(U) \hookrightarrow \prod_j \mathcal E^k(U_j)
\]
は部分位相空間となる. 
\end{enumerate}
\end{lem}
\end{tcolorbox}
\begin{proof}
(1) $\Phi : U'(x^1,\dots,x^m) \to U(y^1,\dots,y^m)$ diffeoとして

\[
\Phi^{*} : \mathcal E^k(U) := \prod_{J}\mathcal{E}(U)dy^{J} \ \longrightarrow\ \mathcal E^k(U'):=\mathcal{E}(U')dx^{J} 
\]
は以下のような形になる. 
\[
\varphi = \sum_J \varphi_J\, d y^J
\ \longmapsto\ 
\sum_I \Bigl( \sum_J  \Phi^{*}\varphi_J\, \frac{\partial y^J}{\partial x^I} \Bigr)\, d x^I
\]
となる. ここで
\[
\frac{\partial y^J}{\partial x^I} 
= \det \Bigl( \frac{\partial y^{j_k}}{\partial x^{i_l}} \Bigr)
\]
とする. 
よって\ref{lem-E-1}で$\varphi_J \mapsto \Phi^{*}\varphi_J$が連続であることを言っており, 和をとるのも連続なので, 連続性が言える.

(2):
$$\prod_I \mathcal E(U) dx^{I} \cong
\mathcal E^k(U) \ \hookrightarrow\
\prod_{i} \mathcal{E}^k (U_i) 
\underset{(1)}{\cong}
\prod_{i} \left( \prod_I \mathcal E(U)\, d\bar z^I \right)
\cong
\prod_I \Bigl( \prod_j \mathcal E(U_j)\, d\bar z^I \Bigr)
$$
となる. \ref{lem-E-1}でから, $\mathcal E(U)\subset  \prod_j \mathcal E(U_j)$が部分位相になっているのでいえた. 
% [page 2]
\end{proof}

\begin{tcolorbox}[mybox]
\begin{defn}
\label{defn-E-5}
$M$ 2nd countable $m$次元$C^\infty$級多様体について
$C^k(M),\ \mathcal E^k(M)$ に次のように位相を入れる:

$M = \bigcup U_j$を座標近傍$U_j$のcountable coveringとし, 
\[
\mathcal E^k(M) \hookrightarrow \prod_j \mathcal E^k(U_j)
\]
による部分位相を入れる. 

% [page 1]
$A \subset M$ compact subsetについて, 
$\mathcal{D}_A^k(M)  \subset \mathcal{E}^k(M) $はclosed subspaceである.(下のremark参照)
$\mathcal{D}_A^k(M)$には$\mathcal{E}^k(M)$のsubtopologyををいれる  



$\mathcal{K}^k(M),\mathcal{D}^k(M)$ には、
\begin{align*}
\mathcal B := 
&\Bigl\{\, W \subset \mathcal{D}^k(M)\ ;\ W\text{ = non-empty convex balanced} \\
 &\text{s.t.}\ \forall A \subset M\ \text{cpt},\
W \cap \mathcal{D}_A^k(M)\ \text{is open in }\mathcal{D}_A^k(M) \Bigr\}
\end{align*}
を0の local baseとなる位相を入れる.
\end{defn}

\end{tcolorbox}
\begin{rem}
\ref{lem-E-1}から$\mathcal E^k(M)$の位相は座標近傍の$\{U_j\}$ の取り方に依らない.
さらにこの位相は
\[
\mathcal E^k(M) \to \mathcal E^k(U) \quad\Big|\quad U \subset M\ \text{loc. coord.}
\]
に関する weak top. である(weak topologyに関しては\ref{lem-H-11}参照)

$\mathcal{D}_A^k(M)  \subset \mathcal{E}^k(M) $はclosed subspaceなのは
$x \in U \subset M$ loc coord. に対し, 代入写像
\[
ev_x \mathcal E^k(M) \to \C^{\oplus J}
\quad
\varphi \longmapsto (\varphi(x))_J
\]
が連続で, 
$\mathcal{D}_A^k(M) = \bigcap_{x \in M \setminus A} \Ker ev_x
$
とかけるのでclosed subspaceになる. 

$\mathcal{K}^k(M),\mathcal{D}^k(M)$の位相を詳しくいうと, \ref{defn-M-8.4}のように, 
集合族$\tau$を, "$\bigcup_{i \in I} (\varphi_i + W_i )$とかけるもの"の集まりとする. 
ただし$i \in I$について, $\varphi_i \in \mathcal{D}^k(M), W_i \in \beta$とする. 
\end{rem}

% [page 1]

\[
\begin{array}{c|c|c}
\text{dom} 
& \text{local }(U\subset \mathbb R^n) 
& \text{global }(M\text{ mfd.}) 
\\ \hline
\text{form} 
& C^k(U),\ \mathcal E^k(U) 
& C^k(M),\ \mathcal E^k(M) 
\\ \hline
\text{form support $\subset A$} 
& K_A^k(U),\ D_A^k(U) 
& K_A^k(M),\ D_A^k(M) 
\\ \hline
\text{form support compact} 
& K^k(U),\ D^k(U) 
& K^k(M),\ D^k(M)
\end{array}
\]

という対応がある. 

\begin{tcolorbox}[mybox]
\begin{prop}
\label{prop-E-6}
% [page 1]
$M$ 2nd countable$m$次元$C^\infty$級多様体について以下が成り立つ.
\begin{itemize}
\item $\mathcal E^k(M),\ D_{A}^k(M)$ locally convex, complete metrizable  
位相$\C$ ベクトル空間 Heine-Borel property.
\item  $C^k(M),\ K_{A}^k(M)$  locally convex, complete, metrizable  
位相$\C$ ベクトル空間.
\end{itemize}
\end{prop}
\end{tcolorbox}
\begin{proof}
$k=0$, $M = \cup_{i =1}^{\infty} U_{i}\subset \mathbb R^m$ などに関してはすでに示している

% [page 1]

\ref{Prop-M-7.3}によって$\mathcal E^0(U) = C^\infty(U)$ について, 
コンパクト集合の列$K_1 \subset K_2 \subset \cdots $であって, 
seminorm の separating family 
\[
P_N(f) = \max \{\, |D^\alpha f(x)|\ ;\ x \in K_N,\ |\alpha| \le N \}
\]
によって位相を入れると, metrizable 位相$\C$ ベクトル空間であることがわかる. 
(ここにopen coverがcountableがいる)

また\ref{Prop-M-7.4}でcomplete と Heine-Borel propertyを示した.
(ここに平均値の定理を使う. つまり微分可能性がいる. )
よって$C(U)$ しても同じで, Heine-Borel property以外 は 同様にできる。

% [page 1]

k>0,$\mathcal E^k(U),\ C^k(U)$ に関しては, seminormを
\[
P_N(\varphi) := \max \Bigl\{\, |D^\alpha \varphi_J(x)|\ \Big|\ 
x \in K_N,\ J=(j_1<\cdots<j_k),\ |\alpha| \le N \Bigr\}
\]
に変えれば同様に言える


$\mathcal E^k(M),\ C^k(M)$ に関しては 前回 \ref{lem-E-1} (2) で与えられた  
コンパクト集合の列をとる. 具体的に
$(M = \bigcup U_j $とし
$K_{i1} \subset K_{i2} \subset \cdots \subset U_i$となるコンパクト集合列について
$$
K_N := \bigcup_{i=1}^{N}K_{i, N}
$$
\[
P_N(\varphi) := \max \Bigl\{\, |D^\beta \varphi_J(x)|\ \Big|\ 
x \in K_{iN} \subset U_i,
J = (j_1<\cdots<j_k),\ |\omega| \le N \Bigr\}
\]
とすると$K_{1} \subset K_{2} \subset \cdots \subset M$であり, 同様の議論が回る. 

% [page 1]
$K_A^k(M),\ D_A^k(M) $は それぞれ$ C^k(M),\ \mathcal E^k(M) $の閉部分空間なので, 諸性質は保たれる.

\end{proof}

% [page 1]
\begin{tcolorbox}[mybox]
\begin{prop}(cf. \ref{thm-M-8.7})
\begin{enumerate}[label=$(\alph*)$]
\item $V \subset D^k(M)$ convex balanced について, $V$がopenであることは, 任意のcompact $A \subset M$について, $V \cap D_A^k(M) \subset D_A^k(M)$であることと同値.
\item $A \subset M$ compactとしたとき,  
$D_A^k(M) \subset D^k(M)$部分位相空間.
\item $E \subset D^k(M)$ boundedならば, ある
compact $A \subset M$であって, $E \subset D_A^k(M)$となる. 
\item $D^k(M)$は Heine-Borel property を持つ.
\item $D^k(M) \cong \operatorname{colim}_{A\subset M\ \text{cpt}} D_A^k(M)$. ここでこの同型はlocally convex 位相$\C$ ベクトル空間としての同型である. 
\end{enumerate}
また上の主張は$D_A$を$K_A$に変えても成り立つ. 
\end{prop}
\end{tcolorbox}
\begin{proof}
[(a)] \ref{thm-M-8.6}が$D^k(M)$でも言える(\ref{thm-M-8.7} (a)の議論が回る)

[(b)] $D_A^k(M) \hookrightarrow D^k(M)$ は, (a)より連続である.

一方,$D^k(M) \hookrightarrow \mathcal E^k(M)$ も 連続である.
なぜならば$\mathcal E^k(M)$ は locally convex より, convex balanced な open base at 0 となるものがある.
それらをひとつ取っても convex balanced であり, 各 $D_A^k(M)$ に制限しても open なので, $D^k(M)$ の上で openとなる. 
\end{proof}







