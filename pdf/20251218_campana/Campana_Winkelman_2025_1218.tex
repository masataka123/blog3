%\documentclass[dvipdfmx,a4paper,12pt]{amsart}
\documentclass[dvipdfmx,a4paper,11pt]{article} %titleとe-mailをコメントアウトする.


%%% Packages %%%
\setlength{\lineskip}{0pt}

% --- 和文フォント設定 (ゴシックを使うなら) ---amsartを使う時はコメントアウト
\renewcommand{\kanjifamilydefault}{\gtdefault}
\usepackage{otf}          % min10を避けるため
\usepackage{pxrubrica}    % 和文ルビ

% --- 基本パッケージ ---
\usepackage{graphicx}
\usepackage[all]{xy}
\usepackage{wrapfig}
\usepackage{pgfplots}
\usepackage{color}
\usepackage[dvipsnames]{xcolor}

% --- 数学関連 ---
\usepackage{amsmath,amssymb,amsthm,amsfonts,mathtools}
\usepackage{amscd,dsfont,bigdelim,braket,physics,mathrsfs,bm}

% --- 書式・リスト関連 ---
\usepackage{latexsym}
\usepackage{setspace}
\usepackage{multirow}
\usepackage{enumerate}
\usepackage{enumitem}

% --- コメント・取消線など ---
\usepackage{comment}
\usepackage[normalem]{ulem} % \emph の下線化を抑止(cancelと共存)

% --- URL・文字コード ---
\usepackage{url}
% \usepackage[utf8]{inputenc}  % ← 使用しているエンジンがuplatexなら不要、pdflatexなら有効に

% --- showkeys(常に表示) ---
%\usepackage{showkeys}
%\renewcommand*{\showkeyslabelformat}[1]{%
%  \fbox{\parbox{1.6cm}{\normalfont\tiny\sffamily#1\vspace{6mm}}}%
%}
% --- hyperrefは最後に読み込む ---
\usepackage[dvipdfmx,colorlinks,linkcolor=blue,anchorcolor=blue,citecolor=blue]{hyperref}

%%% レイアウト調整 %%%
%%% レイアウト調整(geometryに統一) %%%
\usepackage[
  top=30mm,        % 上余白
  bottom=30mm,     % 下余白
  left=25mm,       % 左余白
  right=25mm,      % 右余白
  headheight=12pt, % ヘッダー高さ(必要なら)
  headsep=10mm,    % ヘッダーと本文の間
  footskip=32pt,   % 本文とフッターの距離
  includehead,     % ヘッダー分も高さに含める
  includefoot      % フッター分も高さに含める
]{geometry}

%%% 行間調整(適宜 1.2 などに変更) %%%
\usepackage{setspace}
\setstretch{1.1}

% --- 段落設定 --- 文字の行間ならここを変更する
\setlength{\parskip}{5pt}   % 段落間のスペース
\setlength{\parindent}{0pt}   % 段落先頭の字下げをなくす


%%% 追加(重複なし)パッケージ・設定 %%%

% --- 目次の体裁調整 ---
\usepackage{tocloft}
%\renewcommand{\contentsname}{目次} % 日本語化
\setlength{\cftbeforesecskip}{0pt}
\setlength{\cftbeforesubsecskip}{0pt}
\setlength{\cftbeforesubsubsecskip}{0pt}

% --- セクション見出しの体裁調整 ---
%\usepackage{titlesec}
%\titleformat*{\section}{\Large\bfseries}
%\titleformat*{\subsection}{\large\bfseries}
%\titlespacing*{\section}{0pt}{1.5ex plus .2ex minus .2ex}{0.8ex plus .1ex}
%\titlespacing*{\subsection}{0pt}{1.0ex plus .2ex minus .2ex}{0.5ex plus .1ex}

% --- ヘッダー/フッター設定 ---
%\usepackage{fancyhdr}
%\pagestyle{fancy}
%\fancyhf{}
%\rhead{岩井 雅崇}
%\lhead{大阪大学 数学専攻}
%\cfoot{\thepage}

% -- enumerate, itemize行間設定
\usepackage{enumitem} % デフォルト設定
\setlist[itemize]{itemsep=3pt, parsep=0pt}
\setlist[enumerate]{itemsep=3pt, parsep=0pt}
% "変更する際は右を使う" \setlength{\parskip}{0cm} % 段落間\setlength{\itemsep}{5pt} % 項目間

% --- tcolorbox 設定 ---%\begin{tcolorbox}[mybox]と使う
\usepackage[most]{tcolorbox}
\tcbuselibrary{breakable, skins, theorems}
\tcbset{
  mybox/.style={
    colback = white,
    colframe = green!35!black,
    fonttitle = \bfseries,
    breakable = true
  }
}

%--緑枠自動化
\AtBeginEnvironment{prop}{\begin{tcolorbox}[mybox]}
\AtEndEnvironment{prop}{\end{tcolorbox}}
\AtBeginEnvironment{lem}{\begin{tcolorbox}[mybox]}
\AtEndEnvironment{lem}{\end{tcolorbox}}
\AtBeginEnvironment{thm}{\begin{tcolorbox}[mybox]}
\AtEndEnvironment{thm}{\end{tcolorbox}}
\AtBeginEnvironment{defn}{\begin{tcolorbox}[mybox]}
\AtEndEnvironment{defn}{\end{tcolorbox}}
\AtBeginEnvironment{cor}{\begin{tcolorbox}[mybox]}
\AtEndEnvironment{cor}{\end{tcolorbox}}
\AtBeginEnvironment{ques}{\begin{tcolorbox}[mybox]}
\AtEndEnvironment{ques}{\end{tcolorbox}}
\AtBeginEnvironment{conj}{\begin{tcolorbox}[mybox]}
\AtEndEnvironment{conj}{\end{tcolorbox}}


% --- TikZ 設定 ---
\usepackage{tikz}
\usetikzlibrary{positioning, arrows.meta, fit, calc, backgrounds}
\pgfdeclarelayer{background}
\pgfdeclarelayer{foreground}
\pgfsetlayers{background,main,foreground}

% --- footnote がページをまたがない設定 ---
\interfootnotelinepenalty=10000

% --- 目次に表示する階層の深さ ---
\setcounter{tocdepth}{2}

% --- 日本語目次---
\usepackage{pxjahyper}

%--newtheorem%--newcommand----

\newtheorem{thm}{Theorem}[section] 
\newtheorem{theo}[thm]{Theorem}
\newtheorem{cor}[thm]{Corollary}
\newtheorem{prop}[thm]{Proposition}
\newtheorem{conj}[thm]{Conjecture}
\newtheorem*{mainthm}{Theorem}
\newtheorem{deflem}[thm]{Definition-Lemma}
\newtheorem{lem}[thm]{Lemma}
\theoremstyle{definition} 
\newtheorem{defn}[thm]{Definition}
\newtheorem{propdefn}[thm]{Proposition-Definition} 
\newtheorem{lemdefn}[thm]{Lemma-Definition} 
\newtheorem{thmdefn}[thm]{Theorem-Definition} 
\newtheorem{eg}[thm]{Example} 
\newtheorem{ex}[thm]{Example} 
\newtheorem{ques}[thm]{Question}
\newtheorem{remin}[thm]{Reminder}
\theoremstyle{remark}
\newtheorem{rem}[thm]{Remark}
\newtheorem{setup}[thm]{Setup}
\newtheorem{obs}[thm]{Observation}
\newtheorem{notation}[thm]{Notation}
\newtheorem{cl}{Claim}
\newtheorem{claim}[thm]{Claim}
\newtheorem{assup}[thm]{Assumption}
\newtheorem{step}{Step}
\newtheorem*{clproof}{Proof of Claim}
\newtheorem{cln}[thm]{Claim}
\newtheorem*{ack}{Acknowledgements} 
\numberwithin{equation}{section}
\newtheorem{case}{Case}



\newcommand{\rk}[0]{\operatorname{rk}}
\newcommand{\supp}[0]{\operatorname{Supp}}
\newcommand{\Rad}[0]{\operatorname{Rad}}
\newcommand{\Sha}[0]{\operatorname{Sha}}
\newcommand{\sha}[0]{\operatorname{sha}}
\newcommand{\eend}[0]{\operatorname{End}}
\newcommand{\codim}[0]{\operatorname{codim}}
\newcommand{\nd}[0]{\operatorname{nd}}
\renewcommand{\rank}[0]{\operatorname{rank}}
\newcommand{\degree}[0]{\operatorname{deg}}
\newcommand{\Exc}[0]{\operatorname{Exc}}
\newcommand{\pr}{{\rm pr}}
\newcommand{\id}{{\rm id}}
\newcommand{\Sym}{{\rm Sym}}
\newcommand{\End}[0]{\operatorname{End}}
\newcommand{\Coker}[0]{\operatorname{Coker}}

\newcommand{\Supp}{{\rm Supp}}
\newcommand{\Hom}[0]{\mathscr{H}\!\textit{om}}
\newcommand{\Ext}[0]{\mathscr{E}\!\textit{xt}}
\newcommand{\GL}[0]{\operatorname{GL}}
\newcommand{\SheafHom[1]}{\mathscr{H}\!\!\!\text{\calligra om}_{\,{#1}}}
\newcommand{\PGL}[0]{\mathbb{P}\GL(r,\C)}

\newcommand{\Alb}{{\rm Alb}}
\newcommand{\alb}{{\rm alb}}
\newcommand{\verti}{{\rm vert}}
\newcommand{\hor}{{\rm hor}}
\newcommand{\univ}{{\rm univ}}
\newcommand{\Tor}{{\rm tor}}
\newcommand{\shaf}{\mathrm{sha}}
\newcommand{\Shaf}{\mathrm{Sha}}
\newcommand{\reg}{{\rm{reg}}}
\newcommand{\sing}{{\rm{sing}}}
\newcommand{\qt}{{\rm{qt}}}
\newcommand\sO{{\mathcal O}}
\newcommand{\Div}[0]{\operatorname{div}}
\newcommand{\ddbar}{dd^c}
\newcommand{\cV}{\mathcal{V}}
\newcommand{\deldel}{\sqrt{-1}\partial \overline{\partial}}
\newcommand{\dbar}{\overline{\partial}}
\newcommand{\I}[1]{\mathcal{I}(#1)}
\newcommand{\Aut}[1]{\mathrm{Aut}(#1)}
\newcommand{\Ker}[1]{\mathrm{Ker}(#1)}
\newcommand{\Image}[1]{\mathrm{Im}(#1)}
\DeclareMathOperator{\Ric}{Ric}
\DeclareMathOperator{\Vol}{Vol}
 \newcommand{\pdrv}[2]{\frac{\partial #1}{\partial #2}}
 \newcommand{\drv}[2]{\frac{d #1}{d#2}}
  \newcommand{\ppdrv}[3]{\frac{\partial #1}{\partial #2 \partial #3}}
\newcommand{\underalign}[2]{\quad \underset{\mathclap{\strut #1}}{#2}\quad}
\newcommand{\polar}{\beta}
  
\newcommand{\R}{\mathbb{R}}
\newcommand{\Z}{\mathbb{Z}}
\newcommand{\N}{\mathbb{Z}_+}
\newcommand{\C}{\mathbb{C}}
\newcommand{\Q}{\mathbb{Q}}
\newcommand{\D}{\mathbb{D}}
\newcommand{\mP}{\mathbb{P}}
\newcommand{\mO}{\mathcal{O}}
\newcommand{\B}{\mathds{B}}
\newcommand{\tl}{\hspace{-0.8ex}<\hspace{-0.8ex}}
\renewcommand{\tr}{\hspace{-0.8ex}>}

\newcommand{\xb}[1]{\textcolor{blue}{#1}}
\newcommand{\xr}[1]{\textcolor{red}{#1}}
\newcommand{\xm}[1]{\textcolor{magenta}{#1}}



\title{Campana-Winkelman 15, Oka多様体のまとめ}
\author{Masataka IWAI}
%\address{Department of Mathematics, Graduate School of Science, Osaka City University 3-3-138, Sugimoto, Sumiyoshi-ku Osaka, 558-8585Japan} 
%\email{{\tt masataka.math@gmail.com}}
%\email{{\tt masataka.math@gmail.com, masataka@sci.osaka-cu.ac.jp}}



\date{\today, version 0.01}


\renewcommand{\thefootnote}{\arabic{footnote}}

\baselineskip = 15pt
\footskip = 32pt

\begin{document}



\maketitle
%
\begin{abstract}
岡多様体の勉強会のためにCampana-Winkelman \cite{CW15}とOka多様体に関してまとめたものです. 
\end{abstract}
%
\setcounter{tocdepth}{3}
\tableofcontents

\section*{はじめに}
このpdfは次の内容に基づいています.
\begin{itemize}
\item Campana-Winkelman 15 のまとめ. \cite{CW15}をそのまま読んだ内容を書いています. 内容は"special多様体と$h$-principle"について(雑に??)まとめた(?)論文の解読書です. 
\item Oka多様体まとめと問題. Oka多様体に関する日下部さんのサーベイやForstneric先生のサーベイをまとめた内容です. 
\item Voison先生のサーベイまとめ. \cite{Voi}のサーベイをまとめています. Iitaka-fibration, MRC fibration+有理連結, Shafarevich map, Core fibration (+special多様体)の内容を(玄人向けに?)まとめたサーベイの解読書です. (が, あまりよくわからなかったです. )
\item その他は期限内に読めなかったが, 和訳だけしたものを書いています. ただ著作権的にまずい可能性もあるので, その場合は削除します.  
\end{itemize}

また\xr{赤字}がありますが, これはわからなかったところです. 
わかったことがあれば教えてください. 

\newpage
\section{Campana-Winkelman 15 まとめ}

\subsection*{はじめに}
以下\cite{CW15}の論文を現論文の順番通りに解読していく. 
いろんな内容があり面白い反面, 繋がりがよくわからなかったのでかなり苦労した. 
%ただ読んだ感じいろんな内容があってつながりがよくわからなかった. 
%またレフェリーから何も指摘されなかったのか?と思った.


\subsection{1章 Introduction}
\begin{defn}\cite[Defintion 1.1]{CW15}
\label{defn-CW15-1.1}
解析空間(complex space)$X$が$h$-principleを満たすとは, 
任意のStein 多様体$S$, 任意の連続関数$f : S \to X$について, ある正則関数$F : S \to X$があって, $F$と$f$はhomotopicである. 
\end{defn}

Historyはこちら:
\begin{itemize}
\item Oka-Grauert: Stein 多様体 $S$上のcomplex Lie group $G$をfiberとする主$G$束を$X$とする. その連続なセクション$s : X \to G$は, ある正則切断とhomotopicになる
\item Gromov: Elipticならば, $h$-principleを満たす.
\item Forstneric: Subeliptic ならば$h$-principleを満たす.
\item 等質空間. (homogenerous complex manifold. 群$G$が$X$に推移的に作用しているもの)は$h$-principleを満たす. よって$\C\mathbb{P}^n$, Grassmanian, Torusも$h$-principleを満たす. 
\item $A \subset \C^m$をAlgebraic subvarietyで余次元2以上なものをとると, $\C^m \setminus A$も$h$-principleを満たす.
\item Oka多様体は$h$-principleを満たす. \ref{cor-Oka-hP}参照. 
\end{itemize}
Steinの定義は次の通りである. 

\begin{defn}[{正則凸・正則分離・Stein空間, \cite[IV.2.1, 2.2, 3.12, V.4.2]{GR}}]
$X$を(2nd countable)解析空間とする.  (reduced, irreducible, normalは不要).

$X$が正則凸(holomorphically convex)であるとは同値な以下の条件を満たすこと.
\begin{enumerate}
\item 任意の相対コンパクト$K \subset X$について, その正則凸包
$$
\widehat{K}:= 
\{  x \in X ||f(x)| \le  \max_{y \in K}| |f(y)| \text{が任意の$f \in \mathcal{O}_{X}(X)$で成り立つ}\}
$$
もまた相対コンパクトになる. 
\item 任意の集積点を持たない点列$\{ y_n\}_{n=1}^{\infty}$について, $X$上のある正則関数$f$が存在して$\sup_{n \in \N}|f(y_n)|=+ \infty$となる. 
\end{enumerate}

$X$が正則分離可能(homomorphically separable)とは, 任意の$x, y \in X$についてある$X$上の正則関数$f$があって$f(x) \neq f(y)$となること. 

$X$がStein空間とは次の同値な条件を満たすこと.
\begin{enumerate}
\item $X$は正則完備, つまり$X$は弱正則凸(任意のコンパクト集合$K$について$\widehat{K} \cap W$がコンパクトになるような$K$の開近傍$W$が存在する)かつ$X$内のコンパクト解析的部分集合は有限集合である.
\item $X$内の任意の閉部分集合$P$について定理B, つまり$H^q (P, \mathcal{F})=0$が任意の解析的連接層$\mathcal{F}$と$q \ge 1$について成り立つ.
\item $\mathcal{I} \subset \mathcal{O}_X$を連接ideal層としたとき, $H^1 (X, \mathcal{I})=0$が成り立つ.
\item $\Gamma(X,\cdot) : Coh(X) \to Ab$は完全関手.
\item $X$は正則凸, 正則分離可能, かつ任意の$x \in X$についてその埋め込み次元$e:=\dim_{\C} m_x /m_{x}^{2}$とすると, ある$e$この$f_1, \ldots, f_e \in \mathcal{O}_{X, x}$があって, $m_x$を生成する.
\item $X$は正則凸かつ正則分離可能,
\end{enumerate}

\end{defn}
埋め込み次元の部分がわかりづらい. 
もし簡単のため$X \subset \C^N$, $x=0$で$X = (g_1 = \cdots = g_k=0)$と書けているときは
$$
e:=\dim_{\C} m_x /m_{x}^{2} 
= N - \rank J_{0}(g_1, \ldots, g_k)
$$
となる. 陰関数定理から$0 \in X$の周りで$\C^{N-\rank J_0}=\C^e$に埋め込むことができる.
特に$\dim X \le e$である. 

正則ならば$=$がなりたつ. よってSteinならば局所座標を全体に伸ばすことができるので次を得る. 
\begin{cor}[局所座標公理]
$X$がStein 多様体とする. 任意の$x \in X$についてある$X$上の正則関数$f_1, \ldots, f_n$があって$f_i |_{U}$が$z$を含むある小さな開集合$U$上の局所座標となる.
\end{cor}

この論文の主定理は次の通り. 
\begin{thm}\cite[Main Theorem]{CW15}
$X$をirreducible projective varietyとする. 
\begin{enumerate}
\item (=\ref{thm-CW15-6.1}) $X$がprojective manifoldで, $h$-principleを満たすならばspecial.
\item (=\ref{thm-CW15-5.8}) $X$が$h$-principleを満たすならば, Brody hyperbolic K\"ahler 多様体 $Y$への正則写像$f : X \to Y$について, $f$は定数になる.
\end{enumerate}
\end{thm}

\begin{thm}\cite[Theorem 1.2]{CW15}
\label{thm-CW15-1.2}
$X$を解析空間とする. 
\begin{enumerate}
\item  (=\ref{thm-CW15-5.4}) $h$-principleを満たすならば, homotopically $\C$-connected
\item (=\ref{subsec-CW15-7}) $X$がalgebraic varietyまたはcompact K\"ahlerとする. このとき(quasi-)Albanese mapは全射.
\end{enumerate}
\end{thm}

\begin{defn}\cite[Definition 1.3]{CW15}
\label{defn-CW15-1.3}
$X$を解析空間とする.
\begin{enumerate}
\item $X$が$\C$-connectedとは, $X$の任意の2点がentire curveのchainで結ばれること. つまり$x, y \in X$について, ある$f_1, \ldots, f_{N}: \C \to X$があって
$$
x\in f_1(\C) \quad f_i(\C) \cap f_{i+1}(\C) \quad y \in f_N(\C)
$$となること. 
\item  $X$がBrody hyperbolicとは, 任意の正則写像$f: \C \to X$が定数となること. 
\item $X$がhomotopically $\C$-connectedとは, 任意のunramified covering $X' \to X$と, Brody hyperbolic解析空間$Y$への正則写像$X' \to Y$について, ホモトピー群の写像
$$
\pi_{k}(f) : \pi_{k}(X') \to \pi_{k}(Y)
$$
が$k>0$について0写像になること. 
\end{enumerate}
\end{defn}
$\C$-connectednessは"covering", "正則写像での像"によって保たれる. また$X$がsmoothならばproper modificationでも保たれる.
また解析空間の写像$\pi : X \to Y$がunramifiedとは$\Omega_{X/Y}^{1}=0$となること. (\cite[Ch.2 Subsection 2-4]{SCV7}参照.) \footnote{ただこの定義はあまり良くないと思う. おそらく\'etale(flat かつ unramified)の方がいいと思う. なお$X,Y$smoothであればflatは常に言えるのであんまり差はない.}

ホモトピー群に関して復習する. 

\begin{defn}[ホモトピー群]
\label{defn-CW15-homotopy}
$X$を位相空間, $x_0 \in X$とする. 
$S^n:= \{ (x_0, \ldots, x_n) \in \R^{n+1} \mid \sum_{i=0}^{n} (x_i)^{2}=1\}$を$n$次元球面, $P=(1, 0, \ldots, 0)$とする. 
$X$の$n$次ホモトピー群を
$$
\pi_{n}(X, x_0):= \{ f : S^{n}\to X \mid f(P)=x_0\}/\sim
$$
と定義する.

ここで$f \sim g$を"$f$と$g$はhomotopic", つまり
「ある連続写像$H : S^{n}\times [0,1] \to X$で$H(x, 0)=f(x), H(x, 1)=g(x)$なものが存在する」として
同値関係を入れる. 
\end{defn}
$I=[0,1]$として
$S^{n} \sim I^n/\partial I$を使って定義する方法もある. 

次のことがわかっている
\begin{itemize}
\item $X$が$\C$-connected, $Y$がBrody hyperbolicならば, 正則写像$f : X \to Y$は定数.
特に$\C$-connectedならばhomotopically $\C$-connected.\footnote{この部分\cite{CW15}では"weak $\C$-connected"と書いてあった. }
\item $X$可縮ならばhomotopically $\C$-connected. $\pi_k(X)=1$であるので. 
\item \cite[Theorem 5?]{CW09} smooth projective 3-foldでhomotopically $\C$-connectedだが$\C$-connectedでない例がある. 
\end{itemize}
まとめると次が言える. 
\begin{center}
Subelliptic manifold $\Rightarrow$ $\C$-connected $\Rightarrow$ homotopically $\C$-connected.
\end{center}

\begin{ques}\cite[Question 1.4]{CW15}
$X$ 連結複素多様体とする.
$h$-principleを満たすならば, $X$は$\C$-connected 解析空間と正則写像によってホモトピー同値か?
\end{ques}
理由としては「知っている例で$h$-principleを満たす連結複素多様体は全てSubelliptic 解析空間と正則写像によってホモトピー同値だから」. 結構安易な質問な気もする. 
が, これはどうも良くない問いで,  コンパクト複素多様体はコンパクト多様体のproper analytic subsetとhomotopicになり得ない. よってこの問いよりも次の問のほうが良い. \footnote{ここは\cite{CW15}の原文そのまま. }
\begin{conj}\cite[Question 1.5, 1.6, 1.7, 1.8]{CW15}
\begin{itemize}
\item (1.5) コンパクト連結複素多様体が$h$-principleを満たすならば, $\C$-connectedか?
\item (1.6) Projective manifoldが$h$-principleを満たすならば, 基本群はAlmost Abelianか??なお\cite[Theorem 6.1]{CW15}で"Projective manifoldについて$h$-principle $\Rightarrow$ Special"が言えている 「Specialならば基本群はAlmost Abelianか?」も未解決. 
\item (1.7) 任意のcompact K\"ahler manifoldについて$h$-principleを満たすならば, Specialか? \cite[Theorem 6.1]{CW15}の証明はJouanoulou trickを使うのでprojectiveを使う. 
\item (1.8) $X$ smooth (or normal) quasi-projective varietyとする. $X$がspecialか$\C$-connectedを仮定する.  このとき$h$-principleを満たすか? またOkaか?
\end{itemize}
\end{conj}

まとめるとこういうことである.
\[
\xymatrix@C=40pt@R=40pt{
  \hbox{elliptic} \ar@{=>}[d]\ar@{=>}[r] 
  & \hbox{Ell}_\infty \ar@{=>}[d] \ar@{=>}[r] 
  & \hbox{Ell}_2 \ar@{=>}[r] \ar@{=>}[l]
  & \hbox{Ell}_1 \ar@{=>}[l] \ar@{=>}[d]\\
\hbox{subelliptic} \ar@{=>}[r] 
& \hbox{Oka} \ar@{=>}[r] \ar@{=>}[d]  \ar@{=>}[u]
&\hbox{$h$-principle}   \ar@{-->}[ld]^{Conj 1.5}\ar@{=>}[r]_{\cite[Thm6.1]{CW15}\text{projのみ}} \ar@{=>}[d]^{\cite[Thm 1.2]{CW15}}\ar@/^1pc/@{-->}[r]^{Conj 1.7}\ar@{-->}[rd]_{Conj 1.6} 
&\hbox{special} \ar@{-->}[d]^{\cite{Cam04} Conj} \ar@/_3pc/@{-->}[l]_{Conj 1.8}\\
 \hbox{$\begin{array}{c} \text{Rationally} \\ \text{chain connected}\end{array}$} \ar@{=>}[r]&  \hbox{$\C$-connected} \ar@{=>}[r]\ar@/^1pc/@{-->}[ru]^{Conj 1.8}\ar@{=>}[d]
& \hbox{$\begin{array}{c}\text{homotopicall}y \\ \C\text{-connected}\end{array}$} & \hbox{$\pi_1$ almost Abel}\\
&  \hbox{$\begin{array}{c} \text{Kobayashi} \\ \text{pseudometric} \equiv 0\end{array}$} & & 
}
\]

\cite [Section 8]{CW15} (\ref{subsec-CW15-8})では次の例をあげる
\begin{itemize}
\item "non-normal projective" curveで, $h$-principleを満たさないが, specialや$\C$-connectedである例. 
\item "compact no-K\"ahler" surfaceで, $h$-principleを満たさないが, specialである例. 
\item "non-compact non-algebraicな$\C^2$の領域"で, $h$-principleを満たさないが, $\C$-connectedである例. 
\end{itemize}
つまりConjecture 1.8には\underline{normal性, K\"ahler性, コンパクト性は必要}である. 

\begin{rem}\cite[Remark 1.9]{CW15}
\label{rem-CW15-1.9}
(1)自明な議論から以下が成り立つ. 
$$
\text{可縮} \Rightarrow \text{$h$-principle}
$$
\begin{proof}
$X$ 可縮なので, $x \in X$であって, 定数関数$g : X \to \{ pt \}$と包含写像$ i : \{x \} \hookrightarrow X$であって, $i \circ g \underset{homotopic}{\sim} id_{X}$となるものがある. 

よって任意のStein 多様体$S$, 任意の連続関数$f : S \to X$について, 
$i \circ g \circ f: S \to X$は定数関数で特に正則関数である. 
そして$i \circ g \circ f \underset{homotopic}{\sim} f$であるのでいえた. 
\end{proof}

よって$h$-principleは可縮でない多様体にのみ興味がある話題となる. 
またコンパクト複素多様体は可縮になり得ない.
$X$は向きづけ可能であるので, 基本類が存在して$H_{n}(X, \Z) \cong \Z$であるためである. 

また可縮 なaffineでlog general typeな多様体が存在する. (引用は\cite[Remark 1.9]{CW15}参照)
つまり$h$-principleをみたすけどspecialではないAffine varietyが存在する. (つまりコンパクト性は必要.)


(2). $X' \to X$をunramified coveringで$X, X'$が連結な複素多様体とする.
\ref{lem-CW15-6.6}から$X$が$h$-principleを満たすなら, $X$'も満たす. 
ただし逆は成り立たない.

反例として, コンパクトBrody-Hyperbolic manifold $X$でEilenberg-maclane $K(\pi, 1)$空間なものとする.\footnote{$\pi_1(X) \cong \pi$かつ$\pi_{n}(X)=0$となる弧状連結な空間のこと. \url{http://pantodon.jp/index.rb?body=Kpi1}参照. 例えば種数$g \ge 2$の曲線は$K(\pi, 1)$空間である. ($\pi$を自由群とする.)}
例えばcompact ball quotientや種数2以上の曲線とする. 
すると$X$が可縮ではない.
よって\ref{cor-CW15-5.5}より, $X$は$h$-principleを満たさない. 
一方で$X_{univ}$は可縮であるので$h$-principleを満たす. 

%%%%%%%%%%%%%%%%%%%
\begin{comment}

例えば$X$を種数$g \ge 2$の曲線とする.  $X$は$h$-principleを満たさない. 
\begin{proof}
背理法. $h$-principleを満たすならば, \ref{thm-CW15-1.2}からhomotopically $\C$-connectedなので, $X$がBrody-hyperbolicより
$$
\pi_k(id_X) : \pi_k (X) \to \pi_k (X)
$$
は0写像となる. 特に$\pi_1(X)=0$となり矛盾する. 
\end{proof}
\end{comment}
%%%%%%%%%%%%%%%%%%%%%%%%
(3) \xr{以下の文は本当?? にわかには信じられない.}
\ref{defn-CW15-1.1}における$f : S \to X$を考える. 
$\dim S \ge 3$ならば$S$の複素構造$J_0$を$J_1(f)$に取り替えることができて, \ref{defn-CW15-1.1}は常に成り立つ
\cite[Section 9.10]{For18}から, $\dim S =2$ならば$S$の同相写像を保つ向き付けを考える必要がある(らしい?) 
\end{rem}

\subsection{2章 Specialness}
\subsubsection{Specialness and the core map}
\cite[Definition 2.1]{CW15}では$\kappa^{sat}$が出てきたが, その後使われていないので, \cite{Cam04}に基づいて定義をする. 
またコンパクト解析空間$X$がFujiki classとは, compact K\"ahler 多様体とbimeromorphicとなることとする. 
\begin{defn}\cite[Definition 2.1]{CW15}, 
$X$ 連結コンパクト複素多様体 in Fujiki's classとする.
\begin{enumerate}
\item \cite[Definition 2.24]{Cam04} $p>0$とする. line bundle $L \subset \Omega_{X}^{p}$がBogomolov sheafとは, $\kappa(F)=p$となること. 
ここでKodaira-Iitaka 次元を
$$
\kappa( F):= \limsup_{m \to \infty}\frac{\log h^0(X, F^{\otimes m})}{ \log m}
\in \{- \infty, 0, 1, \ldots, \dim X \}
$$
として定義する. 
\item \cite[Theorem 2.27]{Cam04} $X$がspecialとはBogomolov sheafを持たないこと. 
\item \cite[Theorem 2.25]{Cam04} $L$がBogomolov sheafのとき, ある$p$次元コンパクト複素多様体$Y$へのmeromorphic dominant map $f : X \dashrightarrow Y$があって, $L = f^{*}(K_Y)$が$X$のgeneral pointで成り立つ. 
\end{enumerate}
\end{defn}
Specialであることはbimeromorophicな性質である. 

2個目の主張はCastelnuovo-De Franchisの定理である(Bogomolovによって一般化された).
Bogomolov-Sommese vanishingによって, 任意のline bundle $L \subset \Omega_{X}^{p}$は常にから
$$
\kappa( F) \le \nu(X, F) \le p
$$
を満たす. この等号が成り立つときがBogomolov sheafである.

\begin{rem}\cite[Remark 2.2]{CW15} 
\begin{enumerate}
\item $X$ specialならば, 一般型(of general type) $Y$へのdominant meromorphic map $f : X \dashrightarrow Y$は存在しない. もし存在したら$f^{*}(K_Y) \subset \Omega_{X}^{\dim Y}$のsaturationがBogomolov Sheafとなってしまうため
\item\cite[Chapter 2]{Cam04}  $X$ specialならば. 任意のdominant meromorphic map $f : X \dashrightarrow Y$について, $Y$もspecial.
\item\cite[Chapter 5]{Cam04} $X$ specialならば, finite unramified covering $X' \to X$について$X'$もspecial.\footnote{"この証明はかなり難しい"と\cite{CW15}に書いてあった. }
特に以下が成り立つ. 
\begin{center}
special $\Rightarrow $ weakly special
\end{center}
なお逆は成立しない. (3次元の反例がある)
ここで$X$がweakly specialとは「任意のfinite unramified covering $X' \to X$について, $X'$はof general type $Y$へのdominant meromorphic map $f : X' \dashrightarrow Y$を持たない」こと
\item Weakly SpecialだけではCore mapなどの存在が言えないので不十分である. ただ\cite[Chapter 9]{Cam04} Kobayashi Pseudo metricとの予想もある. またweakly specialでKobayashi pseudometricが消えない例もある. \cite{CW09} \footnote{この文章よくわからなかったが, 要するに"special"が大事だということ?? あとweakly specialだけでは不十分なこともわかっている. \url{https://arxiv.org/abs/2410.06643}参照. }
\item \cite[Definition 2.1]{Cam04} $X$がspecialであることは, of general type orbifold pair $(Y, \Delta)$へのdominant rational mapが存在しないことと同値である. 
\item RC または $\kappa(K_X)=0$ならばspecial. 
\item \cite[Section 6.5]{Cam04} Special 多様体は$n-1$以下の小平次元を取りうる.
\item \cite[Chapter 5]{Cam04} $n=1$のときは
$$
\text{Special} \Leftrightarrow \text{weakly Special}  \Leftrightarrow \text{Brody hyperbolicではない}
$$
$n=2$のときは
$$
\text{Special} \Leftrightarrow \text{weakly Special}  \Leftrightarrow \text{$\kappa(K_X)<2$かつ$\pi_1(X)$ almost Abelian}
$$
よってSpecial surfaceならば以下のいずれかが成り立つ
\begin{itemize}
\item $\kappa(K_X) = -\infty$かつ$q(X) \le 1$
\item $\kappa(K_X)= 0$
\item $\kappa(K_X) = 1$かつ$\widehat{q}(X) \le 1$. ($\widehat{q}(X) := \sup\{ q(X') \mid \text{$X' \to X$ finite \'etale}\}$)
\end{itemize}
\item \cite{BL00}より $n=2$のとき, non-elliptic K3を除いて, 
$$
\text{Special} \Leftrightarrow  \text{$\C^2$-dominable}
$$
詳しくいうとnon-elliptic K3はspecialであることがわかっているが, $\C^2$-dominableがわからない. \footnote{$\C^n$-dominableならばSpecialである. 逆はおそらく正しくないと予想されている. というのも\cite{CW23}にて, "This ($\C^n$ dominable implies specialness) might however be a low-dimensional phenomenon, and it is not expected to remain true in dimension 3."とあった. そのためOka$\Rightarrow $ Specialはもしかしたら正しくない??}
%\item $n \ge 3$のとき$$\text{Special} \Rightarrow \text{Weakly Special} $$は成り立つが逆は成り立たない. weakly specialでKobayashi pseudometricが消えない例もある\cite{CW09}
\end{enumerate}
\end{rem}

任意のFujiki 多様体はSpecialとof log general type$(C(X), \Delta_{c_X})$に分解できる. 
\begin{thm}\cite[Chapter 3]{Cam04}
$X$ コンパクト複素多様体 in Fujiki's class.
このとき$c_X : X \dashrightarrow C(X)$というファイバー連結なdominant almost holomorphic meromorphic mapが存在して次を満たす.
\begin{enumerate}
\item 一般fiberはspecial.
\item Orbifold Base $(C(X), \Delta_{c_X})$はof general type. つまり$\kappa(K_{C(X)} + \Delta_{c_X})=\dim C(X)$である. 
\end{enumerate}
この$c_X$をcore mapという. 
特にCore mapが自明(つまり$\dim C(X)=0$)ならば, $X$はspecialである.
\end{thm}
meromorphic map $\varphi : X \dashrightarrow Y$がalmost holomorphicであるとは, 
$f$の不確定点除去
   \[
\xymatrix@C=50pt@R=20pt{
\widetilde{X} \ar@{->}[r]^{\widetilde{\varphi}}\ar@{->}[d]^{\pi} &Y\\
X \ar@{-->}[ru]_{\varphi}  & \\
}
\]
をとったとき, 不確定点$I(\varphi) \subset X$として, $\widetilde{\varphi}(\pi^{-1}(I(\varphi))) \neq Y$となること. 
同値な言い換えとして, あるZariski open set $Y_0 \subset Y$があって$\varphi : \varphi^{-1}(Y_0) \to Y$がwell-definedであること. 

\begin{rem}
core map $c_X : X \dashrightarrow C(X)$について$C(X)$はMoishezonであることがわかっている. (Algebraic reduction を経由するから)
よって$C(X)$はbimeromorphicなmodelを取り替えてprojectiveであると仮定して良い.  
\end{rem}


\begin{conj}\cite[Conjecture 2.4]{CW15}
\label{conj-CW15-2.4}
$X$ コンパクト複素多様体 in Fujiki's class.
このとき$c_X =(J \circ r)^{\dim X}$となる. ここで
\begin{itemize}
\item $J$ Orbifold version of Moishezon map (Iitaka mapの一種)
\item $r$ Rational quotient 
\end{itemize}
とする. 特にspecial 多様体は$\kappa=0$と$\kappa_{+}=0$で構成される. 
ここで$\kappa_{+}(X)$を次で定める. 
$$
\kappa_{+}(X):=\sup\{ \kappa( \det\mathcal{F}) \mid \mathcal{F} \subset \Omega_{X}^{p} \text{coherent subsheaf }\}
$$
\end{conj}
$\kappa_{+}(X)$は\cite{Cam95}によって普遍被覆やShafarevich mapとの関連がある.

\begin{thm}
Orbifold version of Iitaka Conjecture が成り立つならば, 
\ref{conj-CW15-2.4}も成り立つ. 
\end{thm}

\begin{conj}\cite[Conjecture 2.7]{CW15}
コンパクト複素多様体 in Fujiki's classについて次は同値.
\begin{enumerate}
\item special
\item 基本群が almost Abelian.
\item Kobayashi pseudometric が消える.
\item $\C$-connected.
\end{enumerate}
\end{conj}

\subsubsection{Kobayashi pseudometricのおさらい}
ここからはKobayashi pseudometricに関するおさらいである.
\cite{Kob98}参照のこと. 
山ノ井先生のサーベイ\cite{Yamanoi}も見やすいらしい. 

\begin{defn}\cite[Section 1.3 and 1.4]{NO90}\cite[Chapter 3]{Kob98}
\label{defn-Kob98-Ch-3}
$X$ 解析空間, \( p, q \in X \)について,$p$から$q$への円板鎖(\textit{chain of analytic disks from \( p \) to \( q \)})を次で定める:
$f_j : \Delta := D(0,1) \to X$を正則写像として, 
\[
p = f_0(0),  \quad f_j(z_j) = f_{j+1}(0), \quad 0 \le j \le k-1, \quad q = f_k(z_k),
\]
このとき, 小林擬距離(Kobayashi pseudo-distance) 
\( d_{\mathrm{Kob}} \) を
\[
d_{\mathrm{Kob}}(p,q) 
:= \inf_{\{f_j,  z_j\}} \left( d_{\mathrm{Poin}}(0, z_0) + \cdots + d_{\mathrm{Poin}}(0, z_k) \right).
\]
として定める. ここでPoicare 計量は
\[
 d_{\mathrm{Poin}}(0, z) = \log \frac{1 + |z|}{1 - |z|} = 2 \tanh^{-1} |z|
\]
とする.  ($\rho(0, z)$と書くこともある.)

$M$が多様体の場合は同値な定義もある.
\textit{Kobayashi-Royden infinitesimal pseudometric} を
$\xi \in T_{X,x}$について,  
\begin{align*}
F_M(\xi) &= \inf \{ \lambda > 0 ; \ \exists f : \Delta \to X, \ f(0) = x, \ \lambda f'(0) = \xi \}\\
&= \inf \left\{  \frac{1}{r} \mid  \exists g : D(0,r) \to X \ g(0) = x, \  g'(0) = \xi \right\}
\end{align*}
とし, $\gamma : [0, 1] \to M$という$C^\infty$級曲線に関して, Finsler pseudometricを
$$
L_{M}(\gamma):=\int_{0}^{1}F_{M}(\frac{d \gamma}{dt}) dt
$$
とし\footnote{$v = \xi + \overline{\xi} \in T_{X, x} \oplus \overline{T_{X, x} }$と分解して, $F_{M}(v)=F_{M}(2 \xi)$として定義する. 2倍する理由計算上の都合から. \cite[1.3.1]{NO90}参照}
$$
d_{\mathrm{Kob}}(p,q) := \inf_{ \gamma}L_{M}(\gamma)
$$
として定義する. 
この二つは一致する.
\end{defn}

一般のPoincare 計量に関しては次のとおり:
メビウス変換$\varphi_{w}(z):=\frac{z - w}{1 - \bar{w} z}$として
$$
d_{\mathrm{Poin}}(w, z):=\log\frac{1 + |\varphi_{w}(z)|}{1- |\varphi_{w}(z)|}
=
 \log \frac{|1 - z\bar{w}| + |z-w|}{| 1- z\bar{w}| - |z - w|}
= 2 \tanh^{-1} \left|\frac{z - w}{1 - \bar{w} z} \right|.
$$
である. これは$\Delta$上の$\frac{dz \otimes d\overline{z}}{(1 - |z|^2)^2}$という計量で与えられる.




以下成り立つことを述べておく.

\begin{prop}\cite[Chapter 3]{Kob98}
\label{prop-Kob98-Ch-3}
$X, Y$を解析空間とする. 
\begin{enumerate}
\item \cite[Proposition 3.1.6]{Kob98} $f : X \to Y$を正則写像とするとき
$$
d_{\mathrm{Kob}, Y} (f(p), f(q)) \le d_{\mathrm{Kob}, X} (p,q)
$$
特に$f$が双正則ならば上は等号が成立する.
\item \cite[Proposition 3.1.6]{Kob98}. 単位円板 $D$において, $d_{\mathrm{Kob}, D}$は
Poincare metric $\rho$に等しい. 
\item \cite[Proposition 3.1.7]{Kob98}. $d_{\mathrm{Kob}, X}$は擬距離であり, 任意の単位円板$D$からの正則写像$f : D \to X$について
$$
d_{\mathrm{Kob}, X} (f(z), f(w)) \le \rho(z,w)
$$
となる. また任意の$X$上の擬距離$\delta$で上のように
$\delta (f(z), f(w)) \le \rho(z,w)$を満たすならば, $\delta \le d_{\mathrm{Kob}, X}$となる. 
\item \cite[Theorem 3.1.9]{Kob98} $$d_{\mathrm{Kob}, X \times Y}((x_1, y_1), (x_2, y_2))=\max\{d_{\mathrm{Kob}, X}(x_1, x_2), d_{\mathrm{Kob}, Y}(y_1, y_2) \}$$
\end{enumerate}
\end{prop}
\begin{proof}

[1] $f_j : \Delta := D(0,1) \to X$を正則写像で
\[
p = f_0(0),  \quad f_j(z_j) = f_{j+1}(0), \quad 0 \le j \le k-1, \quad q = f_k(z_k),
\]
となるものについて, $f \circ f_j$が$f(p)$と$f(q)$を結ぶ正則写像たちになることから. 

[2]
まず単位円板上のPoincare metric $\rho$について, $D$と正則写像$f : D \to D$について, 
\begin{equation}
\label{eq-2-prop-Kob98-Ch-3}
\rho(f(a), f(b)) \leq \rho(a, b)
\quad \text{for } a, b \in D.
\end{equation}
が成り立つ. 
なぜならば$z \in D$についてSchwarz Lemmaより
\begin{equation}
\label{eq-prop-Kob98-Ch-3}
|\varphi_{f(0)}(z)|
\underset{\text{\ref{defn-Kob98-Ch-3}}}{=}
\left|\frac{f(z) - f(0)}{1 - \bar{f(0)} z} \right|
\underset{Schwarz}{\le} |z|
\end{equation}
であるので, $b=0$の場合は
$$
\rho(f(a), f(0))
=
\log\frac{1 + |\varphi_{f(0)}(f(a))|}{1- |\varphi_{f(0)}(f(a))|}
\underset{\text{(\ref{eq-prop-Kob98-Ch-3})}}{\le}
 \log\frac{1 + |a|}{1- |a|}
 =\rho(a, 0)
$$
となる. $b \neq 0$の場合はメビウス変換を考えて, $b=0$の場合に帰着する. 

簡単のため, $p=0$とする. 
$0$から$q$への円板鎖$f_{0}, \ldots, f_{k}$について
$$
\rho(0, q)
=\rho(0, f_{k}(z_k))
\underset{\text{$\rho$距離+$f_j(z_j) = f_{j+1}(0)$}}{\le}
\sum_{i=1}^{k} \rho(f_{i}(0), f_{i}(z_i))
\underset{\text{(\ref{eq-2-prop-Kob98-Ch-3})}}{\le}
\sum_{i=1}^{k} \rho(0, z_i)
$$
となる. 
右のinfがKobayashi擬距離の定義より$\rho(0,q) \le d_{\mathrm{Kob}}(0,q)$.
一方, $id : D \to D$は$0$から$q$への円板鎖を与えるので定義から
$\rho(0,q) \ge d_{\mathrm{Kob}}(0,q)$.
よって等号成立する. 

[3] 擬距離になるのは定義などから. $d_{\mathrm{Kob}, X} (f(p), f(q)) \le \rho(p,q)$は上の二つによる. 
また$\delta$がそのような擬距離ならば, $p$から$q$への円板鎖$f_{0}, \ldots, f_{k}$について
$$
\delta(p,q) 
\underset{\text{$\delta$擬距離}}{\le}
\sum_{i=1}^{n}\delta(f_i(0), f_i (z_i))
\underset{\text{仮定}}{\le}
\sum_{i=1}^{n}\rho(0, z_i)
$$
よって右のinfがKobayashi擬距離の定義よりいえる. 

[4] 込み入るので省略. 円板鎖を取り換えまくって定義から愚直に示していく.  
\end{proof}

\begin{ex}
(1). $\Delta=D(0,1)$の場合は
$$
d_{\mathrm{Kob}, D(0,1)}(0, z) = \rho(0, z)= \log \frac{1 + |z|}{1 - |z|}
$$

(2). 半径$r>0$の円板$D(0,r)$の場合は, $f : D(0,r) \to D(0, 1)$を$f(z)=\frac{z}{r}$が双正則写像になるので
$$
d_{\mathrm{Kob}, D(0,r)}(0, z) 
\underset{\text{\ref{prop-Kob98-Ch-3} (1)}}{=} 
d_{\mathrm{Kob}, D(0,1)}(f(0), f(z))
=\rho(0, f(z))
= \log \frac{r + |z|}{r - |z|}
$$

(3) $\C$の場合は(2)において, $r \to \infty$して$d_{\mathrm{Kob}, \C}\equiv \log 1=0$となる. 
よって\ref{prop-Kob98-Ch-3}から次が言える. 
$$
d_{\mathrm{Kob}, \C^n} 
\underset{\text{\ref{prop-Kob98-Ch-3} (4)}}{=} 
\max d_{\mathrm{Kob}, \C} \equiv 0 
$$

(4) $\C^{*}$について, 全射正則写像$f : \C \to \C^{*}$を$f(z)=e^z$とすれば
$$
d_{\mathrm{Kob}, \C^*} \underset{\text{\ref{prop-Kob98-Ch-3} (1)}}{\le} 
d_{\mathrm{Kob}, \C} \equiv 0 
$$

(5) 解析空間 $X$が$\C$-connected(\ref{defn-CW15-1.3}参照)ならば$d_{\mathrm{Kob}, X} \equiv 0$.
証明は以下の通り: $p, q \in X$について,
$f :\C \to X$で$f(a)=p, f(b)=q$となる$f$が存在する場合は
$$
d_{\mathrm{Kob}, X}(p, q)
\underset{\text{\ref{prop-Kob98-Ch-3} (1)}}{\le}
d_{\mathrm{Kob}, \C}(a, b)=0
$$
となりいえる. 一般の場合は, $p, q \in X$について,
$p=p_0, p_1, \ldots, p_{l}=q$で$p_i$と$p_{i+1}$が$\C$で結べるものをとれば
$$
d_{\mathrm{Kob}, X}(p, q)
\underset{\text{擬距離}}{\le}
\sum_{i=1}^{l}d_{\mathrm{Kob}, X}(p_{i-1}, p_i)=0
$$

よって次が言える
\[
\xymatrix@C=30pt@R=40pt{
\hbox{Oka} \ar@{=>}[r]&
\hbox{$\begin{array}{c} \C\text{-dominable} \\ (\exists \C^n \twoheadrightarrow X)\end{array}$} \ar@{=>}[r]&
\hbox{$\C$-connected} \ar@{=>}[r]&
\hbox{$\begin{array}{c} \text{Kobayashi} \\ \text{pseudometric} \equiv 0\end{array}$} \\
\hbox{$\begin{array}{c} \text{Rationally} \\ \text{connected}\end{array}$}  \ar@{=>}[r]&
\hbox{$\begin{array}{c} \text{Rationally} \\ \text{chain connected}\end{array}$}  \ar@{=>}[ru]&
&
\\
}
\]
特に複素連結Lie 群$X$について, $d_{\mathrm{Kob}, X} \equiv 0$である. 
\end{ex}



\begin{thm}[Brody 78]
コンパクト複素多様体について次は同値. 
\begin{enumerate}
\item Kobayashi hyperbolic, つまり$d_{\mathrm{Kob}}$は距離になる. ($d_{\mathrm{Kob}}(x, y)=0 \Leftrightarrow x=y$ということ)
\item Brody hyperbolic, つまり任意の正則写像\( f : \mathbb{C} \to X \) は定数. 
%\item The Kobayashi \textit{infinitesimal pseudometric} \( k_x \) is everywhere non degenerate.
\end{enumerate}
\end{thm}

他には以下のことが知られている. 
\begin{lem}\cite{KL22}
$X, Y$をコンパクト解析空間とする. 
 性質 \( P \)を“non-Kobayashi hyperbolic”または“\( d_{\mathrm{Kob}, X} \equiv 0 \)”とするとき, 次が成り立つ. 
\begin{enumerate}
%\item Holomorphic maps \( f : X \to Y \) are distance decreasing for the Kobayashi metric.
\item \( f : X \to Y \)がfiniteとする. $X$が性質$P$を持つなら, \( Y \)も性質\( P \)を持つ.  finite étale なら逆も成り立つ. 
\item\( f : X \dashrightarrow Y \) dominant meromorphic mapについて, \( d_X = 0 \)ならば, \( d_Y = 0 \).
\item  \( X = X_1 \times \cdots \times X_n \) とし, 全ての\( X_i \) が性質$P$を持つなら, $X$も性質$P$を持つ
\item \( f : Y \to X \) bimeromorphic morphismで\( X \)をsmoothとする. 
 \( d_X \le \varepsilon \) ならば \( d_Y \le \varepsilon \)である.
\end{enumerate}
\end{lem}
(1)の主張に関しては , \cite[Theorem 3.2.8]{Kob98}による: 
$X$ 解析空間, $\pi : \widetilde{X} \to X$がcovering spaceならば
$$
d_{\mathrm{Kob}, X} (p, q) = \inf_{\pi(\widetilde{p})=p, \pi(\widetilde{q})=q, }
d_{\mathrm{Kob}, \widetilde{X}} (\widetilde{p}, \widetilde{q}).
$$

最後の主張については\cite[Theorem 3.2.19]{Kob98}による:
Zariski closed subset \( V \subset X \) で \(\codim_{X}V \ge 2 \)ならば, 
\[
(d_{\mathrm{Kob}, X})|_{X \setminus V} = d_{\mathrm{Kob}, X \setminus V}
\]



\begin{thm}[Demailly-Lempert-Shiffman]
 \( X \)をquasi-projective varietyする. 
 このときKobayashi 擬距離\( d_{\mathrm{Kob}, X} \)は代数曲線で定義される. 
 つまり代数曲線の族$C_i$で
\[
d_{\mathrm{Kob}, X}(p, q) = \inf \sum_{i=1}^{k} d_{\mathrm{Kob}, C_i}(p_{i-1}, p_i),
\]
となる. ただし$p=p_0, p_1, \ldots, p_{l}=q$で$p_i$と$p_{i+1}$は代数曲線$C_i$で結べるものとする. 
\end{thm}
これからもRationally chain connectedならば$d_{\mathrm{Kob}, X} \equiv 0$が言える. 



\subsubsection{Orbifold Kobayashi-Ochiai}

\begin{thm}\cite[Theorem 8.2]{Cam04}
\label{thm-CW15-2.8}
$X$をコンパクト複素多様体 in Fujiki's class, $c_X : X \dashrightarrow C(X)$をcoremapとする. 
$\overline{M}$を連結複素多様体とし, $M \subset \overline{M}$ Zariski open subsetとする. 

このとき任意のmeromorphic map $\varphi : M \dashrightarrow X$で$g:= c_X \circ \varphi: M \dashrightarrow C(X)$が非退化(dominant)になるものについて, $g$は$\overline{M} \dashrightarrow C(X)$にmeromorphicallyに拡張する.
   \[
\xymatrix@C=50pt@R=20pt{
M  \ar@{-->}[r]^{\forall \varphi}   \ar@{->}[d] \ar@{-->>}[rd]_{g}  &X  \ar@{-->>}[d]^{c_X}\\
\overline{M} \ar@{-->}[r]_{\exists \overline{g}} & C(X) \\
}
\]
\end{thm}
$g:= c_X \circ \varphi: M \dashrightarrow C(X)$が非退化とは, $M$のある点でsubmersive, つまりヤコビ行列がの階数が$\dim C(X)$と等しいこと. サードの定理よりこれは$g$がdominant(全射)と同じである. 
また$c_X$の部分は, 任意のof general typeな射$f : X \dashrightarrow Y$でも良い. (\cite[Chapter 2]{Cam04}参照.)
 
\begin{cor}\cite[Corollary 2.9]{CW15}
\label{cor-CW15-2.9}
$X$ コンパクト複素多様体 in Fujiki's classとする. 
dominant meromorphic map$\varphi : \C^n \dashrightarrow X$が存在するならば, $X$はspecialである.
特にOkaならばspecial.
\end{cor}
\begin{proof}
$M:=\C^n \subset \C\mathbb{P}^n=:\overline{M}$とすると, \ref{thm-CW15-2.8}よりdominant meromorphic map $\overline{g} : \C\mathbb{P}^n \dashrightarrow C(X)$を持つ. 
$\C\mathbb{P}^n$がspecialなので, \cite[Corollary 2.17]{Cam04}から$X$もspecialとなる

最後の主張はOkaならば$\C^n$-dominable, つまり dominant meromorphic map $\varphi : \C^n \dashrightarrow X$を持つため. 
\end{proof}

\begin{rem}
\ref{thm-CW15-2.8}の状況において, $D = \overline{M} \setminus M$は$\overline{M}$のreduced divisorであるとする. 
すると\cite[Theorem 8.2]{Cam04}(Kobayashi-Ochiai)によって
$$
g^{*} : H^0(C(X), m(K_{C(X)} +\Delta(c_X)))\to H^0(\overline{M}, (\Omega_{\overline{M}}^{\dim C(X)})^{\otimes m} ((m-1)D))
$$
という写像を得る. 
$g$が非退化なので, あるlocusで$g$は全射になることから, $g^{*}$は単射になる. 

これを用いると\ref{cor-CW15-2.9}の別証明を与えることができる. 
$M:=\C^n \subset \C\mathbb{P}^n=:\overline{M}$とおくと, 
\begin{equation}
\label{eq-P^nvanish}
H^0(\overline{M}, (\Omega_{\C\mathbb{P}^n}^{\dim C(X)})^{\otimes m} ((m-1)D))=0
\end{equation}
である. よって$\dim C(X)=0$である.

\ref{eq-P^nvanish}の証明に関しては次のとおり:\footnote{\cite[Example 8.8]{Cam04}には"次の事実は容易に確認できる"とあったが, これは容易じゃないと思う. }
$D=\C\mathbb{P}^n \setminus \C^n $は$\C\mathbb{P}^n$のhyperplaneと同一視でき
$D \sim \mathcal{O}_{\C\mathbb{P}^n}(1)$である. 
$p=\dim C(X)$とすると, $D^{n-1}$に関するslope $\mu_{D}$は以下のように計算できる. 
\begin{align*}
\mu_{D}\left((\Omega_{\C\mathbb{P}^n}^{p})^{\otimes m} ((m-1)D)\right)
&\underset{\text{slopeの加法性}}{=}
\mu_{D}\left((\Omega_{\C\mathbb{P}^n}^{p})^{\otimes m} \right) 
+ \mu_{D}((m-1)D)) \\
&\underset{D^n =1}{=} m \mu_{D}(\Omega_{\C\mathbb{P}^n}^{p})  + m-1\\
&\underset{\cite[Prop. 4.2]{Iwa21}}{=}  - m \frac{p(n+1)}{n} + m -1 <0 \\
\end{align*}
また今$\C\mathbb{P}^n$はKE計量を持つので
$(\Omega_{\C\mathbb{P}^n}^{p})^{\otimes m} ((m-1)D)$は$D$-semistableである. 

一般にample line bundle $A$について, vector bundle $E$が$A$-semistableかつ$\mu_{A}(E) <0$ならば, $H^0(X, E)=0$である(例えば\cite[Chapter 5]{Kob14}参照)
よって\ref{eq-P^nvanish}が言える. 
\end{rem}

\begin{thm}\cite[Theorem 2.10]{CW15}
\label{thm-CW15-2.10}
$X, Z$をprojective manifold, 
$M$をsmooth algebraic varietyとする. 
次を仮定する.
\begin{itemize}
\item $\tau : M \to Z$という全射なalgebraic射を持つ. 
\item 任意の$\tau$のfiberは$\C^k$と双正則なaffine空間とする. 
\end{itemize}

任意のmeromorphic map $G : M \dashrightarrow X$で
$g = c_X \circ G : M \dashrightarrow C(X)$が非退化(dominant)になるものについて, 
ある$\varphi : C(Z) \dashrightarrow C(X)$が存在し, 次の図式が可換になる. 

\[
\xymatrix@C=50pt@R=20pt{
M \ar@{->>}[r]^{\tau}\ar@/^2pc/@{-->}[rr]^{\forall G} &Z \ar@{-->}[d]_{c_Z} & X\ar@{-->>}[d]^{c_X} \\
&C(Z) \ar@{-->}[r]^{\exists \varphi}  & C(X)
}
\]

\end{thm}
\begin{proof}
$M =\overline{M} \setminus D$となるような
projective variety $\overline{M}$とnormal crossing な超曲面$D$をとる. 
すると$\overline{\tau} : \overline{M} \rightarrow Z$と言う拡張が取れる. \footnote{Nagataのコンパクト化. \url{https://stacks.math.columbia.edu/tag/0F3T?utm_source=chatgpt.com} 参照. ここに代数性を使っている. (Steinだと無理だと思う)}
%\footnote{affine variery$V$は$\C^n \hookrightarrow \C\mathbb{P}^n$を使って, Zariski 閉包をとればこのことができる. 一般の場合は有限個の被覆をとってprojectiveに埋め込んで閉包を取るはず.}

また \ref{thm-CW15-2.8}から$g$の拡張 $\overline{g} : \overline{M} \dashrightarrow C(X)$が存在する.
core mapのfunctoriality \cite[Chapter 2]{Cam04}も用いて次の図を得る. 
\[
\xymatrix@C=50pt@R=20pt{
\overline{M} \ar@{-->}[d]^{c_{\overline{M}}} \ar@{->}[r]^{\overline{\tau}}\ar@/^2pc/@{-->}[rr]^{G} 
&Z \ar@{-->}[d]_{c_Z} 
& X\ar@{-->}[d]^{c_X} \\
C(\overline{M})\ar@{-->}[r]^{c_{\overline{\tau}}}  
\ar@/_2pc/@{-->}[rr]_{c_{\overline{G}}} 
&C(Z) 
& C(X)
}
\]
$c_{\overline{\tau}} : C(\overline{M}) \dashrightarrow C(Z)$がbimeromorphicであることを示す.
もしこれが示されれば
$\varphi:=c_{\overline{G}} \circ c_{\overline{\tau}}^{-1} : C(Z) \dashrightarrow C(X)$とおけば証明が完了する. 

さて$\overline{\tau} : \overline{M} \rightarrow Z$のgeneral fiberは有理連結(rationally connected)である.
これは
$$
\tau : M \hookrightarrow \overline{M} \underset{\overline{\tau}}{\to} Z 
$$
より, 一般の$z \in Z$について$\C^{k} \cong \tau^{-1}(z) \subset \overline{\tau}^{-1}(z)$である. よって$\overline{\tau}^{-1}(z)$は$\C\mathbb{P}^k$と双有理であり, 特にrationally connectedである.

よって$r : \overline{M} \dashrightarrow R(\overline{M})$をrational quotient(MRC fibration)とすると, 
$r(\overline{\tau}^{-1}(z))$は一般の$z \in Z$でpointとなる. 
よって$\alpha : Z \dashrightarrow R(\overline{M})$があって, $\alpha \circ  \overline{\tau}= r$となる. 

\[
\xymatrix@C=50pt@R=20pt{
\overline{M} \ar@{-->}[d]^{c_{\overline{M}}} \ar@{->}[r]^{\overline{\tau}}\ar@/^2pc/@{-->}[rr]^{r} 
&Z \ar@{-->}[d]_{c_Z} \ar@{-->}[r]_{\exists \alpha} 
& R(\overline{M})\ar@{-->}[d]^{c_{R(\overline{M})}} \\
C(\overline{M})\ar@{-->}[r]^{c_{\overline{\tau}}}  
\ar@/_2pc/@{-->}[rr]_{c_{r}} 
&C(Z)  \ar@{-->}[r]_{\exists c_{\alpha}} 
& C(R(\overline{M}))
}
\]

coremapのfunctorialityより
$$
c_\alpha: C(Z) \dashrightarrow C(R(\overline{M}))
\quad \text{and} \quad 
c_{\alpha} \circ c_{\overline{\tau}} = c_r
$$
である. よって\cite[Theorem 3.26]{Cam04}(下参照)より, 
$c_r: C(\overline{M}) \dashrightarrow C(R(\overline{M}))$はbimeromorphicであるので言えた. 
%これより$c_{\overline{\tau}} : C(\overline{M}) \dashrightarrow C(Z)$のgeneral fiberもRationally connectedである.(RCの像はRCであるので.) \footnote{meromorphic map $f : X \dashrightarrow Y$の$y \in Y$のfiberとは$\pi : \widetilde{X} \to X, \widetilde{f} : \widetilde{X} \to Y$と言う$f$の不確定点除去をとって$f^{-1}(y) := \pi(\widetilde{f}^{-1}(y))$と定義する. } さて$r : \overline{M} \dashrightarrow R(\overline{M})$をrational quotient(MRC fibration)とすると, $\$下の定理より
%の「」という定理から, rational quotientを介することで言えた. \footnote{この定理使っているとき点で$Z$がMoishezonは必要}
\end{proof}
\begin{thm}\cite[Theorem 3.26]{Cam04}
$Y$をコンパクト複素多様体 in Fujiki's class, $r : Y \dashrightarrow R(Y)$をrational quotientとするとき
$c_{r} : C(Y) \dashrightarrow C(R(Y))$はbimeromorphicである.
\end{thm}


\begin{rem}\cite[Remark 2.11]{CW15}
もっと強く次が成り立つ.

\begin{itemize}
\item $X, Z$ projective manifold. 
\item  $F : X \dashrightarrow Y$ general type orbifold baseへのfibration. 
\item $M$ smooth algebraic varietyで, projective manifold $\overline{M}$と, normal crossing hypersurface $D$によるコンパクト化$M =\overline{M} \setminus D$をもつとする. 
\item $\overline{\tau} : \overline{M} \to Z$のfiberはrationally connectedであり, 任意の$D$の既約成分は$Z$に全射でうつる. 
\item meromorphic map $G : M \dashrightarrow X$で$g = F \circ G : M \dashrightarrow Y$がdominantであるとする.
\end{itemize}

このときある$\varphi : C(Z) \dashrightarrow Y$が存在し, 次の図式が可換になる. 
\[
\xymatrix@C=50pt@R=20pt{
M \ar@{->}[r]^{\tau}\ar@/^2pc/@{-->}[rr]^{\forall G} &Z \ar@{-->}[d]_{c_Z} & X\ar@{-->}[d]^{F} \\
&C(Z) \ar@{-->}[r]^{\exists \varphi}  & Y
}
\]
証明には\cite{GHS03}と\cite[Theorem 3.26]{Cam04}を使う.
\end{rem}


\subsection{3章 Jouanoulou's trick}
\begin{prop}\cite[Proposition 3.1]{CW15}
\label{prop-CW15-3.1}
$X$をprojective manifoldとする(もっと強くvarietyでも良い).

このときある \underline{affine} 複素多様体$M$と正則全射$\tau : M \to X$があって次を満たす.
\begin{enumerate}
\item $\tau : M \to X$ ホモトピー同値
\item $\tau$の全てのファイバーは$\C^n$と同相. 特に任意のファイバーでKobayashi 擬距離が消えている.
\item $\tau$は局所的に自明なファイバー束である. 
\item $\tau$はreal analytic sectionを持つ.
\end{enumerate}
\end{prop}

$M$がAffineとは$\C^n$の閉代数多様体と同型なもの. 同じことだが$M \cong \mathrm{Spec}(\C[X_1, \ldots, X_N]/(f_1, \ldots, f_l))$となること.
この命題において"$M$がaffine(Stein)"であることが重要である. 
\begin{rem}\cite[Remark 3.2]{CW15}
上は\cite{Lar05}で紹介されたテクニック.
"Good manifold"("quasi-projective manifold上のaffine bundleでファイバーが$\C^n$")でも上の主張は成り立つ. 
\end{rem}

\begin{proof}

[$X = \C\mathbb{P}^N$のとき.]
$(\C\mathbb{P}^{N})^{\vee}$をdual projective space, つまり$\C\mathbb{P}^N$のhyperplaneの集合とする. 
$P:=\C\mathbb{P}^N \times (\C\mathbb{P}^{N})^{\vee}$とし
$$
D:= \{ (x, H) \in P \mid x \in H\} \subset P=\C\mathbb{P}^N \times (\C\mathbb{P}^{N})^{\vee}
$$
とおく. $\C\mathbb{P}^N$のhyperplaneのなるuniversal familyのgraphである. 

$D \subset P$はample である.\footnote{\cite{CW15}には"Projection $pr_1 : P \to \C\mathbb{P}^N$, $pr_2 : P \to (\C\mathbb{P}^{N})^{\vee}$のfiberに含まれるlineのなす二つのfamilyとpositiveに交差するからampleである"と書いてあったがわからなかった. }
まず$(\C\mathbb{P}^{N})^{\vee} \cong \C\mathbb{P}^{N}$である. 
これは
$$
(\C\mathbb{P}^{N})^{\vee}
\ni H = \{[z_0 : \cdots : z_N] \in \C\mathbb{P}^{N} \mid a_0z_0 + \cdots + a_Nz_N =0 \}
\mapsto [a_0 : \cdots a_N] \in \C\mathbb{P}^{N}
$$
で対応する. よって以後$P=\C\mathbb{P}^{N} \times \C\mathbb{P}^{N}$とみなす. 
この同一視によって, 
$$
D =\{ ([z_0: \cdots : z_N], [w_0: \cdots : w_N]) \in \C\mathbb{P}^{N} \times \C\mathbb{P}^{N}\mid z_0 w_0 + \cdots + z_Nw_N=0 \}
$$
となる. よって$D$は$\C\mathbb{P}^{N} \times \C\mathbb{P}^{N}$のhyperplaneになるのでampleである. \footnote{セグレ埋め込み$\C\mathbb{P}^N \times (\C\mathbb{P}^{N})^{\vee} \hookrightarrow \C\mathbb{P}^{N^2 + 2N}$によって, $D$は$\C\mathbb{P}^{N^2 + 2N}$の2次超曲面内のhypersurfaceになることからもわかる. セグレ埋め込みは\url{https://mathlog.info/articles/1265}参照}

$V:= P \setminus D$, $\tau:=pr_{1}|_{V} : V\to \C\mathbb{P}^N$とおけば, これが(1-4)の条件を満たすaffine多様体であることを示す. これを見ていく. 

(1-3). $U:=\{ z_N \neq 0\} \subset \C\mathbb{P}^N$とする. $U \cong \C^{N}$であり,  $U$という座標近傍で見ると, $\eta_i :=\frac{z_i}{z_N}$として
$$
 \{ z_N \neq 0\} \times \C\mathbb{P}^{N-1} \cong \tau^{-1} (\{ Z_N \neq 0\} \cap D)
= \{((\eta_0, \ldots, \eta_{N-1}),[w_0: \cdots : w_N] )\mid w_N = \eta_0 w_0 + \cdots +\eta_{n-1}w_{N-1}  \} 
$$
$$
(\eta_0, \ldots, \eta_{N-1}), [w_0: \ldots :w_{N-1}]
\mapsto (\eta_0, \ldots, \eta_{N-1}) , [w_0: \ldots:  w_{n-1}: \sum_{i=0}^{N-1}\eta_i w_i]
%=\{ ([z_0: \cdots : z_N], [w_0: \cdots : w_N] \mid z_0 w_0 + \cdots z_Nw_N \}
$$
である.  よって$\tau^{-1}(U) \cong U \times\mathbb{P}^{N} \setminus  U \times\mathbb{P}^{N-1} \cong U \times\C^{N}$である. これより(2), (3)が言える. 
(1)は(3)より両者のホモトピー群が等しいのでWhiteheadの定理 \ref{thm-Whitehead}からわかる. 


(4)に関しては, 
$$
\C\mathbb{P}^N \ni [z_0 : \cdots : z_N] 
\mapsto ( [z_0 : \cdots : z_N], [\overline{z_0} : \cdots : \overline{z_N}] ) \in V
$$
がreal analyticなsectionとなる. 
%real analytic sectionは$\C^{n+1}$のHermitian metricを入れて, lineについてhyperplaneを対応させる対応となる. 
%つまり簡単に言えば

%について, $x \subset \C^{n+1}$のlineを得るので, これと直行しているhyperplaneを$H_x$といて, $(x, H_x) \in V$を対応させる \xr{これはreal analytic?}

$M$がaffineになることに関しては, 一般論から.\footnote{projective varietyの基本開集合$D_{+}(f)$がaffineになる. \url{https://stacks.math.columbia.edu/tag/01MB?utm_source=chatgpt.com}}
Projective variety $W=\mathrm{Proj} \C[X_0, \ldots, X_n]/(f_1, \ldots, f_l) = \mathrm{Proj} \oplus_{l \in \N} S_l$について, $g(X_0, \ldots, X_n)$を$d$次斉次多項式としたとき
$$
\{ w \in W \mid g(w) \neq 0\} \cong 
\mathrm{Spec}\left( \left\{ \frac{S_{kd}}{g^k} \mid k \in \N \right\} \right)
$$
となるのでaffineである. 

一般論を使わなくても, セグレ埋め込み
$$
\C\mathbb{P}^{N} \times \C\mathbb{P}^{N} \hookrightarrow  \C\mathbb{P}^{N^{2}-1}
\quad (z, w) \mapsto (z_i w_j)_{1 \le i, j \le N}
$$
を使ってもわかる. $\C\mathbb{P}^{N^{2}-1}$の座標を$T_{ij}$とすると
$P\cong \{ T_{ii}T_{jj} - T_{ij} T_{ji}=0\}_{1 \le i, j \le N} \subset \C\mathbb{P}^{N^{2}-1}$という零点集合になる. 
この同一視によって$D \cong P \cap \{ \sum_{i=1}^{N}T_{ii}=0\}$となる. 
よって
$$
M \cong \{ T_{ii}T_{jj} - T_{ij} T_{ji}=0\}_{1 \le i, j \le N} \cap   \{ \sum_{i=1}^{N}T_{ii} \neq 0\}
$$
となる. 
$M$はAffine多様体$\{  S_{ii}S_{jj} - S_{ij} S_{ji}=0 \} \cap \{ \sum_{i=1}^{N}S_{ii} =1\} \subset \C^{N^2}$と同型になる. 


一般には$X \subset \C\mathbb{P}^N$を閉埋め込みとして, $M:={pr_{1}}^{-1}(X)$とすれば良い. 
これは$X \subset \C\mathbb{P}^N, V \to \C\mathbb{P}^N$のbase changeなので, と$M$は$V=\C\mathbb{P}^N \setminus D$のclosed algebraic subsetになり, $V$がaffineなのでaffineとなる. 
(1-4)に関しても$V$上で成り立っているので, $M$でも成り立つ. (base changeでファイバーは不変である. $X$がsingularでも$C^\infty$性はこれからわかる).
\end{proof}

\begin{ex}
$X=\C\mathbb{P}^1$のとき. 
\begin{itemize}
\item $P:=\C\mathbb{P}^1 \times (\C\mathbb{P}^{1})^{\vee}=\C\mathbb{P}^1 \times \C\mathbb{P}^1$
\item $D\cong \{ (x, x) \in \C\mathbb{P}^1 \times \C\mathbb{P}^1\}$. 特に対角線集合と同型. 
これは
$$
D =\{ ([z_0: z_1], [w_0: w_1]) \in \C\mathbb{P}^1 \times \C\mathbb{P}^1\mid z_0 w_0 + z_1w_1=0\}
=\{ ([z_0: z_1], [-z_1: z_0]) \in \C\mathbb{P}^1 \times \C\mathbb{P}^1\}
$$
なので, 後ろの座標を$[-z_1: z_0] \mapsto [z_0: z_1]$と入れ替えれば言える. 
\item $V = P \setminus D=\{(x, y) \in \C\mathbb{P}^1 \times \C\mathbb{P}^1\mid x \neq y\}$
\item $\tau={pr_1}|_{V} : V \to X$ $(x, y) \mapsto x$. これは$\C$-bundleになる.
\item real analytic section $X \to V$は例えば$x \mapsto (x, \overline{x})$などがある. 
\item セグレ埋め込みは
$$
\C\mathbb{P}^1 \times \C\mathbb{P}^1 \to\C\mathbb{P}^3
\quad 
[a: b], [c,d] \mapsto [ac:ad:bc:bd]
$$
であり, $P \cong \{xw - yz=0 \} \subset \C\mathbb{P}^3$, 
$D \cong \{ x+w=0\}$となる. 
よって
$$
P \setminus D \cong \{xw - yz=0 \} \cap \{ x+w \neq 0\} \subset \C\mathbb{P}^3
$$
である. これは$\C^3$内の$x(1-x)-yz=0$と同型である.\footnote{$x+w=1$を$xw - yz=0$に代入する. }
\end{itemize}
$C$が代数曲線のときはfinte map
$f : C \to \mathbb{P}^1$として, そのbase change$M \to C$をとれば, これが欲しいやつとなる. \footnote{\xr{と書いてあったがこれ本当?singularになると思うんだけど...}}
\end{ex}

\begin{ques}\cite[Ques 3.3]{CW15}
任意の複素多様体 $Z$について, あるStein manifold $S$と正則写像 $f : S \to Z$があって, そのfiberが$\C^n$となるものはあるか?
\end{ques}
$Z$がprojectiveなら上より存在, コンパクト K\"ahlerのときでも未解決. 

\subsection{4章 Opposite complex structures and associated cohomological integrals}

\subsubsection{dominant rational mapでの引き戻し }
下の補題に関して現論文では"in Fujiki's class"の仮定はなかったが, 多分必要だと思う. 
\begin{lem}\cite[Lemma 4.1]{CW15}
\label{lem-CW15-4.1}
$X, Y$ コンパクト複素多様体 \underline{in Fujiki's class}, 
$f: X \dashrightarrow Y$ dominant meromorphic map, 
$I(f) \subset X$ 不確定点とする. 

このとき任意の$c \in H^{k,k}(Y)$について, ただ一つの$c' \in H^{k,k}(X)$で次を満たすものが存在する:

任意の$X$上のclosed $(n-k, n-k)$-form $\alpha$, $Y$上のclosed $(k, k)$-form $\beta$で, 
 $c=[\beta]  \in H^{k,k}(Y) $となるものについて, 
$$
[\alpha] \cdot c'
=\int_{X \setminus I(f)} \alpha \wedge f^{*}\beta
$$
上のおいて$[c]$の$f$での引き戻しを$f^{*}([c]):=c'$として定義する.
\end{lem}
%\xr{これdeRhamか?Doulbeautではなく?}

$H^{k,k}(X)$については次のとおり (\cite[Section 6.8, 6.12]{Demagbook}参照).
Dolbeault cohomology
$$
H^{p,q}(X):=H^q (X, \Omega_{X}^{p}):=
 \{ \text{$\overline{\partial}$-closed $(p,q)$-forms}\} 
/ \{\text{$\overline{\partial}$-exact $(p,q)$-forms} \}
$$
で定める.
K\"ahlerならば, 
%$X$ 複素多様体なので, 
Bott-Chern cohomology group
$$
H^{p,q}_{BC}(X)
:=\{ \text{$d$-closed $(p,q)$-forms}\} 
/ \{\text{$\partial \overline{\partial}$-exact $(p,q)$-forms} \}
$$
として, $H^{p,q}(X) \cong H^{p,q}_{BC}(X)$である.
またK\"ahler indentityから
$$H^{k}_{dR}(X)\cong \oplus_{p+q=k}H^{p,q}(X)
\quad \text{and} \quad
\overline{H^{p,q}(X)} \cong H^{q,p}(X)
$$
である.
これはFujikiでも成り立つ. 
またSerre dualityによって(これにはK\"ahlerは不要)
$$
<> : H^{p,q}(X) \times H^{n-p,n-q}(X) \to \C
$$
は非退化である. 

\begin{proof}

[$c'$の唯一性]
$(c_1 - c_2) \cdot [\alpha]=0$が
任意のclosed $(n-k, n-k)$ form $\alpha$ で成り立つので, Serre dualityより, $c_1=c_2$である. 


[$c'$の存在性]
$\tau : X' \to X, F: X' \to Y$を
$f$の不確定点除去とする. 
$F^{*} :  H^{p,q}(Y) \to H^{p,q}(X')$によって, 
$F^{*} [\beta] \in H^{k, k}(X')$が定義できる. 
そこで$\alpha$を$X$上のclosed $(n-k, n-k)$-closed form として
$$
\alpha \mapsto \int_{X'}\tau^{*}\alpha \wedge  F^{*}\beta 
$$
と定義する. 
すると
$$F^{*} [\beta]  : H^{n-k, n-k}(X) \to \C
\quad
[\alpha] \mapsto \int_{X'}\tau^{*}\alpha \wedge  F^{*}\beta 
$$が定義できる.
$\tau^{*} : H^{n-k, n-k}(X)  \hookrightarrow H^{n-k, n-k}(X')$なので, $F^{*} \beta $は非退化である. 
よって, Serre dualityより, $F^{*} [\beta] \in H^{k, k}(X)$が定義できる. 
これを$c' \in H^{k, k}(X)$とする. 
定義から
$$
[\alpha] \cdot c' 
\underset{\text{def}}{=}\int_{X'} \tau^{*}\alpha \wedge F^{*} \beta
\underset{\text{smooth}}{=} \int_{X' \setminus \tau^{-1}(I(f))}\tau^{*}\alpha \wedge F^{*} \beta 
\underset{\text{isom}}{=} \int_{X \setminus I(f)}\alpha \wedge f^{*} \beta 
$$ 
となる. 
\end{proof}

\subsubsection{Opposite complex structure}
以下概複素構造などの復習. 

$X$を$n$次元複素多様体とする.
$X_{\R}$を$X$を$2n$次元実多様体とみなしたものとし, 
$T_{\R}X$を$X_\R$の実接ベクトル束とする. 
$z_i := x_i + \sqrt{-1}y_i$として
\begin{equation}
\label{eq-cpxstr}
J : T_{\R}X \to T_{\R}X
\quad
J\left(\frac{\partial}{\partial x_i} \right):=\frac{\partial}{\partial y_i}
\quad
J\left(\frac{\partial}{\partial y^i}\right):=-\frac{\partial}{\partial x_i}
\end{equation}
と定義する. 
これは$J \circ J = -Id$となる. 

一般に$J \circ J = -Id$となるものを概複素構造という. 
概複素構造$J$について
$$
T_{\R}X \otimes_{\R}\C 
=T'X \oplus T''X
$$
と分解できる, ここで$T'X$ (resp. $T''X$)は$J$の固有値$\sqrt{-1}$ (resp. $-\sqrt{-1}$)に対応する部分束である.
同様にして余接ベクトル束$T_{\R}^{*}X$について
$$
T_{\R}^{*}X \otimes_{\R}\C 
={T'}^{*}X \oplus {T''}^{*}X
$$
と分解できる. 
同様にして, 
$$
\wedge^{r} T_{\R}^{*}X \otimes_{\R}\C 
=
\sum_{p+q=r} 
\wedge^{p} {T'}^{*}X \otimes\wedge^{q} {T''}^{*}X
$$
である. 

$J$の捩れを, $u,v \in T_{\R}X$について, Nijenhuisテンソル
$$
N(u,v):=[Ju, Jv]-[u,v]-J[u, Jv]-J[Ju,v]
$$
と定義する. 
$N(fu, v)=fN(u,v)=N(u, fv)$なので, $N_x :  T_{\R, x}X \times T_{\R, x}X \to T_{\R, x}X$という交代双線形写像を得る.


\begin{thm}[Newlander-Nirenberg]
$N=0$($J$が可積分)ならば複素構造である
\end{thm}

\begin{defn}
\label{defn-CW15-opposite}
$X$を$n$次元複素多様体, $J$を$X$に対応する可積分な概複素構造, 
$X_\R$を$X$の$2n$次元実多様体構造とする.

$X$のopposite (conjugate) complex structure $\overline{X}$を以下で定める.
$$
\overline{X} :=
(X_\R, -J)
$$
\end{defn}
$\overline{X}$が複素多様体となるのは, $-J$が可積分になるから. 実際
$$
N_{-J}(u, v)
=[-Ju, -Jv]-[u,v]-(-J)[u, -Jv]-(-J)[-Ju,v]
=[Ju, Jv]-[u,v]-J[u, Jv]-J[Ju,v]
=N_{J}(u, v)
$$
なので$J$が可積分と$-J$可積分は同値である. 

\begin{ex}
$X=(\C^n, z_1, \ldots, z_n)$について
$\overline{X}=(\C^n, \overline{z}^1, \ldots, \overline{z}^n)$である. 
この理由を定義から説明する. 

以下\ref{eq-cpxstr}のあたりのように, 
$z_i = x_i + \sqrt{-1}y_i$と実係数で表示し, $X_\R$に$(x_1, y_1, \ldots, x_n, y_n)$という実座標を取る. すると
$$
T_{\R}X = \R \left( 
\frac{\partial}{\partial x_1}, \frac{\partial}{\partial y_1}, \ldots, \frac{\partial}{\partial x_n}, \frac{\partial}{\partial y_n}
\right)
$$
となる.\footnote{$R( \cdot )$とは$\R$で生成されるベクトル空間とする.}. 
よって
$$
T_{\R}X \otimes \C=
 \C \left( \frac{\partial}{\partial x_1}, \frac{\partial}{\partial y_1}, \ldots, \frac{\partial}{\partial x_n}, \frac{\partial}{\partial y_n}\right)
 ={T'}_X \oplus {T''}_{X}
$$
である. 定義から次がわかる. 
$$
{T'}_X :=\C\left( \underbrace{\frac{\partial}{\partial x_1} - \sqrt{-1} \frac{\partial}{\partial y_1} }_{=: \frac{\partial}{\partial z_1}}, \ldots, \underbrace{\frac{\partial}{\partial x_n} - \sqrt{-1} \frac{\partial}{\partial y_n}}_{=: \frac{\partial}{\partial z_n}}, \right)
\quad
{T''}_X :=\C\left( \underbrace{\frac{\partial}{\partial x_1} + \sqrt{-1} \frac{\partial}{\partial y_1} }_{=: \frac{\partial}{\partial \overline{z_1}}}, \ldots, \underbrace{\frac{\partial}{\partial x_n} +\sqrt{-1} \frac{\partial}{\partial y_n}}_{=: \frac{\partial}{\partial \overline{z_n}}} \right)
$$

さて$\overline{X}=(X_\R, -J)$の複素構造においての$T_{\overline{X}}$とは, 
\begin{center}
$T_{\overline{X}}$ $=$ ($-J : T_{\R}X \otimes \C \to T_{\R}X \otimes \C$の固有値$\sqrt{-1}$の固有空間)
\end{center}
すると\ref{eq-cpxstr}から
$$
(-J)\left( \frac{\partial}{\partial \overline{z_i}} \right)
\underset{\text{定義}}{=}
-J\left( \frac{\partial}{\partial x_i}\right) - \sqrt{-1} J\left(\frac{\partial}{\partial y_i}\right)
\underset{\ref{eq-cpxstr}}{=}
-  \frac{\partial}{\partial y_i} + \sqrt{-1} \frac{\partial}{\partial x_i}
= \frac{\partial}{\partial \overline{z_i}}
$$
である. よっていえた. 
\end{ex}

さて$X$を$n$次元複素多様体, $J$を$X$に対応する可積分な概複素構造, 
$X_\R$を$X$の$2n$次元実多様体構造とする.
また$X$の正則接ベクトル束を, ${T'}_X$と同一視する. 

$h : {T'}_X \times {T'}_X \to \C$を$X$のエルミート計量とする.\footnote{以下は中野 多変数函数論 9章を参照した.}
そして
$$
h_{i \overline{j}}:= h\left( \frac{\partial}{\partial z_i}, \frac{\partial}{\partial z_j}\right)
$$
とする. すると以下のように$h$から$\widetilde{h} : T_{\R}X \otimes \C \times T_{\R}X \otimes \C\to \C$という拡張を得る. 
\begin{itemize}
\item ${T'}_X$と${T''}_{X}$は$\widetilde{h}$で直交している.
\item $u,v \in {T''}_{X}$について, 
\begin{equation}
\label{eq-hermite}
\widetilde{h}(u, v):=\overline{h(\overline{u}, \overline{v})}
\end{equation}
\end{itemize}
簡単な計算から
$h_{i \overline{j}}={g'}_{ij} + \sqrt{-1}{g''}_{ij}$とすると
$$
{g'}_{ij} = \widetilde{h}\left( \frac{\partial}{\partial x_i}, \frac{\partial}{\partial x_j}\right) = \widetilde{h}\left( \frac{\partial}{\partial y_i}, \frac{\partial}{\partial y_j}\right)
\quad
{g''}_{ij} = \widetilde{h}\left( \frac{\partial}{\partial x_i}, \frac{\partial}{\partial y_j}\right)=-\widetilde{h}\left( \frac{\partial}{\partial x_j}, \frac{\partial}{\partial y_i}\right)
$$
となる. 
特に$g$はRiemann 計量で$g(X, Y)=g(JX, JY)$ ($X, Y \in T_{\R}X$)となるものを誘導する. 
逆にこのようなRiemann 計量からエルミート計量$h$を誘導できる

さて$\overline{X} :=(X_\R, -J)$においてエルミート計量は\ref{eq-hermite}より
\begin{equation}
\label{eq-opher}
\widetilde{h}\left( \frac{\partial}{\partial \overline{z_i}}, \frac{\partial}{\partial \overline{z_j}}\right)
=\overline{h_{i \overline{j}}} (= h_{j \overline{i}})
\end{equation}
で与えられる. 以下これを$\overline{h}$とかく.
$h$の基本2次形式$\omega_h$は
$$
\omega_h := \sqrt{-1} \sum_{i, j} h_{i \overline{j}} dz_i \wedge d \overline{z_j}
$$
で与えられるので, $\overline{h}$の基本2次形式$\omega_{\overline{h}}$は
$$
\omega_{\overline{h}}
:= \sqrt{-1} \sum_{i, j} \overline{h_{i \overline{j}}} d \overline{z_i} \wedge dz_j
\underset{\ref{eq-opher}}{=}  - \sqrt{-1} \sum_{i, j}  h_{j \overline{i}} dz_j \wedge d \overline{z_i} 
= - \omega_h
$$
である. 特に$\omega_h$がK\"ahlerならば, $\omega_{\overline{h}}$もK\"ahlerとなる. 

また$X=(X_\R, J)$の向きを$\omega_{h}^n$で入れ, $\overline{X}=(X_\R, -J)$の向きを$\omega_{\overline{h}}$で入れる.\footnote{$2n$次元実多様体$X_\R$に向きを与えるとは, どこでも消えない$2n$-formを与えることと同じ.}
すると$\omega_{h}^n= (-1)^n \omega_{\overline{h}}$なので, $(-1)^n$のambiguityが出る. 
つまり任意の$X_\R$上の$C^{\infty}$級$2n$-form $\alpha$について, 
$$
\int_{X}\alpha = (-1)^{n} \int_{\overline{X}}\alpha 
$$
となる. 
以上をまとめると次のとおり.
\begin{lem}
\label{lem-opposite-Kahler}
$X$を$n$次元複素多様体, $J$を$X$に対応する可積分な概複素構造, 
$X_\R$を$X$の$2n$次元実多様体構造, 
$\overline{X}=(X_\R, -J)$をopposite complex structureとする. 

$h$を$X$のエルミート計量, $\omega_h$を$h$の基本2次形式とするとき次が成り立つ. 
\begin{itemize}
\item $\overline{h}$で$\overline{X}$のエルミート計量が与えられる.
\item $\omega_{\overline{h}}=-\omega_h$. 特に$h$がケーラー計量ならば, $\overline{h}$もそう. 
\item $\omega_{\overline{h}}^n$で$\overline{X}$に向きを入れると, $X$の向きと$(-1)^n$の違いが出る. 
特に任意の$X_\R$(=$X, \overline{X}$)上の$C^{\infty}$級$2n$-form $\alpha$について, 
$$
\int_{X}\alpha = (-1)^{n} \int_{\overline{X}}\alpha 
$$
\end{itemize}

\end{lem}



%%%%%%%%%%%%%%%%
\begin{comment}

$(z^1, \ldots, z^n)$を$(X, J)$の複素座標とするとき, $(\overline{z}^1, \ldots, \overline{z}^n)$は$(X, \overline{J})$の複素座標となる. 

$$
\C\mathbb{P}^n \to \overline{\C\mathbb{P}^n}
\quad
[z^0: \cdots :z^{n}] \mapsto [z^0: \cdots :z^{n}] 
$$
と定義するとこれは正則になる. 特に$X$ projectiveならば, $\overline{X}$もそうなる. 

$X$をK\"ahler 多様体, $g$をK\"ahler 計量(リーマン計量)\footnote{$g$はHermitian metricなので$g(Ju, Jv)=g(u,v)$である}, 
$\omega$をK\"ahler formとする. 
すると
$$
\omega(u, v)=g(Ju, v)
$$
となる. よって$-\omega$によって$\overline{X}$もK\"ahlerになる. 

$X$は0点を持たない実$2n$形式$\omega^{n}$によって向きづけられる. 
すると$\overline{X}$も0点を持たない実$2n$形式$(-\omega)^{n}$によって向きづけられる. 
今
$$
id_{X_0}: X=(X_0, J, \omega) \to \overline{X}=(X_0, -J, -\omega)
$$
は$n$が偶数のときは向きは不変だが, $n$が奇数のときは向きが反対になる.

\end{comment}
%%%%%%%%%%%%%%%%

\subsubsection{opposite structure and meromorphic map}
以下の補題に関しても$X$の条件がなかったが, 多分K\"ahlerの仮定は必要だと思う. 
\underline{下線}がついているところは私が付け加えた仮定である.\footnote{証明を見ると必要な仮定などが欠如していた. また色々と誤植がひどかった. 例として$\omega_{\overline{X}}$が何か明記がなかった. }
\begin{lem}\cite[Lemma 4.2, Corollary 4.3]{CW15}
\label{lem-CW15-4.2}
$X$をコンパクト$n$次元K\"ahler多様体, $\overline{X}$をそのoppositeとする.
\begin{itemize}
\item $\zeta : \overline{X} \to X$ $C^{\infty}$級写像.
\item $c : X \dashrightarrow Y$ を複素多様体$Y$へのmeromorphic map.
\item $\varphi:=c \circ \zeta : \overline{X} \dashrightarrow Y$
\item $\alpha$: $Y$上の$d$-closed $C^{\infty}$級 $2d$-form.
\item $\omega_{X}$: $X$上の\underline{K\"ahler form}. 

\end{itemize}
$\zeta \underset{\text{homotopic}}{\sim} id_{X}$であるならば, 
\underline{$\omega_{\overline{X}}:=- \omega_{X}$}として, 
$$
I':=\int_{\overline{X}} \omega_{\overline{X}} \wedge \varphi^{*}\alpha
=(-1)^d \int_{X}\omega_{X}^{n-d} \wedge c^{*}\alpha
=:(-1)^{d} I
$$
%$$
%I':=\int_{\overline{X}}\zeta^{*}(\omega_{X}^{n-d} \wedge c^{*}\alpha)
%=(-1)^d \int_{X}\omega_{X}^{n-d} \wedge c^{*}\alpha
%$$
となる. ($c^{*}\alpha$は\ref{lem-CW15-4.1}の定義)

さらに$\dim Y >0$かつ$c : X \dashrightarrow Y$が非退化(dominant)ならば, 
$\varphi:=c \circ \zeta : \overline{X} \dashrightarrow Y$はmeromorphicではない.
\end{lem}

簡単な例は$\zeta = id_{X} : \overline{X} \to X, c=id_{X} : X \to X$である. 
明らかに$\varphi:=c \circ \zeta = id_{X} : \overline{X} \to X$は正則ではない.

\begin{proof}
$\omega_{\overline{X}}:=- \omega_{X}$とする. 
\ref{lem-opposite-Kahler}より$(\overline{X}, \omega_{\overline{X}})$はK\"ahler多様体で, 向きづけは$(-1)^n$の違いが出る. 

$\zeta$と$id_{X}$はhomotopicであるので
$$
\zeta^{*}=id_{X}
:H^{2(n-d)}(X) \to H^{2(n-d)}(\overline{X})
$$
%\xr{ここにdeRham使っているのでFujikiが必要.}
となる. 
よって
\begin{align*}
I=\int_{X}\omega_{X}^{n-d} \wedge c^{*}\alpha
&\underset{\text{pullback}}{=}\int_{X_{\R}}\zeta^{*}\omega_{X}^{n-d} \wedge c^{*}\alpha \\
&\underset{\text{orientation}}{=}(-1)^{n}\int_{\overline{X}}id_{X}^{*}\omega_{X}^{n-d} \wedge c^{*}\alpha \\
&\underset{\text{$\omega_{\overline{X}}$の定義}}{=}(-1)^{n+n-d}\int_{\overline{X}} \omega_{\overline{X}}^{n-d} \wedge c^{*}\alpha \\
&= (-1)^{d} I'
\end{align*}
となりいえる.

最後の主張に関して, $Y$をK\"ahlerと仮定して良い\footnote{Varochrosの定理から$X \to Y$がflatかつ$X$がK\"ahlerなら$Y$もK\"ahlerになる. よってflatteningをとると$Y$のbimeromorphic modelがK\"ahlerになる. よって$Y$をK\"ahlerとして良い. }
$\alpha=\omega_Y$をK\"ahler formとすると
$$I:=\int_{X}\omega_{X}^{n-1} \wedge c^{*}\omega_Y \underset{\text{非退化}}{>}0$$である. 
一方, もしmeromorphicならば, 
$I'=\int_{\overline{X}} \omega_{\overline{X}} \wedge \varphi^{*}\omega_Y \ge 0$でもある. これは$I=(-1)I'$に矛盾する.
\end{proof}

\subsection{5章 $h$-principle and Brody-hyperbolicity}

\begin{prop}\cite[Proposition 5.1]{CW15}
\label{prop-CW15-5.1}
$n$次元球面$S^{n}:= \{ (x_0, \ldots, x_{n}) \in \R^{n+1} \mid \sum_{i=0}^{n} x_{i}^{2}=1\}$は$n$次元affine quadric
$$
Q_n :=  \{ (z_0, \ldots, z_{n}) \in \C^{n+1} \mid \sum_{i=0}^{n} z_{i}^{2}=1\}
$$
とhomotopicである. 

また$Q_n$の任意の2点は$\C^{*}$によって結ばれる. 特にKobayashi pseudo-metricは常に0である.
\end{prop}
\begin{proof}
まず$z_{i}= x_i + \sqrt{-1}y_i$とすると$\sum_{i=0}^{n} z_{i}^{2}=1$という条件は
\begin{equation}
\label{eq-prop-CW15-5.1}
\sum_{i=0}^{n} x_{i}^{2} - y_{i}^{2}=1
\quad \text{and} \quad
\sum_{i=0}^{n} x_{i} \cdot y_{i}=0
\end{equation}
となる. 
今$\R^{n+1}$について, 
$q(x, y):=\sum_{i=0}^{n}x_i \cdot y_i$とする. 
$$
\begin{matrix}
\rho: & Q_n &\to &N_n:=\{(x, y) \in S^{n} \times \R^{n+1} \mid q(x, y)=0 \} \\
&z=x + iy = (x_0 + i y_0, \cdots, x_n + i y_n)&
\mapsto&
\frac{1}{\sqrt{1 + q(y, y)}}(x, y)
=
\left( \frac{(x_0, \ldots, x_{n} ) }{\sqrt{1 + q(y, y)}}, 
\frac{(y_0, \ldots, y_{n} )}{\sqrt{1 + q(y, y)}} \right)\\
\end{matrix}
$$
とする. \ref{eq-prop-CW15-5.1}よりWell definedで$\frac{x}{\sqrt{1 + q(y, y)}} \in S^{n}$となる. 

$D^{n+1}= \{ y \in \R^{n+1} \mid q(y,y)<1\}$とすると. 
$\rho(Q_n)=S^{n} \times D^{n+1}$であり
$\rho : Q_n \to S^{n} \times D^{n+1}$は連続全単射である. そして逆写像は
$$
\begin{matrix}
\rho^{-1}: & S^{n} \times D^{n+1} &\to &Q_n\\
&(s, t)&
\mapsto&
\frac{1}{\sqrt{1 -  q(t, t)}}(s, t)
=
\left( \frac{(s_0, \ldots, s_{n} ) }{\sqrt{1 - q(t, t)}}, 
\frac{(t_0, \ldots, t_{n} )}{\sqrt{1 - q(t, t)}} \right)\\
\end{matrix}
$$
となる. よって$Q_n$と$S^{n} \times D^{n+1}$が同相になる.
よって$Q_n$と$S^{n}$はhomotopic.

$Q_n$が$\C^{*}$-connectedを示す.\footnote{CW15の説明はよくわからんかったので別証}
$$
M=\{ (z_0 :  \cdot: z_{n} : z_{n+1})  \in \C\mathbb{P}^{n+1} \mid \sum_{i=0}^{n}z_{i}^{2} =z_{n+1}^{2}\}
=
\{ (z_0 :  \cdot: z_{n} : z_{n+1})  \in \C\mathbb{P}^{n+1} \mid \sum_{i=0}^{n}z_{i}^{2}  + (\sqrt{-1}z_{n+1})^{2}=0\}
$$
とおく. 
$Q_{n}=M\setminus \{ z_{n+1} =0\}$である.
%また$x, y \in M$は何本かの$\C\mathbb{P}^{n+1}$内の有理直線で結べる.これは
%$$\C\mathbb{P}^1 \to \C\mathbb{P}^2\quad[a:b] \mapsto [\frac{a + b}{2} : \frac{a -  b}{2 \sqrt{-1}} : ab]$$によって$[1:0:0]$と$[0:1:0]$
%$M \subset \C\mathbb{P}^{n+1} $はsmoothな2次の超曲面である

今$x, y \in M$とする. すると$x, y \in H \cong \C\mathbb{P}^3 \subset \C\mathbb{P}^{n+1}$で
$M \cap H \subset \C\mathbb{P}^{3}$が2次曲面になるものが存在する. ($\sum_{i=0}^{n}z_{i}^{2}  + (\sqrt{-1}z_{n+1})^{2}=0$という2次形式が退化しないようにとる)
よって$\C\mathbb{P}^{3}$内の2次曲面$M \cap H$について, 
$$M \cap H =\{ z_{0}^2 + z_{1}^2 + z_{2}^2 + z_{3}^2=0 \} \cong \mathbb{P}^1 \times \mathbb{P}^1
$$であるので\footnote{セグレ埋め込み$[a: b], [c,d] \mapsto[ac:ad:bc:bd]$で同型になる}, これから$x, y \in M$は次数 2 の滑らかな有理曲線で結ぶことができる.

以上より$x, y \in Q_{n}=M\setminus \{ z_{n+1} =0\}$について, ある次数 2 の滑らかな有理曲線 $C$があって$x, y \in C$となる. 
$C \cap \{ z_{n+1} =0\}$は交点数を計算すると2となるので, 
$x, y$は$\mathbb{P}^1$から2点引いた$\C^*$で結ぶことができる. 

%ある正則写像$f :  \C\mathbb{P}^{1} \to M$で$x,y \in f(\C\mathbb{P}^{1})$となる. $f(\C\mathbb{P}^{1})$は$\{ z_{n+1} =0\}$と2点交差する. \xr{これ証明を埋める}よって$f' :  \C^{*} \to M\setminus \{ z_{n+1} =0\}=Q_{n}$という正則写像を得る. 

Kobayashi-pseudo metricが0に関しては, $x, y \in Q_n$について, $f : \C^{*}\to Q_n$で$f(0)=x, f(1)=y$をとると
$$
d_{\mathrm{Kob},Q_n}(x, y)
= d_{\mathrm{Kob},Q_n}(f(0), f(1)) 
\underset{\text{Prop. \ref{prop-Kob98-Ch-3}}}{\le} d_{\mathrm{Kob}, C^{*}}(0, 1) 
\underset{d_{\mathrm{Kob},\C^{*}} \equiv 0}{=}0 
$$
となりいえる. 
\end{proof}

% [page 3]


\begin{ques}\cite[Question 5.2]{CW15}
\label{ques-CW15-5.2}
$Z$を連結実多様体もしくは有限次元\( CW \)複体とする. 
このとき「$Z$とhomotopicで, Kobayashi擬距離が0となるStein多様体 \( S \)の存在」を特徴づけるtopological obstructionはあるか?
\end{ques}
例えば, 実次元2のときは種数$g$がそれに当たる. 
なおKobayashi擬距離の部分の仮定を外せば常に正しい

\begin{prop}\cite[Proposition 5.3]{CW15}
\label{prop-CW15-5.3}
$Z$ 連結実多様体もしくは有限次元\( CW \)複体とする. 
$Z$とhomotopicなStein多様体 \( S \)が存在する.
\end{prop}

GrauertによるSteinの特徴付けである.( \cite[Corollary 3.5.3]{For18}参照).
Eliashbergの定理からもわかる.
\xr{ただこの証明壊滅的にわからなかったので, 教えてください. なおこの\ref{prop-CW15-5.3}はこの後全く使わないです.}
\begin{proof}
\( Z \) が\( CW \)-複体の場合は連結実多様体の場合に帰着できる. 
というのも \( Z \) が\( CW \)-複体ならば埋め込み\( Z \hookrightarrow \R^n \)が存在する.
 よって, ある$Z$の開近傍$U \subset \R^n$があって, 
  \( Z \)は$U$とhomotopicである. \xr{これは有名な事実??}
$U$は連結実多様体より良い.

以下$Z$が連結実多様体のときを考える. 
 \( \R^3 \)と直積を取ることで, \(\dim_\R(Z) > 2\)を仮定して良い. 
$M:=T^{*}_{\R}Z$を$Z$の余接束とする. 
\( M \)はシンプレクティック多様体(2-form$\omega$で$\omega^{\dim Z}$が非退化なもの)が存在する. 
よってalmost complex structure$J_{\omega}$が存在する.\xr{???}

 \( M = T^*Z \) の計量$h$と $Z$上のexhaustive Morse 関数\( \rho \) をとる. 
すると \( p(v) = \rho(\pi(v)) + h(v) \) が$M$上のexhaustive Morse 関数となる. 
\( p \) のcritical pointは全て$M:=T^{*}_{\R}Z \to Z$の0sectionにあり, それは
$\rho$のcritical pointと一致する. \xr{???}

よって,  critical pointでindex \(>\dim(Z) = \tfrac{1}{2}\dim(M)\)となるものは存在しない. 
下のEliashberg の定理(下の\cite[Theorem 1.3.1]{Eli90})から$M$はStein 多様体になる. \xr{???}
そして$M:=T^{*}_{\R}Z  \to Z$はhomotopy 同値である. 

\begin{thm}\cite[Theorem 1.3.1]{Eli90}
Let \( X \) be a \( 2n \)-dimensional smooth manifold with an almost complex structure \( J \).
Let \( \varphi : M \to \R \) be a proper Morse function such that indexes of all its critical points are \( \le n \).
Suppose that \( n > 2 \).
Then \( J \) is homotopic to a genuine complex structure \( \widetilde{J} \) such that \( \varphi \) is \( \widetilde{J} \)-convex.
In particular, the complex manifold \( (M, \widetilde{J}) \) is Stein.
\end{thm}
\end{proof}



\begin{thm}\cite[Theorem 5.4]{CW15}
\label{thm-CW15-5.4}
\( X \) を\( h \)-principleを満たす解析空間とする. 
このとき \( X \) はhomotopically \(\C\)-connected.
\end{thm}
\begin{proof}

%%%%%%%%%%%%
\begin{comment}


まず家庭とゴールは次のとおり.
\begin{itemize}
\item[仮定 \( hP(X) \)] 任意のStein manifold$S$, 任意の連続関数$f : S \to X$について, ある正則関数$F : S \to X$があって, $F$と$f$はhomotopicである. 
\item[示すこと.]  任意のunramified covering $X' \to X$と, Brody hyperbolic解析空間$Y$への正則写像$\varphi : X' \to Y$について, ホモトピー群の写像
$$
\pi_{k}(\varphi) : \pi_{k}(X') \to \pi_{k}(Y)
$$
が$k>0$について0写像になること. 
\end{itemize}
\end{comment}
%%%%%%%%%%%

背理法で示す. あるunramified covering $X' \to X$と, Brody hyperbolic解析空間$Y$への正則写像$\varphi : X' \to Y$で, $\pi_{k}(\varphi) : \pi_{k}(X') \to \pi_{k}(Y)$
が$k>0$について0写像でないとする.
\ref{lem-CW15-6.6}より\( X'\) も$h$-principleを満たすことに注意する. 
%Then there exists a holomorphic map \( g : Y \to X \) with \( Y \) Brody-hyperbolicand such that there exists a non-zero induced morphism on ホモトピー群s\( \pi_k(g) : \pi_k(X) \to \pi_k(Y) \) with \( k > 0 \).

背理法の仮定から \( f : S^k \to X \) を連続写像で
$$
[f] \neq 0 \in \pi_k(X)
\quad
\pi_{k}(\varphi)([f]) \neq 0 \in \pi_{k}(Y)
$$
となるものがある. 
\( Q_k \)を$k$次元のaffine quadricとすると, 
\ref{prop-CW15-5.1}より, 連続写像 \( \rho : Q_k \to S^k \)でホモトピー同値を与えるものがある. 

よって$Q_k$はSteinより, $h$-principleから, 正則写像$\widetilde{f} : Q_n \to X$で
$$
\widetilde{f} \sim  f \circ \rho : Q_k \to X
$$
となるものが存在する. 
よって以下の図式を得る. 
\[
\xymatrix@C=70pt@R=40pt{
\pi_{k}(S_k)=\Z\ar@{->}[rd]_{\pi_{k}(f) }  \ar@{->}[rrd]^{\neq 0}  & & \\
\pi_{k}(Q_k)=\Z\ar@{->}[u]^{\pi_{k}( \rho)  \text{isom}} \ar@{->}[r]_{\pi_{k}( f \circ \rho) = \pi_{k}(\widetilde{f}) } &\pi_{k}(X)  \ar@{->}[r]_{\pi_{k}(\varphi)}& \pi_{k}(Y)
}
\]
よって$\pi_{k}(\varphi \circ \widetilde{f}) : \pi_{k}(Q_k) \to \pi_{k}(Y)$はnonzero mapである. 
一方で, 正則写像$\varphi \circ \widetilde{f} : Q_k \to Y$は定数(後述)である. よって$\pi_{k}(\varphi \circ \widetilde{f})$はzeromapになり矛盾.

[正則写像$\varphi \circ \widetilde{f} : Q_k \to Y$が定数について]
$x, x' \in Q_k$について, \ref{prop-CW15-5.1}より$h : \C^{*} \to Q_k$で$h(a)=x, h(a')=x'$が存在する. 全射正則写像$\mathrm{exp} : \C \to \C^{*}$, $z \mapsto e^z$をとれば$\varphi \circ h \circ \mathrm{exp} : \C \to Y$が取れる. $Y$がBrody hyperbolicなので定数となり$\varphi (x)=\varphi (x')$.
\end{proof}

\begin{cor}\cite[Corollary 5.5]{CW15}
\label{cor-CW15-5.5}
\( X \)をBrody-hyperbolic複素多様体とする. 
\( X \) が \( h \)-principleを満たすことと, $X$が可縮であることは同値である. 
\end{cor}
\begin{proof}
可縮ならば\( h \)-principleは\ref{rem-CW15-1.9}から.
%%%%%%%%%%%%%
\begin{comment}
[一応証明]
$X$が一点集合ならば自明. 一般の場合は$x_0 \in X$と連続写像
$$
F : X \times [0,1] \to X
\quad 
F(x, 0)=x
\quad 
F(x, 1)=x_0
$$
が存在する. 
よって(Stein)manifold$S$と連続関数$f : S \to X$について, 
$$
F':=F \circ (f \times id_{[0,1]}): S \times [0,1] \to X
$$
を考えると
$F'(s, 0)=f(s)$, $F'(s, 1)=x_0$
で$f$は定数写像とホモトピー同値である. 定数写像は正則なのでOK.
\end{comment}
%%%%%%%%%%%%%%%%%%%%%
$X$が$h$-principleを満たすとする. 
\ref{thm-CW15-5.4}より, homotopically \(\C\)-connected. よって$X$がBrody hyperbolicなので
任意の$k \in \N$で$\pi_k(\mathrm{id}_X) : \pi_k(X) \to \pi_k(X)$はzero mapである. 
特に任意の$k \in \N$で\(\pi_k(X) = 0\). 以上よりWhiteheadの定理(下参照)より\( X \)は可縮である. 
\end{proof}
%最後はWhiteheadの定理である. 
\begin{thm}[Whitehead]
\label{thm-Whitehead}
CW複体の間の連続写像$f : X \to Y$について
$\pi_{k}$が任意の$k \in \N$で同型(弱ホモトピー同値)ならば, $X$と$Y$はホモトピー同値(ホモトピック)である. 
\end{thm}



\begin{cor}\cite[Corollary 5.6]{CW15}
\label{cor-CW15-5.6}
コンンパクトなBrody hyperbolic 多様体は\( h \)-principleを満たさない.
\end{cor}
特に種数2以上のコンパクトリーマン面は$h$-principleを満たさない
\begin{proof}
正の次元のコンパクトな複素多様体は基本類$[X] \in H_{2n}(X, \Z)$が0ではないので, 特に可縮でない. 
よって\ref{cor-CW15-5.5}より. 
\end{proof}

\begin{rem}
\label{rem-CW15-5.7}
"homotopically \(\C\)-connected"
から"Brody hyperbolic Kahler 多様体への写像が定数になる"は従わない. 
(Projectiveなら\ref{thm-CW15-5.8}から正しい.)

%\( f : X \to Y \) をsmooth projective manifoldの間の正則写像 で定数写像と homotopicでないものとする. その場合でもhomological degree($H_n(f)$の次数?)が0でなくても. 任意の$k \in \N$について, \(\pi_k(f) = 0\) となりうる. 

$C$を超楕円曲線とする. 
$C$は種数2以上で次数2のfinite map $C \to \mathbb{P}^1$を持つ.
$$
\pi_2(C)=\pi_1(\mathbb{P}^1)=0
$$
であるから常に \(\pi_k(f) = 0\)である. 
一方で$f$は定数ではない. 
つまり
\begin{center}
\(\pi_k(f) = 0\) でも$f$が定数にならないことはありうる. 
\end{center}
なおこの例では, $H_2(f) \neq 0$でもある. 
%よって「 homotopically \(\C\)-connected $\Rightarrow$ 任意のBrody-hyperbolic解析空間へのが定数写像とホモトピック」は自明ではない. 
%なおProjectiveなら\ref{thm-CW15-5.8}から正しい. 
\end{rem}
ちなみに$\pi_2(C)=0$に関しては次の命題からわかる.($C$の普遍被覆は可縮である)
\begin{prop}\cite[Proposition 4.1]{Hat02}
$\widetilde{X} \to X$を位相空間のcovering spaceとする.
2以上の整数$n$について$\pi_n(\widetilde{X}) \cong \pi_{n}(X)$.
\end{prop}

\begin{thm}\cite[Theorem 5.8]{CW15}
\label{thm-CW15-5.8}
\( X \) 既約projective varietyで\( h \)-principleを満たすとする. 
\( f : X \dashrightarrow Y \)をBrody hyperbolic Kähler 多様体\( Y \)へのmeromorphic mapとする. 

\( f \) が正則, または\( X \)がsmooth(KLTでも可)ならば,  \( f \) は定数である.
\end{thm}

\begin{proof}
次の主張により$f : X \to Y$は正則写像であると仮定して良い. 
\begin{claim}
$X$がsmooth(KLTでも可)のときは, meromorphic map \( f : X \dashrightarrow Y \)は常に正則である. 
\end{claim}
\begin{proof}[Claimの証明]
\(\pi : \widetilde{X} \to X \)を$X$をresolutionしてから, $f$の不確定点解消したものとする.  $\widetilde{f} : \widetilde{X} \to Y$を$f$から誘導された射とする. 

$x \in X$を$f$の不確定点とすると, $X$はKLTなので Hacon-Mckernanの定理\cite{HM07}から\footnote{$X$がsmoothなら, 不確定点解消はsmooth centersのblowing upでできる.}
$\pi^{-1}(x)$はrationally connectedである. 
一方, $Y$はBrody-hyperbolicなので特にrational curveを含まない. よって$\widetilde{f}(\pi^{-1}(x))$は一点である. よって\cite[Lemma 1.15]{Deb01}のrigidity lemmaから, \( f \)は正則である. 
\end{proof}


%以下 \( f  : X \to Y\)を正則として良い.
\( f \) を定数でないと仮定して矛盾を示す. 
\underline{\( X \) はprojectiveなので}
代数曲線\( C  \subset  X \)で\( f|_C \) が定数でないものを取る. 

$J_C$を$C$の複素構造として次のものを取る.
\begin{itemize}
\item \( \overline{C} := (C, -J_C)\) $(C, J_C)$のopposite  complex structure (\ref{defn-CW15-opposite}参照)
\item \( j : \overline{C} \to C \) "集合"としての恒等写像. 
\item \( \tau : E \to \overline{C} \) affine \( \C \)-bundle (\ref{prop-CW15-3.1}参照)
\end{itemize}
すると\( h \)-principleから, 連続写像\( j \circ \tau : E \to X \)は, 正則写像 \( h : E \to X \)とホモトピックである. 

\[
\xymatrix@C=50pt@R=20pt{
E \ar@{->}[d]_{\tau \text{ $\C$bundle}} \ar@{->}[rd]^{ j \circ \tau } \ar@/^2pc/@{->}[rd]^{\exists h} & &  \\
\overline{C} \ar@{->}[r]_{j}\ar@/_2pc/@{-->}[rr]^{\exists  \varphi}&X \ar@{->}[r]^{f}  & Y
}
\]


任意の$c \in \overline{C}$について, $\tau^{-1}(C) \cong \C$より, \( Y \) はBrody-hyperbolicなので, $f \circ h( \tau^{-1}(C))=pt$となる.
よって正則写像$\varphi : \overline{C} \to Y$で, 
 \( f \circ h = \varphi \circ \tau \) となるものが存在する. 
 
$\tau : E \to \overline{C}$はホモトピー同値を与えるので,
 \( \varphi \sim  f \circ j \)である.
 特に
 $$
 \varphi^{*}=(f \circ j )^{*} :
 H^2(Y, \R) \to H^2 (\overline{C}, \R) \underset{\int_{\overline{C}} \cdot}{\cong} \R
 $$
 である.  今$Y$上のK\"ahler formを$\omega$とする.
$\varphi : \overline{C} \to Y$はmeromorphicなので.
 $$
  \varphi^{*}\omega \underset{\text{$\int_{\overline{C}} \cdot $による同一視}}{\ge} 
0
 $$
 一方で, \ref{lem-CW15-4.2}より
  $$
 (f \circ j )^{*}\omega \underset{\text{$\int_{\overline{C}} \cdot $による同一視}}{=} 
\int_{\overline{C}} f^{*}\omega \underset{\ref{lem-CW15-4.2} \& \omega \text{2-form}}{=}
= - \int_{C} f^{*}\omega 
\underset{\text{\( f|_C \) が定数でない}}{<}0.
 $$
これは$ \varphi^{*}=(f \circ j )^{*}$に矛盾である.
\end{proof}



\subsection{6章 \emph{h}-principle implies specialness for projective manifolds}

\begin{thm}\cite[Theorem 6.1]{CW15}
\label{thm-CW15-6.1}
 \(X\) projective manifold.
 \(X\) が\(h\)-principleを満たすならば,  \(X\) はspecial.
\end{thm}
\begin{proof} 
 \(\bar X \)を$X$のopposite complex structureとし, 
 \(\iota\colon \bar X \to X\) を集合としての恒等写像とする.
 %$\iota$は$X$を実$2n$次元の多様体と見た場合は$C^{\infty}$級である. 
 \ref{prop-CW15-3.1}より\underline{$\bar X$ projectiveなので}, あるaffine多様体$M$と正則全射$\tau : M \to \bar X$がある. 
 %($M \to X$は反正則である)
% \(\bar X  \)もまたprojectiveなので, \ref{prop-CW15-3.1} is also projective. Hence we can find a Stein manifold \(M\) together with a holomorphic fibre bundle \(\tau\colon M \to \bar X\) with some \(\C^k\) as fibre (Proposition~3.1).
%\ar@/_2pc/@{-->}[rr]^{\varphi}
\[
\xymatrix@C=70pt@R=30pt{
M \ar@{->}[d]^{\tau \text{ 正則 }} \ar@{->}[rd]^{h \text{ 正則 }}  \ar@/^2pc/@{-->}[rrd]^{g=c_X \circ h  \text{ 有理  }}& &  \\
\overline{X} \ar@/^2pc/@{->}[u]^{\sigma \text{ $C^{\infty}$ section }} \ar@{->}[r]^{\iota \text{ $C^{\infty}$ }} \ar@/_2pc/@{->}[r]_ {\zeta =h \circ \sigma \text{ $C^{\infty}$ }} &X \ar@{-->}[r]^{c_X \text{ 有理 }}  & C(X)
}
\]
$M$はSteinなので, $\iota \circ \tau : M \to X$に\(h\)-principleを使って, ある正則写像 \(h\colon M \to X\)で,  \(\iota \circ \tau\)とhomotopicなものが存在する. 

\ref{prop-CW15-3.1}より, $C^{\infty}$級section \(\sigma\colon \bar X \to M\) を取り, 
 \(\zeta := h \circ \sigma\colon \bar X \to X\)とおくと
 $$
\zeta := h \circ \sigma \underset{\text{homotopic}}{\sim} \iota \circ \tau \circ \sigma \underset{\text{$\sigma$ section}}{=}\iota \circ id_{\bar{X}} = \iota
 $$
 よって$\zeta$は$\iota=id_{X} : \bar{X} \to X$とhomotopicである. 

今 \(c_X \colon X \dashrightarrow C(X)\)を$X$のcoremapとする.\footnote{ここに$X$が多様体であることを使っている. } 
以下$\dim C(X)=0$(つまり$X$がspecial)であることを背理法で示す. 
 \(d := \dim(C) > 0, n := \dim(X), g := c \circ h\colon M \dashrightarrow C(X)\).とおく. 
 %We claim that \(c \circ \zeta\colon \bar X \dashrightarrow C\) is non-degenerate, and thus, that so is \
\(\omega_C\) , \(\omega_X\) を\(C\) , \(X\)のK\"ahler formとして, 
\[
I := \int_X \omega_X^{\,n-d} \wedge c^*(\omega_C^{\,d}) > 0.
\]
とおく. 
\ref{lem-CW15-4.2}より, $\omega_{\bar X}:=-\omega_X$とすると
\[
I' := \int_{\bar X}  \omega_{\bar X}^{\,n-d} \wedge (c_X \circ \zeta)^*(\omega_C^{\,d})\bigr)
   \underset{\ref{lem-CW15-4.2}}{=}(-1)^d \cdot I \neq 0.
\]
特に \((c \circ \zeta)^*(\omega_C^{\,d}) \in H^{d, d}(\bar{X})\)は0ではない. 
\ref{lem-CW15-4.1}より, \(c \circ \zeta : \bar{X} \dashrightarrow C(X) \)は非退化である. 
%non-degenerate(ある点$x \in \bar{X}$でsubmertive)である.(これは定義されている部分でsmoothになるようにして, 座標計算にすれば良い.) 
よって
 $$
g \underset{\text{def}}{:=}  
 c_X  \circ h 
 \underset{\text{$\sigma$ section}}{=}
  c_{X} \circ ( h \circ \sigma) \circ \tau
  \underset{\zeta := h \circ \sigma}{=}  
  \underbrace{c \circ \zeta}_{\text{non-degenerate}} \circ \underbrace{\tau}_{\text{fiber bundle}}
 $$
 より$g=c_X  \circ h : M \dashrightarrow C(X)$も非退化である. 
%is not of measure zero. By Sard's theorem, this implies that \(c \circ \zeta\) is non-degenerate, and therefore so is \(c \circ h\).
%By Theorem~2.10, it follows that 
% [page 2]

\ref{thm-CW15-2.10}から, meromorphic map \(\varphi\colon \bar X \dashrightarrow C\) で 
\(\varphi \circ \tau = g = c_X\circ h\)となるものがある. 
\[
\xymatrix@C=50pt@R=20pt{
M \ar@{->}[r]^{\tau \text{ 正則 } }\ar@/^2pc/@{->}[rr]^{ h \text{ 正則 } } &\bar{X}\ar@{-->}[d]_{c_{\bar{X}}} \ar@{-->}[rd]^{\exists \varphi  \text{ 有理 }} & X\ar@{-->}[d]^{c_X } \\
&C(\bar{X}) \ar@{-->}[r]^{\exists c_{\varphi} \text{ 有理 }}  & C(X)
}
\]
すると
\[
\varphi 
= \varphi \circ \underbrace{\tau \circ \sigma}_{=id_{\bar{X}}} 
\underset{(\varphi \circ \tau = c_X\circ h)}{=} c_{X} \circ \underbrace{h \circ \sigma }_{=: \zeta} 
= c_{X} \circ \zeta
\]
よって$c_{X} \circ \zeta : \bar{X} \dashrightarrow C(X)$はmeromorphicである. これは\ref{lem-CW15-4.2}に矛盾する.
%We now consider the integral \(J = \int_{\bar X} \omega_X^{n-1} \wedge c^*(\omega_C)\). We have \(J > 0\), giving a contradiction by Corollary~4.3. Hence \(X\) cannot satisfy the \(h\)-principle unless \(\dim(C)=0\), that is, unless \(X\) is special.
\end{proof}

\begin{conj}\cite[Conjecture 6.2]{CW15}
 \(X\)をprojective manifoldとする.
$X$が \(h\)-principleを満たすならば,  \(\pi_1(X)\) はalmost Abelianか?
\end{conj}
これはCampanaの予想"Specialならば \(\pi_1(X)\) はalmost Abelianか?"の亜種である. 
\(\pi_1(X)\)がlinear($GL$の部分群)かsolvableならば解決済み.\footnote{Arapura-Noriの定理から\(\pi_1(X)\)はvirtually nilpotentになり, Albanese mapが全射ファイバー連結なので\cite[Section 7]{Cam04}の議論から\(\pi_1(X)\)はvirtually abelianになる.}

\begin{conj}\cite[Question 6.3.]{CW15}
\begin{enumerate}
\item compact K\"ahler 多様体において, \(h\)-principleならばspecialか? 
これは曲面では正しい. %This is true, at least, for コンパクト Kähler surfaces (see Theorem~6.4 and its corollary below).
\item \(X\) quasi-projective manifoldとする. 
\(h\)-principleを満たし, \(X\) がいかなるproper subvariety \(Z \subset X\)ともhomotopy同値になり得ないとする. 
このとき\(X\) はspecialか?
\end{enumerate}
\end{conj}


%\begin{thm}\cite[Theorem 6.4]{CW15}
%\label{thm-CW15-6.4}
%compact K\"ahler多様体\(X\) が\(h\)-principleを満たすならば,  Albanese map of \(X\) は全射. 
%\end{thm}
%証明は\ref{thm-CW15-7.1}の証明をなぞる.

\begin{cor}\cite[Corollary 6.5]{CW15}
\label{cor-CW15-6.5}
compact K\"ahler曲面において, \(h\)-principleならばspecial.
\end{cor}
\begin{proof}
$K_X$ bigならばprojectiveになる. この場合はわかっているので$\kappa(K_X) \le 1$と仮定して良い. 

$X$がspecial でないとする. \cite[Proposition 9.29]{Cam04}より, \underline{$X$が曲面なので}
 weakly specialでもない.
 つまり, あるfinite etale cover $X' \to X$と種数2以上のcurve $C$へのsurjective 正則写像 \(f'\colon X' \to C\)が存在する. 
\ref{lem-CW15-6.6}から$X'$も\(h\)-principleを満たすが, これは\ref{cor-CW15-7.3}に矛盾する. 
\end{proof}

\begin{lem}\cite[Lemma 6.6]{CW15}
\label{lem-CW15-6.6}
 \(\pi \colon X' \to X\) を解析空間のunramified coveringとする.   
 \(X\)が\(h\)-principleを満たせば, $X'$も\(h\)-principleを満たす.
\end{lem}
unramified coveringとはおそらくetale (flat かつunramified)のことだと思う. 
また逆は成り立たない(が, finite etaleの場合は不明)
\begin{proof}
%Galois被覆をとって, $\pi$をGaloisとして良い\footnote{$\pi_{*}\pi_{1}(X') \subset \pi_{1}(X)$が正規部分群になるようにしておく. そうでないとhomotopy lifting propertyが使えない}
 \(f\colon S \to X'\)を Stein 空間 \(S\)からの連続写像とする. 
 $\pi \circ f : S \to X$は連続なので, $h$-principleから, ある正則写像\(F\colon S \to X\) で\(\pi \circ f\)と
 homotopicなものがある. 
つまり
$$
H : S \times [0,1] \to X
\quad
H(\cdot, 0)=\pi \circ f 
\quad
H(\cdot, 1)=\pi \circ F
$$
となるものが存在する.
  \[
\xymatrix@C=50pt@R=20pt{
\{0\}\ar@{->}[r]^{f}\ar@{^{(}-_>}[d] &\mathcal{C}(S,X') \ar@{->}[d]^{\pi^{*}}  \\
[0,1]\ar@{->}[r]_{H}\ar@{-->}[ru]^{\exists H'}  & \mathcal{C}(S,X)
}
\]
よってHomotopy Lifting propertyから\footnote{調べた限り被覆空間とかファイバー束ならば満たすらしい. Serre fibrationだから?}
$$
H' : S \times [0,1] \to X
\quad
H'(\cdot, 0)=f 
\quad
\pi \circ H'=H
$$
となるものが存在する. 
よって$F':=H'(\cdot, 1)$とすると$\pi \circ F' = F : S \to X$が正則なので, $F'$も正則.
この$F'$は$f$とhomotopicである
 %\xr{これ本当か?homotopy lifting propertyを調べる. OKa}
 \end{proof}
% homotopy lifting propertyとは被覆写像 \(\pi \colon X' \to X\) 

\begin{rem}
Oka性がunramified coveringで移る理由は, (例えばCAPを考えると)定義に出てくる$S$みたいなものが可縮なので, Liftが常に取れるから. 
$h$-principleで同様にいかないのは"任意の連続写像$S \to X$が$S \to X'$にLiftするかがわからない"部分にある($S$は単連結とは限らない.)
\end{rem}


\subsection{7章 Necessary conditions on the quasi-Albanese map}
\label{subsec-CW15-7}



\subsubsection{コンパクト K\"ahler manifoldの場合}

compact K\"ahlerの場合を考え, quasi-projectiveはどのように補正したらいいかを述べる. 
\begin{thm}\cite[Theorem 6.4]{CW15}
\label{thm-CW15-7.1}
 %\noindent\textbf{Theorem 7.1.}
$X$をcompact K\"ahlerとする.
もしAlbanese map が全射でないならば, \(X\) は \(h\)-principleを満たさない.
\end{thm}


\begin{proof}
\(h\)-principleを満たすとして矛盾を示す.  

Albanese map \(a : X \to A\)とする.
\(Z=\overline{a(X)}^{\mathrm{zar}}\)とする. 
平行移動して, $A$の単位元$e_A$が \(e_A \in Z\)となっているとして良い.

$a$が全射でないので, $A\setminus Z \neq \varnothing$である. 
よって\underline{$(\star 1)$\cite[Theorem 4]{Kaw80}}より, 
 % (see also \cite[Theorem 5.6.19]{NW14} \ref{thm-NW14-5.6.19}参照) 
高々有限個数のsub-semitorus \(T_i \subset A\)と \(T_i\)-orbit \(S_i \subset A\)があって次を満たす.
\begin{enumerate}
\item \(S_i \subset Z\) 
\item 任意の \(A\) のsub-semitorus $W$について, ある$x \in A$で$x + W \subset Z$ならば, \(W \subset S_i\)となる$i$がある.  
\end{enumerate}
 
\ref{lem-CW15-7.2}より, 
\(\gamma_0 \in \pi_1(A)\setminus \bigcup \pi_1(S_i)\)が取れる. 
 Albanese mapの性質から
 $$
 a_{*} : \pi_1(X) \twoheadrightarrow \pi_1(A)
 $$
 なので, $a_{*}(\gamma) = \gamma_0$が取れる. 
 よって$\gamma : S^1 \to X$なる連続写像が存在する. 
 $S^1 \underset{\text{homotopic}}{\sim}\C^{*}$なので, 連続写像$\gamma : \C^{*}\to X$が存在する(記号の濫用だが, これも$\gamma$と表す)
 
 $X$は\(h\)-principleを満たすので, 正則写像
 $$
 f : \C^{*} \to X
 $$
 で $\gamma\underset{\text{homotopic}}{\sim} f$となるものが存在する. 
 これより正則写像
 \[
F = a \circ f \circ \exp : \C  \to \C^{*} \to X \to \rightarrow Z \subset A .
\]
を得る. 
\underline{$(\star 2)$\cite[Theorem 3]{Kaw80}より} 
%Noguchi’s logarithmic version of the theorem of Bloch–Ochiai (\cite[Theorem 4.8.17]{NW14} 下に引用)より. 
analytic Zariski closure \(\overline{F(\C)}^{\mathrm{zar}}\) を\(Z\)上でとると,  
\(\overline{F(\C)}^{\mathrm{zar}}\) は$A$のsub-semitorusの平行移動となる.
 
よって(2)からある$S_i$があって, \(\overline{F(\C)}^{\mathrm{zar}}\subset S_i\)となる.
よって, $a \circ f (\C^{*})\subset S_i$より$\pi_1(\C^*) \cong \Z$の単位元を$1_{\C^{*}}$とすると
\[
a _{*}(\gamma) 
=
(a \circ \gamma)_*(1_{\C^{*}}) 
\underset{\gamma\underset{\text{homotopic}}{\sim} f}{=}(a \circ f)_*(1_{\C^{*}}) 
\underset{a \circ f (\C^{*})\subset S_i}{\in} \pi_1(S_i),
\]
これは$a _{*}(\gamma)=\gamma_0 \in\pi_1(A)\setminus \bigcup \pi_1(S_i)$に矛盾する.
\end{proof}


また証明に使った補題を示しておく. 
\begin{lem}\cite[Lemma 7.2]{CW15}
\label{lem-CW15-7.2}
 \(\Gamma_1, \dots, \Gamma_k\)を\(G = \Z^n\)の部分群とする. 
  \(\mathrm{rank}_\Z \Gamma_i < n\)ならば \(\cup_i \Gamma_i \neq G\)
\end{lem}

$\Gamma_i$は有限生成(ネーター環$\Z$の有限生成加群の部分加群より)なので, 有限生成アーベル群の基本定理より, $\gamma_i \cong \Z^{\mathrm{rank}_\Z \Gamma_i } \oplus \mathrm{(Torsion part)}$となる. 


\begin{proof}
$G =\Z^{r} \hookrightarrow \R^{n}$とユークリッド空間に埋め込む.
$x \in G$について$||x||$をそのユークリッド距離とする. 
部分群$H \subset G $のランクを$d$とする. 
$r \in \N$について
$$
N(H, r):= \{ x \in H \mid || x|| \le r\}
$$
とおくと, $r \gg 0$について$|N(H, r)|=O(r^{d})$である.

以上より, 仮定から\(N(\Gamma_i,r) = O(r^{n-1})\)かつ, \(N(G,r) = O(r^n).\) 
である. 
よって, $r$を十分大きく取れば(torsion partの部分も無視できるくらい大きくすれば), $\gamma \in G \setminus \cup_i \Gamma_i $が存在する. 
\end{proof}


\begin{cor}\cite[Corollary 7.3]{CW15}
\label{cor-CW15-7.3}
\(X\) をcompact K\"ahlerとし, 種数2以上の曲線$C$への全射$F : X \to C$を有するとする. 
このとき \(X\)は\(h\)-principleを満たさない.
\end{cor}
\begin{proof}
 \(A\) (resp. \(J\)) を$X$ (resp. $C$)のAlbanese varietyとする.  
\ref{thm-CW15-7.1}より$X \to A$が全射出ないことを示せば良い.

背理法. $X \to A$が全射とする. Albanese mapの関手性より,  \(F_1 : A \to J\)がある. 
 \[
\xymatrix@C=50pt@R=20pt{
X \ar@{->>}[r]^{a_X} \ar@{->>}[d]_{F} &A \ar@{->>}[d]^{F_1} \\
C\ar@{->}[r]^{a_C} & J
}
\]
すると\(F_1 : A \to J\)は全射となる.
これは$F$は全射なので, $F^{*}  : H^0(C, \Omega_{C}^{1}) \hookrightarrow H^0(X, \Omega_{X}^{1})$は単射である. 
 よって
 $${F^{*}}^{\vee} :
 H^0(X, \Omega_{X}^{1})^{\vee}\twoheadrightarrow  H^0(C, \Omega_{C}^{1})^{\vee}
 $$
 は全射である. Albanese varietyの構成から$J=H^0(C, \Omega_{C}^{1})^{\vee}/H_{1}(C, \Z)$であるので, \(F_1 : A \to J\)の全射性が言える(\ref{prop:A6}参照)

よって$F_1\circ a_X$は全射で, $F_1\circ a_X = a_{C} \circ F$より$a_C$は全射である. 
しかし$C$は種数2以上なので,  \(g=\dim(J) > \dim(C)=1\)であるので矛盾. 
 \end{proof}

%\cite{NWY07}より次が言える

\begin{prop}\cite[Proposition 7.4]{CW15}
\label{prop-CW15-7.4}
\(X\) compact K\"ahler 多様体で, complex torus $A$へのfinite map $f : X \to A$を有するとする. 

\(X\) が\(h\)-principle を満たすならば,  \(X\) はcomplex torusである.
\end{prop}

証明がなかったが多分これで言えると思う. \footnote{\cite{CW15}には\cite{NWY07}を使えば言えると書いていたが, よくわからなかった.}
\begin{proof}
$X \to \mathrm{Alb}(X)$をAlbanese mapとする. 
Albanese map の普遍性から$\mathrm{Alb}(X) \to A$が存在する. 
そのStein 分解を$\mathrm{Alb}(X) \to A' \to A$とする. 
 \[
\xymatrix@C=50pt@R=20pt{
X \ar@{->}[r] \ar@{->>}[d]_{f \text{ with con. fibre}} & \mathrm{Alb}(X) \ar@{->>}[d]^{\text{ with con. fibre}} \\
A& A' \ar@{->}[l]^{\text{ finite }}
}
\]
ここで上に出てくる全ての多様体は次元が同じことに注意する. 
よって$\mathrm{Alb}(X) \to A'$はbimeromorphicである. しかし$\mathrm{Alb}(X)$に有理曲線は存在しないので, $\mathrm{Alb}(X) \cong A'$である. 

よって, $X \to \mathrm{Alb}(X)$はfinite map になる. 一方でファイバー連結なので同型写像になる. 
\end{proof}

\subsubsection{quasi-Albanese の復習}
quasi-Albanese mapに関して\cite{Wang22}のpreprintのAppendixを参照した\footnote{\url{https://arxiv.org/abs/2005.05782} Preprint versionにはAppendixがあってかなりわかりやすい}(Fujino先生の論文でも良い)

\begin{defn}[{\cite{Fuj15}, Definition 2.8}]
%\label{def:A1}
\(G\) を連結な代数群とする.
\[
1 \to H \to G \to A \to 1
\]
を \(G\) の Chevalley decompositionとする. 
ここで \(H\) は 線形代数群 であり,\(A\) は Abelian variety である.
もし \(H \simeq G_m^{\dim H}\)(ここで \(G_m\) は乗法群 \(\C^\ast\) を表す)であれば,\(G\) を \emph{semi-Abelian variety} と呼ぶ.
\end{defn}

semi-Abelian variety のいくつかの初等的性質を次のようにまとめておく:

\begin{prop}[{\cite{Fuj15}, Lemma 2.11, Lemma 2.13}]\label{prop:A2}
\(G\) を semi-Abelian variety とし,
\[
1 \to G_m^{d} \to G \to A \to 1
\]
を \(A\) を Abelian variety とする \(G\) の Chevalley decomposition とする.
このとき次が成り立つ. 
\begin{enumerate}
\item \(G\) は \(A\) 上の主 \(G_m^{d}\)-束である.
\item \(G\) は可換群である.
\item \(G\) の普遍被覆は \(\C^{\dim G}\) であり, \(\pi_1(G)\)を\(\C^{\dim G}\) の格子とみなせば
      \(G \simeq \C^{\dim G}/\pi_1(G)\) となる.
\end{enumerate}
\end{prop}

%Abelian variety(あるいは複素トーラス)の場合と同様に(cf.\ \cite{Uen75}, Lemma 10.1, Theorem 10.3, pp.\ 116--119),semi-Abelian variety の閉部分代数多様体は次の剛性性質をもつ:

\begin{prop}[{\cite{Fuj15}, Theorem 4.4}]\label{prop:A3}
\(G\) を semi-Abelian variety とし,\(W\) を \(G\) の閉部分代数多様体とする.
このとき log Kodaira 次元 \(\tilde{\kappa}(W)\ge 0\) . 

また\(\tilde{\kappa}(W)=0\) となるのは, \(W\) が \(G\) の semi-Abelian subvariety の平行移動である場合に限る.
\end{prop}

% logarithmic Kodaira dimension は次のとおり, 

%\begin{defn}\label{def:A4}
%\(V^\circ\) を smooth quasi-projective variety(あるいはより一般に algebraic variety)とする.
%\(V^\circ\subset V\) が smooth compactification であって
%\[
%D_V:=V\setminus V^\circ
%\]
%が(reduced な)SNC divisor となるような \(V\) をとる
%このとき \(V^\circ\) の \emph{logarithmic Kodaira dimension} \(\tilde{\kappa}(V^\circ)\) を,
%\(K_V+D_V\) の Iitaka--Kodaira dimension
%\[
%\tilde{\kappa}(V^\circ):=\kappa\bigl(V,K_V+D_V\bigr)
%\]
%として定める.
%\end{defn}
%上の\(V\) の存在は Nagata の compactification theorem と Chow の補題,および Hironaka の解消定理から従う. 
%次に smooth quasi-projective variety の Albanese map について述べる:

\begin{prop}[{\cite{Fuj15}, Theorem 3.16}]\label{propdef:A5}
\(U\) を smooth quasi-projective variety とし,\(u\in U\) を固定点とする.
このとき semi-Abelian variety \(\Alb_U\) と 代数射
\[
\alb_U:U\to\Alb_U
\]
が存在して次を満たす. 
\begin{itemize}
\item \(\alb_U(u)=0\).
\item 任意の semi-Abelian variety \(G\) への代数射\(\alpha:U\to G\) で \(\alpha(u)=0_G\) を満たすなら, 
ただ一つの 代数群 の準同型 \(f:\Alb_U\to G\) が存在して
\(\alpha=f\circ\alb_U\) を満たす.
\end{itemize}
\(\alb_U\) はこの普遍性によって一意に決まり,
\(\alb_U\) を \(U\) の \emph{quasi-Albanese map} と呼び,
\(\Alb_U\) を \(U\) の \emph{quasi-Albanese variety} と呼ぶ.


\end{prop}
\(U\) がコンパクトであれば,\(\alb_U\) は通常の Albanese map と一致する.
\(\Alb_U\) と \(\alb_U\) の構成については \cite[3]{Fuj15} を参照.
Albanese map の基本的性質は次のとおり. 

\begin{prop}[{\cite{Fuj15}, Lemma 3.11}]\label{prop:A6}
\(U\) を smooth quasi-projective variety とし,
\(\alb_U:U\to\Alb_U\) をその quasi-Albanese map とする.
すると誘導される写像
\[
(\alb_U)_\ast:H_1(U,\Z)\longrightarrow H_1(\Alb_U,\Z)
\]
は全射であり,\(\ker(\alb_U)_\ast\) は \(H_1(U,\Z)\) の torsion 部分に一致する.
\end{prop}

%%%%%%%%%%%%%%
\begin{comment}


\begin{prop}\label{prop:A7}
\(U\) を smooth quasi-projective variety とし,
\(\alb_U:U\to\Alb_U\) をその quasi-Albanese map とする.
\(V\) を \(U\) の compactification であって \(V\setminus U\) が SNC divisor となるものとする.
\[
1 \to G_m^{d} \to \Alb_U \xrightarrow{p} A_U \to 1
\]
を \(\Alb_U\) の Chevalley decomposition とする.
また \(Z\) を \(\alb_U\) の像の閉包とする.このとき
\begin{enumerate}
\item \(A_U\) は \(V\) の Albanese variety \(\Alb_V\) と同型であり,
      制限写像 \(\alb_V|_U\) は \(p\circ\alb_U\) に等しい;
\item \(Z\) は \(\Alb_U\) を生成する.
\end{enumerate}
\end{prop}

\begin{proof}
(a) は \cite{Fuj15}, Lemma 3.7--3.8 の構成から直ちに従う.

(b) については,まず \(Z\) により生成される \(\Alb_U\) の algebraic subgroup を \(G\) とおく.
つぎに \(Z\) の \(A_U\) への像を \(W\) とおけば,\(W\) は \(\alb_V\) の像である.
\cite[Proposition 3.1(b)]{Uen75}(Lemma 9.14, pp.\ 108--110)より,
\(W\) は \(A_U=\Alb_V\) を生成する.したがって包含写像
\[
G\hookrightarrow\Alb_U\longrightarrow A_U
\]
は全射である.ゆえに \(G\) の Chevalley decomposition を
\[
1 \to H \to G \to A_U \to 1
\]
と書くことができる.ここで \(H\subseteq G_m^{d}\) である.
\(H\) は diagonalizable であるから,\cite{Spr98}, 3.2.7 Corollary, p.\ 45 により,
\(H\) は有限 Abelian 群と algebraic torus の直積である.
しかし \(G\) は連結であるから \(H\) も連結であり,従って \(H\) は algebraic torus である.
よって定義から \(G\) は semi-Abelian variety となる.
このとき \(U\to G\) は Albanese map の普遍性を満たすので,
とくに \(G=\Alb_U\) が従う.
\end{proof}
\end{comment}
%%%%%%%%%%%%%%%%%%%%

\subsubsection{quasi-projectiveの場合}
\begin{thm}
%\label{thm-CW15-7.1-q}
 %\noindent\textbf{Theorem 7.1.}
$X$をquasi-projective manifoldとする.
もしquasi-Albanese map がdominantでないならば, \(X\) は \(h\)-principleを満たさない.
\end{thm}

\begin{proof}
\ref{thm-CW15-7.1}の証明において次のように補正すれば良い.

$(\star 1)$の部分は\cite[Theorem 5.6.19]{NW14}や\cite[Lemma 5.5]{NWY02}を参照する.

$(\star 2)$の部分はNoguchi's logarithmic version of the theorem of Bloch-Ochiai (\cite[Theorem 4.8.17]{NW14})を引用する. 
%\footnote{\cite[Theorem 4.8.17]{NW14}(Bloch-Ochiai)からなぜ$\overline{f(\C)}^{zar}$がtorusの平行移動になることについて, $\overline{f(\C)}^{zar}$を含む最小のtorusを取ると, }
これで証明が回る. 
\end{proof}

\begin{cor}\cite[Corollary 7.3]{CW15}
\label{cor-CW15-7.3-q}
\(X\) をalgebraic variety とし, 種数2以上の曲線$C$への全射$F : X \to C$を有するとする. 
このとき \(X\)は\(h\)-principleを満たさない.
\end{cor}
\begin{proof}
証明は全く同じ. 途中で"Albanese varietyの構成から$J=H^0(C, \Omega_{C}^{1})^{\vee}/H_{1}(C, \Z)$であるので全射性が言える"という部分に\ref{prop:A6}を使うだけ
 \end{proof}
 

\begin{prop}\cite[Proposition 7.4]{CW15}
\label{prop-CW15-7.4-q}
\(X\) quasi-projective manifold で, semi-abelian variety $A$へのfinite map $X \to A$を有するとする. 

このとき\(X\) が\(h\)-principle を満たすならば,  \(X\) はsemi-abelian varietyである
\end{prop}
"\cite{NWY07}より次が言える"と書いていたが証明がわからなかった.

\subsubsection{引用されていた定理たち}
以下\cite{NW14}の引用されていた定理を述べておく. 


\begin{thm}\cite[Theorem 5.6.19]{NW14}
\label{thm-NW14-5.6.19}
\(T\) を semi-torus とし,\(T \hookrightarrow \overline{T}\) を equivariant compactification とする。\(\overline{T}\) の closed analytic subset を \(\overline{Z}\) とし,\(Z = \overline{Z} \cap T\) に含まれる \(T\) の sub-semi-torus の正次元軌道をすべて集めたものの合併を \(W\) とする。

このとき,\(\overline{T}\) の closed analytic subset \(\overline{W}\) が存在して \(W = \overline{W} \cap T\) を満たす。さらに,\(W\) の任意の既約成分は正次元の stabilizer group をもつ。
\end{thm}

\begin{thm}\cite[Theorem 4.8.17]{NW14} (Logarithmic Bloch--Ochiai Theorem)
\(M\) を コンパクト Kähler manifold とし,\(D\) を \(M\) 上の reduced divisor とする。
もし \(\overline{q}(M \setminus D) > \dim M\) ならば,任意の entire curve \(f : \mathbb{C} \to M \setminus D\) の像は \(M\) の中で Zariski dense にはならない。
\end{thm}

\begin{cor}\cite[Corollary 4.8.18]{NW14} (Bloch--Ochiai Theorem)
もし \(q(M) > \dim M\) ならば,任意の entire curve \(f : \mathbb{C} \to M\) は algebraically degenerate である。
\end{cor}


\begin{thm}\cite[Theorem 6.6.1]{NW14}(Noguchi--Winkelmann--Yamanoi)
\(X\) を complex algebraic variety とし,\(\pi : X \to A\) を semi-abelian variety \(A\) への有限写像とする。
\(f : \mathbb{C} \to X\) を任意の entire curve とする。もし \(\overline{\kappa}(X) > 0\) ならば,\(f\) は algebraically degenerate である。

さらに,\(f(\mathbb{C})\) の Zariski 閉包の正規化は semi-abelian variety となり,それは \(A\) のある真の semi-abelian subvariety の平行移動の有限 étale 被覆である。
\end{thm}

\begin{cor}\cite[Corollary 6.6.2]{NW14}
\(X\) を quasi-Albanese map が proper map であるような complex algebraic variety とする。
\(\overline{\kappa}(X) > 0\) かつ \(\overline{q}(X) \ge \dim X\) であると仮定する。このとき,任意の entire curve \(f : \mathbb{C} \to X\) は algebraically degenerate である。
\end{cor}




%%%%%%%%%%%%%%%%%%%%%%%%%%%%%%%%%%%%%%
\begin{comment}


\begin{thm}\cite[Theorem 13:]{Kaw81}

 Let $f : X \to A$ be a finite morphism from a complete
normal algebraic variety to an abelian variety. Then $\kappa(X)\geq 0$
and there are an abelian subvariety $B$ of $A$, etale covers
$\tilde{X}$ and $\tilde{B}$ of $X$ and $B$, respectively, and a
complete normal algebraic variety $\tilde{Y}$ such that:
\begin{enumerate}
\item $\tilde{Y}$ is finite over $A/B$.
\item $\tilde{X}$ is isomorphic to $\tilde{B}\times \tilde{Y}$.
\item $\kappa(\tilde{Y})=\dim \tilde{Y}=\kappa(X)$.
\end{enumerate}
\end{thm}


\subsubsection{quasi-projectiveの場合}
\begin{thm}
\label{thm-CW15-7.1-q}
 %\noindent\textbf{Theorem 7.1.}
$X$をquasi-projective manifoldとする.
もしquasi-Albanese map がdominantでないならば, \(X\) は \(h\)-principleを満たさない.
\end{thm}

\begin{proof}
\ref{thm-CW15-7.1}の証明において次のように補正すれば良い.
$(\star 1)$\cite{Kaw80}
[]

\(A\)を$X$の quasi-Albanese varietyとし,  quasi-Albanese map \(a : X \to A\)とする.
\(Z=\overline{a(X)}^{\mathrm{zar}}\)とする. 
平行移動して, $A$の単位元$e_A$が \(e_A \in Z\)となっているとして良い.

$a$がdominantでないので, $A\setminus Z \neq \varnothing$である,. 
よって\cite{Kaw80} (see also \cite[Theorem 5.6.19]{NW14} \ref{thm-NW14-5.6.19}参照)より, 
高々有限個数のsubsemi-torus\(T_i \subset A\)と \(T_i\)-orbit \(S_i \subset A\)があって次を満たす.
\begin{enumerate}
\item \(S_i \subset Z\) 
\item 任意の \(A\) のsub-semitorus$W$について, その平行移動$x + W$が\(Z\)に入っているならば,  それは\(S_i\)に入る. 
\end{enumerate}
 
\ref{lem-CW15-7.2}より, 
\(\gamma_0 \in \pi_1(A)\setminus \bigcup \pi_1(S_i)\)が取れる. 
 quasi-Albanese mapの性質から
 $$
 a_{*} : \pi_1(X) \twoheadrightarrow \pi_1(A)
 $$
 なので, $a_{*}(\gamma) = \gamma_0$が取れる. 
 よって$\gamma : S^1 \to X$なる連続写像が存在する. 
 $S^1 \underset{\text{homotopic}}{\sim}\C^{*}$なので, 連続写像$\gamma : \C^{*}\to X$が存在する(記号の濫用だが, これも$\gamma$と表す)
 
 $X$は\(h\)-principleを満たすので, 正則写像
 $$
 f : \C^{*} \to X
 $$
 で $\gamma\underset{\text{homotopic}}{\sim} f$となるものが存在する. 
 これより正則写像
 \[
F = a \circ f \circ \exp : \C  \to \C^{*} \to X \to \rightarrow Z \subset A .
\]
を得る. Noguchi’s logarithmic version of the theorem of Bloch–Ochiai (\cite[Theorem 4.8.17]{NW14} 下に引用)より. analytic Zariski closure \(\overline{F(\C)}^{\mathrm{zar}}\) を\(Z\)上でとると,  
\(\overline{F(\C)}^{\mathrm{zar}}\) は$A$のsub-semitorusとなる. 
これは上の(2)からある$S_i$があって, \(F(\C) \subset S_i\)となる.
よって, $a \circ f (\C^{*})\subset S_i$より$\pi_1(\C^*) \cong \Z$の単位元を$1_{\C^{*}}$とすると
\[
a _{*}(\gamma) 
=
(a \circ \gamma)_*(1_{\C^{*}}) 
\underset{\gamma\underset{\text{homotopic}}{\sim} f}{=}(a \circ f)_*(1_{\C^{*}}) 
\quad
(a \circ f)_*(1_{\C^{*}}) \underset{a \circ f (\C^{*})\subset S_i}{\in} \pi_1(S_i),
\]
これは$(a _{*}(\gamma)=\gamma_0 \in\pi_1(A)\setminus \bigcup \pi_1(S_i)$に矛盾する.
\end{proof}


引用した\cite[Theorem 5.6.19, 4.8.17]{NW14} は以下の通り. 
\begin{thm}\cite[Theorem 5.6.19]{NW14}
\label{thm-NW14-5.6.19}
\(T\) を semi-torus とし,\(T \hookrightarrow \overline{T}\) を equivariant compactification とする。\(\overline{T}\) の closed analytic subset を \(\overline{Z}\) とし,\(Z = \overline{Z} \cap T\) に含まれる \(T\) の sub-semi-torus の正次元軌道をすべて集めたものの合併を \(W\) とする。

このとき,\(\overline{T}\) の closed analytic subset \(\overline{W}\) が存在して \(W = \overline{W} \cap T\) を満たす。さらに,\(W\) の任意の既約成分は正次元の stabilizer group をもつ。
\end{thm}

%\begin{proof}
%\(P = \mathbb{P}(\Lie T)\) とおく。次に,\(\exp(v)\)-軌道が \(p\) において少なくとも次数 \(k\) で \(Z\) に接する,という条件で \((p,[v]) \in Y_k\) と定めて,部分集合 \(Y_k \subset \overline{Z} \times P\) を定義する。これは closed analytic subset であり,実際,
%\((p,[v]) \in Y_k\) であることと,すべての \(f \in \mathscr{O}(\overline{Z})_p\) と \(i \leq k\) に対して\((v^i(f))_p = 0\) が成り立つこととは同値である。\(Y_k\) を \(\overline{Z}\) への射影の像を \(V_k\) とし,
%\[V = \bigcap_{k \in \mathbb{N}} V_k\]とおく。すると \(V\) は \(\overline{T}\) の closed analytic set であり,\(W = V \cap T\) が成り立つ。

%最後に,\(W_0\) を \(W\) の既約成分とする。\(W_0\) が正次元の stabilizer group をもつことを示したい。これは,ある semi-torus がもつ sub-semi-torus は高々可算個であるという事実(Corollary~5.1.9)から従う。□
%\end{proof}

%Kawamata Structure Theorem のためには,任意の analytic subvariety \(Z\) を考えるのではなく,ある equivariant compactification \(\overline{T}\) における閉包 \(\overline{Z}\) が再び analytic になるような \(T\) の subvariety \(Z\) のみを考える必要がある。

\begin{thm}\cite[Theorem 4.8.17]{NW14} (Logarithmic Bloch--Ochiai Theorem)
\(M\) を コンパクト Kähler manifold とし,\(D\) を \(M\) 上の reduced divisor とする。
もし \(\overline{q}(M \setminus D) > \dim M\) ならば,任意の entire curve \(f : \mathbb{C} \to M \setminus D\) の像は \(M\) の中で Zariski dense にはならない。
\end{thm}

%ここで,\(M\) 上の Zariski 位相とは,analytic subset が閉集合となるような位相を意味する。

%\begin{proof}
%semi-torus に関する第 5 章からいくつかの事実を用いる。

%\(\alpha : M \setminus D \to A\) を quasi-Albanese map とし,\(A = A_{M \setminus D}\) とおく。すると \(\alpha\) は meromorphic map \(\overline{\alpha} : M \to \overline{A}\) へと延長される(Proposition~4.5.13)。\(\overline{X} = \overline{\alpha}(M)\),\(X = \overline{X} \cap A\) とおくと,\(\overline{\alpha}(M)\) は \(\alpha(M \setminus D)\) の閉包と一致する。仮定より
%\[\dim X < \overline{q}(M \setminus D) = \dim A\]が成り立つ。

%\(\beta : A \to A' := A / \St(X^0)\) を商写像とし,\(X' = \beta(X)\) とおく。すると \(A'\) は semi-torus であり,\(\beta\) は compactification の間の meromorphic map \(\overline{\beta} : \overline{A} \to \overline{A}'\) へと延長され,\(\overline{\beta}(\overline{X}) = \overline{X}'\) かつ
%\[0 < \dim X' < \dim A'\]
%を満たす(Proposition~5.3.13 (vii), Proposition~5.3.24 を参照)。\(X'\) は \(A'\) を生成することに注意する。Proposition~5.6.21 から \(X'\) は logarithmic general type であることが従う。特に \(\overline{X}'\) は Moishezon である。したがって,\(A'\) の最大の affine subgroup \(T'\) による商 \(A'/T'\) は Moishezon であり,従って abelian である(Noguchi--Winkelmann~[02], Lemma~2.11 参照)。よって \(A'\) は semi-abelian である。したがって,議論は代数的な場合に帰着される。

%もし \(f\) の像が \(M\) の中で Zariski dense であるならば,
%\[g = \beta \circ \alpha \circ f : \mathbb{C} \to X'\]は algebraically non-degenerate である。一方,Theorem~4.8.2 から任意の \(\delta > 0\) に対して
%\[T_g(r) = S_g(r) = O(\log T_g(r)) + \delta \log r \,\big\|_{E(\delta)}\]
%が従う。Theorem~2.5.4 により,\(g\) は定数写像でなければならず,これは矛盾である。□
%\end{proof}

\begin{cor}\cite[Corollary 4.8.18]{NW14} (Bloch--Ochiai Theorem)
もし \(q(M) > \dim M\) ならば,任意の entire curve \(f : \mathbb{C} \to M\) は algebraically degenerate である。
\end{cor}



また証明に使った補題を示しておく. 
\begin{lem}\cite[Lemma 7.2]{CW15}
\label{lem-CW15-7.2}
 \(\Gamma_1, \dots, \Gamma_k\)を\(G = \Z^n\)の部分群とする. 
  \(\mathrm{rank}_\Z \Gamma_i < n\)ならば \(\cup_i \Gamma_i \neq G\)
\end{lem}

$\Gamma_i$は有限生成(ネーター環$\Z$の有限生成加群の部分加群より)なので, 有限生成アーベル群の基本定理より, $\gamma_i \cong \Z^{\mathrm{rank}_\Z \Gamma_i } \oplus \mathrm{(Torsion part)}$となる. 


\begin{proof}
$G =\Z^{r} \hookrightarrow \R^{n}$とユークリッド空間に埋め込む.
$x \in G$について$||x||$をそのユークリッド距離とする. 

部分群$H \subset G $のランクを$d$とする. 
$r \in \N$について
$$
N(H, r):= \{ x \in H \mid || x|| \le r\}
$$
とおくと, $r \gg 0$について$|N(H, r)|=O(r^{d})$である.

以上より, 仮定から\(N(\Gamma_i,r) = O(r^{n-1})\)かつ, \(N(G,r) = O(r^n).\) 
である. 
よって, $r$を十分大きく取れば(torsion partの部分も無視できるくらい大きくすれば), $\gamma \in G \setminus \cup_i \Gamma_i $が存在する. 
\end{proof}


\begin{cor}\cite[Corollary 7.3]{CW15}
\label{cor-CW15-7.3}
\(X\) をalgebraic variety とし, 種数2以上の曲線$C$への全射$F : X \to C$を有するとする. 
このとき \(X\)は\(h\)-principleを満たさない.
\end{cor}
\begin{proof}
 \(A\) (resp. \(J\)) を$X$ (resp. $C$)のquasi-Albanese varietyとする.  

背理法. 全射であるとする. 
quasi-Albanese mapの関手性より,  \(F_1 : A \to J\)がある. 
すると  \(F_1 : A \to J\)は全射となる.
 \[
\xymatrix@C=50pt@R=20pt{
X \ar@{-->}[r]^{a_X} \ar@{->>}[d]_{F} &A \ar@{->>}[d]^{F_1} \\
C\ar@{->}[r]^{a_C} & J
}
\]
\ref{thm-CW15-7.1}より, $a_X : X \to J$が全射でないことを示せば良い.

なぜならば, $F$ dominantなので, $F^{*}  : H^0(C, \Omega_{C}^{1}) \hookrightarrow H^0(X, \Omega_{X}^{1})$は単射である. 
 よって
 $${F^{*}}^{\vee} :
 H^0(X, \Omega_{X}^{1})^{\vee}\twoheadrightarrow  H^0(C, \Omega_{C}^{1})^{\vee}
 $$
 は全射である. Albanese varietyの構成から$J=H^0(C, \Omega_{C}^{1})^{\vee}/H_{1}(C, \Z)$であるので全射性が言える(\ref{prop:A6}参照)

よって$F_1\circ a_X$は全射で, $F_1\circ a_X = a_{C} \circ F$より$a_C$は全射である. 
しかし$C$は種数2以上なので,  \(g=\dim(J) > \dim(C)=1\)であるので矛盾. 
 \end{proof}

\cite{NWY07}より次が言える

\begin{prop}\cite[Proposition 7.4]{CW15}
\label{prop-CW15-7.4}
\(X\) quasi-projective manifold で, semi-abelian variety$A$へのfinite map$X \to A$を有するとする. 

このとき\(X\) が\(h\)-principle を満たすならば,  \(X\) はsemi-abelian varietyである
\end{prop}
\begin{proof}
\xr{projectiveの場合はKawamataの定理からわかる?}
\end{proof}

\begin{thm}\cite[Theorem 13:]{Kaw81}

 Let $f : X \to A$ be a finite morphism from a complete
normal algebraic variety to an abelian variety. Then $\kappa(X)\geq 0$
and there are an abelian subvariety $B$ of $A$, etale covers
$\tilde{X}$ and $\tilde{B}$ of $X$ and $B$, respectively, and a
complete normal algebraic variety $\tilde{Y}$ such that:
\begin{enumerate}
\item $\tilde{Y}$ is finite over $A/B$.
\item $\tilde{X}$ is isomorphic to $\tilde{B}\times \tilde{Y}$.
\item $\kappa(\tilde{Y})=\dim \tilde{Y}=\kappa(X)$.
\end{enumerate}
\end{thm}


引用されている\cite{NWY07}は次のものだと思う. 

\begin{thm}\cite[Theorem 6.6.1]{NW14}(Noguchi--Winkelmann--Yamanoi)
\(X\) を complex algebraic variety とし,\(\pi : X \to A\) を semi-abelian variety \(A\) への有限写像とする。
\(f : \mathbb{C} \to X\) を任意の entire curve とする。もし \(\overline{\kappa}(X) > 0\) ならば,\(f\) は algebraically degenerate である。

さらに,\(f(\mathbb{C})\) の Zariski 閉包の正規化は semi-abelian variety となり,それは \(A\) のある真の semi-abelian subvariety の平行移動の有限 étale 被覆である。
\end{thm}

\begin{cor}\cite[Corollary 6.6.2]{NW14}
\(X\) を quasi-Albanese map が proper map であるような complex algebraic variety とする。
\(\overline{\kappa}(X) > 0\) かつ \(\overline{q}(X) \ge \dim X\) であると仮定する。このとき,任意の entire curve \(f : \mathbb{C} \to X\) は algebraically degenerate である。
\end{cor}

%Theorem~6.6.1 の証明は次のように進められる。

%\begin{itemize}
%  \item[(i)] Kawamata の Theorem~5.6.22 により,容易に log general type の場合に帰着される。
 % \item[(ii)] (下で述べる)ステップ (iii) のためには,\(X\) の compactification とその非特異模型を構成し,そこでの log canonical divisor を特定するような,かなり繊細な構成が必要になる(Theorem~6.6.6 を参照)。
 % \item[(iii)] \(f\) の algebraic non-degeneracy を仮定して,Theorem~6.5.1 と Theorem~6.5.17 を適用することで矛盾を導く。
%\end{itemize}


% [page 2]
%%%%%%%%%%%%%%%%%%%%%%%%%%%%%%%%
\end{comment}

\subsection{8章 (Counter-)examples}
\label{subsec-CW15-8}
%We now present examples showing that the desired implications “special 
%\(\Longrightarrow\) \(h\)-principle” and “C-connected \(\Longrightarrow\) special” certainly do not hold without imposing some normality and algebraicity, Kählerness, or completeness conditions on the manifold in question.
$$
\text{special} \Longrightarrow \text{\(h\)-principle}
\quad
\text{$\C$-connected} \Longrightarrow \text{special}
$$
という予想は$X$がnormal, K\"ahler, コンパクトという条件がいる.

\begin{ex}\cite[Example 8.1]{CW15}
\label{ex-CW15-8.1}
non-normal 代数曲線\(X\) で$\C$-connected(もっと強くrational)を満たすが, \(h\)-principleを満たさない例\footnote{normalではないのでspecialなどは論じることができない. }

\[
X = \{ [z_0 : z_1 : z_2]\in \C\mathbb{P}^2 : z_0 z_1 z_2 = z_1^3 + z_2^3 \}.
\]
とする. これは$(1:0:0)$に特異点があるので, smooothではない. よってnon-normalである. 

さらに$\widehat{X}:= \mathbb{P}^1$とすると, 
$$
h : \widehat{X}= \mathbb{P}^1 \to X \quad
[x_0 : x_1] \mapsto [x_0^3 + x_1^3 : x_0^2 x_1 : x_0 x_1^2]
$$
という写像で$X$はrationalである. 特に$\C$-connected. 

またこの$h$は$X$の正規化である. 
ここは\cite[Chapter 4 exercise1.8]{Har77}を用いる.
$\pi : \overline{X} \to X$を正規化とすると, 
$p =(1:0:0) \in X$について
$$
\delta_p := \mathrm{length}
(\pi_{*}\mathcal{O}_{\overline{X}}/\mathcal{O}_{X})
$$
である. $p$の近くで$X$は$uv=u^3 + v^3$となるので, $\delta_p=1$である. 
一方$X$は$\C\mathbb{P}^2$の3次曲線なので
$h^1(X, \mathcal{O}_{X})=1 (=\frac{(3-1)(3-2)}{2})$
よって\cite[Chapter 4 exercise1.8]{Har77}から
$$
h^1 (\overline{X}, \mathcal{O}_{\overline{X}})
=h^1(X, \mathcal{O}_{X})-\delta_p =0
$$
となり$\overline{X} \cong \mathbb{P}^1$となる. 

 \(\widetilde{X} \to X\)を$X$の普遍被覆とすると, 
\(\widetilde{X}\) は可算無限個の$S^2$がくっついた感じになっている.
$h : \widehat{X}= \mathbb{P}^1 \to X $ は
$h([0:1])=h([1:0])=[1:0:0]$
でありその他の点は1:1に移るので, 
\begin{center}
$X$は$S^2$の北極と南極をつなげた図形と同相
\end{center}
である.
これは$S^2 \lor S^1$とホモトピー同値である\footnote{これ有名な例らしい\url{https://math.stackexchange.com/questions/20282/fundamental-group-of-s2-with-north-and-south-pole-identified} Hatcherのp.11にわかりやすい図がある. }
よって普遍被覆は$S^1$の部分が直線に伸びて, その直線に$S^2$が$\Z$こつながる.

Hurewiczの定理から
 \[ \pi_2(\widetilde{X}) \simeq H_2(\widetilde{X}, \Z) \simeq \Z^\Z
 \]
 となる. よって
   \(\pi_2(X) \simeq \Z^\Z \)である. 
特に$h : \widehat{X} \to X $によって誘導される
\begin{equation}
\label{eq-ex-CW15-8.1}
\Z \simeq \pi_2(\widehat{X}) \overset{h_{*}}{\rightarrow} \pi_2(X) \simeq \Z^\Z
\end{equation}
は全射ではない. 

今 \(Q_2\)を\ref{prop-CW15-5.1}のような2次元 affine quadricとする. 
\(Q_2\) はSteinで$S^2$とhomotopicである.
そこで$\gamma \in \pi_2(X) \setminus h_{*}(\pi_2(\widehat{X}))$を選ぶと,
\ref{eq-ex-CW15-8.1}から
$$
\gamma : Q_2 \to X
\quad\text{and} \quad 
\text{$\gamma$ は $h : \widehat{X} \to X$を経由しない}
$$
という連続写像を構成できる .

もし$X$が$h$-principleを満たすならば, $\gamma \underset{\text{homotopic} }{\sim} f_\gamma$となる正則写像$f_\gamma : Q_2 \to X$が存在する. 
 $h : \widehat{X} \to X$は$X$の正規化より$f_\gamma$ は $h : \widehat{X} \to X$を経由する. 
 しかしこれは$\gamma$の取り方に矛盾する. 
 \end{ex}
\xr{この構成使ったらnormalでも反例作れない?}

\begin{ex}\cite[Example 8.2]{CW15}
 non-Kähler コンパクト複素曲面(井上曲面)で \(h\)-principleを満たさないが, specialとなるものがある. 

井上曲面$X$は次を満たす.
\begin{itemize}
\item  コンパクト複素曲面かつ代数次元$a(X)=0$ \footnote{"OT manifold(井上曲面の高次元版)は$a(X)=0, \kappa(K_X)=-\infty$で有理曲線を持たない"とのこと. }
\item 普遍被覆が\(\Delta \times \C\)で, 複素直線でfoliatedされる. 多分これは$\pi : \Delta \times \C \to X$という商写像として, $\pi(\{ x\} \times \C)$の形のleafを持つfoliationが存在することだと思う. 
%\xr{これはBrunellaの$\nu=1, \kappa=-\infty$のfoliationの例?}
%\item 任意の正則写像$f : \C \to X$について像がdence entireになる. 
\end{itemize}
\cite[Section 2.1]{Cam04}により, $a(X)=0$ならばSpecialである
%$a(X)=0$なので, Bogomolov sheafが存在し得ない\xr{これは存在したら代数次元が1以上になるから?}

一方任意の正則写像$\C^{*} \to X$の像は$X$のleafに含まれる. \footnote{多分これから任意の正則写像$f : \C \to X$について像がdence entireになることも言えると思う. }
これはexpをかまして, $\C \to X$を考えると, $\C \to \Delta\times \C$にLiftして, $\C \to \Delta$は定数になるので, $\C$の像は$\pi(\{ x\} \times \C)$の形になる. 

なので
$\Z=\pi_1(\C^{*}) \to \pi_{1}(X)$という群準同型で, 正則写像から来ないものが存在する. 
よって\(h\)-principleを満たさない.


\xr{と書いてあったがこれよくわからなかった.}$\pi_1(X)$は$\Z^3$と$\Z$のextensionで$\pi_1(L)$は$\Z$と同型らしい.  最後に関しては$L \subset X$をleafとして
$i_{*}\pi_1(L) \neq \pi_1(X)$であることから来ていると思う. 
もしそれが言えたら, $\pi_1(X) \setminus i_{*}\pi_1(L)$の元$\eta$に送るmapが連続写像で正則写像から来ないものの例になる. つまり$1_{C} =[S^1 \to \C^{*}] \mapsto \eta = [S^1 \mapsto X]$なので
$$
F : \C^{*} \underset{\text{homotopic}}{\sim} S^1 \overset{\eta}{\to} X
$$
が連続写像でいかなる正則写像ともhomotopicにならないものになる.(ということ??)
\end{ex}

\begin{ex}\cite[Example 8.3]{CW15}
non-compact 複素多様体で$\C$-connectedだが, \(h\)-principleを満たさない例がある. 

Rosay-Rudinの定理(\cite[Theorem 4.5]{RR88}) から
離散部分群\(S \subset \C^2\)であって, 任意の非退化な正則写像 \(F : \C^2 \to \C^2\)
%\footnote{\(F\) が non-degenerate であるとは, ある点 \(p \in \C^2\)で \(\mathrm{rank}(DF)_p = 2.\)となるものが存在すること.}
 について 
\[
F(\C^2) \cap S \neq \emptyset
\]
となるものが存在する. 
\(X := \C^2 \setminus S.\) とする. 
$S$は離散なので, \(X\) は$\C$-connectedである.($S$の濃度は高々可算なので, 任意の2点は折れ線で結べるから)

\(G := SL_2(\C).\)とする.
$G$はaffine(Stein)である. そして\(S^3\)と homotopicである.
\footnote{$P$を正定値Hermite行列の空間とすると, $SL(2, \C) = SU(2, \C) \times P$という分解が存在するらしい. \url{https://homepage.ntu.edu.tw/~cjtsai/teaching/23dg/HW2.pdf?utm_source=chatgpt.com} これを認めると$P$は可縮(単位行列とむすぶ)であり,$SU(2, \C) \cong S^3$(diffeo)が言えるので言える. }

% \(p \in G=SL_2(\C)\),  \(v, w \in T_p G \cong \C^2\)とする.  exponential mapによって正則写像 \[\mathrm{exp}:  \C^2 \cong T_p G \to G\] を考える.
% from \(\C^2\) to \(G\) for which \(v\) and \(w\) are in the image. 
\(p \in G\),  \(v, w \in T_p G \)として, 
正則写像 \[f: \C^2 \to G\] で $exp(v),  exp(w) \in f(\C^2)$となるものがある($exp : T_pG \to G$を指数写像とする.この構成はexp を$v, w$が張る$\C^2$に制限すれば良い.)

 任意の正則写像\(F : G \to X\)について, 上のexp使って$\C^2 \to G \to X \hookrightarrow \C^2$を考えれば, 
$S$の定義から
$$
\mathrm{rank}(DF)_p \le 1 \quad \text{for any \(p \in G,\)}
$$
が成り立つ.
よって実多様体の写像$F_\R : G_\R \to X_\R$と見たときのJacobi行列$J_{F_\R}$のランクは2以下である. 
以上より任意の d-closed 3-form $\omega$について \(F^*\omega \equiv 0\)である. 
%$$
%F^{*} : H^3(X, \R)=H^3_{dR}(X,R) \overset{F^{*}}{\to} H^3(G, \R)\quad [\omega] \mapsto [F^*\omega]=0
%$$
つまり\(F^* : H^3(X, \R) \to H^3(G, \R)\) は自明である. 

一方で連続写像\(f : S^3 \to X\) で 
\[
f^* : H^3(X, \R) \to H^3(S^3, \R)
\quad \text{が0ではない}
\] 
となるものが存在する. 
これは\(p \in S\)について, 
半径$r$のopen ball $B_{p}(r) $ 
$B_{p}(r) \setminus \{ p\} \subset X$
となるものをとれば良い. 

この$f$を$G \underset{\text{homotopic}}{\sim} S^3$にかまして, 連続写像$ G \to X$を考えれば, これが$h$-princpleを壊す例となる. (もし正則写像とhomotopicなら$ H^3(X, \R) \to H^3(G, \R)$は0射になってしまう.)
\end{ex}

% [page 3]

\subsection{9章 Does special imply the \(h\)-principle?}
この辺りは\cite{CW15}そのまま書いた. 

\begin{ques}
\(X\)  smooth projectiveとする. 
Specialならば$h$-principleを満たすか? これは2次元でも不明. 
\end{ques}
1次元ならOK. 種数と対応する. 

2次元で次のものは$h$-principleを満たすことがわかっている.
\begin{itemize}
\item rational surfaces,
\item  minimal surfaces 
\item ruled over an elliptic curve
\item  blown-up abelian surfaces and their étale undercovers (bielliptic)
\end{itemize}

2次元で次のものはspecialだが, $h$-principleを満たす課はわかっていない. 

\begin{itemize}
\item blown-up K3 and Enriques surfaces
\item the blown-ups of surfaces with \(\kappa = 1\)これらは以下の2通り
\begin{enumerate}
\item elliptic fibrations over an elliptic base without multiple fibre
\item elliptic fibrations over a rational base with at most four multiple fibres, where the sum of the inverses of the multiplicities is at least two (respectively, one) if there are four (respectively, three) multiple fibres.
\end{enumerate}
\end{itemize}

3次元以上だと全くわからない. 
例えばFano, RC, rational manifolds (e.g. \(\mathbb{P}^3\)のdegree3以上の曲線に沿ったblow up)など.

3次元以上のFano, RCに関しては, non-degenerate meromorphic map \(\C^n \to X\) も未解決.
もし存在しないならば
「SpecialだがOkaではない」例である. 
これは unirationalでも不明. 

次の操作で \(h\)-principle が保存されるかがわからない. 
\begin{itemize}
\item smooth blow-ups and blow-downs,
\item products\footnote{これは簡単にわかると思う.}
\item  (finite) étale coverings, for which only one direction is known (cf. [For11]).\footnote{finite etale $X' \to X$で$X'$が$h$-principleを満たすときに$X$が満たすかがわからない. }
\end{itemize}
なおこれらの操作でSpecialは保存される

Gromovの定理「elliptic $\Rightarrow$ Oka $\Rightarrow$ \(h\)-principle」である. 
ellipticの例に関して
Homogeneousな複素多様体 (\(\mathbb{P}^n\), Grassmannians, tori)や\(\C^n \setminus A\) ($A$はcodim 2以上のalgebraic subvarieties \(A\)) などがある. 

%Oka(任意の正則写像 \(h : K \to X\) で \(K \subset \C^n\)をコンパクト凸集合とする. このとき\(H : \C^n \to X.\)という正則写像で近似できる)
なおOkaとの関連は"projective manifold"の場合は次のとおり. 
% [page 4]

\[
\text{elliptic} \Longrightarrow \text{Oka} \Longrightarrow h\text{-principle} \Longrightarrow \text{special}
\]

「\(h\)-principle$\Rightarrow$Oka」は一般に嘘である. (単位円板を考える)
ただ次はどうなるだろうか?

\begin{ques}\cite[Question 9.1]{CW15}
\label{ques-CW-9.1}
 projective manifoldに関して \(h\)-principle $\Rightarrow$ Oka?
\end{ques}

\subsection{補足・ちょっと思ったこと}

\subsubsection{Hyperkahlerの場合}
Kobayashi pseudometricが消えるという条件はHyperK\"ahlerの方で調べれれている. 
Kamenove-lehn\cite{KL22}を見ると次の通り

\begin{thm}\cite[Theorem 1.1]{KL22}
%\noindent\textbf{Theorem 1.1.}
$X$primitive symplectic varietyとする. 
任意のprimitive symplectic varietyで$X$とlocally trivial deformationを持つものがrational SYZ conjectureを満たすとする. 
このとき次が成り立つ. 
\begin{enumerate}
\item $b_{2}(X)\ge 5$ならば, $X$はKobayashi hypertbolicではない. 
\item $b_{2}(X)\ge 7$ならば, Kobayashi pseudometric $d_{X}$は常に0.
\end{enumerate}
\end{thm}

\begin{ques}
上の多様体はOka, $\C$-connected, $h$-principleを満たす?
\end{ques}
$\kappa=0$なのでSpecialは確定である. 

また"singular"の場合は本当にOka, $h$-principleを満たすかな?と思った.

\begin{ques}
Fujikiの複素2次元symplectic orbifoldは全て$h$-principleを満たすか?
\end{ques}
Fujki先生が82年にsymplectic orbifoldの例を出している.
\url{https://arxiv.org/abs/2503.23373}参照のこと. 


%\begin{ques}LCもしくはそれより悪い特異点だと\ref{ques-CW-9.1}は成り立たないのでは?
%\end{ques}
%例えば種数2以上の曲線を$C$として, そのprojective cone $X$をとると, これはrationally chain connectedである. 
%しかしblowup $\widetilde{X} \to X$は$C$上の$\mathbb{P}^1$ bundleとなる. 
%よって$\widetilde{X}$はspecialではないしまたKobayashi psedo-metricも消えてない

\subsubsection{\cite{CW15}関連}
\cite[Section 7]{CW15}見てて思ったこと

\begin{ques}
$X$をcompact K\"ahler多様体とする. 
$h$ principleを満たすならばweakly special??

また$h$ principleを満たすならば任意のBrody hyperbolic 多様体への正則写像は定数?
\end{ques}
$X$がprojectiveの場合は正しい. specialならばweakly specialなので.
$X$がcompact K\"ahlerの場合はどうなる?
もし正しいなら\cite[Section 7]{CW15}は自明に正しい. 

\begin{ques}
$h$-principleはbimeromorphic invariant?

finite etale $X' \to X$で$X'$が$h$-principleを満たすときに$X$が満たす?
\end{ques}
もしそうなら面白いし, そうでないならspecialでない反例になる. 

\begin{ques}
$h$-principleを満たすならば, 任意の表現 $\rho : \pi_1(X) \to GL(r, \C)$の像はalmost Abelian??
\end{ques}
specialならば上は満たす. 

\begin{ques}
LCもしくはそれより悪い特異点だと\ref{ques-CW-9.1}は成り立たないのでは?

また\ref{thm-CW15-5.8}は$X$がsingularだと下の例が反例になっている?
\end{ques}
$Y $を$\mathbb{P}^2$内にある種数2以上のリーマン面とする.
$X \subset \mathbb{P}^3$を頂点$P$とする$Y$上のconeをとする. 
$f : X \dashrightarrow Y$を$P$から$Y$への射影とする. これは定数ではない.

ちなみに$\pi : \widetilde{X} \to X$を$P$でのblow upとすると
$\widetilde{X} \to Y$は$\mathbb{P}^1$束となる.
$\pi^{-1}(x) \cong Y$であり, $X$の特異点はLCより悪い. 
 \xr{$h$-principleを満たすかがわからない.}
 
なおこの例は非常に有名な例で, Rationally connectedではないが, Rationally chain connectedである例である. (環論でも有名な例らしい). 
また$\widetilde{X}$はspecialではないしまたKobayashi psedo-metricも消えてない

\ref{ex-CW15-8.1}を使えばnormalでも反例があるような気がする.(多分KLTよりsingularityが悪いと色々と不備が出そうな気がする. )
% [page 1]


\newpage

\section{Oka多様体まとめと問題}
%2024年11月の集会で日下部くんと話してた内容や
Oka多様体の基本性質などをまとめる. 
これらは日下部さんのサーベイ(日本数学会 2021 \cite{Kus21})やForstnevicのサーベイ\cite{For13}などをそのまま移した.

また用語に関しては時代によって変わっている可能性がある. 適宜ForstnericのICM2026のサーベイや日下部さんの岡シンポジウム2024のサーベイなど最近のものを参照してほしい. 

\subsection{Oka多様体の定義}
\begin{defn}
$X$を複素多様体, $f : X \to Y$を正則写像とする.
\begin{enumerate}
\item $f : X \to Y$上の\underline{spray}とは, 正則ベクトル束$E \to X$と, $s : E\to Y$で任意の$x \in X$について$s(0_x)=f(x)$となるもの.
\item $f : X \to Y$上のspray $E \to X, s:E \to Y$が\underline{dominate}とは, $s|_{E_x} : E_x \to Y$が$0_x \in E_x$で沈め込みになること.
\item $f :  X \to Y$に\underline{dominateなspray}が存在するとは, 正則ベクトル束$E \to Y$と正則写像$s:E \to Y$が存在して, 任意の$x \in X$で$s(0_x)=f(x)$かつ$s|_{E_X} : E_x \to Y$が$0_x \in E_x$で沈め込みになること.
\end{enumerate}
\end{defn}

\begin{defn}
$X$を複素多様体とする. 
\begin{itemize}
  \setlength{\parskip}{0cm} % 段落間
  \setlength{\itemsep}{0cm} % 項目間
%\item $X$ is dominable (resp. strongly-dominable) とはある点$x \in X$ (任意の点$x \in X$)についてある正則写像$s : \C^N \to X$があって$s(0)=x$かつ$0$で$s$は沈め込みになる. 
\item $X$が\underline{(Gromov)-Elliptic}であるとは, $id_X$にdominateなsprayが存在すること. つまり正則ベクトル束$E \to X$と正則写像$s:E \to X$が存在して, $s(0_x)=x$かつ$s|_{E_x} : E_x \to X$が$0_x \in E_x$で沈め込みになること.
\item $X$が\underline{subelliptic}であるとは, $X$ 上の有限個の正則ベクトル束 $\pi_j : E_j \to Y$ と零切断に制限すると恒等写像になる正則写像 $s_j : E_j \to Y,\ j = 1, \dots, m$ で、任意の $y \in Y$ に対して
  \[
  \sum_{j=1}^{m} (ds_j)_0|_{(E_j)_y} = T_y Y
  \]
  となるものが存在すること.
\medskip
\item  $X$が\underline{$Ell_1$(相対楕円的)}であるとは, 任意のStein多様体$Z$と任意の正則写像$f :  Z \to X$にdominateなsprayが存在すること. つまり正則ベクトル束$E \to Z$と正則写像$s:E \to X$が存在して, $s(0_z)=f(z)$かつ$s|_{E_z} : E_z \to X$が$0_z \in E_z$で沈め込みになること.
% \item $X$が$Ell_2$ であるとは, Stein 多様体の解析的部分集合の近傍から $Y$ への正則写像に対して、ジェット拡張に関する岡の原理が成り立つことである。  
 % \item $X$が$Ell_\infty$ であるとは、定理1における拡張の部分をジェット拡張にした岡の原理の多面体パラメータ版が成り立つことである(cf.~[5, §3.1])。
\item $X$が\underline{Oka}とは, 任意の$n \in \N$と任意のコンパクト凸集合$K \subset \C^n$について$\mathcal{O}(\C^n, X)|_{K} \subset \mathcal{O}(K,Y)$がdenseであること. この性質をCAP(Convex Approxmation Property)という.
\medskip
\item $X$が\underline{$\C$-dominable}とは, ある$x \in X$と正則写像$s : \C^{\dim X} \to X$があって, $s(0)=x$かつ$0$で沈め込みになること. (つまり$s$が非退化であること.) 
\item $X$が\underline{$\C$-strongly-dominable}とは, 任意の点$x \in X$についてある正則写像$s : \C^{\dim X} \to X$があって$s(0)=x$かつ$0$で沈め込みになる. 
\medskip
\item $X$が\underline{$\C$-connected}であるとは, 任意の $a, b \in Y$ に対し, 有限個の $f_j \in \mathcal{O}(\mathbb{C}, Y)$, $j = 1, \dots, m$ で
  \[
  a \in f_1(\mathbb{C}), \quad b \in f_m(\mathbb{C}), \quad f_j(\mathbb{C}) \cap f_{j+1}(\mathbb{C}) \neq \emptyset, \quad j = 1, \dots, m-1
  \]
  となるものが存在すること. %($\C$-chain connectedって感じ)
  \item $X$が\underline{strongly $\C$-connected}であるとは, 上で $m=1$ としたものが成り立つこと.
  \item$X$ がZariski dense entire curveを持つとは, $f \in \mathcal{O}(\mathbb{C}, Y)$ で $\overline{f(\mathbb{C})} = Y$ となるものが存在すること. 
\end{itemize}
\end{defn}

図にするとこんな感じである. 
\[
\xymatrix@C=50pt@R=30pt{
  \hbox{elliptic} \ar@{=>}[d]\ar@{=>}[r] & \hbox{Ell}_\infty \ar@{=>}[d] \ar@{=>}[r] & \hbox{Ell}_2 \ar@{=>}[r] \ar@{=>}[l]& \hbox{Ell}_1 \ar@{=>}[l] \ar@{=>}[d]\\
\hbox{subelliptic} \ar@{=>}[r] & \hbox{Oka} \ar@{=>}[ld]\ar@{=>}[rr] \ar@{=>}[d]  \ar@{=>}[u]& &\hbox{$\C$-strongly dominable}  \ar@{=>}[d] \ar@{=>}[ld]\\
  \hbox{dense entire curve} & \hbox{strongly $\C$-connected} \ar@{=>}[r] & \hbox{$\C$-connected}& \hbox{$\C$-dominable} 
}
\]

上に図に関して2点補足すると以下の通りである.

\begin{thm}\cite{Kus21}\cite{For13}
複素多様体 $X$について次は同値
\begin{enumerate}
  \setlength{\parskip}{0cm} % 段落間
  \setlength{\itemsep}{0cm} % 項目間
  \item Oka
 \item $Ell_1$ (他にも$Ell_2$, $Ell_{\infty}$とかあるらしい.)
 \item convex elliptic. つまり任意のコンパクト凸集合 $K \subset \C^n$から$X$への正則写像に支配的スプレーが存在する.
 \item 任意のStein 多様体(空間?) $Z$, コンパクト正則凸集合 $K \subset Z$, 閉部分多様体$Z' \subset Z$について, 連続写像$f_0 : Z \to X$で$K$の近傍と$X'$上で正則ならば, あるホモトピー$f_t : Z \to X$が存在して次を満たす. 
 \begin{itemize}
  \setlength{\parskip}{0cm} % 段落間
  \setlength{\itemsep}{0cm} 
  \item $f_t : Z \to X$は$K$の近傍で正則で, $K$上に一様に近似する.
  \item $f_t|_{Z'} = f_{0}|_{Z'}$
  \item $f_1 :  Z \to X$ は正則
 \end{itemize}
 \item 任意のコンパクト凸集合$K \subset \C^n$の近傍からの$X$への正則写像について, 正則写像$\C^n \to X$で$K$上に一様近似できる. 
 \end{enumerate}
\end{thm}


\begin{thm}
Gromov ellipticならばOkaである. 逆は必ずしも成り立たない
\end{thm}
\begin{proof}
上の同値性(これも定理だが)からGromov ellipticならば$Ell_1$を示せば良い.
Stein 多様体$Z$と任意の正則写像$f :  Z \to X$について, $f^{*} E \to Z$と$f^{*} E  \to E \to X$を考えればこれがdominate sprayを与える. 
\end{proof}

Oka多様体の他の同値性に関しては次のとおり. 以下の同値性から$h$-principleとの関係が言える. 
\begin{thm}\cite{FL11}
複素多様体 $X$について次は同値
\begin{enumerate}
  \setlength{\parskip}{0cm} % 段落間
  \setlength{\itemsep}{0cm} % 項目間
  \item Oka. つまりCAP(Convex Approxmation Propety)を満たす. ここでCAPとは, 任意の$n \in \N$と任意のコンパクト 凸集合$K \subset \C^n$について$\mathcal{O}(\C^n, X)|_{K} \subset \mathcal{O}(K,Y)$がdenseであること.(一様近似できる).
\item Convex Interpolation Property (CIP)を満たす.  
つまり任意の閉部分多様体\(T \subset \mathbb{C}^n\)で, ある$\mathbb{C}^k$のconvex domainと双正則なものと, 任意の正則写像\(f : T \to X\)について, $f$は正則写像\(\mathbb{C}^n \to X\)への拡張を持つ.
  \item Basic Oka Property Approxmation and Interpolation (BOPAI)を満たす. 
  つまり任意のStein inclusion \(T \hookrightarrow S\), 任意の正則凸コンパクト集合\(K \subset S\), そして連続関数\(f : S \to X\)で \(K \cup T\) 上で正則なものについて, $f$は正則写像\(S \to X\)に変形できる(homotopicである). またこの変形は$T$を保つ. 
 \item Parametric Oka Property Approxmation and Interpolation (POPAI)を満たす. 
 つまり,  \(T \hookrightarrow S\) \(K \subset S\) をBOPAIのものとし, 
 \(Q \subset P\subset\mathbb{R}^m\)をコンパクト部分集合とする. 

 任意の連続関数\(f : S \times P \to X\)であって, 
\begin{itemize}
\item 任意の$x \in Q$について, \(f(\cdot,x) : S \to X\)は正則
\item 任意の$x \in P$について, \(f(\cdot,x)\)は\(K \cup T\)上で正則
\end{itemize}
このとき連続な変形
\(f_t : S \times P \to X\) ($t \in [0,1]$)であって, 
\(f=f_0\) かつ次を満たす. 
\begin{itemize}
\item $\{ f_t \}_{t \in [0,1]}$は\((S \times Q) \cup (T \times P)\)を保つ. 
\item 任意の\(t \in [0,1]\), $x \in P$について
\(f_t(\cdot,x)\)は$K$上で正則. さらに \(f_t\)は\(K \times P\)上に$f$に一様に近づく.(多分$\lim_{t \to 0}\sup_{K \times P}||f_t - f||_{X} =0$だと思う)
\item 任意の\(x \in P\)について, \(f_1(\cdot,x) : S \to X\)は正則 
\end{itemize}
\end{enumerate}
\end{thm}
他にもBOBIとかいっぱいあるがBOPとPOP以外は全て同値とのこと\footnote{\cite{FL11}のPOPは今でいうPOPAIのことだと思う. }

最後のPOPAIがわかりずらい. $K=\varnothing$にするとこの図のようになる.
ここで$\mathcal{O}(S,X) $は$S \to X$なる正則写像の集合, $\mathcal{C}(S,X) $は$S \to X$なる連続写像の集合とする.
\[
\xymatrix@C=80pt@R=30pt{
Q \ar[r]^{q \mapsto f(\cdot,q)} \ar@{^{(}-_>}[d]& \mathcal{O}(S,X) \ar@{^{(}-_>}[r] \ar[d]& \mathcal{C}(S,X) \ar[d]^{\cdot |_{T}} \\
P \ar[r]_{p \mapsto f(\cdot,p)} \ar@{.>}[ur]^{p \mapsto f_1(\cdot, p)}  \ar@{->}[urr]_{ \quad \quad p \mapsto f(\cdot, p)} & \mathcal{O}(T,X) \ar@{^{(}-_>}[r] & \mathcal{C}(T,X)
}
\]
つまり, 
\begin{itemize}
\item 連続関数で\(f : S \times P \to X\)を与える. これは$p \in P$ごとに$f(\cdot,p)  \in \mathcal{C}(S,X)$を与える($P$から右右上の矢印)
\item $q \in Q$ごとに$f(\cdot,q)  \in \mathcal{O}(S,X)$を与える(上の水平の矢印)
\item$p \in P$ごとに$f(\cdot,p)  \in \mathcal{O}(T,X)$を与える(下の水平の矢印)
\end{itemize}
となるものを与えると, \(f_t : S \times P \to X\)という変形で
\begin{itemize}
\item (POPの3つ目の条件と同値) $p \in P$について, $f_1(\cdot,p)  \in \mathcal{O}(S,X)$($P$から右上の矢印)
%\item (POPの3つ目の条件と同値)$p \in P$ごとに$f_{t}(\cdot,p)  \in \mathcal{C}(S,X)$を与える($P$から右右上の矢印)
\item (POPの1つ目の条件と同値) $Q \to P \to \mathcal{C}(S,X)$と$P \to \mathcal{C}(S,X) \to  \mathcal{C}(T,X)$の部分が可換
\end{itemize}
となるものが存在する. ($K=\varnothing$より2番目の条件はnull condition)

$K=\varnothing, T=\ast$にするともっと簡単になる. 
連続写像$f : P \to \mathcal{C}(S,X) $で$f(Q) \subset \mathcal{O}(S,X) $となるものは
$f _1: P \to \mathcal{O}(S,X)$にhomotopyで連続変形できる.
ただし, $\mathcal{O}(S,X), \mathcal{C}(S,X)$にはコンパクト開位相を入れる. 
%(ただしコンパクト開位相に関しては\cite{Iwa}参照)
図式にすると次のとおり
 \[
\xymatrix@C=50pt@R=20pt{
Q\ar@{->}[r]\ar@{^{(}-_>}[d] &\mathcal{O}(S,X) \ar@{^{(}-_>}[d]  \\
P\ar@{->}[r]_{f}\ar@{-->}[ru]^{f_1}  & \mathcal{C}(S,X)
}
\]
ただこの図の可換性に関しては注意が必要で, 
\begin{itemize}
\item $Q \to P \to \mathcal{O}(S,X) $の部分は可換.
\item $ P \to \mathcal{O}(S,X) \to \mathcal{C}(S,X) $の部分は"連続変形"をかませば可換.
\end{itemize}


\begin{cor}
\label{cor-Oka-hP}
Stein多様体\(S\)とOka多様体\(X\)について, 
\(
\mathcal{O}(S,X) \hookrightarrow \mathcal{C}(S,X)
\)
という包含写像は, コンパクト開位相に関して弱ホモトピー同値. (この性質を弱岡という)

つまり任意の$k \in \N$について次の同型($k=0$のときは全単射)が言える. 
$$\pi_{k}(\mathcal{O}(S,X)) \cong \pi_{k}(\mathcal{C}(S,X))$$

特にOka多様体は$h$-principle(任意のStein manifold $Z$からの連続写像$Z \to X$が正則写像とhomotopicである) を満たす.
\end{cor}

ホモトピー群の定義は\ref{defn-CW15-homotopy}参照.
$S^n$を$n$次元球面, $P=(1, 0, \ldots, 0)$, $x_0 \in X$をとって
$x_0\in \mathcal{O}(S,X)$を$S \to \{ x_0\}$という定数写像としたとき
$$
\pi_{n}(\mathcal{O}(S,X) ):= \{ f : S^{n}\to \mathcal{O}(S,X) \mid f(P)=x_0\}/\sim
$$
とする.
 ここで$f \sim g$を"$f$と$g$はホモトピック", つまり
「ある連続写像$H : S^{n}\times [0,1] \to \mathcal{O}(S,X)$で$H(x, 0)=f(x), H(x, 1)=g(x)$なものが存在する」として同値関係を入れる. 

\begin{proof}
%$k=1$の場合を考える(一番わかりやすいから)$\pi_{1}(\mathcal{O}(S,X))=\{ f : S^1 \to \mathcal{O}(S,X) \text{conti}\}/\underset{\text{homotopic}}{\sim}$である. 
\[
\pi_k(i) : \pi_{k}(\mathcal{O}(S,X) ) \rightarrow \pi_{k}(\mathcal{C}(S,X))
\]
とする. 全単射を示せば良い.

(単射性) 
$\gamma : S^{k}\to \mathcal{O}(S,X)$, $f(P)=x_0$かつ
$\pi_k(i)(\gamma)=0$とする. 
すると 
連続写像
$H : S^{k}\times [0,1] \to \mathcal{C}(S,X)$で$H(x, 0)=\gamma (x), H(x, 1)=x_0 \in \mathcal{C}(S,X)$なものが存在する

前の議論で, $Q=S^{k} \times \{ 0, 1\}, P=S^{k}\times [0,1] , K=\varnothing, T=\ast $とすると
 \[
\xymatrix@C=50pt@R=20pt{
S^{k} \times \{ 0, 1\}\ar@{->}[r]^{ \gamma, x_0}\ar@{^{(}-_>}[d] &\mathcal{O}(S,X) \ar@{^{(}-_>}[d]  \\
S^{k}\times [0,1] \ar@{->}[r]_{H}\ar@{-->}[ru]^{G}  & \mathcal{C}(S,X)
}
\]
となるので, ある$G : S^{k}\times [0,1] \to \mathcal{O}(S,X)$で$G(x, 0)=\gamma (x), G(x, 1)=x_0 \in \mathcal{O}(S,X)$なものが存在する. よって定義から$\gamma=0 \in \pi_{k}(\mathcal{O}(S,X) )$

(全射性)
$\delta : S^{k}\to \mathcal{C}(S,X)$, $f(P)=x_0$とする. 
前の議論で, $Q=\varnothing,  P=S^{k} , K=\varnothing, T=\ast $とすると \[
\xymatrix@C=50pt@R=20pt{
\varnothing\ar@{->}[r]\ar@{^{(}-_>}[d] &\mathcal{O}(S,X) \ar@{^{(}-_>}[d]  \\
S^{k}\ar@{->}[r]_{\delta }\ar@{-->}[ru]^{\widetilde{\delta}}  & \mathcal{C}(S,X)
}
\]
となるので, $\widetilde{\delta} : S^{n}\to \mathcal{O}(S,X)$で
$\delta$とhomotopicとなるものが存在する. 
よって$\pi_k(i)(\widetilde{\delta})=\delta$である.  

最後の主張に関しては(全射性)の証明において$k=0$, $S^0 =pt$とすれば言える.  
\end{proof}

\subsection{成り立つこと}
Oka多様体の性質をまとめると次のとおり.
\begin{enumerate}
  \setlength{\parskip}{0cm} % 段落間
  \setlength{\itemsep}{0cm} 
\item Stein OkaはGromov elliptic.
\item \cite{Kus21} OkaであるZariski開集合で被覆される多様体はOka
\item \cite[Prop2.8]{For13} 正則な covering map $\tilde{X} \to X$についてOkaである性質はうつりあう.\footnote{ここは$h$-principleと違う点. 種数2のリーマン面は$h$-principleを満たさないが, 普遍被覆は可縮なので$h$-princioleを満たす. この例はどちらもOkaではない. }
\item \cite[Cor 2.9]{For13} Okaならばstrongly Liouville(任意のnegative pshは定数のみ.)
\item \cite{For18} Okaならば$\C$-strongly dominable. つまり$f : \C^{\dim X} \to X$という全射正則写像がある.
\item \cite[Theorem 2.11]{For13} 正則ファイバー束$f : E \to X$でファイバーがOkaとする. このとき$E$がOkaと$X$がOkaは同値. ただsubmersionでは分かってないらしい. \footnote{ここもspecialと違う. $X$とファイバーがspecialだが$E$がspecialではないsubmersionの例がある. ファイバー束の場合はわからん.} 
\item Okaの1点blow upはOka. ただblow up locusの次元が1次元以上だと分かってないらしい. 
\item  \cite[Theorem 2.13]{For13} $\C^n, \C\mathbb{P}^n$, Grassmann多様体について, codim 2以上の$\C^n$内のalgebraic subvarietyの補集合はOka. 
\item RationalならGromov ellipticが2024年11月に示されたらしい\footnote{\url{https://arxiv.org/abs/2411.17892}}
\end{enumerate}

\begin{ex}
(1). 等質多様体$X$はGromov elliptic. 特にOka.
\begin{proof}
群$G$が$X$に正則かつ推移的に$X$に作用するとして
$$
s : X \times T_{e}G \to X
$$
を$s(x,v):=exp(v)x$とすればこれが$id_X$のdominateなsprayになる.
\end{proof}

(2). $T_X$が完備正則ベクトル場$V_1, \ldots, V_N$で生成される\footnote{正則柔軟(holomorphic soft?)というらしい}ならば$X$はGromov Ellipticである.
特にOkaである. 
\begin{proof}
$\varphi^{i}_{t} \in Aut(X)$を$V_i$のフローとして
$s : X \times \C^N \to X$を
$$
s(x, t_1, \ldots, t_N)=\varphi^{1}_{t_1} \circ \cdots \circ \varphi^{N}_{t_N}(x) 
$$
とすればこれが$id_X$のdominateなsprayになる.
\end{proof}
特に$X$ コンパクトかつ$T_X$がglobally generatedならOkaである.

(3) toricならばOka(\cite[Theorem 2.17]{For13}, Larussonによる結果)
特に$\C\mathbb{P}^n$やminimal Hirzebruch surfaceもOka

\begin{proof}$\C^{*}$がOkaであることを使い$X$がtorus factorを持たないとして良い. 
このとき$X = (\C^m \setminus Z)/G$で$Z$がalgebraic subvarietyでcodim2以上なので, $\C^m \setminus Z$はOkaであり, $G$が推移的に作用しているので従う. 
\end{proof}

(4) トーラスはOka. $\C^n$がOkaであるので. 
実は\cite[Cor 2.25]{For13}で$\C^n/\Gamma$から有限個の点を取り除いてもOkaになることがわかっている. 
これは「$\C^n \setminus A$で$A$がdiscreteなtame集合というもの」からのcovering mapを持つからである. 

(5) コンパクトリーマン面がOka であることは$g \le 1$と同値.
これは$g\ge2$ならば$\C \to X$という全射はないから. 

(6) 2次元surface$X$について, RCならばOka.
これはMMP$X \to X_{min}$を考えると
$X_{min}$はHirzebruch surface, $\C\mathbb{P}^2$となるからである. 

突き詰めると$X$uniruledで$X_{min}$が種数2以上のruled surfaceでないなら, $X$はOkaである

(7) Hopf surfaceもOka.

普遍被覆が$\C^n \setminus \{0\}$であるので. 

(8) \cite[Section 8]{Cam04} Okaならばspecial. これはOrbifold Kobayashi-Ochiaiを使う. 

%\begin{proof}
%$f : X \to (Y, \Delta)$をcore fibrationとする. 
%$X$はOkaなので, $s: \C^n \to X$というsurjective mapがあり, $f \circ s : \C^n \to (Y,\Delta)$を得る.
%しかし$(Y,\Delta)$はof general typeなのでKobayashi Ochiaiの定理より, 定数に限る(本来の主張は有限こという主張だが, 定数でないと無限こ作れる. )
%よって$\dim Y =0$より$X$special.
%\end{proof}
\end{ex}

\begin{ex}\cite[Subsection 2.8]{For13}
minimal Surfaceに関しては, Kodaira 次元で分けると次がわかる

$\kappa=- \infty$. 分類より$\C\mathbb{P}^2$かruled surface
であるので, 「ruled surface over curve with $g\ge 2$」以外はOkaである

$\kappa=0$.
torus, biellipticならOka. 
K3, Enriquesは不明.

$\kappa=1$ 
Buzzard-Lu \cite{BL00}より, $\C$-dominableとdense entire curveがあることは同値である.
その他全く不明. 

$\kappa =2$
SpecialでないのでOkaでもない.
\end{ex}

一つ面白い結果があったので, 書いておく
\begin{prop}\cite[Cor 2.39]{For13}
Kummer ならば $\C$-strongly dominable
\end{prop}
\begin{proof}
statified Okaならば$\C$-strongly dominableであることがわかっている.
ここでstatified Okaとは$X=X_0 \supset X_1\supset \cdots \supset X_m = \varphi$という閉部分代数多様体のstratificationがあって, $X_i \setminus X_{i+1}$がOkaになるものである.

Kummer 曲面 $X$はトーラスを$(z.w) \mapsto (-z,-w)$の作用でわり, そこででた16点の特異点でblow upしたものである.つまり16点の特異点でblow upしたrational curveの集合を$C=\cup_{i=1}^{16}R_i$とすれば
$X \setminus C $は$ T \setminus{\text{16点}}$ を$\Z_2$で割ったものと同型であり, $T \setminus{\text{16点}}$はOkaである.

$C$はrational curveの集まりよりOkaである. よって$X$はstatified Okaであり, $\C$-strongly dominableである
\end{proof}

\subsection{Oka多様体に関する未解決問題たち}

\begin{ques}
specialならばOkaか?
\end{ques}
ただこれ成り立つか微妙である. 
なおCampana Winkelmanは次のことを示した
\begin{thm}\cite{CW15}
$X$をprojective manifoldとする.
$X$が$h$-principle, つまり「任意のStein manifold $Z$からの連続写像$Z \to X$が正則写像とhomotopicである」を満たすとする.

このとき$X$はspecialでBrody hyperbolic K\"ahler manifold $Y$への正則写像$X \to Y$は定数である.
\end{thm}

$h$-principleであってもOkaではない. 
例えば可縮 ならばどんな写像も定値写像とhomotopicなので, 単位円板が反例になる. 

\begin{ques}
specialならば$h$-principleを満たすか? 
\end{ques}


\begin{ques}
$T_X$ nefなFanoはOka多様体か?
\end{ques}
Campana-Peternell予想が正しいなら$T_X$はglobally generatedなので, Okaとなる.
(Rational homogenusからOkaでもいい)

突き詰めるとこういうことになる.
\begin{ques}
\begin{itemize}
\item$T_X$ psefなrationally connectedはOkaか?
\item FanoやRCはOkaか?
\item $\kappa=0$ならOkaか?
\end{itemize}
\end{ques}

%正直ちょっと難しい感じがするが, ある程度は何かできなかなーとは思う. 



なおCampana-Winkelman 23 \cite{CW23}で, "RCはdence entire curveを持つこと"がわかっている. 
specialならばdence entire curveを持つは未解決だが, semiabelian varietyとbirationalならば言えている(らしい?).\footnote{これは2024年の多変数冬セミナーで山ノ井先生が講演していた.}


\newpage
\section{Voisonのサーベイ}
\subsection*{はじめに}
Claire VoisinのFibrations in algebraic geometry and applicationsのサーベイ\cite{Voi}をまとめた. 
が, このサーベイかなり難しく, 読めない部分が多かった. なので読んでいるときにわからなかった部分をあえて\xr{赤字}で残すことにした. \footnote{ただ著作権的にいいのかどうかわからないので, やばいのであれば削除します. }

\subsection{Introduction}
\begin{defn}
$f : X \dashrightarrow Y$をprojective manifoldかコンパクト複素多様体の射とする. 
$f : X \dashrightarrow Y$が\textit{fibration}とは, 
ファイバー連結なdominant mapとなること.
\end{defn}

fibration$ f : X \rightarrow Y$のいいところは
$X$の性質が$Y$やfiber $F$に遺伝する(逆も然り)ところである. 

\begin{enumerate}
\item $Y$と全てのファイバー$F$が Brody (resp. algebraically) hyperbolicならば, $X$もそう. \footnote{\xr{これは一般に言えてる??}}

\item $Y$と一般ファイバー$F$がgeneral typeならば, $X$もそう. (Iitaka conjectureがこのときは成り立つから)
\item $Y$と一般ファイバー$F$がRCならば, $X$もそう. (多分Graber-Harris-Star \cite{GHS03}??) 
\end{enumerate}

このサーベイでは3つのfibrationを扱う
\begin{enumerate}
\item Iitaka Fibration
\item MRC fibrationとShafarevich map ($\Gamma$-reduction)
\item Core fibration
\end{enumerate}

3つ目については次の予想から来ている
\begin{conj}
general typeならば, Kobayashi pseudometricが一般点で非退化?

$K_{X} \equiv 0$ならば, Kobayashi pseudometricが常に0?
\end{conj}
%Verbitsky が hyper-Kähler manifoldのときに何かやってるらしい. 
%Kamenova-Lehn 22も参照.
\cite{KL22}などHyper-Kahlerの場合に研究が進んでいるらしい. 
\begin{conj}
Special ならば, Kobayashi pseudometricが常に0?
\end{conj}

% [page 3]

projective/コンパクト K\"ahlerの仮定は, 「コンパクト閉代数(解析)集合のパラメーター空間(CHow-Barlet space)がコンパクトになるという部分で役にたつ(Fujikiの定理)
K\"ahlerの仮定を外してもlocal deformationの部分は大丈夫だが, コンパクト性はもはや成り立たない.
%(twistor family of a \( K3 \) surface など, Section 2.1, Example 2.2).

\subsection{Fibrations and holomorphic forms}

\subsubsection{General facts on fibrations and holomorphic forms}
 \( X \),  \( Y \) projective manifold/コンパクト Kähler manifoldとし, 
 \( f : X \to Y \)を全射とする. 

\begin{rem}
 \( f : X \to Y \)のStein分解\( f_{st} : X \to Y_{st} \)を取る.
 $f_{st} : X \to Y_{st}$はwith connected fiberとなる.
 よってほとんどの場合において$f$はfibrationであると仮定して良い. 
 
 ただ \( Y_{st} \)はnormalなだけなので, $Y$の滑らかさが必要な場合はこの議論は適応できない. 
\end{rem}


\begin{lem}\cite[Lemma 1.2]{Voi}
\label{lem-Voi-1.2}
 \(  f : X \to Y \ \) をproper射とするとき次を満たす. 
\begin{enumerate}
\item \( f_{*} : \pi_1(X) \to \pi_1(Y) \) について, $ f_{*}(\pi_1(X)) \subset \pi_1(Y)$は有限位数を持つ. さらにwith connected fiberならば,  $f_{*}(\pi_1(X)) =  \pi_1(Y)$
\item  \( f^* : H^i(Y,\mathbb{Q}) \to H^i(X,\mathbb{Q}) \) はHodge構造における単射である. 
\item  一般ファイバーが\( X_s \)が連結かつ, 任意の \( i>0 \)で\( H^0(X_s,\Omega^i_{X_s}) = 0 \)
ならば, \( f^* : H^0(Y,\Omega^i_Y) \to H^0(X,\Omega^i_X) \)は \( i \ge 0 \)で同型である. 
\end{enumerate}
\end{lem}

\begin{proof}

[1] $U \subset Y$をdense Zariski openで
\( f_U : X_U \to U \)がsmoothになるものをとる. 
$f_U$は位相的にlocally trivialなので
$$
1 \to \pi_{1}(F) \to \pi_1(X_U) \to \pi_1(U) \to 1
$$
である. 
 $Y$がsmoothだと \( \pi_1(U) \to \pi_1(Y) \) が全射になる. これより言える. \footnote{\xr{$\pi_1(X_U) \to \pi_1 (X)$の部分は?}}

[2, 3]
この状況下では, 
$\R$係数のcohomologyのleft inverse
$$
f^{*} : H^{i}(X, \R) \to H^{i}(Y,  \R) 
\quad
\alpha \mapsto f_*(\omega^d \cup \alpha)
$$
が存在する. 
ここで \( d := \dim X - \dim Y \) かつ,  \( \omega \) を \( X \)のK\"ahler formでファイバー上でvolume 1となるものとする. 
($H^{i}(Y,  \R) \cong H^{\dim Y - i}(Y,  \R)^{\vee}$とpoincare dualityで同一視しているので, 上のようなpushforwardが取れる.)
 よって\(f^* : H^i(Y,\mathbb{Q}) \to H^i(X,\mathbb{Q}) \)は単射になる.
 
 また一般点$s \in Y$について,
\begin{equation}
\label{eq-lem-Voi-1.2}
0 \to f^*\Omega_Y \to \Omega_{X|X_s} \to \Omega_{X_s} \to 0,
\end{equation}
から, $\Omega_{X|X_s}^{i}$のfiltrationを誘導する. 
仮定から$H^0(X_s,\Omega^i_{X_s}) = 0$なので
\[
H^0(X_s, \Omega^i_{X|X_s}) = H^0(X_s, f^*\Omega^i_Y).
\]
である. 
$X$上の正則 $i$-form $\alpha$は, 正則値$Y^0 \subset Y$上の \( f^*\Omega^i_Y \)の正則 $i$-form\( \beta \) と同一視される. 
この \( \beta \) は \( Y \) 上に
 \[ \beta' = f_*(\omega^d \wedge \alpha) \]
として, 拡張される. 
\( \alpha = f^*\beta' \)が\( X^0 = f^{-1}(Y^0) \)上で成り立つので, $X$上でも成り立つ.
\end{proof}

% [page 4]

%\begin{rem}\cite[Remark 1.3]{Voi}
%この議論はPluricanonical formについては適応できない. 
%Section 4で, \( \Omega^i_X \)のsectionで,   dense Zariski open set \( X_U = f^{-1}(U) \subset X \)上で\( f^*\Omega^i_Y \subset \Omega^i_X \)のsectionになるが,  \( \Omega^i_Y \)のsectionのpullbackにならないものを示す.
%しかし\ref{lem-Voi-1.2}の3はpluridifferential formsでも成り立つ(らしい?). 
%\end{rem}

\begin{rem}\cite[Remark 1.4]{Voi} 
\ref{eq-lem-Voi-1.2}の完全系列は, \[
0 \to N^*_{X_s/X} \to \Omega_{X|X_s} \to \Omega_{X_s} \to 0,
\tag{2}
\]
の特別な場合である. $s \in Y$がgeneralならば \( N_{X_s/X} = f^*T_{Y,s} \)となる. 
\end{rem}

\( X \)の \( d \)-次元多様体\( \mathcal{X}_s \subset X \)による covering familyによる 
以下のような図式を考える. ここで\( f \)を fibration, \( \phi \)をgenerically finite dominant mapとする. 
\[
\xymatrix{
\mathcal{X} \ar[r]^{\phi} \ar[d]_f & X \\
Y &
}
\]


\begin{lem}\cite[Lemma 1.5]{Voi}
\label{lem-Voi-1.5}
fibration \( f : X \to Y \)について, \( K_{X|_{X_s}} = K_{X_s} \)となる. 
また covering family \( (\mathcal{X}_s)_{s \in Y} \)について, general member \( \mathcal{X}_s \)について, 
\[ K_{\mathcal{X}_s} = K_{X|_{ \mathcal{X}_s}} + D 
\]
となる. ここで \( D \) は\( \mathcal{X}_s \)上effective divisorである.
\end{lem}
\begin{proof}
一般点$s \in Y$について,
\begin{equation}
\label{eq-lem-Voi-1.5}
0 \to f^*\Omega_Y \to \Omega_{X|X_s} \to \Omega_{X_s} \to 0,
\end{equation}
であり,  \( f^*\Omega_Y \) は$X_s$に沿って自明であるので,  detをとれば
$$
 K_{X|_{X_s}} 
  \underset{\text{def}}{\cong} \det \Omega_{X|X_s} 
 \underset{(\ref{eq-lem-Voi-1.5})}{\cong}
 \det f^*\Omega_Y|_{X_{s}} \otimes \det \Omega_{X_s}
\underset{\text{$f^*\Omega_Y |_{X_s}$が自明}}{\cong}
  \det \Omega_{X_s} \underset{\text{def}}{\cong} K_{X_s} 
$$
$\phi$はgenerically finiteなので, ($X$がsmoothであることも使って), ある ramification divisor $R$があって,  \( K_{X'} = \phi^*K_X + R \)となる. よって, \( X_s \not \subset R \)となる$X_s$について$D=R|_{X_s}$とすれば良い. 
\end{proof}

\begin{lem}\cite[Lemma 1.6]{Voi}
\label{lem-Voi-1.6}
 holomorphic pluridifferential forms(多重正則形式?), つまり$k \in \N$について\( H^0(X, \Omega_X^{\otimes k})\)はcompact K\"ahler多様体の上の双有理不変量である.
 \end{lem}
\begin{proof}
$X, Y$を compact K\"ahler 多様体とし,  \( \phi : X \dashrightarrow Y \)を双有理射とする. 
 Zariski open \( U \subset X \) で,  \( \mathrm{codim}\, X \setminus U \ge 2 \) かつ
  \( \phi|_{U} : U \to Y \) がwell-definedであるものを取る. 
  すると
  \[
\phi^* : H^0(Y,\Omega_Y^{\otimes k}) \to H^0(U,\Omega_U^{\otimes k}).
\]
という写像は, $\phi : U \to Y$が非退化であり
%がgenerically of maximal rankである(多分言っている意味は"一般点で$U \cap \phi^{-1}(V) \to V$がsmooth(biholo)になるので, Jacobi行列のrankが$\dim Y$となる"ということだと思う) 
微分が消えないので単射である. 
Hartogsの定理から, \( H^0(U,\Omega_X^{\otimes k}) =H^0(X,\Omega_X^{\otimes k}) \)である.
同様のことを$X$と$Y$を入れ替えれば, 同型が言える. 
\end{proof}

\subsubsection{Iitaka fibration}

\( X \) projective manifold, \( L \) line bundleとする.
\[
M(L) := \{ k \in \mathbb{N} \mid H^0(X, kL) \ne 0 \}
\]
という部分集合を定義する.
これは\( \mathbb{N} \) の部分monoidとなる.\footnote{$N$がmonoidとは$(a \cdot b) \cdot c = a \cdot (b \cdot c)$と単位元の存在を満たすもの. $M \subset N$が部分monoidとは$x, y \in M$ならば$x \cdot y \in M$なるもの}
よって, ある$k_0 \in \N$があって, 十分大きな$M(L)$の元は$k_0$の倍数となる. 

以下 \( k_0 \ne 0 \)とする. 
定義から$m \gg0$かつ$k_0$の倍数ならば,  \( H^0(X, kL) \ne 0\)である. 
次の例から$k_0 >1$となりうる
\begin{ex}
\(E\) 楕円曲線, $Y$ projective manifoldとする.
 \(L_0\) を 位数 \(k_0 >1\) の$E$上のtorsion line bundle, 
  \(L_Y\) を$L_Y$のample line bundleとする. 
\(X = E \times Y\),   \(L = L_0 \boxtimes L_Y\)とおくと$k_0 >1$となる. 
\end{ex}
% [page 1]
Iitaka 次元 \(\kappa(L)\) を
\(k \in M(L)\)において \(|kL|\)の線形系で定義される写像
\[
\phi_{kL} : X \dashrightarrow \mathbb{P}^N
\]
の像の最大次元とする. 
ただし\(M(L)=\{0\}\) のときは \(\kappa(L)= -\infty\) とする.
\(L=K_X\)のIitaka 次元を $X$のKodaira 次元という

例えば\(\kappa(L)=0\)  であることは, 
 \(M(L) \neq \{0\}\) かつ任意の \(k \in M(L)\)で \(h^0(X,kL)=1\)が成り立つことと同値である. 
 

\begin{thm}\cite[Theorem 1.7  (Iitaka)]{Voi}
\label{thm-Voi-1.7}
あるfibration
\[
\phi_L : X \dashrightarrow Y
\]
で次を満たすものが存在する
\begin{enumerate}
\item \(\dim Y = \kappa(L).\)
\item \(\tilde{\phi}_L : \tilde{X} \to Y, \ \tau : \tilde{X} \to X\) を \(\phi_L\)の不確定点解消とし, \(\tilde{\phi}_L\) の一般ファイバーを$F$とする.  
 \(\tau^*L|_{F}\) の Iitaka 次元は\(0\)となる.
\[
\xymatrix@C=50pt@R=20pt{
\widetilde{X} \ar@{->}[r]^{\tilde{\phi}_L} \ar@{->}[d]_{ \tau} &Y \\
X \ar@{-->}[ru]_{\phi_L} & 
}
\]
\end{enumerate}
さらにこのfibration $\phi_L$ は $Y$のbirationalを除いて一意である.
このfibration $\phi_L : X \dashrightarrow Y$ をIitaka fibrationという. 
\end{thm}

\(L=K_X\) の場合, \ref{lem-Voi-1.6}から任意の$r \in \N$について
$H^0(\widetilde{X}, K_{\widetilde{X}}^{\otimes r}) \cong H^0(X, K_{X}^{\otimes r}) $
である. 
つまり\(\tilde{\phi}_{K_X}: \tilde{X} \to Y\)が \(K_{\tilde{X}}\)のIitaka Fibrationと同一視できる. 
さらに\ref{lem-Voi-1.5}から, 一般ファイバー$F$について\(K_{\tilde{X}}|_{F}=K_F\) となる.
よって以上より下を得る. 
\begin{cor}\cite[Corollary 1.8]{Voi}
0以上の Kodaira 次元 \(\kappa\) を持つ多様体において, canonical fibration $\phi_{K_X}: X \dashrightarrow Y$で$Y$の次元は \(\kappa\) で, 一般ファイバーは Kodaira 次元 \(0\)を持つものが存在する.
\end{cor}

\begin{proof}[Proof of \ref{thm-Voi-1.7}]
任意の\(k \in M(L)\)について,  ある\( 0 \neq \sigma \in H^0(X, k_0L)\)をとって
包含写像\(\sigma : H^0(X,kL) \hookrightarrow H^0(X,(k+k_0)L)\)がある. 
よって次の可換図式を得る:
\[
\xymatrix@C=50pt@R=20pt{
X \ar@{-->}[r]^{\phi_{(k+k_0)L}} \ar@{=}[d] & \mathbb{P}^{N'} \ar[d]^{\pi} \\
X\ar@{-->}[r]^{\phi_{kL}} & \mathbb{P}^N 
}
\]
右の$\pi : \mathbb{P}^{N'} \to \mathbb{P}^{N}$は linear projectionである. 
よって\(k\) を十分大きくすれば, 
 \begin{equation}
 \label{eq-thm-Voi-1.7}
 \dim \mathrm{Im}\,\phi_{kL} = \kappa(L)
  \end{equation}
 となり\(\pi\)がgenerically finite rational map 
 \[\mathrm{Im}\,(\phi_{(k+k_0)L}) \dashrightarrow \mathrm{Im}\,\phi_{kL},\] 
 を引き起こす. これらは$X$で支配されるので, 
 $k$が十分大きければbirationalとなる.
 
 さて$k_0$で割り切れる十分大きな$k$を取り, \(Y\)を\(\mathrm{Im}\,\phi_{kL}\)のsmooth model とする. 
 さらに \(\phi : \tilde{X} \to Y\) を\(\phi_{kL}: X \dashrightarrow Y\)の不確定点解消とする. 
 
 示すべきことは以下の二つである. 
 \begin{itemize}
 \item[(a)]\(\phi\)の一般ファイバーが既約となること. 
 \item[(b)]  $\widetilde{X}$上のline bundle \(\tilde{L}=\tau^{*}L\)を一般ファイバーに制限したとき, そのIitaka次元は0.\footnote{semiampleのときと違い, $\tilde{L}$は$Y$のline bundleの引き戻しになるとは限らない}
 \end{itemize}

 
 \(Y_k = \mathrm{Im}\,\phi_{kL}\)とし,  \(\mathcal{O}_{Y_k}(1)\)を$Y_k \subset \mathbb{P}^N$としたときの\(\mathcal{O}_{\mathbb{P}^N}(1)\)の引き戻しとする. 
すると定義から$H^0 (X, L^{\otimes k} \otimes \phi_{kL}^*\mathcal{O}_{Y_k}(-1))$ 
に0でないsectionがある.($\sigma$を使う.)
よって0でないsection
\[ 
\alpha \in H^0 (\widetilde{X}, \tilde{L}^{\otimes k} \otimes \phi^*\mathcal{O}_Y(-1))
\] がある. 
ただし, \(\mathcal{O}_Y(1)\)は$\mathcal{O}_{Y_k}(1)$の$Y$への引き戻しとする. 
\(\mathcal{O}_Y(1)\)はbigなので, 任意の\(r \ge 0\)について, ある\(s > 0\)があって, 
\[
\phi_*(\tilde{L}^{\otimes r}) \otimes \mathcal{O}_Y(s)
\]
は$Y$上でgenerically globally generatedである. 
(Kodaira分解$\mathcal{O}_Y(1) \sim_\Q A + E$を使えば$E$以外でampleなので)
よって\(\phi\)の一般ファイバー $F$ について, 
$$
H^0(\widetilde{X}, \tilde{L}^{\otimes r} \otimes \phi^{*}\mathcal{O}_Y(s))
\twoheadrightarrow
H^0(F, \tilde{L}^{\otimes r} \otimes \phi^{*}\mathcal{O}_Y(s)|_{F})=
H^0(F, \tilde{L}^{\otimes r}|_{F})
$$
は全射である. 

さて(a), (b)を示す.
(b)が成り立たないとすると, 1次独立な section \(\sigma_1, \sigma_2 \in H^0(F, \tilde{L}^{\otimes r}|_{F})\)
がある. 
これより, $X$への拡張\(\tau_1,\tau_2 \in H^0(\widetilde{X}, \tilde{L}^{\otimes r} \otimes \phi{*}\mathcal{O}_Y(s))\)がある. 
\(\alpha^s\) をtwistして1次独立な section\(\tau_1',\tau_2' \in H^0(\widetilde{X}, \tilde{L}^{\otimes r + sk})\) が作れる. これは(\ref{eq-thm-Voi-1.7})付近のように$k$を十分大きくとったことに矛盾する. 
よって$H^0(F, \tilde{L}^{\otimes r}|_{F})$はたかだか一つのsectionを持つ. (sectionを持つのは$\alpha$を制限すれば良い)よって(b)がいえた.
(a)も同様で, もし既約でないのであれば上のように1次独立な sectionが作れてしまうためである. 
\end{proof}

%\begin{rem}\cite[Remark 1.9]{Voi}
%\(L=K_X,\)のとき, 上の構成で用いた\(\mathcal{O}_Y(1)\)は$K_Y$と何の関係もない. 
%\end{rem}

% [page 2]

\subsubsection{Castelnuovo-de Franchis and Bogomolov theorems}

\begin{lem}\cite[Lemma 1.10]{Voi}
 \label{lem-Voi-1.10}
\(X\) をprojective manifold/ コンパクト Kähler manifold とする. 
\(\alpha, \beta \in H^0(X,\Omega_X^1)\) を二つの1次独立な 1-formで
\[\alpha \wedge \beta = 0 \text{ in } H^0(X,\Omega_X^2)\]
であるものとする.
このとき, 種数\(2,\)以上のsmooth projective curve \(C\)への射\(\phi : X \to C\)と, 
$C$上の正則 1-forms\(\alpha_0,\beta_0\) があって, 
 \[
 \alpha = \phi^*\alpha_0 \quad \text{and} \quad \beta = \phi^*\beta_0
 \]
 となる. 
 より一般に \(g\)個の$X$上の1次独立な  \((1,0)\)-form \(\alpha_i\)で, 
  \[\alpha_i \wedge \alpha_j = 0 \text{ in } H^0(X,\Omega_X^2)
  \]
  を満たすならば,  種数\(g\)以上のsmooth projective curve \(C\), morphism \(\phi : X \to C\)があって, 
  $\alpha_i$は$C$上の正則 1-formsの引き戻しになる. 
\end{lem}

\begin{proof}
$X$がコンパクト Kähler なので, 正則1次形式ならばclosedである.
これは$\overline{\partial}$-closedならばKahlerよりharmonicとなり, $d$-closedとなるから. 

2つの1-form \(\alpha, \beta\) は各点で平行である. 
よってrational map \(f : X \dashrightarrow \mathbb{P}^1\) があって, 
 \[\alpha = f\beta\]
 とかける. (可縮な近傍をとってドラームコホモロジーを見る)
正則 1-formは$d$-closedなので, \(d\alpha=df \wedge \beta = 0\)となる. 
よって\(df\)もまた\(\alpha, \beta\)と各点で平行である.

 そこで\(f : X \dashrightarrow \mathbb{P}^1\)を考える. %(要は$x \mapsto (\alpha(x) : \beta(x))$)
 すると, $f$のfiber上で \(\alpha\)と\(\beta\)は消えている. \footnote{\xr{これはなぜかわからんかった.}}
 そこで次を考える.
 \begin{itemize}
 \item \(\tilde{f} : \tilde{X} \to \mathbb{P}^1\) を$f$のresolution
 \item \(F : \tilde{X} \to C\),  \(r : C \to \mathbb{P}^1\) を$\widetilde{f}$の Stein factorization
 \item   \(\tilde{\alpha}, \tilde{\beta}\)を \(\alpha, \beta\) の引き戻しとする. これは閉正則1次形式であり, これらは, \(F|_{U}: U \to C\)のヤコビ行列の階数が最大となる$U$上では
 \(F^*\Omega_C \subset \Omega_{\tilde{X}}\)のsectionとなる. 
 \end{itemize}
 \[
\xymatrix@C=50pt@R=20pt{
\widetilde{X}\ar@{->}[rd]^{\widetilde{f}}\ar@{->}[r]^{F} \ar@{->}[d]^{\pi} & C \ar[d]^{r} \\
X\ar@{-->}[r]^{f}& \mathbb{P}^1
}
\]

今\(\tilde{X}\) 上のK\"ahler form $\omega$で$F$のfiberで体積が1となるものをとって
$$
\alpha_0 := F_*\omega^{d-1}\wedge \tilde{\alpha}
$$
とする.
すると $\alpha_0$は$C$上の正則1次形式であり, \(\tilde{\alpha} = F^*\alpha_0\) となる. \footnote{\xr{これ$C$上正則?}}
よって, \(g(C) \ge 2\)である. $C$は有理曲線を含まないので, 
\(F \circ \pi^{-1}: X \to C\) はmorphismとなる. (\cite[1.1節参照]{KM98})
\end{proof}

\begin{thm}\cite[Theorem 1.11, Catanese]{Voi}
 \label{thm-Voi-1.11}
\(X\)をコンパクト Kähler manifoldとする. 
2つの1次独立な\(\beta,\beta' \in H^1(X,\C)\) で
\[ \beta \wedge \beta' = 0 \text{ in } H^2(X,\C)\]
となるものがあるとき, 
ある種数2以上の曲線$C$への射\(f : X \to C\)で
\(\beta,\beta'\) は\(H^1(C,\C)\)の元の引き戻しになるものが存在する. 

 より一般に$C$は種数$g$以上の曲線への射\(f : X \to C\)の存在は, 
 $g$次元ベクトル空間 \(V \subset H^1(X,\C)\) であって
 \[\wedge^2 V \equiv 0  \text{ in } H^2(X,\C)\] 
 となるものの存在と同値である. (上に関しては$\wedge^{2} H^1(X,\C) \rightarrow H^2(X,\C)$と見ている)
\end{thm}
\begin{proof}
 簡単のため\(g=2\)とする.
 \(\beta, \beta'\)の Hodge 分解を
\[
\beta = \beta^{1,0} + \beta^{0,1} \in H^{1,0}\oplus \overline{H^{1,0}}
\quad \beta' = \beta'^{1,0} + \beta'^{0,1} \in H^{1,0}\oplus \overline{H^{1,0}}
\]
とする. 
ここで \(\beta^{1,0}, \overline{\beta^{1,0}}, \beta'^{1,0}, \overline{\beta'^{1,0}} \in H^{1,0}\), つまり
正則1形式である.\footnote{$(p,0)$ 調和形式と正則$p$-formはcompact K\"ahlerでは同じ}
$\beta \wedge \beta' =0$の仮定から
\[
\beta^{1,0} \wedge \beta'^{1,0} = 0 
\quad 
\overline{\beta^{0,1}} \wedge \overline{\beta'^{0,1}} = 0 \text{ in } H^0(X,\Omega_X^2).
\]
\begin{equation}
\label{eq-thm-Voi-1.11}
\beta^{1,0} \wedge \beta'^{0,1}  + \beta^{0,1} \wedge \beta'^{1,0}  = 0 \text{ in } H^{1,1}(X).
\end{equation}
である. 
\ref{lem-Voi-1.10}より以下の場合に帰着できる. (そうでない場合は\ref{lem-Voi-1.10}から$X \to C$が存在する).
\begin{tcolorbox}[mybox]
\begin{center}
 \(\beta'^{1,0} = \lambda \beta^{1,0},\ \beta'^{0,1} = \mu \beta^{0,1}\) 
 となる$\lambda, \mu \in \C$が存在する.
 \end{center}
 \end{tcolorbox}
%\xr{この場合なぜpullbakになる?})
%背理法. 存在したと仮定する. %もしどちらも平行とすると
 %\(\beta'^{1,0} = \lambda \beta^{1,0},\ \beta'^{0,1} = \mu \beta^{0,1}\) 
 %となる$\lambda, \mu \in \C$がある, 
 \(\beta\) と \(\beta'\)は一次独立なので \(\lambda \neq \mu\) である.
\ref{eq-thm-Voi-1.11}に代入すると, 
$$
\mu \beta^{1,0} \wedge \beta^{0,1}+ \lambda \beta^{1,0} \wedge \beta^{0,1} = 0 \text{ in } H^{1,1}(X)
$$
を得る.  \(\mu \neq \lambda,\) なので, 
\begin{equation}
\label{eq2-thm-Voi-1.11}
\beta^{1,0} \wedge \beta^{0,1} = 0 \text{ in } H^{1,1}(X).
\end{equation}
である. そこで
\[
\eta:=\beta^{1,0} \wedge \overline{\beta^{0,1}}  \text{ in } H^{2,0}(X).
\]
とおく. 実は$\eta=0$である. なぜなら
\ref{eq2-thm-Voi-1.11}から,  \(\eta \wedge \overline{\eta} = 0 \text{ in } H^{2,2}(X)\)である. 
すると$\omega$をK\"ahler formとして, Hodge-Riemann bilinear relation から
%(例えばVoisonの本 6.3.3とかRoss-Toma21とか)より
$$
H^{2, 2}(X, \R) \overset{\wedge \omega^{n-2}}{\to} H^{n, n}(X, \R) \
$$
が正定値であるので, $\eta=0$となる. 
よって\ref{lem-Voi-1.10}より, $X \to C$が存在する.
\end{proof}


% [page 6]

\begin{cor}\cite[Corollary 1.12, Beauville, Siu]{Voi}
 \label{cor-Voi-1.12}
\(X\)をコンパクト Kähler manifoldとする. 
種数2以上の曲線$C$への定数でない写像 \(F : X \to C\)が存在することは, 
群準同型
\(\alpha : \pi_1(X) \to \pi_1(C)\) で像が有限指数であるものが存在することと同値. 
\end{cor}

\begin{proof}

 \(C\) は\(K(\pi,1)\)空間である\footnote{基本群が$\pi$でそれ以外のホモトピー群が全て自明な空間.  \url{http://pantodon.jp/index.rb?body=Kpi1} 離散群の分類空間ともみれて, $X$上の主$\pi$束と連続写像$X \to K(\pi,1)$が一対一に対応する.}
 よって\(\alpha\)は連続写像 \(f : X \to C\)を誘導する. 
 (作り方は $X$上の主$\pi$束$X_{\univ} \times C_{\univ}/\pi_1(X) $を作れば, 分類空間の定義から$X \to C$が作れる).
 
仮定から\(\mathrm{Im}\,\alpha \subset\pi_1(C) \)は有限指数なので, 
\[
f_{*} : H_1(X,\Z)=\pi_1(X)^{ab} \to H_1(C,\Z)=\pi_1(C)^{ab} 
\]
は有限のcokernelを持つ. ($\otimes_{\Z}\C$したら全射になる)
これよりpull-back \(f^* : H^1(C,\C) \to H^1(X,\C)\) は単射になり, 次の図式を得る.($\wedge$はcup-productを意味する)
 \[
\xymatrix{
H^1(C,\C)\otimes H^1(C,\C) \ar[r]^{\wedge} \ar[d]_{f^*\otimes f^*} & H^2(C,\C) \ar[d]^{f^*} \\
H^1(X,\C)\otimes H^1(X,\C) \ar[r]^{\wedge}  & H^2(X,\C)
}
\]

さて\(V := f^*H^{1,0}(C) \subset H^{1,0}(X)\)を考えると, \(f^*\) は単射なので, 
これは\(g(C)\)次元である.  
さらに\(\alpha,\alpha' \in H^{1,0}(C)\)について
\[\alpha \wedge \alpha' = 0 \text{ in } H^2(C,\C)\]
であるので上の図式から
 \[ f^*\alpha \wedge f^*\alpha' = 0 \text{ in }  H^2(X,\C) \]
よって \ref{thm-Voi-1.11}より種数2以上の曲線$C$への定数でない写像 \(F : X \to C\)が存在する.
逆に関しては, \ref{lem-Voi-1.2}より.
\end{proof}


\begin{thm}\cite[Theorem 1.13. Bogomolov and Campana]{Voi}
 \label{thm-Voi-1.13}
\(X\)をコンパクト Kähler manifold,  \(L \subset \Omega_X^k\)をline bundleとする. (ただしsaturatedであるとは限らない). 

\(\kappa(L) \ge k\)ならば, あるrational map \(\phi : X \dashrightarrow B,\)があって, 
 \(B\) は$k$次元のprojective manifoldであり, 
\(L\) はある$X$のZariski open上で\(\phi^*K_B \subset \Omega_X^k\)と一致する. 
\end{thm}

%\begin{rem}\cite[Remark 1.14]{Voi}
% \(B\)はof general typeであるとは限らない. 
%ただ \(\phi\) が"of general type in the sense of Campana"である. 
%理由としては
 %\(\phi^*K_B \subset \Omega_X^k\)がsaturatedとは限らないから. 
 %$L$ の saturation \(L\) の方がIItaka次元が\(\phi^*K_B\)より大きくなる. 
%\end{rem}

% [page 7]

%\begin{rem}\cite[Remark 1.15]{Voi}
%\ref{thm-Voi-1.13}から,  \(L\) は"generically"に
%ランク \(k\)の部分層 \(F\subset  \Omega_X\)を用いて\(\bigwedge^k F\) に等しいことがわかる. ($F=\phi^*\Omega_B \subset  \Omega_X$とする.)
%\end{rem}

\begin{proof}
Step1. 
$L$のIitaka fibrationが\(0 \ne s_0,\dots,s_N \in H^0 (X, L)\)で与えられる場合を考える. 
そのdominant rational map \(\phi : X \dashrightarrow B \subset \mathbb{P}^{N}\) で \(\dim B = k\)なものがある. 
\(H^0(X, L) \subset H^0(X, \Omega_X^k)\)によって, ある正則$k$形式$\alpha_i$があって, 
$$
s_i = \alpha_i \text{ in } H^0(X, \Omega_X^k)
$$
するとある有理関数\(\phi_i\)で \(\alpha_i = \phi_i \alpha_0\) となる. ($\phi_i =\frac{s_i}{s_0}$である)

\ref{lem-Voi-1.10}と同様に$X$がcompact K\"ahlerなので,  \(d\alpha_i=d\alpha_0=0\)である.
よって\(d\phi_i \wedge \alpha_0 = 0\)なので
$$
\mathcal{F}:=(d\phi_1, \ldots, d\phi_N) \subset \Omega_{X}
$$
とすると, これは$X$の一般点で\(\Omega_X\)のランク \(k\)の部分束である. 
つまり \(\mathcal{F}=\phi^*\Omega_B\)が一般点で成り立つ.

\begin{lem}\cite[Lemma 1.16]{Voi}
 \(W\)をベクトル空間, \(0 \ne u \in \wedge^k W\)とする
このとき
\[V := \{v \in W,\, v\wedge u = 0\} \subset W\]
は次元$k$以下である. 
さらに$\dim V=k$であることは, \(u\) が\(\wedge^k V\)の生成元であること(つまり\(u\) が分解可能)と同値である.
\end{lem}
よって$X$の一般点で, \(L\)と
\(\wedge^k(\phi^*\Omega_B) = \phi^*\Omega_B^k,\)
が一致する. 

Step2. 一般の場合. %これはCampanaの論文\cite{Cam04}の方法を用いる.
$s \in H^0(X, L^{\otimes N})$について, 
%In general (and this is the case considered by Campana), we use the well-known but important fact that given a line bundle \(L\) on \(X\) and a section \(s\) of \(L^{\otimes N},\) there exists a 
generically finite dominant map \(r : X' \to X\)と \(s' \in H^0(X', r^*L)\) であって, 
 \[ r_*(\mathrm{div}\,s') = \mathrm{div}\,s.\]
 となるものが存在する. 
 これは$X$の位数$N$で \(\mathrm{div}\,s.\)に沿って分岐するcyclic coverをとって, それをresolutionしたものとして構成する. 
 
 これを繰り返すと次をえる
 \begin{itemize}
 \item generically finite cover \(r : X' \to X\)
 \item \(r^*L \subset \Omega_{X'}^k\). さらに\(r^*L\)のIitaka fibraitonが\(H^0(X', r^*L)\)で与えられる.
 \item (Step 1の議論から) \(\phi' : X' \dashrightarrow Y'\)で\(Y'\)は$k$次元のprojective manifoldで,  \(r^*L=\phi'^*\Omega_{Y'}^k\) が\(X'\)の一般点で成り立つ. 
 \item $\phi_{L} : X \dashrightarrow Y$を$L$のIitaka fibrationとする. 
 \end{itemize}
 $s \in Y$を一般点としたとき, \(r^{-1}(X_s)\)の既約成分上において, 
 \(r^*L\) は Iitaka 次元 \(0\) である. よって次の可換図式をえる. 
\[
\xymatrix{
X' \ar@{-->}[r]^{\phi'=\phi_{|r^*L|}} \ar[d]_{r \text{ gen. fin.}} & Y' \ar[d]^{r'} \\
X \ar@{-->}[r]_{\phi_L} & Y
}
\]
そして\(L = \phi_{L}^{*}\Omega_Y^k.\)が\(X\)の一般点で成り立つ. 
%\xr{この構成ちょっとわからんのでCam04見るしかないかも}
\end{proof}

\begin{defn}\cite[Definition 1.17]{Voi}
$k >0$を整数とする. 
rank \(1\) subsheaf \(\mathcal{L} \subset \Omega_X^k\)で
\(\kappa(\mathcal{L}) = k\) となるものを, \emph{Bogomolov subsheaf}と呼ぶ
\end{defn}

\begin{rem}\cite[Remark 1.18]{Voi}
\ref{lem-Voi-1.6}から,  Bogomolov sheafの存在はbimeromorphically/birationally invariant である. 
%(もちろんコンパクト or projectiveは仮定する).
詳しく言うと, rank \(1\)の saturated subsheaf
\(\mathcal{L} \subset \Omega_X^k,\), 任意の birational map \(\phi : X' \dashrightarrow X,\)
について, \(\phi^*\mathcal{L} \subset \Omega_{X'}^k\) のsaturationが,  \(\mathcal{L}\)と同じIitaka次元を持つ. 
\end{rem}

% [page 3]


\subsection{Fibrations from families of cycles}

\subsubsection{Generalities about Hilbert schemes and Chow varieties}


\begin{thm}\cite[Theorem 2.1]{Voi}
 \label{thm-Voi-2.1}
 \( X \)をcomplex projective varietyとする. 
 このとき高々可算個の, projective schemes \( \mathcal{Z} \) とprojective scheme \( Y \) へのflat射$f: \mathcal{Z} \to Y$と射\( \phi : \mathcal{Z} \to X \)があって, 
\[
\xymatrix{
\mathcal{Z} \ar[r]^{\phi} \ar[d]_{f} & X \\
Y
}
\]
$f$のファイバー上で$\phi : \mathcal{Z}_y \hookrightarrow X$は埋め込みであり, 任意の部分scheme \( Z \subset X \) について, ある点$y \in Y$があって, \(Z= f(\mathcal{Z}_y) \) となる. 
\end{thm}

ちょっとわかりずらいので, \cite{SCV7}を参照すると以下の通り. 
\begin{cor}{\cite[Ch.8 Corollary 1.2]{SCV7}}
\label{cor-SCV7:Ch8:1.2}
$X$を解析空間とする. 
このときある解析空間 $\mathcal{D}:=\mathcal{D}(\mathcal{O}]_X)$と
subspace $Y \subset \mathcal{D} \times X$が存在して次を満たす.
\begin{enumerate}
\item $Y$は$\mathcal{D}$上でflatかつ$pr_{2}|_{Y} : Y\to X$はproper.
\item (universal property) $S$が有限次元解析空間, $Z \subset S \times X$をsubspaceとする. 
さらに$Z$は$S$上でflatかつ$pr_{1}|_{Z} : Z\to X$はproperとする. 
このとき, 正則写像$f : S \to \mathcal{D}$がただ一つ存在して, $Z \cong S \times_{\mathcal{D}} Y$となる
\end{enumerate}
\end{cor}
要は\ref{thm-Voi-2.1}の$Y$がHilbやChowに相当して, $\mathcal{Z}$がuniversal familiy $\{(y, x) \in  Y \times X \mid x \in y \}$ ($Y$の元は$X$のsubvarietyに相当する)である. 



この結果は\( X \) がコンパクト Kählerでも成り立つ. その場合"subschemes"の部分を"closed analytic subsets"にかえる必要がある. 
またその場合$Y$はK\"ahlerとは限らず, Fujiki classになる. 

%%%%%%%%%%%%%%%
\begin{comment}


コンパクト analytic space with Kähler desingularization となる. (たぶんFujiki classになると思う)


\ref{thm-Voi-2.1}の証明はanalyticだとかなりむずい. 
\xr{以下はそのまま翻訳をしたが, 正直よくわからなかった.}
\begin{proof}[\ref{thm-Voi-2.1}の証明の概略]


algebraicな場合は, \( X \subset \mathbb{P}^N \) とprojectiveに埋め込むと, $X$のHilbert schemesが
 \(\mathrm{Hilb}_{\mathbb{P}^N}\) のclosed algebraic subsetで, $X$の定義式
 ($X=\{f_1=\cdots f_r=0  \}$なる$f_i$)で消えるものとして定義できる. 

\( \mathbb{P}^N \)の場合, 
初めにdegreeが一様に有界なsubschemeはbounded familyをなすことを示す.
つまり高々有限この族でquasiprojective varietyで与えられたdegreeを持つsubschemeをパラメトライズするものの存在を示す. 
これはdegree\( d \) なるsubscheme\( Z \subset \mathbb{P}^N \) は
 あるdegree \( d' \)の方程式の解になることを示す.($d'$は$d$のみに依存. 実は$d'=d$ととれる)
 このboundednessは次の消滅定理から言える:
任意の$d \in \N$について,  ある$k_ 0\in \N$があって, 任意の$i>0, k \ge k_0$および
degree \( d \)のsubscheme \( Z \subset\mathbb{P}^N \)について, 
\[
H^i(\mathbb{P}^N, I_{Z / \mathbb{P}^N}(k)) = 0 
\]
である. (\( k_0 \) は$d$にのみ依存する. )
これより, \( P_{I_Z} \) を \( I_{Z / \mathbb{P}^N} \)のHilbert多項式として
\[
h^0(\mathbb{P}^N, I_{Z / \mathbb{P}^N}(k)) = P_{I_Z}(k),
\]
となることが言える. 

 Hilbert多項式は高々有限個なので,  Hilbert多項式$P$を一つfixすると, 
 \[
 \{Z \subset\mathbb{P}^N \mid  P_{I_Z} = P \}
 \hookrightarrow
 G(r_k, M) 
 \quad
 [Z] \mapsto I_Z(k) \subset H^0(\mathbb{P}^N, \mathcal{O}_{\mathbb{P}^N}(k)).
 \]
 と一気に Grassmannian $ G(r_k, M) $で\( r_k := P_{I_Z}(k) ,  M = h^0(\mathbb{P}^N, \mathcal{O}_{\mathbb{P}^N}(k)) \),
 となるものに埋め込める.
 
 最後に$k$を大きくとると, 
 上の像がGrassmannianのclosed algebraic subsetになることを示す. 
 これは\[
I_Z(k) \otimes H^0(\mathcal{O}_{\mathbb{P}^N}(1)) \to H^0(\mathcal{O}_{\mathbb{P}^N}(k+1))
\]
からなる\( I_Z(k+1) \)に入るimage( rank \( \le P_{I_Z}(k+1) \)である)ものから定義される方程式の0点に一致している
\end{proof}
\end{comment}
%%%%%%%%%%%%%%%%

\ref{thm-Voi-2.1}において, "$X$が"コンパクト"や, "projective/K\"ahler"は必要である
詳しくいうとclosed subvariety/analytic spaceの"localな存在"には必要ない(Barletなど)
しかしコンパクト性にはFujiki Classが必要(Bishopの定理を使うから)

%%%%%%%%%%%%%%%%%%%%%%%%%%%%
% [page 2]
\begin{comment}

closed analytic subspaces (in fact, rational curves)でcompact な変形空間を持たない例は以下の通り. これは$K3$曲面の twistor familyの例でもある

\begin{ex}\cite[Example 2.2]{Voi}
 \label{ex-Voi-2.2}
 $S$K3, $h$をそのKE計量, \( \omega \)を$h$に対応するclose positive K\"ahler formとする.

\( \sigma_S \)を$S$の正則2形式(nowhere vanishing), 
\( \sigma_S, \omega \) そして概複素構造\( I \) が$g = \mathrm{Re}\, h $のLevi–Civita connection に関して並行なものとする. 
今\( [\alpha, \beta, \gamma] \in  \mathbb{P}^2 \)\) としてその中のconic$Z \subset \mathbb{P}^2 $を
\[
\int_S (\alpha \omega + \beta \sigma_S + \gamma \overline{\sigma_S})^2 = 0.
\]
として定義する. 
すると\( 0 \neq t = (\alpha, \beta, \gamma) \in Z\) について
2-form 
\[ \omega_t := \alpha \omega + \beta \sigma_S + \gamma \overline{\sigma_S} \]
は定義から$S$上でparallelである. 
また上の定義式から\( \omega_t \)は$S$上でどこでもdegenerateする. 
一方, nowhere vanishingなので
ence it defines a rank 2 vector subbundle \( F \subset T_{S, \mathbb{C}} \) which is in fact transverse to \( T_{S, \mathbb{R}} \). To see this last point, observe that if \( \omega_t \neq 0 \) then \( \omega_t \wedge \overline{\omega_t} > 0 \) (this is true pointwise as forms on \( S \) as one easily checks). We then have an almost complex structure \( I_t \) such that \( F \) is the sheaf of (1,0)-tangent vectors for \( I_t \) and it is easy to show, using the flatness for the Levi–Civita connection, that \( I_t \) is integrable, giving rise to a deformation \( S_t \) of the complex structure on \( S \). Working a little more, this construction provides a compact complex threefold \( \mathcal{X} \) with a morphism \( f : \mathcal{X} \to \mathbb{P}^1 = C \) whose fibers are the \( K3 \) surfaces \( S_t \). \( \mathcal{X} \) is diffeomorphic to \( S \times C \) and the curves \( x \times C \), for \( x \in S \) are holomorphic \( \mathbb{P}^1 \)’s contained in \( X \). The family of these \( \mathbb{P}^1 \)’s is not holomorphic however; it is the real part of a 4-dimensional holomorphic family \( \mathcal{M} \) of deformations of these \( \mathbb{P}^1 \)’s. The fact that there is no compact family \( \mathcal{M} \) of such curves and their degenerations follows from the observation (used by Campana in [18]) that the deformation space \( \mathcal{M}_x \) of such \( \mathbb{P}^1 \) passing through one point \( x \in \mathcal{X} \) is 2-dimensional. Assume \( \mathcal{M} \) is compact, then we have accordingly a compact 2-dimensional family \( \overline{\mathcal{M}}_x \) of rational curves in \( X \) passing through \( x \). This implies by Mori’s argument [25] that there is a reducible \( \mathbb{P}^1 \) passing through this point. As the degree of these \( \mathbb{P}^1 \)’s over \( C \) is 1, this is possible only if the limiting rational curve has an irreducible component which is a rational curve in \( S \) passing through \( x \). Choosing for \( x \) a point which does not lie on a rational curve contained in \( S \), we get a contradiction.
\end{ex}

\begin{rem}
Campana used in [18] these rational curves and their Brody reparametrization to show that in the twistor family of any hyper-Kähler manifold, at least one fiber is not Kobayashi hyperbolic. This result is now essentially subsumed by Verbitsky’s work [54]. The limits are thus entire curves and not closed analytic subsets.
\end{rem}

\xr{上を解読する}
\end{comment}
%%%%%%%%%%%%%%%%%%%%%%%%%%%%%

\begin{thm}\cite[Theorem 2.4]{Voi}
 \label{thm-Voi-2.4}
\( X \) をprojective manifold/compact Kähler manifoldとし, \( x \in X \) をvery general pointとする.  
\( Z \subset X \) が$x$を通る closed subvariety ならば, 
ある covering family
\[
\xymatrix{
\mathcal{X} \ar[r]^{\phi} \ar[d]_{f} & X \\
Y
}
\]
で次を満たすものが存在する
\begin{itemize}
\item \( f  : \mathcal{X} \to Y\) fibration
\item \(\mathcal{X} ,  Y \) projective manifold /コンパクト Kähler, 
\item \( \phi : \mathcal{X} \to X\) 全射
\item ある\( s \in Y \)があって,  \( \phi|_{\mathcal{X}_s} : \mathcal{X}_s \to X\)は\( \widetilde{Z} \to X \)と同一視できる. ここで  \( \widetilde{Z} \) は \( Z \subset X \)の特異点解消である.
\end{itemize}
\end{thm}

よく使う方法は以下の通り(ただし$X$は全て$\C$上で定義されているとする. $\overline{\Q}$だと可算なので使えない)
\begin{itemize}
\item \cite{CH24}. $X \dashrightarrow Z$をMRCとする. 一般点$x \in X$について, $X_{x}$をファイバーとすると$Chow(X)$は可算個の成分しか持たないので, $Y \subset Chow(X)$でvery generalな$x \in X$について$X_x$をパラメトライズするものがある. よってuniversal family を$\mathcal{Z} $とすれば上の写像が作れる(よく使うのはそっから正規化をとるもの)
\item \cite{AD14}, \cite{Wang22}. $\mathcal{F} \subset T_X$というalgebraically integrable  foliationが存在するとき. この場合定義から, 任意の(もしくは一般の)$x \in X$について$L_x$というleafで$L_x \subset \overline{L_x}^{\mathrm{zar}}$がopenなので, $x \in X$についsubvariety $\overline{L_x}^{\mathrm{zar}}$を対応できる.
よって上と同じ方法で $Y \subset Chow(X)$でvery generalな$x \in X$について$\overline{L_x}^{\mathrm{zar}}$をパラメトライズするものがある. 
\end{itemize}

\begin{proof}

 \ref{thm-Voi-2.1}より高々可算個のprojective scheme $\psi : \mathcal{Z}\to X$で, $X$のsubvariety/subschemeをパラメトライズするものが存在する. 
$\mathcal{Z}_i$を$\mathcal{Z}$の既約成分する. 
\( \psi : \mathcal{Z}_i \to X\) はprojective射なので, 像はZariski closedである.
そこで
$$
B:= \bigcup_{i \mid \psi(\mathcal{Z}_i) \subsetneq X}\psi(\mathcal{Z}_i) 
$$ 
とする. $X \setminus B$の点は$X$のvery general point となる. 

 \( x \in X \setminus B\),  \( Z \subset X \)を$x$を通る subvarietyとする. 
 すると
\[
\xymatrix{
\mathcal{Z} \ar[r]^{\psi} \ar[d]_{f} & X \\
Y
}
\]
で\( Z \) はあるfiber \( \mathcal{Z}_y \hookrightarrow X \)と同一視できる.

そこで特異点解消をとって, \( Y \)はsmoothにして良い.
また\( \psi : \mathcal{Z} \to X \)は全射である. 
よって\( \mathcal{Z} \)の特異点解消を\( \mathcal{X} \) とすれば欲しいものが得られる.
%( \( Z \) reducedなので, generically smoothより,  \( \mathcal{Z} \) は \( \mathcal{Z}_y \)に沿ってgenerically smooth となる)
\end{proof}

\begin{rem}
もし \( Z \subset X \)がsmoothならば, \( \mathcal{Z} \)は\( \mathcal{Z}_y \)に沿ってsmoothであるので, 
\( \mathcal{Z}_y \)に沿って \( \mathcal{X} \cong \mathcal{Z} \) であり, つまり\(\mathcal{X}_y \cong   \mathcal{Z}_y  \cong Z \)である. 
\end{rem}

\begin{rem}
\( \mathcal{Z} \to X \) はgenerically finiteになる.(必要ならば$Y$を制限し, 特異点解消させることにより)

\( z \in \mathcal{Z} \) を
\begin{itemize}
\item  \( Y \)は \( y = f(z) \) で smooth
\item \( \mathcal{Z}_y \)で$z$を通るものもsmooth
\item  \( \psi : \mathcal{Z} \to X\) は\( z \)でsubmersion
\end{itemize}
このとき\( \psi|_{\mathcal{Z}_y}  : \mathcal{Z}_y \hookrightarrow X\) はimmersionであり
\( k := \mathrm{codim}\, \mathcal{Z}_y \)次元ベクトル部分空間 \( V \subset T_{Y,y} \) について
  \[ \psi_* : f^{-1}_*(V) \to T_{X,\phi(z)}
  \] 
  が同型になる. 
  
  つまり任意の$k$次元subvariety \( Y' \subset Y \) でtangent spaceが$y$で \( V \)になるものについて,
  \( \phi|_{\mathcal{Z}_{Y'}} : {\mathcal{Z}_{Y'}} \to X\)はgenerically finiteとなる. 
\end{rem}



%%%%%%%%%%%%%%%%%%%%%%%%%%%%
% [page 2]
\begin{comment}
\ref{thm-Voi-2.4}を使うと次がわかる. 
\begin{thm}\cite[Theorem 2.7]{Voi}
 \label{thm-Voi-2.7}
 \( X \)  smooth projective,  \( f : X \dashrightarrow X \) dominant rational map. 
とする. (例えば,  hyper-Kähler, Calabi–Yau manifold with Picard number 1など)

 次のどちらかが成り立つ
 \begin{enumerate}
\item very general point \( x \in X \)において, orbit \( \{x, f(x), \ldots, f^k(x), \ldots\} \) が$X$上でZariski dense
\item あるnontrivial fibration \( \phi : X \dashrightarrow B \) で \( \phi \circ f = \phi\)となるものが存在する. 
 \end{enumerate}
\end{thm}
証明はよくわからなかったので省略

\subsubsection{First application: fibrations associated with a dominant self-map}


\begin{proof}

簡単のため\( f : X \to X\)というmorphismの場合を考える. 
\( x \in X \)とする. 
\[ O_x := \overline{\{x, f(x), \ldots, f^k(x), \ldots\} }^{zar} \subset X
\]
とする. \( O_x \) は$f$でstableなので, 
その既約成分 \( \Gamma \) は preperiodic, つまりある\( l \ge 0, k > 0 \)があって, 
\(f^l(\Gamma) \) が\( f^k \) でstableとなる. 
$x$をvery generalに置き換えることで, 上の$k, l$などは$x$に依存しないようにできる. 
 
 \( \Gamma_x \) を$x$を通る \( O_x \) の既約成分とする. 
 \( \Gamma_x = X \)ならばそれで良い. 
 以下\( \Gamma_x \neq X \)とする. 
\(f^l(\Gamma) \) が\( f^k \) でstableかつ, \( f^l(x) \in f^l(\Gamma) \)である. $f^l(x)$もまたvery generalである
% ここで, Note that, as \( x \) is very general and \( f \) is dominant, the point \( f^l(x) \) is also a very general point of \( X \). 
% We thus know that through the general point \( f^l(x) \) of \( X \), there is a variety \( f^l(\Gamma_x) \) which is invariant under \( f^k \) for some \( k > 0 \). 
 
\ref{thm-Voi-2.4}より, 
これらのvariery($f^l(\Gamma)$たち)が族をなす. 
つまり\( X \) はs\( f^k \)でinvariantなvarietyの族で覆われる. 
そのvarietyのうち, 一番次元が小さいもので, 既約かつある\( f^k \)でinvariant, そして一般点 を通るものを考えると, これは$X$のmeromorphic fibration を作る. 
 This means that there is only one such variety passing through the general point of \( X \). This is obvious since each irreducible component of \( \Gamma_1 \cap \Gamma_2 \) passing through \( x \) satisfies these properties if \( \Gamma_1 \) and \( \Gamma_2 \) do (maybe up to changing \( k \)). We thus construct a fibration of \( X \) which is invariant under some power \( f^k \), for some \( k > 0 \). It is then immediate that \( f \) acts meromorphically on the base \( B \) of this fibration, providing \( g : B \dashrightarrow B \) such that \( g^k = \mathrm{Id}_B \). Then the composite rational map \( X \dashrightarrow B \dashrightarrow B / \langle g \rangle \) satisfies the desired properties. 
\end{proof}
\end{comment}
%%%%%%%%%%%%%%%%%%%%%%

% [page 1]

\subsection{MRC and $\Gamma$-fibrations}

\subsubsection{Rationally connected varieties}


\begin{defn}\cite[Definition 3.1]{Voi}
projective variety $X$がRCC(rationally chain connected)であるとは, 
任意の2点$x, y \in X$について, ある有理曲線のchain, 
つまり
\[ f_i : \mathbb{P}^1 \to X  \text{ $i = 1, \dots, N$}
\quad
 f_1(0) = x, 
 f_N(\infty) = y, 
 f_i(\infty) = f_{i+1}(0) 
 \text{ $i = 1, \dots, N-1$}
 \]
 となるものが存在すること.
 

$X$ がRC(rationally connected)とは任意の2つの一般点が一つの有理曲線で結べること. 
\end{defn}

\begin{ex}\cite[Example 3.2]{Voi}
projective variety $X$のprojective cone $C_X$はRCC. (vertex使えば二つで結べる)
$C_X$のresolution$\widetilde{C_X}$は$X$上の$\mathbb{P}^1$束になるので, $\widetilde{C_X}$はRCCとは限らない.(例えば$X$が種数2のリーマン面の時など)
よってRCCは birationally invariantではない. 

また$C_X$はRCとは限らない. 
例えば$X$そのものが有理曲線を持たないとき, RCとは限らない. (例えば$C_X$のresolutionを取って$X$に落とせば$X$に有理曲線が存在してしまう)
なおRCはbirationally invariantである. 
\end{ex}

%$X$が$\C$(多分標数0の体でも)上のsmooth varietyならば次が言える. 

\begin{thm}[{KMM92, Campana, \cite[Theorem 3.3]{Voi}}]
 \label{thm-Voi-3.3}
$X$を$\mathbb{C}$上のprojective manifoldとする.
次は同値
\begin{itemize}
\item $X$ RCC.
\item $X$ RC 
\item ある有理曲線 \( f : \mathbb{P}^1 \to X \) で \( f^*T_X \)が$\mathbb{P}^1$上のpositive ベクトル束である.
\end{itemize}
%Then for $x, y$ two general points of $X$, there exists an irreducible rational curve \( f : \mathbb{P}^1 \to X \)passing through $x$ and $y$, that is \( f(0)=x, f(\infty)=y \), so $X$ is rationally connected.Equivalently, there exists a rational curve \( f : \mathbb{P}^1 \to X \) such that \( f^*T_X \) is a positive vector bundle on $\mathbb{P}^1$. 
特に$\mathbb{C}$上のprojective varietyがRCであることは, ある(任意の)特異点解消$\widetilde{X} \to X$について$\widetilde{X}$がRCC(RC)であることと同値である. 
\end{thm}
ここで$\mathbb{P}^1$上のベクトル束
$E$がpositiveとは, Grothendieckの定理から$E=\bigoplus \mathcal{O}_{\mathbb{P}^1}(a_i)$と分解したときに$a_i >0$となることとする.

$x \in X$とし, 有理曲線$\mathbb{P}^1 \to X$で$f(0)=x$となるものが存在するとする. 
以下はDebarreの本\cite[Section 2.3]{Deb01}を参照.
$$
\textit{Mor}(\mathbb{P}^1,X):=\{ g : \mathbb{P}^1 \to X\} 
\quad
\textit{Mor}(\mathbb{P}^1,X, f|_{0}):=\{ g : \mathbb{P}^1 \to X \mid g(0)=f(0)\} 
$$
とする. これらはscheme/analytic spaceとなる. 
$\textit{Mor}(\mathbb{P}^1,X, f|_{0})$は別の見方ができる. 
evaluation map
$$
\rho : \textit{Mor}(\mathbb{P}^1,X) \to \textit{Mor}_(\{0\},X)\cong X
\quad g \mapsto g(0)
$$
の$f(0)=x$でのファイバーともみれる. 
さて$[g] \in \textit{Mor}(\mathbb{P}^1,X)$での接空間は\cite[Proposition 2.4]{Deb01}
より
\begin{equation}
\label{eq3-Deb01-Subsec2.3}
T_{\textit{Mor}(\mathbb{P}^1,X), [g]}
\underset{\cite[Proposition 2.4]{Deb01}}{\cong} 
H^0(\mathbb{P}^1, \mathcal{H}om(g^{*}\Omega_{X}^{1}, \mathcal{O}_{\mathbb{P}^1}))
\underset{\text{$X$ smooth}}{\cong} 
H^0(\mathbb{P}^1, g^{*}T_X)
\end{equation}
となる. 同様にして, $\textit{Mor}_(\mathbb{P}^1,X, f|_{0})$は
\begin{equation}
\label{eq2-Deb01-Subsec2.3}
T_{\textit{Mor}(\mathbb{P}^1,X, f|_{0}), [g]}
\underset{\text{tangent map}}{\cong} 
\mathrm{Ker} H^0(\mathbb{P}^1, g^{*}T_X) \to H^0(\{0\} , g^{*}T_X)
\underset{0 \to \mathcal{I}_0 \to \mathcal{O}_{\mathbb{P}^1} \to \mathcal{O}_{\mathbb{P}^1} /\mathcal{I}_0 \to 0 }{\cong}  H^0(\mathbb{P}^1, g^{*}T_X \otimes \mathcal{I}_0 )
\end{equation}
となる.\footnote{$H^0(\mathbb{P}^1, g^{*}T_X) \to H^0(\{0\} , g^{*}T_X)$は要するに$0 \in \mathbb{P}^1$を代入する写像. $H^0(\{0\} , g^{*}T_X) \cong \C^{\rk g^{*}T_X}$である.}
さて\cite[Theorem 2.6]{Deb01}より
$$
\dim_{[g]} \textit{Mor}(\mathbb{P}^1,X)
\underset{\text{$X$ smooth}}{\ge} 
\dim H^0(\mathbb{P}^1, g^{*}T_X) - \dim H^1(\mathbb{P}^1, g^{*}T_X)
$$
となる.\footnote{おそらく$\textit{Mor}(\mathbb{P}^1,X, f|_{0}), [g]$は$[g]$の周りでsingularになりうるのでtangent spaceの次元とは一致しない. \cite[Theorem 2.6]{Deb01}の主張では, $\textit{Mor}(\mathbb{P}^1,X)$は$[g]$の周りで$\dim H^0(\mathbb{P}^1, g^{*}T_X)$次元のnonsingular varietyの$\dim H^1(\mathbb{P}^1, g^{*}T_X)$個の方程式の0点集合となる.}
同様にして, (\cite[Section 2.3]{Deb01}参照)
\begin{equation}
\label{eq-Deb01-Subsec2.3}
\dim_{[g]}\textit{Mor}(\mathbb{P}^1,X, f|_{0})
\ge 
\dim H^0(\mathbb{P}^1, g^{*}T_X \otimes \mathcal{I}_0)
- \dim H^1(\mathbb{P}^1, g^{*}T_X \otimes \mathcal{I}_0)
\end{equation}


\cite{Voi}に戻ると, 
\( f(0)=x \in X\)なる有理曲線で$f^{*}T_X$がpositiveなものが存在するとする.
すると
\begin{align*}
\dim_{[f]}\textit{Mor}(\mathbb{P}^1,X, f|_{0})
&\underset{\ref{eq-Deb01-Subsec2.3}}{\ge}
\dim H^0(\mathbb{P}^1, f^{*}T_X \otimes \mathcal{I}_0)
- \dim H^1(\mathbb{P}^1, f^{*}T_X \otimes \mathcal{I}_0) \\
&= 
\dim H^0(\mathbb{P}^1, f^{*}T_X \otimes \mathcal{O}_{\mathbb{P}^1}(-1))
- \dim H^1(\mathbb{P}^1, f^{*}T_X \otimes \mathcal{O}_{\mathbb{P}^1}(-1)) \\
&\underset{f^{*}T_X\text{positive}}{=}
\dim H^0(\mathbb{P}^1, f^{*}T_X \otimes \mathcal{O}_{\mathbb{P}^1}(-1))
\end{align*}
となる. つまり$\textit{Mor}(\mathbb{P}^1,X, f|_{0})$は$[f]$の近くで
 \( h^0(\mathbb{P}^1, f^*T_X \otimes \mathcal{I}_0) \)次元(以上)を持っている.
\footnote{ \cite[Theorem 2.6]{Deb01}の主張から, 実は$\textit{Mor}(\mathbb{P}^1,X, f|_{0})$は$[f]$でsmoothであることがわかる. $\dim H^0(\mathbb{P}^1, g^{*}T_X)$次元のnonsingular varietyの$0=\dim H^1(\mathbb{P}^1, g^{*}T_X)$個の方程式の0点集合となるので} 
 そこで
evaluation map
\begin{equation}
\label{eq4-Deb01-Subsec2.3}
\rho_{\infty} : \textit{Mor}(\mathbb{P}^1,X, f|_{0})\to \textit{Mor}(\{\infty\},X)\cong X
\quad g \mapsto g(\infty)
\end{equation}
を考えると, でのdifferential map (tangent map)は
\ref{eq2-Deb01-Subsec2.3}と同様にして
\[
T_{\rho} : H^0(\mathbb{P}^1, f^*T_X \otimes \mathcal{I}_0 ) 
\to (f^*T_X \otimes \mathcal{I}_0)|_{\infty}
\quad
s \mapsto s(\infty)
\]
となる.  $f^*T_X$ positiveより上の写像は全射である.

よってSardの定理から$\rho_{\infty}$の正則値(ヤコビ行列のランクがfullランクである$X$の点たち)は一般点となる\footnote{一般のSardの定理だと臨界値($X$ から正則値を抜いたもの)がルベーグ測度0ということだが, 一般点となるのは代数的/解析的だから?}
これより代入写像
$\rho_{\infty} : \textit{Mor}(\mathbb{P}^1,X, f|_{0})\to X$
は$X$の一般点において全射になる. (正則値ならばlocalに射影になるので)
これは
つまり
\begin{center}
一般点$y \in X$について, ある$g \in \textit{Mor}(\mathbb{P}^1,X, f|_{0})$で$g(\infty)=y$となるものが存在する
\end{center}
ということである. $g : \mathbb{P}^1 \to X$かつ$g(0)=f(0)=x$より$X$はRCとなる. 

逆にRCだったら, $f^*T_X $ positiveな有理曲線の存在も同じで示す. (\cite[Proposition 4.9]{Deb01}がそれにあたる 代入写像$\rho_{\infty} : \textit{Mor}(\mathbb{P}^1,X, f|_{0})\to X$が$X$の一般点において全射を言い変えて, differential $T_{\rho} $が全射をいう.)
% is surjective, which by Sard’s theorem will be the case for a general $f$ if the general rational curve through $x$ which is a deformation of $f$ passes through the general point of $X$, the evaluation map (16) is surjective, which says that \( f^*T_X(-0) \) is generically generated by sections, hence \( f^*T_X \) is positive. The first statement in Theorem 3.3 (passing from chains to reducible curves) is more tricky as one needs to smoothify the chain of $\mathbb{P}^1$’s. One ingredient is the following lemma:

\ref{thm-Voi-3.3}の残っているのはRCCならばRCの部分である. 
まず次のLemmaを見る
\begin{lem}\cite[Lemma 3.4]{Voi}
\label{thm-Voi-3.4}
$X$を$\mathbb{C}$上のsmooth projective  varietyでRCC, $x \in X$  very general pointとする. 
このとき\( f : \mathbb{P}^1 \to X \)を$x$を通る有理曲線とするとき, \( f^*T_X \) semi-positive.
\end{lem}
\begin{proof}
$X$ RCCなので, 任意の$x \in X$について$x$を通る有理曲線が存在する.
よって
$$
ev : \mathbb{P}^1 \times \textit{Mor}(\mathbb{P}^1,X) \to X
\quad (s, f) \mapsto f(s)
$$
は全射になる. 
$\textit{Mor}(\mathbb{P}^1,X) =\bigcup B_i$と既約分解する. 
今$\textit{Mor}(\mathbb{P}^1,X)$は高々可算個のschemeなので, 
\ref{thm-Voi-2.4}と同じく
$$C=\bigcup_{i \mid ev(\mathbb{P}^1 \times B_i)\subsetneq X}ev(\mathbb{P}^1 \times B_i)
 \cup  \bigcup_{i \mid ev(\mathbb{P}^1 \times B_i)= X}
 \{ x \in X  \mid ev: \mathbb{P}^1 \times B_i \to X \text{において$x$は臨界値}\}
$$とすると , $X \setminus C$は$X$のvery general pointとなる. 

さて$x \in X \setminus C$, \( f : \mathbb{P}^1 \to X \)かつ$f(0)=x$とする. 
$C$の定義から, 既約成分$B \subset \textit{Mor}(\mathbb{P}^1,X)$ で$[f] \in B$かつ
\[
 ev :  \mathbb{P}^1 \times B \to X 
 \] 
 は全射かつ$(0, [f])$で正則点となるものが存在する. 
$ev$の$(0, [f])$での微分は
\[
T_{B,f} \to H^0(\mathbb{P}^1, f*T_X) \to (f*T_X)|_{\infty}
\]
でこれが全射になるので,  \( f^*T_X \) はgenerically globally generated となりsemipositiveとなる\footnote{\xr{ここわからんかった. $T_{\mathbb{P}^1 \times \textit{Mor}(\mathbb{P}^1,X) , (0, [f])} = \C\times H^0(\mathbb{P}^1, f*T_X) $になるのでは?(\ref{eq3-Deb01-Subsec2.3}参照)}}
\end{proof}

\begin{rem}
\( f^*T_X \) semi-positiveとなる有理曲線をfree rational curveとなる. 
free rational curveの存在はuniruled(任意の一般点に対してそれを通る有理曲線が存在する)と同値なので, 
この補題はRCCならばuniruledを示している. 
\end{rem}

\ref{thm-Voi-3.3}の難しい部分は
有理曲線のchainから有理曲線を作る部分である. 
$x$を $y$を結ぶ有理曲線のchain $C_1, C_2$があるとする(簡単のため2つにする).
\ref{thm-Voi-3.4}より
\( f_i^*T_X \) semipositiveを仮定して良い. \( f_i^*T_X \) semipositiveだけだと, RCのような有理曲線, つまり
(\ref{eq4-Deb01-Subsec2.3})の微分写像が全射になるような曲線は作れない. 

なので有理曲線のnormal bundleのpositivityを増やすようなfree rational curveを作ることで, $C_1, C_2$を繋ぐrational curve("leg")を作っていく. \footnote{\xr{ここもわからなかった. 多分\cite{CW23}でもあるようなComb smoothingの話??}}

\subsubsection{The MRC fibration}
以下基礎体は$\C$とする. 
\begin{thm}[{KMM92, Campana, \cite[Theorem 3.6]{Voi}}]
\label{thm-Voi-3.6}
projective manifold $X$について, あるrational map 
\( \phi : X \dashrightarrow B \)
で次を満たすものが存在する. 
\begin{enumerate}
\item very general point $x \in X$, general point $y \in X_{\phi (x)}$, および任意の有理曲線$C \subset X$で
$y$を通るものについて, $C \subset X_{\phi (x)}$.
\item $\phi$のファイバーはRC
\item $\phi$ almost holomorphic.つまり, general fiberでwell-definedである(不確定点が$B$全体を覆わない)
\end{enumerate}
\end{thm}
3の性質より, $\phi$の一般ファイバーはsmoothである. 
この3の性質が本質的であり, 1の条件よりもっと強いことが言える

\((1')\)  very general point \( x \in X \)と有理曲線 \( C \subset X \) で\(C \cap X_{\phi (x)}\neq \varnothing \)ならば,  \( C \subset X_{\phi (x)}\).

% [page 3]

\textit{Sketch  proof of \ref{thm-Voi-3.6}}
$X$ non-uniruledのときは$\phi =id_X$とすればいいので, $X$ uniruledを仮定する. 
\ref{thm-Voi-3.4}より, very general point \( x \in X \) について, あるfree rational curve \( f : \mathbb{P}^1 \to X \) で\( x \)を通るものがある.

  \( x' \in C_x = f(\mathbb{P}^1) \) となるvery general pointは$X$においても very generalなので, 
\ref{thm-Voi-3.4}を\( x' \)にも適応して, free rational curve\( C'_{x'} := \mathrm{Im} f', f' : \mathbb{P}^1 \to X \)で \( x' \)を通るもので$C$と異なるものがある. 
よって, \( S_x = \bigcup_{x' \in \mathbb{P}^1} C'_{x'} \)という\( x \)を通る曲面ができる. これはRCCとなる.
実は実際にはRCである. (ここには議論がいるがfreeであることが効く)

\begin{lem}\cite[Lemma 3.9]{Voi}
\( f : \mathbb{P}^1 \to X, f' : \mathbb{P}^1 \to X \)を二つの free rational curvesで\( f(0) = f'(0') \)となるものとする. 
このとき\( C'' = C \cup C' \) ( \(0 = 0' \)でくっつける)のある\( (f, f') : C'' \to X \)のsmoothificationがある.

つまり, ある射 \( g : \mathcal{C} \to X \)で \( \phi : \mathcal{C} \to B \) は曲線$B$上の滑らかな曲面で,  smooth fiber は$\mathbb{P}^1$かつ, ある$b_0 \in B$で
 \( C_{b_0} \cong C'' \)かつ\( g|_{C_{b_0}} = (f, f') \)となるものが存在する. 
 
 さらに\( g_b : \mathbb{P}^1 \to X \)は一般点 \( b\in B \)についてfreeである.
 \[
\xymatrix{
\mathcal{C} \ar[r]^{g} \ar[d]_{\phi} & X \\
B
}
\]
\end{lem}
この補題はweak gluing lemmaと言われる(Miyaoka-Peternell\cite{MP97}参照)
これより \( C \cup C' \)を含む曲面$S_x$で任意の一般点は$x$とfree rational curveで結ばれるものが存在する. 

これを繰り返して($C_x$を$S_x$に取り替えて)
very general point $x \in X$について, あるvariety\( X_x \)で$x$を通り, (1)を満たすようなものが存在する. 
また(2)も成り立つ. 実際にvery general point \( x' \in X_x \)について, 
 \( X_{x'} = X_x' \) でgeneral point \( y \in X_x \)は\( x' \) と有理曲線で結ばれる. 
 よって \( X_x \)はRCとなる. (というかこれは構成からそうなるのでは?)
 
あとはfibrationを作れば良い. これは\ref{thm-Voi-3.10}の応用である. 
上の構成から\( \tilde{\phi} : \tilde{X} \to B \)というfibrationが作れる, ここで \( \tau : \tilde{X} \to X \) を$X$のblowupの列とする. \footnote{\xr{ここちょっとわからん. $B$は上の$B$??}}
$\tau$の1次元以上のfiberはRCなので,  $E:= \mathrm{Ex}(\tau)$は$B$にdominateしないか, $\phi : E \to B$が$\tau$を経由するかである. これから$B$の一般点上で$\phi$はwell-definedとなる.
\qed

(3) に関してはCampanaの別証明もある.(多分こっちはDouady space使うはず.)
% [page 4]

%An essential property satisfied by the MRC fibration map, and already used above to prove (iii) in Theorem 3.6 and (i’) in Remark 3.8, has been established by Graber–Harris–Starr \cite{32}.

\begin{thm}[{Graber-Harris-Starr \cite{GHS03}, \cite[Theorem 3.10]{Voi}}]
\label{thm-Voi-3.10}
 $X$ projective manifold $\phi : X \dashrightarrow B$ MRC fibrationとするとき, 
  $B$ は non-uniruled.
\end{thm}


\begin{thm}\cite[Theorem 3.12]{Voi}
\label{thm-Voi-3.12}
$f : Y \to C$ projective 射で一般ファイバーがRC, $C$ をsmooth curveとする.
このとき$f$のsection $C \to Y$がある. 
もっと強く$f$のsectionで$f$の一般ファイバーの一般点を通るものが存在する.
\end{thm}

\textit{Proof of \ref{thm-Voi-3.12} $\Rightarrow$ \ref{thm-Voi-3.10}}
$\phi : X \dashrightarrow B$ MRC fibration, $b \in B$を一般点とする.
resolutionをとって, $\phi : X \rightarrow B$が射であると仮定して良い. 

$B$がuniruledとすると, 一般点$b \in B$について, 
ある定数でない\( \alpha : \mathbb{P}^1 \to B \) で $b \in B$を通るものが存在する.
$\alpha(0)=b$として良い. 
$X_\alpha := X \times_B \mathbb{P}^1$とする. $b \in B$がgeneralなので, 既約成分$X_{\alpha,d} \subset X_\alpha$で
$$
\phi_\alpha : X_{\alpha,d} \twoheadrightarrow \mathbb{P}^1
$$
がdominantになるものがある. 
さらに $X_{\alpha,d} \to\mathbb{P}^1$は$0$のファイバーでsmoothとして良く, $0$のファイバーは$X_b= \phi^{-1}(b)$と同一視される. 
よってresolutionを
\[ 
 \tilde{\phi}_\alpha : \tilde{X}_{\alpha,d} \to \mathbb{P}^1
\]
をとって, \ref{thm-Voi-3.12}を適応すると, 
 $\tilde{X}_{\alpha,d}$には有理曲線$C$で$C \not \subset\tilde{\phi}_{\alpha}^{-1}(0)$かつ$ \tilde{\phi}_{\alpha}^{-1}(0)$の一般点を通るものが存在する. 
 $$
 \tilde{X}_{\alpha,d} \to X_{\alpha} \overset{p_1}{\to} X
 $$
 という合成写像によって, $X$にも 有理曲線$C$で, $C \not \subset X_b$かつ$X_b$の一般点を通るものが存在する. これはMRCの(1)に矛盾する.
 \qed

% [page 1]

\subsubsection{Mumford's conjecture}


\begin{lem}\cite[Lemma 3.13]{Voi}
\label{lem-Voi-3.13}
\( X \) RCならば,  \( H^0(X, \Omega_X^{\otimes k}) = 0 \) for \( k > 0 \).
\end{lem}

\begin{proof}
\ref{thm-Voi-3.3}より, very free rational curve, つまり \( f : \mathbb{P}^1 \to X \)で
  \[ f^* T_X = \oplus_i \mathcal{O}_{\mathbb{P}^1}(a_i) 
  \quad \text{and} \quad
  a_i >0
  \]
なものが存在する. このevaluation map
\begin{equation*}
\rho : \mathbb{P}^1\times \textit{Mor}(\mathbb{P}^1,X, f|_{0})\to \textit{Mor}(\{pt\},X)\cong X
\quad (t, g) \mapsto g(t)
\end{equation*}
の$(t, g) \mapsto g(t)$でのdifferential map (tangent map)は\ref{eq2-Deb01-Subsec2.3}と同様にして
\begin{equation}
\label{eq-lem-Voi-3.13}
(T_{\rho})_{(t, g) } : T_{\mathbb{P}^1, t} \times H^0(\mathbb{P}^1, g^*T_X \otimes \mathcal{I}_0) 
\to (f^*T_X \otimes \mathcal{I}_0)|_{g(t)}
\quad
(v, s) \mapsto v \cdot s(g(t))
\end{equation}
となる.
ここで$s(g(t)) \in \C$とはsection $s$に$g(t) \in X$を代入したもので, 代数的にいえば$\mathcal{O}_{X, g(t)}/m_{X, g(t)} \cong \C$への行き先の値である. 

$g^*T_X \otimes \mathcal{I}_0 \cong \oplus_i \mathcal{O}_{\mathbb{P}^1}(a_i - 1) $ 
で$a_i - 1 \ge 0$なので, 
\ref{eq-lem-Voi-3.13}の$(T_{\rho})_{(t, g) }$は全射である. 
よってSardの定理から, $T_{\rho}$は$X$の一般点で全射になる. 
つまり言い換えると, 一般点$x \in X$について, ある$h \in \textit{Mor}(\mathbb{P}^1,X, f|_{0})$で
$h^{*}T_X$がpositiveなものが存在する. 

さて$s  \in H^0(X, \Omega_X^{\otimes k})$とする. 
一般点$x \in X$ について, 上の$h : \mathbb{P}^1 \to X$をとると, $h^{*}s \equiv 0$である. よって$s_x =0$である. $x$は一般点なので$s \equiv 0$となる. 
%よって$\rho : \mathbb{P}^1\textit{Mor}(\mathbb{P}^1,X, f|_{0})\to X$は$X$の一般点で全射になる. analytic space/scheme$B$で

%が\( \mathbb{P}^1 \)の任意の点で微分が全射になるものが存在する.
% Note that the generic deformation \( f_t : \mathbb{P}^1 \to X \) also satisfies the property that \( f_t^* T_X = \oplus_i \mathcal{O}_{\mathbb{P}^1}(a_i') \), where all \( a_i' \)'s are \( > 0 \), since this is a Zariski open property on families of vector bundles on \( \mathbb{P}^1 \). Next, as \( f^* T_X = \oplus_i \mathcal{O}_{\mathbb{P}^1}(a_i) \), with \( a_i > 0 \) for any \( i \), we get that \( H^0(\mathbb{P}^1, f^* \Omega_X^{\otimes k}) = 0 \) for \( k > 0 \) and similarly \( H^0(\mathbb{P}^1, f_t^* \Omega_X^{\otimes k}) = 0 \) for \( k > 0 \) for a general deformation \( f_t \) of \( f \). Thus \( H^0(X, \Omega_X^{\otimes k}) = 0 \) for \( k > 0 \) since the curves \( f_t(\mathbb{P}^1) \) sweep-out a dense Zariski open set of \( X \). \qedhere
\end{proof}

%The converse statement is the following conjecture attributed to Mumford.

\begin{conj}[{Munford conjecture \cite[Conjecture 3.14]{Voi}}]
\label{conj-Voi-3.14}
\( H^0(X, \Omega_X^{\otimes k}) = 0 \) for \( k > 0 \)ならば, \( X \) RC?
\end{conj}

この予想は下の予想(nonvanishing conjecture)に帰着される. 

\begin{conj}\cite[Conjecture 3.15]{Voi}
\label{conj-Voi-3.15}
 \(\kappa(X)= -\infty \)ならば,  \( X \) uniruled?
\end{conj}
BDPPよりこの予想は「$K_X$ psefならば$\kappa(X)\ge 0$」と同値である. なので激ムズ予想である. 

\begin{proof}[Proof of \ref{conj-Voi-3.15} implies \ref{conj-Voi-3.14}]
\( H^0(X, \Omega_X^{\otimes k}) = 0 \) for \( k > 0 \)とし, 
\( X \dashrightarrow B \) をMRCとする. $\dim B>0$として矛盾を示す. 
すると  \( l > 0 \)について, 
\[ H^0(B, K_B^{l}) \hookrightarrow H^0(X, \Omega_X^{\otimes l \cdot\dim B}) 
\]
となるので, 
 \( H^0(B, K_B^{\otimes l}) = 0 \) for \( k > 0 \)が言える. 
 \ref{conj-Voi-3.15} が正しいとすると, $B$はuniruledになる. 一方\ref{thm-Voi-3.10}より$B$ non-uniruledに矛盾する. 
 \end{proof}

\begin{thm}[{ \cite{BDPP13}, \cite[Theorem 3.16]{Voi}}]
\label{thm-Voi-3.16}
\( K_X \) not psef ならば \( X \)  uniruled.
\end{thm}


% [page 2]

\subsubsection{An application to Calabi-Yau manifolds}

以下, $K_X$が自明なprojective manifoldについて次にように定義する. 
\begin{itemize}
\item $X$がCalabi-Yau(CY)とは次元$k$でholonomy \( SU(k) \)となるもの. 
\item $X$がhyper-Kähler(HK) とは次元 \( 2k \)でholonomy \( Sp(k) \)となるもの.
\end{itemize}


\begin{thm}\cite[Theorem 3.17]{Voi}
%\label{thm:3.17}
\( X \)  既約単連結CYとする.
このとき任意のdominant rational map \( \varphi : X \dashrightarrow Y \)について, 
$Y$がprojective manifoldかつ$0 < \dim Y < \dim X$ならば 
$Y$はRC
\end{thm}

\begin{proof}
Step 1. $Y$がuniruledであることを示す. 

\ref{thm-Voi-3.16}のBDPPより, $K_Y$がpsefでないことを示せば良い. 
背理法. $K_Y$ psefと仮定する. 
rank 1 subsheaf \( \varphi^* \Omega_Y^k \)(\( k = \dim Y \))を考えることで, 
あるline bundle  \( D \subset \Omega_X^k \) で
$$
D = (\varphi^* \Omega_Y^k )^{\mathrm{sat}} \subset \Omega_X^k 
$$
となるものが存在する. 
今ample line bundle \( H \)について, $X$ CYなので, 
\begin{itemize}
\item \( \Omega_X \) is \( H \)-stable 
\item \( \Omega_X^k \) is \( H \)-polystable 
\end{itemize}
となる. これはもっと厳密に言えば
\begin{itemize}
\item  \( SU(n) \) holonomyの場合(CYの場合), \( \Omega_X^k \) stable
\item \( Sp(n) \) holonomyの場合(HKの場合),  \( \sigma_X \)という正則2形式を使って
\[
\Omega_X^k = \bigoplus_{2r \le k} \sigma_X^r \wedge \Omega_{X,0}^{k-2r},
\]
と分解される. ここで\( 2n = \dim X \) かつ\( \Omega_{X, 0}^{k-2r} := 
\mathrm{Ker} (\wedge (\sigma_X)^{n-k+2r+1}: \Omega_X^{k-2r} \to \Omega_X^{2n-k+2r+2}) \)である. 
この \( \Omega_{X,0}^{k-2r} \) はstableである. 
\end{itemize}

今 \( D \subset \Omega_X^k \) psefなので$\mu_{H^{n-1}}(D) \ge 0$である. 
よって
\begin{itemize}
\item CYの場合,  \( \Omega_X^k \) stableより起こり得ない
\item HKの場合, ある$r$があって$D = \sigma_X^r \wedge \Omega_{X,0}^{k-2r}$とならざるを得ない. 
よって$D$がrank 1より$r=\frac{k}{2}$となる($k$は偶数も仮定される)
$D = \sigma_X^{\frac{k}{2}} \wedge \Omega_{X,0}^{0}$は起こり得ない, というのも
$$
\Omega_{X,0}^{0}
 := 
\mathrm{Ker}  (\wedge (\sigma_X)^{n+1} : \Omega_X^{0} \to \Omega_X^{2n+2})
 = \mathcal{O}_{X}
$$
である.
よって$D \cong \sigma_X^{\frac{k}{2}} \cdot \mathcal{O}_{X} \subset \Omega_X^k$である. 
\end{itemize}
以上よりCYの場合は起こり得ない. 

Step 2. $Y$がRCとなること. 
$\varphi : X \dashrightarrow Y$において$Y$ uniruledなので
このとき
\[
 \varphi'=f \circ \varphi : X \overset{\varphi}{\dashrightarrow} 
 Y \overset{f \text{ MRC }}{\dashrightarrow} B 
 \]
 を考える. 
 すると$\varphi : X \to B$についてStep 1を使えば$B$もuniruledとなる. 
 よって\ref{thm-Voi-3.10}より$\dim B=0$となる. 
つまり$Y$はRC.
\end{proof}

% [page 3]

\subsubsection{Shafarevich maps (or $\Gamma$-reductions) and Shafarevich conjecture}

複素多様体\( M \)が正則凸(holomorphically convex)とは, Stein空間への proper 正則写像 \( M \to U \)を有することである. 

\begin{conj}[{Shafarevich conjecture, \cite[Conjecture 3.18]{Voi}}]
%\label{conj:3.18}
\( X \)  projective manifold/コンパクト Kähler manifoldとする. 
その普遍被覆 \( \widetilde{X} \)は正則凸である. 
\end{conj}
普遍被覆ではなく, intermediate coversなら反例がある(Napier 90, due to Narasimhan) 
 general lattice \( \mathbb{Z}^4 \simeq \Lambda \subset \mathbb{C}^2 \),
  \( X :=\mathbb{C}^2 / \Lambda \) general complex torusとする. 
  general rank 3 submodule \( \Lambda' \subset \Lambda \)で, 
   \( \mathbb{C}^2 / \Lambda' \to  X \) というcoverを考えると, これは 正則凸ではない, というのも, $\mathbb{C}^2 / \Lambda'$はコンパクトではないが非定数正則関数を持たない. 

さて普遍被覆 \( \widetilde{X} \)が正則凸であるとする. 
すると \( \widetilde{X} \)の正則関数によって, proper 射 \( F : \widetilde{X} \to U \) で
 \( F \)のファイバーがコンパクト closed analytic subsets となるものが存在する. 
 \(\pi : \widetilde{X} \to X \)とする. 
すると
任意の closed analytic subsets \( Z \subset X \)で
\[ \pi_1(Z') \to \pi_1(X)  \text{ has finite image } \]
となるもの(\( Z' \to  Z \)はresolutionとする)は, $\widetilde{X}$に引き戻すと\( F \)のファイバーに入る.そしてその逆も然りである. \footnote{厳密にいうと\( \pi_1(Z') \to \pi_1(X) \) has finite imageであっても,  \( \widetilde{Z} \subset \widetilde{X} \)という$\widetilde{X}$への引き戻しはclosed コンパクト subsetの高々加算個のunionということしか言えないらしい. 実際Bogomolov-Katzarkov 98ではShafarevich conjectureへのnegativeな意見(!)もある. }
 
より一般に群準同型 \( \rho : \pi_1(X) \to \Gamma \)によって, 
\[ \pi_1(Z') \to \pi_1(X) \xrightarrow{\rho} \Gamma \text{ has finite image } \]
というclosed analytic subsets \( Z \subset  X \)を潰すmapがある. 
 それが Campana-Kollár の \( \Gamma \)-reduction(Shafarevich map) である. 
 
\begin{thm}\cite[Theorem 3.19]{Voi}
\label{thm-Voi-3.19}
%\label{thm:3.19}
 \( X \) projective manifold/コンパクト Kähler manifoldとし, 
群準同型 \( \rho : \pi_1(X) \to \Gamma \)とする. 

このときfibration \( \varphi_\rho : X \dashrightarrow Y \)で次を満たすものが,  $Y$の双有理同値を除いて一意に存在する. 
\begin{enumerate}
\item very general point \( x \in X \)について, $x$を通る \( \varphi_\rho \) のファイバー\( X_x \)は
\[ \pi_1(\widetilde{X_x} ) \to \pi_1(X_x) \to \pi_1(X)\xrightarrow{\rho} \Gamma \text{ has finite image } \]
となる. ここで$\widetilde{X_x}  \to X_x$は特異点解消である. 
\item very general point \( x \in X \) , \(x \in  Z \subset X \)なるsubvarietyについて, 
\[ \pi_1(\widetilde{Z} ) \to \pi_1(Z) \to \pi_1(X)\xrightarrow{\rho} \Gamma \text{ has finite image } \]
ならば,  \( Z \subset X_x \). ここで$\widetilde{Z}  \to Z$は特異点解消である. 
\item  \( \varphi_\rho \) はalmost holomorphicである. 
\end{enumerate}
\end{thm}

\begin{rem}\cite[Remark 3.20]{Voi}
%\label{rem:3.20}
(3)から\( \varphi_\rho \) の一般ファイバーはsmoothである. よって(1)において,  \( \widetilde{X_x} = X_x \)として良い.
\end{rem}

\begin{proof}[Sketch of proof of \ref{thm-Voi-3.19}]
 (3)は(2)からでる. 
\( \tau : \widetilde{X} \to X \) を \( \varphi_\rho \) の不確定点除去とする. 

 \[
\xymatrix@C=50pt@R=20pt{
\widetilde{X}\ar@{->}[d]^{\tau}\ar@{->}[r]^{ \widetilde{\varphi_\rho}} & Y \\
X\ar@{-->}[ru]^{\varphi_\rho}& 
}
\]
  \( \varphi_\rho \) がalmost holomorphicでないと仮定する. 
  ある例外因子\( E \subset \widetilde{X} \) で\(\widetilde{\varphi_\rho}(E)= Y \)となるものがある. 
\( \tau|_{E} : E \to X\)のファイバーは \( \varphi_\rho \)で潰されない. 
そのファイバーはRCより, 有理曲線 \( \mathbb{P}^1 \simeq R \to \widetilde{X} \)で, $R \subset (\tau|_{E})^{-1}(x)$となり, 
 \begin{equation}
\label{eq2-thm-Voi-3.19}
\mathbb{P}^1 \simeq R \to  \widetilde{X} \overset{\varphi_\rho }{\rightarrow}\to Y 
\end{equation}
が$Y$のvery general pointを通るものが存在する. 
そこで
 \[
\xymatrix@C=50pt@R=20pt{
\widetilde{X}_R   \ar@{->}[r]^{\text{resolution}}\ar@/^1pc/@{->}[rr]^{p_1}
& R \times_Y \widetilde{X}  \ar@{->}[d] \ar@{->}[r]& R \ar@{->}[d] \\
& \widetilde{X} \ar@{->}[r]^{\widetilde{\varphi_\rho}} & Y \\
}
\]
とする.
ここで  \( \varphi_\rho \)の
一般ファイバーは既約なので,  \( R \times_Y \widetilde{X} \) はたった一つの既約成分で$R$にdominantするものがある. そいつのresolutionを \( \widetilde{X}_R \) とする. 

\( R^0 \subset R \)を
 \( p_1 : \widetilde{X}_R \to R \times_Y \widetilde{X} \to R \)のregular locusとする. 
 すると
 \begin{equation}
\label{eq-thm-Voi-3.19}
\pi_1(\widetilde{X}_t) \to \pi_1(\widetilde{X}_{R^0})=\pi_1(p_{1}^{-1}(R_0))= \to \pi_1(R^0) \to 1, \quad t \in R^0.
 \end{equation}
である. 
一方$R \subset X$なので, ある曲線\( R' \subset \widetilde{X}_R \) で \( R \) に落ちるものがある. 
% [page 4]

$R \subset (\tau|_{E})^{-1}(x)$なので, 
${R'}^{0} \to R^{0} \to \{x\}$となる, つまり$R'$は$\tau$で潰されるので, 
\[
 \pi_1({R'}^0) \to\pi_1({R'}^0) \to \pi_1(X) 
\] 
というmapは自明である.
一方で
\begin{itemize}
\item \( \pi_1({R'}^0) \to \pi_1(R^0) \)の像はfinite index, 特に$ \pi_1(R^0) \to \Gamma$は有限の像を持つ
\item \( \pi_1(\widetilde{X_t}) \to \Gamma \)は有限の像を持つ
\end{itemize}
であるので, (\ref{eq-thm-Voi-3.19})から
\[ \pi_1(\widetilde{X}_{R^0}) \to \Gamma  \text{ has finite image } \]
である.
 \( \widetilde{X}_R \) smoothなので \( \pi_1(\widetilde{X}_{R}) \to \pi_1(\widetilde{X}_{R^0}) \) は全射になる. よって
\[ \pi_1(\widetilde{X}_{R}) \to \Gamma  \text{ has finite image } \]
である. 

(2)から$\widetilde{X}_{R}$は$ \varphi_\rho : \widetilde{X} \to Y$のあるファイバーに入る.
これは$R$の取り方, \ref{eq2-thm-Voi-3.19}の付近に矛盾する($R$は$Y$の一般点を通る)
これは(2)の最大性に矛盾する. その際に\( \pi_1(\widetilde{X}_{R}) \to \pi_1(\widetilde{X}_{R^0}) \)が全射と \( \widetilde{X}_R \) が smoothを使う.

 fibrationを作るには, 次の補題がいる.
\begin{lem}\cite[Lemma 3.21]{Voi}
%\label{lem:3.21}
\label{lem-Voi-3.21}
\( Z \subset X \) subvariety, 
\( \varphi_W : \mathcal{W} \to X, \pi : \mathcal{W} \to B \) を
$X$のclosed subvariety \( W_b \subset X \)の族で, quasi-projectiveな \( B \)でパラメトライズされているとする. 

さらに次を仮定する. 
\begin{enumerate}
\item 一般の \( b \in B \)について, 
\[ \rho \circ (i_{b})_{*} : \pi_1(\widetilde{W_b}) \to \pi_1(W_b) \to \pi_1(X) \overset{\rho}{\to}\Gamma  \text{ has finite image } \] 
ここで\( i_b : \widetilde{W_b} \to W_b \to X \) をresolutionと包含写像の合成写像とする; 
\item 任意の \( b \in B \)について,  \( W_b \cap Z \neq \varnothing \).
\item \[ \rho \circ j_* : \pi_1(\widetilde{Z}) \to \pi_1(Z) \to \pi_1(X) \overset{\rho}{\to}\Gamma \text{ has finite image } \] 
ここで \( j : \widetilde{Z} \to X \)をresolutionと包含写像の合成写像とする; 
\end{enumerate}
このとき$W:= \overline{\cup_{b \in B} W_b }^{\mathrm{zar}} \subset X$とすると, smooth projective model \( \widetilde{W} \to W\) について,  \( \pi_1(\widetilde{W}) \to \Gamma \) has finite imageである. 
\end{lem}
要は$Z$がhas finite imageかつ$W_b$がhas finite imageならばそれらをunionした$W:= \overline{\cup_{b \in B} W_b }^{\mathrm{zar}} \subset X$もまたhas finite imageということ. 

\begin{proof}
\( \widetilde{\varphi_W} : \widetilde{\mathcal{W}} \to X \)  を$\mathcal{W}$のsmooth projective modelとする. blowupをとりまくって, \( \widetilde{\varphi_W} : \widetilde{\mathcal{W}} \to X \)は \( \psi_W : \widetilde{\mathcal{W}} \to \widetilde{W} \)を経由するとして良い. 

 \[
\xymatrix@C=50pt@R=20pt{
\widetilde{\mathcal{W}} \ar@{->}[d]_{\psi_{W}}\ar@{->}[rr]
\ar@{->}[rrd]^{\widetilde{\varphi_W}}%\ar@/^1pc/@{->}[rr]^{p_1}
& & \mathcal{W}  \ar@{->}[r]^{\pi} \ar@{->}[d]^{\varphi_W} & B\\
\widetilde{W} \ar@{->}[r]&W:= \overline{\cup_{b \in B} W_b }^{\mathrm{zar}} \ar@{^{(}-_>}[r]
& X & \\
}
\]
\ref{lem-Voi-1.2}から
 \( \rho \circ \varphi_{W*} : \pi_1(\widetilde{\mathcal{W}}) \to \pi_1(\widetilde{W}) 
 \text{ has finite image } \)
 なので, 
 \begin{equation}
 \label{eq-lem-Voi-3.21}
  \rho \circ \varphi_{W*} : \pi_1(\widetilde{\mathcal{W}}) \to \Gamma  \text{ has finite image } 
 \end{equation}
 を示せば良い. 
 
\( \widetilde{\mathcal{W}} \) は, Zariski open set \( \widetilde{\mathcal{W}}^0 \) で
\( \mathcal{W} \)のresolutionになり,  \( \widetilde{\mathcal{W}}^0 \to B \) がsmooth proper fibrationとなるものを持つ.
openに制限すると, $\pi_1$は全射\( \pi_1(\widetilde{\mathcal{W}}^0) \twoheadrightarrow \pi_1(\widetilde{W}) \)になるので, 初めから,  \( \widetilde{\mathcal{W}}^0 =\widetilde{\mathcal{W}} \)を仮定して良い. 

 (ii)の仮定から, subvariety \(\Sigma \subset \widetilde{\mathcal{W}} \) で\( B \)へdominantに写るものがある.$\Sigma$をblowupしまくって, $\Sigma$はsmoothかつ, \( \varphi_\Sigma : \Sigma \to Z \) は \( \psi_\Sigma : \Sigma \to \widetilde{Z} \)を経由するとして良い. 
   \[
\xymatrix@C=50pt@R=20pt{
\Sigma\ar@{=}[r] \ar@{->>}[d]^{\psi_\Sigma } 
& \Sigma\ar@{->}[d]^{\varphi_\Sigma} \ar@{->>}[r]^{\pi}
& B  \\
\widetilde{Z} \ar@{->}[r] &Z &  \\
}
\]
\( \pi_* : \pi_1(\Sigma) \to \pi_1(B) \) has finite index image. 
(3)より, 
 \begin{equation}
 \label{eq2-lem-Voi-3.21}
 \rho \circ \varphi_{W*} : \pi_1(\Sigma)\to \rho \circ j_* (\pi_1(\widetilde{Z})) \subset \Gamma 
 \text{ has finite image } 
 \end{equation}
よって,  \( \widetilde{\mathcal{W}} \to B \)はsmoothと仮定した(上の議論)ので, %$B^0 \subset B$を$\widetilde{W} \to W \overset{\pi}{\to} B$のregular locusとして
\[
\pi_1(\widetilde{\mathcal{W}}_b) \to \pi_1(\widetilde{\mathcal{W}}) \to \pi_1(B) \to 1
\]
が成り立つ. 
\ref{eq2-lem-Voi-3.21}(とその上)より$\pi_1(B) \to \Gamma$has finite image, また(1)より 
$\pi_1(\widetilde{\mathcal{W}}_b) \to \Gamma$ has finite imageである.
よって
\[ 
\pi_1(\widetilde{\mathcal{W}}) \to \Gamma  \text{ has finite image } 
\]
よって\ref{eq-lem-Voi-3.21}より言えた. 
\end{proof}

\textit{\ref{thm-Voi-3.19}の証明続き.}
$x \in X$について, 既約closed subvariety \( Z_x \subset X \) で \( x \in Z_x \)かつ, \( \pi_1(\widetilde{Z_x}) \to \Gamma \) has finite imageとなる最大のもの(一番次元が大きいもの)とする
するとvery general point $x \in X$について\( k=\dim Z_x \) となる$k \in \N$が存在する. 

この\( Z_x \) が唯一で, u \( X \dashrightarrow B \) のファイバーとなっていることを示す. 
\ref{thm-Voi-2.4}から
 \[
\xymatrix@C=50pt@R=20pt{
\mathcal{Z} \ar@{->}[d]^{f}\ar@{->}[r]^{ \varphi} & X \\
Y & 
}
\]
で\( f : \mathcal{Z} \to Y \) proper, \( \varphi : \mathcal{Z} \to X\) dominantとなるものがある. 
%(and may be assumed to be finite, and )
very general point $x \in X$について, ある$y \in Y$があって, 
\( Z_x \) と \(\mathcal{Z}_y \)が同一視される. 
%\( f(Z_y) \ne f(Z_y') \) for distinct \( y, y' \).
% [page 5]
%\( y \) general and \( y' \ne y \), as otherwise there is a map \( Y \to Y' \) such that the family of varieties \( f(\mathcal{Z}_{y'}) \) factors through a family parameterized by \( Y' \). 
\( \mathcal{Z} \) のresolution \( \widetilde{\mathcal{Z}} \)を取り, 
$ \widetilde{\mathcal{Z}}  \to \mathcal{Z} \overset{f}{\to} Y$のregular locusを$Y_0$とする. 

 \( x \in X \) がvery generalより, \( Z_x \cong \mathcal{Z}_y \)となる$y \in Y$は$Y^0$の元であると仮定してよく, 
  \[ \pi_1(\widetilde{\mathcal{Z}}_y) \to \pi_1(\mathcal{Z}_y) \overset{\phi_*}{\cong}\pi_1(Z_x)
  \to 
  \Gamma  
  \text{ has finite image for any \( y \in Y^0 \)}
  \] 
  となる. 
  
   \( \varphi : \mathcal{Z} \to X \)がbirationalであることを示す. 
  背理法.
\( f \) がbirationalでないとすると
\(\mathcal{Z}_y \subset  f^{-1}(f(\mathcal{Z}_y)) \subset \mathcal{Z} \)を考えれば, 
ある既約成分$T \subset f^{-1}(f(\mathcal{Z}_y))$ で\(\phi: T \twoheadrightarrow Z_x\)かつ, $B:=f(T)$が1次元以上になるものが存在する. 
%すると\ref{lem-Voi-3.21}の仮定が$T, B$において満たされている. 
%実際begin{itemize}\item $b \in B$general pointについて, \item $b \in B$について$T_b$\end{itemize}
 \( \mathcal{Z}_B := f^{-1}(B) \)とすると, \ref{lem-Voi-3.21}をが適応できて, 
 \( \phi(\mathcal{Z}_B) =:Z' \subset X \)が \( Z_x\)を真に含み,  \( \pi_1(Z') \to \Gamma \) has finite imageとなるものとなる. 
 $x \in Z_x$で\( Z_x \) は最大の次元なので矛盾する. 
\end{proof}

\begin{rem}\cite[Remark 3.22]{Voi}
%\label{rem:3.22}
\( S \) を \( K3 \) surface ,  \( \iota_S \) を自由に作用するinvolutionとする
($S/\iota_S$はEnriques surfaceとなる.)
\( C \)を超楕円曲線でそのinvolutionを \( \iota_C \)とする.
 \[ X := S \times C / \langle (\iota_S, \iota_C) \rangle \]
 とする. すると次がわかる.
 \begin{itemize}
 \item $\pi_1(X)$は無限. 実際$S \times C \to X$がdegree 2のetale coverになり, $\pi_1(S \times C)\cong \pi_1(C)$で無限であるので.
 \item \( f : X \to C / \iota_C \cong \mathbb{P}^1 \) で一般ファイバーが$S$と同型. K3曲面$S$は単連結なので, $id: \pi_1(X) \to \pi_1(X)$に対応する Shafarevich map ($\Gamma$-reduction map)は$f$である.
  \end{itemize}
  
  つまりこの例からShafarevich map $X \to Y$のbase $Y$について
  \begin{itemize}
  \item $Y$はgeneral typeになりえず, 小平次元$- \infty$にもなりうる. また, $f$はsmooth射なので, Orbifold Baseを考えてもgeneral typeになり得ない. 
  \item $Y$の基本群も無限になり得ない. むしろ単連結になりうる. 
  \end{itemize}

\end{rem}



\begin{thm}[{Campana 95, Katzarkov 97, \cite[Theorem 3.23]{Voi}}]
\label{thm-Voi-3.23}
%\label{thm:3.23}
 \( \pi_1(X)\) がnilpotent ならば, 
 Shafarevich fibration は Albanese map \( X \to Y \subset  \operatorname{Alb} X \)のStein factorization\( X \to Y_{\text{st}} \)で与えられる. 
 
 また \( \phi : X' \to Y' \) を \( X \to Y_{\text{st}} \) の特異点解消(つまり \( X' \) と\( Y' \)はsmoothで\( \phi \) がmorphismとなるもの)について, \( \pi_1(X) = \pi_1(X') \) かつ\( \phi_* : \pi_1(X') \to \pi_1(Y') \) が全射かつkernelが有限となる. 
 
    \[
\xymatrix@C=50pt@R=20pt{
X'\ar@{->}[r] \ar@{->}[d]^{\phi } 
& X \ar@{->}[d]^{\text{Sha}} \ar@{->}[rd]^{\text{Alb}} 
&   \\
Y'\ar@{->}[r]_{\text{resol}}& Y_{\text{st}} \ar@{->}[r]_{\text{finite}} & Y \subset \operatorname{Alb} X   \\
}
\]

\end{thm}
 \( X \to Y_{\text{st}} \) の特異点解消はつまり, $Y' \to Y_{\text{st}}$がresolutionで, $X'$を$X' \times_{Y_{st}}Y$の既約成分で$X$にdominantするもののresolutionである. 

$G$がnilpotentであるとは, 
正規部分群からなる有限の列
\[
{1}=G_0 \lhd G_1 \lhd \cdots \lhd G_n = G
\]
で
$G_{i+1}/G_i \le Z(G/G_i)=\{z \in G/G_i  \mid zg=gz \forall g \in G/G_i\}$となるものが存在すること.
\footnote{ $[G, G_{i+1}] \le G_i$でも可.  $[G, G_{i+1}] := \langle \{ [x,g] \mid x \in G, g \in G_{i+1} \} \rangle$である. なおこの$n$の最小値を冪零度(nilpotency class)という. }
同値な定義で$G_{i+1}=[G_i, G]$が有限の長さで自明群になるでもいい. 

$G$がsolvableとは"$G_{i+1}=[G_i, G_i]$が有限の長さで自明群になる”である. 
よってnilpotentならばSolvableである.逆は成り立たない(上三角行列)

\begin{proof}
以下はCampanaの証明. 
基本群などに関して次が言える. 
 \begin{itemize}
 \item \(\phi_* : \pi_1(X') \to \pi_1(Y')\)は全射になる. というのも \( \phi : X' \to Y' \)がファイバー連結なので \ref{lem-Voi-1.2}より.
\item  \( \phi_* : H_1(X', \mathbb{Z}) \to H_1(Y', \mathbb{Z}) \)は全射かつ, finite kernelである. 
 実際$X$ smoothなので, $\operatorname{Alb} X'=\operatorname{Alb} X=Y$であり, 
 \[ 
 \operatorname{alb}_* : H_1(X', \mathbb{Z}) \to H_1(\operatorname{Alb} X', \mathbb{Z}) 
 \] がfinite kernelを持つことから言える. 
\item \( \phi^* : H^2(Y', \mathbb{Q}) \to H^2(X', \mathbb{Q}) \)単射である. \ref{lem-Voi-1.2}(2)から.  \end{itemize}

\( H := \pi_1(Y') \)について
\[ n_{Y'} : H^2(H, \mathbb{Q}) =H^2( \pi_1(Y'), \Q)\to H^2(Y', \mathbb{Q}) \]
が単射であることを示す. (多分これは一般に言える. )
  \( E_H \to B_H = E_H / H \)を$H$の分類空間とする. 
   \( E_H \) は可縮である. 
すると 普遍被覆 \( \widetilde{Y'} \to Y' \) が主$H$束なので, 
$Y \underset{\text{homotopic}}{\sim}(\widetilde{Y'} \times E_H) / H $である. 
そこで \( u_{Y'}: Y \sim (\widetilde{Y'} \times E_H)/H \to B_H \)とする. 
すると\( n_{Y'} \) は
\[
u_{Y'}^* : H^2(B_H, \mathbb{Q}) \to H^2(Y', \mathbb{Q}) = H^2((\widetilde{Y'} \times E_H)/H, \mathbb{Q}).
\]
に等しい.\( u_{Y'} \)のfiberは\( \widetilde{Y'} \)で単連結である. 
よって, \( u_{Y'} \)のLeray spectral sequenceを計算すれば \( u_{Y'}^* : H^2(B_H, \mathbb{Q}) \to H^2((\widetilde{Y'} \times E_H)/H, \mathbb{Q}) \) は単射となる. 

よって\( \phi_* : H_1(X', \mathbb{Q}) \to H_1(Y', \mathbb{Q}) \)は単射なので上と合わせて
 \[ \phi_* : 
 H_2(X', \mathbb{Q}) \to 
  H_2(Y', \mathbb{Q}) \to 
 H_2(H, \mathbb{Q}) 
 \]
 は全射の合成で全射になる. 
 \( G = \pi_1(X') \)とすると,  \( \phi_* : H_2(G, \mathbb{Q}) \to H_2(H, \mathbb{Q}) \)も全射である. 
 これは以下の可換図式が成り立つため: 
\[
\begin{array}{ccc}
H_2(X', \mathbb{Q}) & \longrightarrow & H_2(Y', \mathbb{Q}) \\
\downarrow & & \downarrow \\
H_2(G, \mathbb{Q}) & \longrightarrow & H_2(H, \mathbb{Q})
\end{array}
\]

以上より
\begin{itemize}
\item  \( \phi_* : H_1(G, \mathbb{Q}) \to H_1(H, \mathbb{Q}) \)同型. 
\item   \( \phi_* : H_2(G, \mathbb{Q}) \to H_2(H, \mathbb{Q}) \)全射. 
\end{itemize}
であるので, \ref{lem-Voi-3.24}より
$$\pi_1(X') \to \pi_1(Y')
$$
は有限のkernelを持つ.(全射であることは前から言えてる)
\footnote{なおここで証明が終わっていた.Shafarevich mapになることは Katzarkovの論文\url{https://arxiv.org/pdf/alg-geom/9510009}参照. Albanese morpshim$X \to Y$のStein factorizationに$f : X' \to Y$について, $f(Z)=pt$であることと$H_1(Z, \Z) \to H_1(X, \Z)$ has finite imageであることが同値. (高山先生のサーベイなど参照) よって$\pi_1(X)$がAbelの場合, はAlbanese mapからShafarevich mapが作られる}
\end{proof}

\begin{lem}\cite[Lemma 3.24]{Voi}
\label{lem-Voi-3.24}
%\label{lem:3.24}
\( \alpha : G \to H \) を有限生成群の準同型とする. 
\( H_1(G, \mathbb{Q}) \simeq H_1(H, \mathbb{Q}) \) 同型かつ, 
\( H_2(G, \mathbb{Q}) \twoheadrightarrow  H_2(H, \mathbb{Q}) \)全射ならば, 
任意の \( n \in \N \)について,  \( G/G_n \to H/H_n \)は finite kernel and cokernelを持つ. 

ここで群$\Gamma$について, $n$-th lower central seriesを\( \Gamma_n = [\Gamma, \Gamma_{n-1}] \)とする. 
\end{lem}

\begin{cor}\cite[Corollary 3.25]{Voi}
\label{cor-Voi-3.25}
\( \pi_1(X) \) がvirtually nilpotentならば, Shafarevich conjectureは正しい. 
\end{cor}

\begin{proof}
finite étale coverを取ることで, \( \pi_1(X) \) がnilpotentであるとして良い. 
\ref{thm-Voi-3.23}と同じ記号を使う
  \[
\xymatrix@C=50pt@R=20pt{
X'\ar@{->}[r] \ar@{->}[d]^{\phi } 
& X \ar@{->}[d]^{\text{Sha}} \ar@{->}[rd]^{\text{Alb}} 
&   \\
Y'\ar@{->}[r]_{\text{resol}}& Y_{\text{st}} \ar@{->}[r]_{\text{finite}} & Y \subset  \operatorname{Alb} X =\operatorname{Alb} Y' \\
}
\]
 
\( \widetilde{A} \to \operatorname{Alb} X = \operatorname{Alb} Y' \)を普遍被覆とし, 
 \( Y'_{\operatorname{alb}} := Y' \times_{\operatorname{Alb}(X)} \widetilde{A} \) とする.
 $\widetilde{A} \cong \C^{\dim Y}$である. 
 すると
 \[
 Y'_{\operatorname{alb}} := Y' \times_{\operatorname{Alb}(X)} \widetilde{A}  \to \widetilde{A}  \cong \C^{\dim Y}
 \]
 はproperである. よって$Y'_{\operatorname{alb}}$は正則凸
 
 \ref{thm-Voi-3.23}より
 \( \phi _* : \pi_1(X') \to \pi_1(Y') \) は全射かつ, kernelは有限である. 
 そこで
 \( X'_{\operatorname{univ}}, Y'_{\operatorname{univ}} \)を \( X', Y' \)の普遍被覆とし, 
  \[
   \phi _{\operatorname{univ}} : X'_{\operatorname{univ}} \to Y'_{\operatorname{univ}} 
   \]
をinduced mapとする. 
"kernelは有限"ということから
\( X'_{\operatorname{univ}} \to X' \times_{Y'} Y'_{\operatorname{univ}} \)というfinite etale coverが存在する. 
特に \(    \phi _{\operatorname{univ}} \) はproperである. 
よって$ Y'_{\operatorname{univ}} $が正則凸を示せば良い. 
これは$Y'_{\operatorname{univ}}  \to Y'_{\operatorname{alb}}$もproperなのでいえた.\footnote{\xr{ここのproper性がわからなかった. $Y_{st} \to Y$ finiteだけで言える?} なお\cite{Voi}では"$  Y' \times_{\operatorname{Alb}(X)} \widetilde{A} $がSteinなので..."と続いていたが, これは嘘だと思う. 流石に正則凸くらいしかいえない. あとは$Y \subset Alb(X)$の場合にShafarevich conjectureが成り立つことが言えれば良いような気もする. }
\end{proof}

もっと強く次が言える. これはEyssidieuxらによる.
\begin{thm}\cite[Theorem 3.26]{Voi}
%\label{thm:3.26}
\( \pi_1(X) \)が faithful linear representationを持つならば,  \( \widetilde{X} \) は正則凸
\end{thm}


%Let us say that a projective or compact Kähler variety has a large fundamental group if for any positive dimensional closed algebraic subset \( Z \subset X \), with desingularization \( \widetilde{Z} \to X \), the natural map \( \pi_1(\widetilde{Z}) \to \pi_1(X) \) has infinite image. 
%We will say that the fundamental of \( X \) is generically large if the same is true for varieties passing through the very general point of \( X \). Theorem 3.19 can be rephrased saying that a variety has generically large fundamental group if and only if the Shafarevich fibration is the trivial fibration \( X \to X \). As we mentioned in Remark 3.22, it is unfortunately not the case that the base of the Shafarevich fibration has generically large fundamental group. Furthermore, only its very general fiber is controlled, and it could even be that special fibers have configurations of their components leading to a disproof of the Shafarevich conjecture. The paper \cite{29} constructs in a more controlled way, using nonabelian Hodge theory, the Shafarevich fibration associated with a given linear representation of the fundamental group, and introduces an orbifold structure \( Y_{\mathrm{orb}} \) on the base, such that the given representation of the fundamental group of the original variety \( X \) factors through the orbifold fundamental group. This is the starting point of the works \cite{28}, \cite{29}, the second crucial ingredient being again abelian Hodge theory related to the given faithful representation of \( \pi_1(X) \). This allows the authors to construct a plurisubharmonic exhaustion function on the universal cover of the orbifold or stack \( Y_{\mathrm{orb}} \), in order to show that it is Stein.

% [page 6]

\subsection{Special varieties and the core}

\subsubsection{Morphisms of general type}

Campana \cite{Cam04}の初めにある例から見ていく

\begin{lem}\cite[Lemma 4.1]{Voi}
\label{lem-Voi-4.1}
%\label{lem:4.1}
$C$を種数2以上の超楕円曲線で involutionを\( \iota \)とする. 
さらに$E$を楕円曲線 \( E \) で \( \eta \in E\)をnon-zeroな2-torsion pointとする. 

するとinvolution \( \sigma := (\iota, \eta) \)は\( C \times E \)に自由に作用するので, 
曲面\( \Sigma := C \times E / \sigma \) で楕円曲線 \( E / \eta \)への射と,  \( f : \Sigma \to \mathbb{P}^1 = C / \iota \)でファイバーが楕円曲線 $E$であるものが存在する. 
  \[
\xymatrix@C=50pt@R=20pt{
C \times E \ar@{->}[d]^{\pi} \ar@{->}[r] ^{pr_1}&C \ar@{->}[d]^{\text{degree 2 finite}}_{r}\\
 \Sigma := C \times E / (\iota, \eta)  \ar@{->}[r]^{f}\ar@{->}[d]
&   \mathbb{P}^1 = C / \iota \\
E / \eta &\\
}
\]

\begin{enumerate}
\item $\Sigma$はgeneral type $B$への全射 \( \phi : \Sigma \to B \)を持たない.
\item \( D \subset \mathbb{P}^1 \) を$C \to \mathbb{P}^1$の分岐因子とする. 
このとき$D$のdegreeは\( 2g + 2 \)であり,  \( f : \Sigma \to \mathbb{P}^1\)は$D$上で2重のmultiple fibreを持つ
\item $\pi : C \times E \to \Sigma$を商写像とする.
 このとき\( \pi^{*}f^* K_{\mathbb{P}^1}(\frac{1}{2} D)=\mathrm{pr}_1^* K_C \) である. 
\end{enumerate}
\end{lem}

\begin{proof}
(1) \( f  : \Sigma \to \mathbb{P}^1\)のファイバーは楕円曲線であるので, 
曲面の分類から\( \kappa(\Sigma)=1 \) である. 
よってgeneral type surfaceへの全射は存在しない.

今 \( \phi : \Sigma \to B \)を種数2以上の曲線\( B \)への全射とする. 
$f$のファイバーは楕円曲線なので, 任意の$x \in \mathbb{P}^1$について$\phi(f^{-1}(x))$は$B$上の点となる. 
(楕円曲線から$B$への射は存在し得ない). 
よって剛性定理から, $\mathbb{P}^1 \twoheadrightarrow B$という射が作れるが, それはあり得ない. 

(2) Hurwitzの公式から$2g-2 = 2\cdot( -2)+\sum_{q: \text{$C \to \mathbb{P}^1$の分岐点}}(e_q -1)$で
$e_q=1$より, 分岐点の個数は$2g+2$個である. 
よって$\deg D = 2g+2$である. 
また\( \sigma = (\iota, \eta)  \) は固定点を持たないので, 
 \( f \) はlocally に \( f \circ \pi : C \times E \to \mathbb{P}^1 \)と同じである. 
 よって$p \in D$について\( f \circ q \)はmultiples fibers \( 2E \times p\)を持つ. 
 
(3)  \( r : C \to \mathbb{P}^1 \) を商写像とすると,  Hurwitzの公式から
\( K_C = r^* K_{\mathbb{P}^1}(\tfrac{1}{2} D) \)となり言える. 
\end{proof}

hyperbolicityの観点からすると \( \Sigma \) と \( C \times E \)は同様の振る舞いをする.
%\footnote{一応下に説明があるが, Voison先生が言うような似た性質はわからんかった.}
実際diskや \( \mathbb{C} \) から \( \Sigma \) への正則写像は \( C \times E \)にliftする. 
よって$\Sigma$の小林擬距離は退化するけれども恒等的に0にはなり得ない. 
%it is the pull-back of a certain pseudometric on \( \mathbb{P}^1 \) which takes into account the divisor \( D \).

%Campana introduced special varieties in \cite{20}. The idea is that these varieties should not map in any way to a variety of general type and the conjecture is that these varieties are those which have a trivial Kobayashi pseudometric. However Lemma 4.1 above shows that the surface \( \Sigma \) constructed above should not be considered as special. The crucial corrected definition 4.8 below was introduced by Campana. We first define a \emph{morphism of general type}.

\begin{defn}\cite[Definition 4.2]{Voi}
\label{defn-Voi-4.2}
%\label{def:4.2}
 \( X ,  Y \)をprojective manifold /コンパクト K\"ahler,  
\( \phi : X \to Y \) を dominant proper 射かつ, $k := \dim Y$とする. 

\( \phi \) がgeneral typeであるとは, 次のsaturated rank 1 subsheaf
\[
\mathcal{L}_\phi := (\phi^* K_Y)_{\mathrm{sat}} \subset \Omega_X^k
\]
について$\kappa( \mathcal{L})=k$となること.

rational map \( \phi : X \dashrightarrow Y \)について, \( \phi \)がgeneral typeであるとは, 
ある\( \phi \)の特異点解消 \( \widetilde{\phi} : \widetilde{X} \to Y \) がgeneral typeとすることとする.
\end{defn}
ある\( \phi \)の特異点解消の部分は"任意"にしても実際は問題ない. 

\begin{rem}\cite[Remark 4.3]{Voi}
%\label{rem:4.3}
Bogomolovの定理\ref{thm-Voi-1.13}から \( \kappa( (\phi^* K_Y)_{\mathrm{sat}}) \le k\)はわかっている. 
\end{rem}

\( \phi : X \to Y \) や\( \phi : X \dashrightarrow Y \)がgeneral typeであっても$Y$に関してはあまりよくわからない. 
が, Campanaのアイデアは, $\phi$にある条件をつければ, orbifold構造を考えることで, $Y$がlog general typeになる. 
 
\( \phi : X \to Y \) を射とする. 
$\phi$によって定まる$Y$のorbifold structureを, 
$Y$上の\( \mathbb{Q} \)-divisor \( \Delta_\phi \) を以下に定めることで定義する. 
任意のprime divisor \( D \subset Y \)について
 effective divisorを 
 \[ \phi^*(D) = \sum_E m_E E + D' \]
と分解する.  ここで\( D' \)は$D$をdominantしないものとする. 
 そこで
 \[
 m_D := \inf_{E \subset X} m_E 
 \quad \text{and} \quad
\Delta_\phi := \sum_{D \subset Y} \frac{m_D - 1}{m_D} D.
\]
として定義する. 

簡単な例は$X, Y$が曲線のとき. 
\( \phi : X \to Y \) がlocallyに \( v : U \to V, z \mapsto z^l \)とかけているとき, 
local multiplicity は \( l \)であり, 
\[
v^*(K_V) = K_U(- (l - 1)[0])
 \quad \text{and} \quad
 v^*([0]) = l[0]
\]
であるので,  \( v^*(K_V(\frac{l-1}{l} [0])) = K_U \)となる. 

\( \phi : X \to Y \) が finiteのとき, 構成から
 \( K_X - (\phi^* K_Y(\Delta_\phi)) \) は effectiveである. 
 一方$\dim X  - \dim Y >0$のとき, birational mapで変換したら, 同様のことは成り立つ.
 一般には \( (Y, \Delta_\phi) \) のKodaira次元は \( (X, \phi) \)の birational invariantではない.
 %(何かしら\( \phi \)に条件がいる. この問題もCampana ( \cite{Cam04})にある)

\begin{defn}\cite[Definition 4.4]{Voi}
%\label{def:4.4}
\label{defn-Voi-4.4}
射 \( \phi : X \to Y \) が \emph{neat}であるとは, 
あるbirational morphism \( u : X\to Z\) があって, $Z$がsmoothであり
任意のdivisor \( D \subset X \) が,  \( \phi : X\to Y \)で潰される(つまり$\codim_Y  \phi(D) \ge 2$となる)ならば, 
 $u : X \to Z$でも潰される. 
   \[
\xymatrix@C=50pt@R=20pt{
X  \ar@{->}[r]^{\phi}\ar@{->}[d]^{u} &Y\\
Z  & \\
}
\]
\end{defn}

\begin{lem}\cite[Lemma 4.5]{Voi}
%\label{lem:4.5}
\label{lem-Voi-4.5}
任意のfibration \(\phi :  X \to Y \)は neat birational modelを持つ. 
つまり, 以下のsmooth varietyによる図式が成り立ち, $u, v'$がbirational, $\phi' : X' \to Y'$がneatとなるようなものが存在する. 
 \[
\xymatrix@C=50pt@R=20pt{
X'  \ar@{->}[r]^{\phi'}\ar@{->}[d]^{u} &Y'\ar@{->}[d]^{v'} \\
X  \ar@{->}[r]_{\phi}& Y\\
}
\]
\end{lem}

\begin{proof}
次の手順で構成する.
\begin{enumerate}
\item $\phi$のflatteningをとる. つまり,  \(  X_1 \to X,  Y_1 \to Y \)がprojective birational で \( \phi_1 : X_1 \to Y_1\)がflatとなるものをとる. 
\item \( Y_2 \to Y_1\)をresolutionとする
\item \( X_1 \times_{Y_1}  Y_2\)の既約成分で$X_1$をdominantする成分を取り, そのresolutionしたものを$\tau: X_2 \to X_1$とする.
\end{enumerate}
 \[
\xymatrix@C=50pt@R=20pt{
X_2  \ar@{->}[r]^{\phi_2}\ar@{->}[d]^{\tau}_{\text{bir}} &Y_2 \ar@{->}[d]^{\text{bir}} \\
X_1 \ar@{->}[r]^{\phi_1}_{\text{flat}}\ar@{->}[d]_{\text{bir}} &Y_1\ar@{->}[d]^{\text{bir}} \\
X  \ar@{->}[r]_{\phi}& Y\\
}
\]
$X_2, Y_2$はsmoothである. 
よってあとは \( \phi_2 : X_2 \to Y_2 \)がneatであることを言えば良い. 
\( D \subset X_2\)で$\phi_2$で潰されるとする. 
$X_2 \to X$で潰されることを示す. 

もし$D$が$\tau$で潰されないとすると, 
$\tau(D)\subset X_1$は$\phi_1 : X_1 \to Y_1$で潰される. 
しかしそれは\( \phi_1\) がflatなのであり得ない.
なぜなら
$\phi_1 : \tau(D) \to \phi_1(\tau(D))$に関して, \cite[Thm 2.1.18]{SCV7}や\cite[Prop 2.2.11]{SCV7}より
\begin{align*}
\dim_x \tau(D) 
& \underset{\text{常に成り立つ}}{\le} 
\dim_{\phi_1(x)}\phi_1(\tau(D)) + \dim_x \tau(D)|_{\phi_1^{-1}\phi_1(x)} \\
& \underset{\tau(D)|_{\phi_1^{-1}\phi_1(x)} \subset \phi_1^{-1}\phi_1(x)}{\le} 
\dim_{\phi_1(x)}\phi_1(\tau(D)) + \dim_x \phi_1^{-1}\phi_1(x) \\
& \underset{\text{flatならばequidimensional(open)}}{=} 
\dim_{\phi_1(x)}\phi_1(\tau(D)) + \dim X - \dim Y\\
\end{align*}
よって$\dim_x \tau(D) =\dim X -1$より$\dim Y - 1 \le \dim_{\phi_1(x)}\phi_1(\tau(D)) $を得る. 
\end{proof}

\begin{prop}\cite[Proposition 4.6]{Voi}
\label{prop-Voi-4.6}
%\label{prop:4.6}
\( \phi  : X \to Y\)が neatならば, \[ \kappa(Y, \Delta_\phi)  = \kappa( \mathcal{L}_\phi)\]
ここで\(\mathcal{L}_\phi := (\phi^* K_Y)_{\mathrm{sat}} \subset \Omega_X^p, p := \dim Y \)とする.
\end{prop}

\begin{proof}
十分大きな割り切れる$m$について, 
\(H^0(X,  \mathcal{L}_\phi^{\otimes m}) \)と 
\( H^0(Y, m(K_Y + \Delta_\phi)) \)の$\phi : X \to Y$によるpullbackが同一視できることを示せば良い. 
ここで$D \subset Y$divisorについて, 
 \begin{equation}
\label{eq-prop-Voi-4.6}
 \phi^*(D) = \sum_E m_E E + D' 
  \quad \text{and} \quad
 m_D := \inf_{E \subset X} m_E 
 \quad \text{and} \quad
\Delta_\phi := \sum_{D \subset Y} \frac{m_D - 1}{m_D} D.
 \end{equation}
と定義したが, $m$は$m_D$で割り切れ,  \( m(K_Y + \Delta_\phi) \) がdivisorになることを要請する. 
%As \( \mathcal{L}_\phi \) is saturated, we only have to worry about what happens in codimension 1 on \( X \). 


$$
\mathcal{L}_\phi := 
\mathrm{Ker}:  \Omega_X^p \to \Omega_X^p/(\phi^* K_Y) \to \frac{\Omega_X^p/(\phi^* K_Y) }{\Tor}
$$
であるので, 
\( \mathcal{L}_\phi \)と\( \phi^* K_Y \) 
がcodimension 1でどのような違いが出るかを見る. 
%$mE>1$となるdivisorでdiffer along divisors contracted by \( \phi \) and along divisors of multiple fibers.


(1). $\phi : X \to Y$で潰されるdivisor $E$について.
\( \phi  : X \to Y\)が neatなので\ref{defn-Voi-4.4}のような
birational morphism \( u : X\to Z\) をとる
   \[
\xymatrix@C=50pt@R=20pt{
X  \ar@{->}[r]^{\phi}\ar@{->}[d]^{u} &Y\\
Z  & \\
}
\]
\( E = \sum_i n_i E_i \)で$\phi: X \to Y$で潰されるものを取る. 
\ref{defn-Voi-4.4}より\( E_i \) は \( u : X \to Z\)で潰される. 
$Z$smoothなので, \ref{lem-Voi-1.6}より
$$
H^0(X \setminus \Supp(E), \Omega_X^p)
\overset{u^{*}}{\cong}
H^0(Z\setminus u(E), \Omega_Z^p)
 \underset{\codim u(E) \ge 2}{\cong} 
H^0(Z, \Omega_Z^p) 
\underset{\ref{lem-Voi-1.6}}{\overset{u^{*}}{\cong}} 
H^0(X, \Omega_X^p)
$$
となる. よってこのdivisor $E$では違いが出ない. 
%to conclude that it suffices to compare sections of \( \mathcal{L}_\phi^{\otimes N} \) and \( \phi^*(K_Y(\Delta_\phi))^{\otimes N} \) on \( X \setminus E \), and more generally, over \( Y \setminus W \) for any codimension \( \ge 2 \) closed analytic subset \( W \) of \( Y \). 

(2). $\phi : X \to Y$で潰されないdivisorについて.
$D \subset \Supp(\Delta_\phi ) \subset Y$となるdivisor $D$をとる. 
$D$を既約として良い.
generic pointでどうなるか見る. 
$$
 \phi^{*}(D)
 \underset{(\ref{eq-prop-Voi-4.6})}{=}
 \frac{m_D - 1}{m_D} \phi^{*}D
  \underset{(\ref{eq-prop-Voi-4.6})}{=}
    \sum_{E \subset X, \codim_{Y}\phi(E)=1} 
 m_E  \cdot \frac{m_D - 1}{m_D}  E + 
 \frac{m_D - 1}{m_D}  D' 
$$
($D'$は$\phi$で潰されるもの)となる. 
そこで$E_1 \subset X$で$\codim_{Y}\phi(E_1)=1$かつ$m_E = m_D$となるとする. 
すると,  \( m(K_Y + \Delta_\phi) \) のsectionのpullbackと \( \mathcal{L}_\phi^{\otimes m} \)のsectionは, \( \phi^* D - m_{E_1} E_1 \)が \( \phi^{-1}(D) \to D \)のfiberに沿ってeffectiveになるので, 一致する. 
\footnote{この部分は\cite[Section 1]{Cam04}の方がわかりやすい. $\phi_{0} : X_0 \to Y_0$をsmooth fibrationとなるlocus$Y_0 \subset Y$とすると,\(H^0(X_0,  \mathcal{L}_\phi^{\otimes m}) \)と 
\( H^0(Y_0, m(K_Y + \Delta_\phi)) \)の$\phi : X \to Y$によるpullbackが同一視できる. 
よってあとは$X \setminus X_0$でどうなるかを見る. 
$E \subset X \setminus X_0$のdivisorに関して
 $\phi$で潰されるdivisor$E$は(1)の議論から無視できて, $\phi$で潰されないdivisor$E$は(2)でうまいことなっているんだと思う. neatの条件は $\phi$で潰されるdivisor$E$を制御するために必要らしい. }
\end{proof}


\begin{lem}\cite[Lemma 4.7]{Voi}
\label{lem-Voi-4.7}
%\label{lem:4.7}
\( \phi : X \dashrightarrow Y \)がgeneral typeならば, almost holomorphic.
\end{lem}

% [page 7]

\begin{proof}
\( \phi \)の不確定点除去を \(\widetilde{\phi} : \widetilde{X} \to Y \) とし, 
$\pi : \widetilde{X} \to X$はsmooth center \( Z \subset X \)のblowupで得られるとする. 
 \( E \subset \widetilde{X} \)を例外因子とする. 
 示すことは\(\widetilde{\phi} (E) \neq Y \)である. 
 
背理法. もし\(\widetilde{\phi} (E) = Y \)ならば, 
saturated sheaf 
\[ \mathcal{L}_\phi := (\phi^* K_Y)_{\mathrm{sat}} \subset \Omega_{\widetilde{X}}^{\dim Y}\]
を\( E \) に制限することで, 
 \( \Omega_E^{\dim Y}\)はIitaka次元が$\dim Y$のrank 1 subsheaf $\mathcal{L}$を持つ. 
完全系列
\[
0 \to {\pi|_E}^* \Omega_Z \to \Omega_E \to \Omega_{E/Z} \to 0.
\]
を考える. 
 \( \pi|_{E} : E \to Z \)はprojective bundleなので,  $\mathcal{L} \to \Omega_E^{\dim Y}$は$Z$から来る. 
 (ここには$E \to Z$のfiberが \( \mathbb{P}^l \)であり,  \( \mathbb{P}^l \)はpluridifferential forms を持たないことを使う)%\xr{ここなんとなくわかるけどなんでだろ?}

すると\( \widetilde{\phi}|_{E} E \to Y\)は$Z \dashrightarrow Y$を経由する. 
よって$Z$のgeneral pointで$\phi$が定義されてしまい, これは$Z$が$\phi$の不確定点であることに矛盾する. \footnote{この証明も\cite[Section 2]{Cam04}の方がわかりやすい, \cite{Cam04}の証明は不確定点$Z$から$Y$への写像が誘導されてしまうことからわかる. }
\end{proof}

\ref{lem-Voi-4.5}と\ref{prop-Voi-4.6}より次がわかる. 
\begin{defn}[{\cite[Section 2]{Cam04}, \cite[Definition 4.8]{Voi}}]
\label{defn-Voi-4.8}
%\label{def:4.8}
projective manifold \( X \) が\emph{special}であるとは次の同値な3条件を満たすこと.
\begin{enumerate}
\item $X$の任意のbirational model \(\widetilde{X}\)がgeneral typeな射 \( \widetilde{X} \to Y \)を持たない. (つまり$X$がgeneral type rational mapを持たないということ)
\item $X$の任意のbirational model \(\widetilde{X}\) と$\dim Y>0$となる射 \( \phi: \widetilde{X} \to Y \)について,  \( \kappa(Y, \Delta_\phi) < \dim Y \).
\item \( X \) が\ref{defn-Voi-1.17}でのBogomolov subsheafを持たない
\end{enumerate}
\end{defn}
上の定義からspecialはbirational invariantである. 


\ref{lem-Voi-4.1}でのsurface \( \Sigma \)は\( f : \Sigma \to \mathbb{P}^1 \) で
$D_f :=\frac{1}{2}D$とすれば$\deg(K_{\mathbb{P}^1} + \frac{1}{2}D)=g-1>0$より
$$
\kappa(\mathbb{P}^1, \Delta_f)=\kappa(K_{\mathbb{P}^1} + \frac{1}{2}D) = 1= \dim \mathbb{P}^1 
$$
であるので, \ref{defn-Voi-4.8}の(2)に反してspecialではない. 


\begin{lem}\cite[Lemma 4.9]{Voi}
%\label{lem:4.9}
RCや$K_X \equiv 0$ならばspecial.
\end{lem}

\begin{proof}
$X$をRCとする. 
$\mathcal{L} \subset \Omega_{X}^{p}$ rank 1 saturated sheafについて, 
$\kappa(L)=-\infty$である. なぜならば任意の$s \in H^0(X, L)$について, $s \in H^0(X, \Omega_{p})$だが, $X$ RCなので, \ref{lem-Voi-3.13}と同じ証明により$s=0$である. よってBogomolov sheafは存在しないのでspecial.

$K_X \equiv 0$とし, $H$ ample line bundleとする. 
するとKEが存在するので, 
$\Omega_{X}^{p}$は$H$-polystableである. 
よって$\mathcal{L} \subset \Omega_{X}^{p}$ rank 1 saturated sheafについて, 
$$
L \cdot H^{n-1} \underset{\text{polystable}}{\le} \frac{c_1(\Omega_{X}^{p}) H^{n-1}}{\rk \Omega_{X}^{p}}=0
$$
となる. 
よって常に$\nu(L) \le 0$であるので, $\kappa(L) \le 0$となりBogomolov sheafは存在しない
\end{proof}
もっと強く$\kappa(X)=0$ならばspecialである.
この証明はIitaka予想型の定理\ref{thm-Voi-4.11}をcore fibrationに使う. 

\subsubsection{The core fibration}


\begin{thm}\cite[Theorem 4.10]{Voi}
\label{thm-Voi-4.10}
%\label{thm:4.10}
projective manifold \( X \)について, あるfibration 
\( \phi : X \dashrightarrow Y \) が存在して次を満たす. 
\begin{enumerate}
\item \( \phi \) は general type. 特に \( \phi \)は\ref{lem-Voi-4.7}よりalmost holomorphic.
\item $\phi$の一般ファイバーはspecial.
\item 任意のvery general point \( x \in X \)とsubvariety \( x \in X' \subset X \) について, $X'$の 特異点解消がspecialならば, $X'$は\( \phi \)の$x$を通るfiberに含まれる.
\item (universality) 任意のrational map \( X \dashrightarrow Y' \) で general typeなものは,  \( Y \)を通る. .
\end{enumerate}
\end{thm}

\begin{proof}[\textit{Sketch of the proof.}]
$p$をBogomolov subsheaf\( \mathcal{L} \subset \Omega_X^p \) が存在するようなもので最大の整数とする.
存在しないときは$p=0$とする.\footnote{この構成は\cite[Section 5]{Cam04}での構成方法. 最大のof general type fibrationがcore mapになるというもの. ただこの構成あってるかがわからない.(relative coreなどを帰納法で構成している)}

$p=0$ならば$X$ specialであり, core mapは定数写像である.
$p>0$のとき\ref{thm-Voi-1.13}より, 
rational map \( X \dashrightarrow Y \)で general type かつ $\dim Y=p$のものが存在する. 
 %One then has to prove that it satisfies all the properties (ii) to (iv). The key point is the proof of (ii), which will follow from Theorem 4.11 below by the following argument. 
 
 (2)の性質を示す. 
 general fiber がspecialでないと仮定して矛盾を示す.  
 次元による帰納法を用いて, 
 % するとThen by induction on dimension, it has a well-defined core fibration, so that one can construct a 
 relative core %fibration (over \( Y \))
\[
X \overset{\psi}{\dashrightarrow} Y' \overset{\chi}{\dashrightarrow}Y.
\]
があるとして良い.  relative core fibrationとは, very general point \( t \in Y \)について, 
\( \psi_t : X_t \dashrightarrow Y_t' \) が\( X_t \)のcore fibrationとなるものとする. 
$X_t$がspecialでないので$\dim Y_t' >0$となり, $\dim Y' > \dim Y$である. 

$\psi$のbirational model を取ることで次を仮定して良い. 
\begin{itemize}
\item \( Y' \)はorbifold structure \( \Delta_\psi \) をもち, general fiberで \( (Y_t', \Delta_{\psi_t}) \) はof general type
\item \( \chi : Y' \to Y \) は neat. 
\end{itemize}
よって$X \to Y$がof general typeなので \( Y' \to Y \)もそうなる.
\ref{thm-Voi-4.11}から,  \( (Y', \Delta_\psi) \) がof general typeになるが, それは$\dim Y$の最大性に矛盾する.
\end{proof}

\begin{thm}\cite[Theorem 4.11]{Voi}
\label{thm-Voi-4.11}
%\label{thm:4.11}
\( (Y', D) \) orbifold,  \( Y' \to Y \)  general type neat morphism とする. このとき次が成り立つ.
\[
\kappa(Y', D) \ge \kappa(Y_t', D_t) + \dim Y.
\]
\end{thm}

% [page 8]

\subsubsection{Conjectures}

%number field上の種数2以上の曲線の点は有限個(Faltings finiteness theorem) for points on curves of genus \( \ge 2 \) defined over a number field \cite{30}. There is a beautiful parallel developed by Lang between Brody curves on one hand and points defined over a number field on the other hand. The following conjectures are an example:

\begin{conj}[{Green-Griffiths, \cite[Conjecture 4.12]{Voi}}]
%\label{conj:4.12}
projective manifold $X$が general typeならば, Zariski dense entire curve $\C\to X$は存在しない.
\end{conj}
$K_X$ ampleでもopenである(rational curveが存在するかもしれないので)
%This conjecture is optimal since even when the canonical bundle of \( X \) is ample, there might be rational curves in \( X \). Note that it is not even known that surfaces with ample canonical bundle (e.g. quintic surfaces in \( \mathbb{P}^3 \)) contain only finitely many rational curves (despite positive results by Bogomolov under extra assumptions on Chern numbers, see \cite{9}). 

Abelian varietyのsubvarieties なら, Kawamata, Ochiaiより成り立つ. 
この場合はAbelian variety$A$のBrody curveが, 普遍被覆 のaffine line の射影になる. 
よってその$A$でのZariski closureはabelian subvarietyのtranslateとなる. 

数論的な Green–Griffiths conjectureの類似は次である

\begin{conj}[{Lang-Bombieri, \cite[Conjecture 4.13]{Voi}}]
%\label{conj:4.13}
\( X \) general type / number field \( K \). 
このとき \( K \)-points は Zariski denseではない. 
\end{conj}
Abelian varietyのsubvarietyの場合はFaltingsの定理.

%What should be the generalization of these conjectures to varieties of any Kodaira dimension? Again, it is generally conjectured that for varieties of Kodaira dimension \( \le 0 \), the Kobayashi pseudometric is trivial (see \cite{37}). On the arithmetic side, the following analogue is completely open:
$\kappa \le 0$ならば, Kobayashi 擬距離が0だと予想されている. 数論的な類似は次のとおり. 
\begin{conj}[{Campana, \cite[Conjecture 4.14]{Voi}}]
%\label{conj:4.14}
\( X \) projective variety/number field \( K \)
%function field of a complex curve. T
このとき, $X$の有理点は potentially denseである. つまり有限体拡大 \( L / K \) があって,  \( L \)-点がZariski dense.
\end{conj}

"potential density"はproper étale coverで不変である(Chevalley–Weilの定理). 
Lang–Bombieri 予想は次を意味する. 

\begin{conj}\cite[Conjecture 4.15]{Voi}
%\label{conj:4.15}
\( X \) projective manifold である étale cover \( X' \to  X \) で$X'$が general typeへのdominant rational mapを持つとする. 
このとき$X$は potentially denseではない. 
\end{conj}
上の状況では, \( X' \)並びに$X$はspecialではない(specialもproper étale coverで不変)
%Campana goes further and proposes that the obstruction to potential density lies in the existence of a rational map of general type, or equivalently, in the fact that \( X \) is not special.

\begin{conj}[{Campana, \cite[Conjecture 4.16]{Voi}}]
%\label{conj:4.16}
projective manifold / number fieldにおいて, potentially dense と specialは同値
\end{conj}


\begin{conj}[{Campana, \cite[Conjecture 4.17]{Voi}}]
%\label{conj:4.17}
projective manifold / number fieldにおいて, 小林擬距離が消えることと specialは同値
\end{conj}
Weakly specialという概念もあって, それは"任意のfinite étale coverが of general typeへのdominant rational mapを持たない"という定義である. 
specialならばweakly specialである.  (specialもproper étale coverで不変)
ただ逆は成り立たない. 以下のBogomolov-Tschinkelの例がある.
よって Conjecture 4.15 の仮定はnonspecialより強い. 

\begin{ex}\cite[Example 4.18]{Voi}
%\label{ex:4.18}
\underline{Weakly specialだけどspecialではない例}\footnote{\xr{Bogomolov-Tschinkelの例なのだが代数幾何学が難しすぎて全くわからなかった. 教えてください. }}
elliptic threefold\( \phi : X \to B \) で\( B \) がelliptic surfaceであり次を満たすものがある.
\begin{enumerate}
\item \( \pi_1(X) = \{1\} \).
\item  \( (B, \Delta_\phi) \) of general type.
\end{enumerate}

[作り方]
\begin{itemize}
\item 単連結elliptic surface \( E \to \mathbb{P}^1 \) で$\infty \in \mathbb{P}^1$で重複度2のファイバーを持つものをとる. 
\item elliptic surface \( S \)とrational map \( f : S \dashrightarrow \mathbb{P}^1 \)
 given by an ample enough Lefschetz pencil of curves on \( S \) and smooth fiber \( S_\infty \) over \( \infty \). %\xr{これLefschetz pencil がわからん. }
\item \( B \) を$S$のblowupで $f' : B \to \mathbb{P}^1 \) がwell definedになるものとする.
\item  \( X = B \times_{\mathbb{P}^1} E , \phi : X \to B \)
\end{itemize}

さて\( \Delta_\phi \) は \( B_\infty \) を重複度 \( \tfrac{1}{2} \)で含む. 
\( F \) を\( \tau : B \to S \)のexceptional divisorとすると, 
\[ B_\infty = r^* S_\infty - (F) \quad \text{and} \quad K_B = r^* K_S(F) \]
である. よって
\[ K_B(\tfrac{1}{2} B_\infty) = r^*(K_S(\tfrac{1}{2} S_\infty)) + \tfrac{1}{2} F \]
よって \( S_\infty \) ををporitiveにすると, \( (B, \Delta_\phi) \)の小平次元を2にすることができ, さらに \( X \) を単連結にできる. %\xr{ここ代数幾何が強すぎてわからん}

[Weakly specialだけどspecialではないこと]
Specialではないので(2)より.
 
Weakly specialに関しては$X$は自明なfinite etale coverしか持たない. よって$X$がof general typeへのdominant mapを持たないことを言えば良い. 
今dominant rational map $X \dashrightarrow T$で$T$がof general typeになるものとする. 
eliitptic curveから$T$へのmorpshimは自明%\xr{Kobayashi-Ochiai?}
よって, $B \dashrightarrow T$というdominant morpshimをえるが, やっぱりこれも自明になってしまい, $\dim T=0$となる. %\xr{これもKobayashi-Ochiai?}
\end{ex}

\newpage
\newpage
\section{Campana-Winkelman 23まとめ}

% [page 1]

\subsection{1. Introduction}


\begin{thm}\cite[Theorem A]{CW23}
\label{thm-CW23-A}
\( X \) rationally connected complex projective manifold, 
$M$を高々可算この点からなる集合, 
$D$をreduced hypersurface \( D \), 
 \( \pi : Y \to X \)非自明なramified coverとする.
 
 このときある正則写像\( c : \mathbb{C} \to X \) (entire curve)があって次を満たす. 
 \begin{enumerate}
\item \( c(\mathbb{C}) \) は\( X \)上でdense.
\item \( M  \subset c(\mathbb{C}) \),
\item \( c(\mathbb{C}) \) は\( D \) とどこでもtransversallyに交わる. 
\item  \( c \) は\( Y \)上のentire curveにはliftしない
\item order \( \rho_c \) (Nevanlinna 理論の意味で) は0である. 
\end{enumerate}
\end{thm}

"non-liftability"に関しては,  Corvaja-Zannierの予想の解決(らしい)

Lang予想の逆版として,  $X$をnumber field$k$上のprojective manifoldとするとき, "Zariski-dense entire curveの存在"が"ある有限拡大\( k' / k \)があって, \( X(k') \)はZariski dense"と同値というものがある. 
$X$がRCでも不明である. 

\ref{thm-CW23-A}において, 実はentire curveをgrowth order \( \rho_f = 0 \)としてとることができる. 
実際,  任意のRC\( X \)は dense entire curves \( h : \mathbb{C} \to X \) で \( \rho_h = 0 \)なものを有する.
逆に\cite{CW16}で示した通り. $n$次元のcomplex projective manifold \( X \) が non-degenerate な正則写像\( F : \mathbb{C}^n \to X \)で order \( \rho_F < 2 \) のものを有するとき, $X$ はRCである. 

$X$がKLT surface(normal projective surfaces with only quotient singularities)のとき次が言える:
\begin{tcolorbox}[mybox]
あるeffective non-zero \(\mathbb{Q}\)-divisor \(\Delta\) で \( K + \Delta \sim_\Q 0\) かつ
\((X, \Delta)\) log terminalならば, あるdense entire curve \( h : \mathbb{C} \to X \)  で$h(\C) \subset X_{\reg}$となるものがある. 
\end{tcolorbox}
証明は2次元のMMPを使う. 

% [page 2]

\subsection{2. Review of rationally connected manifolds}

\begin{defn}
 \( X \)をprojective manifoldとする. 

\begin{enumerate}
\item  \( X \) rational とは, \(X \) が\(\mathbb{P}^n \)とbirationalとなること, 
\item \( X \) unirational とは, あるnon-degenerate (dominant) なmeromorphic\( \Phi : \mathbb{P}^n \dashrightarrow X \)が存在すること. 
\item \( X \)がrationally connected とは, 任意の二つの$X$の一般点\( X \) が有理曲線で結ばれること.
\item \( X \)がFano とは, $-K_X$ ampleなること. 
\end{enumerate}
\end{defn}

1~3はbirational invariantである. 
\[
\text{rational} \Longrightarrow \text{unirational} \Longrightarrow \text{rationally connected.}
\]
\[
\text{Fano} \Longrightarrow \text{rationally connected.}
\]
である. 
曲線なら全て同値. 
2次元なら「$\text{rational} \Longrightarrow \text{unirational} \Longrightarrow \text{rationally connected.}$」は同値である. 

3次元ならunirationalなrationalでないものは存在する. 
ただRCだが, unirationalな例は不明. 
%例えば ‘general’ quartics in \(\mathbb{P}^4\), double covers of \(\mathbb{P}^3\) branched over a smooth sextic andstandard conic bundles over \(\mathbb{P}^2\) with smooth discriminant of large degree.
% [page 3]

\subsection{3. Brief review of special manifolds}

\( X \) を\( n \)-次元連結コンパクト complex K\"ahler manifoldとする. 
 Kodaira 次元 を\( \kappa(X) \)とする. 

\begin{defn}[{\cite[Definition 2.24 and Theorem 2.27]{Cam04}}] 
コンパクト Kähler manifold \( X \) が\emph{special} であるとは, 任意の$p>0$, 任意のrank 1 coherent sheaf
\( L \subset \Omega^p_X \)について, \( \kappa( L) < p \) となること. 
\end{defn}

わかっていることは次のとおり
\begin{itemize}
\item RC または\( \kappa(X) = 0 \) は special.
しかし$\kappa= -\infty, 1, \ldots, n-1$について, \( \kappa = k \)となるspecial variety, non-special varietyが共に存在する.
\item Core mapの存在. %We refer to \cite{Cam04} for details on this class of manifolds and for the reasons to introducethem, the main one being the existence of the ‘core map’ which splits any \( X \) into its oppositeparts: special vs general type. In fact, the decomposition \( c = (j \circ r)^n \) of the ‘core map’ (see \cite{Cam04}for details) shows that the ‘building blocks’ of special manifolds should be (smooth) ‘orbifoldpairs’ either ‘rationally connected’,\footnote{Here, rationally connected is in the sense of having \(\kappa^+ = -\infty\). When there is no orbifold structure, a classical conjecture claims that this is equivalent to rational connectedness.}or with \( \kappa = 0 \). 
\item \cite[Corollary 8.11]{Cam04}, \( \mathbb{C}^n \)-dominable (ある meromorphic \( \mathbb{C}^n \to X \) である点でsubmersive)ならばspecial. 逆は成り立たないと思われる(が弱いよそうならある.)
\end{itemize}


\begin{conj}[{\cite[9.2, 9.8, 9.5 and 9.20]{Cam04}}]
\label{conj-CW23-3.2}
コンパクト Kähler manifold \( X \) について次は同値であるだろう
\begin{enumerate}
\item special.
\item Kobayashi pseudo-metric $d_{kob, X}$は常に0
\item entire curve \( h : \mathbb{C} \to X \) でZariski-dense なものが存在
\item entire curve \( h : \mathbb{C} \to X \) でmetrically dense なものが存在
\item 任意の二つの一般点がentire curveで結べる
\item 任意の可算集合があるentire curveの像にふくまれる
\end{enumerate}
\end{conj}


% [page 4]

%This conjecture is motivated by the above-mentioned decomposition \( c = (j \circ r)^n \) of the coremap, which essentially reduces these conjectures to the two cases of orbifold pairs either rationallyconnected in the above sense or with \( \kappa = 0 \).



\begin{conj}[arithmetic analog]
\( X \) projective variety /number field \( K \). 
次は同値
\begin{enumerate}
\item[1. ] \( X \)は複素多様体と見てspecial 
\item[7. ] \( X \) potentially dense, つまりある有限拡大 \( k/K \)があって, \( X(k) \) はZariski dense
\end{enumerate}
\end{conj}
ちなみに 2024年にCampanaらの論文でこの予想はweakly specialに関しては正しくない(weakly specialに関して正しければ,abc予想と矛盾する)ことがわかった. なので, これかなり怪しいと思う. 


\begin{conj}[{\cite[Conjecture 7.1]{Cam04}}]
specialならば\( \pi_1(X) \)  almost abelian
\end{conj}

わかっていることは次のとおり
\begin{itemize}
\item (5)が意味しているのは"special varietyはRCみたいな性質を持つ"ということ. 有理曲線のtranscendental版が成り立つ?
\item 簡単にわかるのは次のとおり
\[
\xymatrix@C=30pt@R=30pt{
   (6)\ar@{=>}[r]\ar@{=>}[d]
   & (4)\ar@{=>}[r]\ar@{=>}[d]
   & (3)\\
   (5)\ar@{=>}[r]
   &(2)
   & \\
}
\]
\item  \( X \) がunirationalならば, 上の3つ全ての予想は正しい.
\item ある全射正則写像\( p : \mathbb{C}^n \to X \)があるならば, \ref{conj-CW23-3.2}は正しい. 
\item RCならば(2), (5)は自明に成り立つ. この論文では残りの(3), (4), (6)が成り立つことを示す. \ref{thm-CW23-4.2}参照. よってRCについて\ref{conj-CW23-3.2}は正しい. 
\item "Nevanlinna analog of the Hilbert irreducibility property"(\cite{CZ17}のCorvaja-Zannierで定義されたもの)"についても調べる.  complex torusでも成り立つ first Chern class が0の場合も調べる. 
\item \cite{BL00}(Buzzard-Lu 00) projective surface がspecialであることは \(\mathbb{C}^2\)-dominable
と同値. (ただし non-elliptic and non-Kummer K3 surfacesは例外). 3次元では正しいと思えないらしい.\footnote{"This might however be a low-dimensional phenomenon, and it is not expected to remain true in dimension 3."とあった. ということは, Oka$\Rightarrow $ Specialはもしかしたら正しくない?}
\end{itemize}


\begin{rem}[Vojta conjecture]
\( X \) projective variety /number field \( K \)とする. 
「\( k' \)-rational pointsがある有限拡大 \( k'/k \) で無限 であることは, \( X \) が entire curveを含む」というVojtaの予想がある. 
Vojta’s analogyとして"entire curves" と "arithmetic geometry"が関連しているらしい. 
上の"Nevanlinna version of the Hilbert property"(Section 9)はこれに近いことらしい. 
\end{rem}



\bibliographystyle{alpha}
\bibliography{ref_MY.bib}
\end{document}



















